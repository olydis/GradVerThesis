% IDEA: separate theoretical/general approach and concrete decisions
% this means, for example, that:
% PART1:
% - requirements for lifted functions, soundness
% - implication (gradual, ...)
% - abstract versions of [w/o], append
% PART2:
% - concrete version of [w/o], append, ... (probably requires notion of normalized env :( )
% - actual implementation

\chapter{Introduction}
\label{ch:introduction}
%% program verification
Program verification aims to check a compute program against its specification.
Automated methods require this specification to be formalized, for example using annotations or assertions in the source code.
Common examples are method contracts and loop invariants.
% Successfully verified programs are guaranteed to comply with  

%% static vs dynamic
There are a variety of approaches to check whether the program behavior complies with given annotations.
They can usually be divided into two categories:\\
\begin{tabular}{l | c | c}
     & Static verification & Dynamic verification \\ \hline
    Approach &
    \begin{minipage}{170pt}
        \vspace{3pt}
        The program is not executed. 
        Instead \textbf{formal methods} (like Hoare logic) are used, trying to derive a proof for given assertions.
        \vspace{5pt}
    \end{minipage}
    &
    \begin{minipage}{170pt}
        \vspace{3pt}
        The specification is turned into \textbf{runtime checks}, making sure that the program adheres to its specification during execution.
        Violations cause a runtime exception to be thrown, effectively preventing the program from entering a state that contradicts its specification.
        Note that in practice this approach is often combined with control flow based testing techniques to detect misbehavior as early as possible.
        \vspace{5pt}
    \end{minipage}\\\hline
    Drawbacks &
    \begin{minipage}{170pt}
        \vspace{3pt}
        The syntax available for static verification is naturally limited by the underlying formal logic.
        Complex conditions/behavior might thus not be expressible, resulting in inability to prove subsequent properties.
        Furthermore, the logic itself might be unable to prove certain properties due to complex behavior of the code and undecidability in general.
        Using static verification usually requires rigorous annotation of the entire source code, as otherwise there might be too little information to find a proof.
        While adding such annotations to own code can be tedious (yet, there are supporting tools), using unannotated libraries can become a problem:
        Even if it is possible to add own annotations, lacking the source code, the verifier is unable to prove those annotations, resulting in inconsistent proves in case the annotation turns out to be wrong.
        \vspace{5pt}
    \end{minipage}
    &
    \begin{minipage}{170pt}
        \vspace{3pt}
        Violations are only detected at runtime, with the risk of going unnoticed before software is released.
        To minimize this risk, testing methods are required, i.e. more time has to be spent after compilation.
        The usage of runtime checks naturally imposes a runtime overhead which is not always acceptable.
        % To get rid of 
        \vspace{5pt}
    \end{minipage}\\\hline
\end{tabular}

Most modern programming languages use static methods to some degree, ruling out at least some types of runtime failure.
%% static typing
Static typing disciplines are among the most common representatives, guaranteeing type safety at compile time, obviating the need for dynamic checks.
Yet, the rigidity and limitations of static type systems resulted in the introduction of dynamic aspects into the otherwise static system:
Casts (e.g. as implemented in C\# or Java) overrule purely static reasoning, allowing the static type system to treat an expression as if it had the claimed type (usually a subtype) instead of the deduced one.
At this location, a dynamic check is introduced, resulting in a cast exception should the programmer's claim turn out wrong.
Note that the necessity of casts is only due one of the drawbacks of static verification, namely the limitation of formal logic.
More sophisticated type systems (e.g. the one in Haskell) might have been able to deduce the claimed type in the first place.

%% static verification
In contrast, general purpose static verification techniques are not common amongst popular programming languages.
Note that such languages are usually driven by cost-benefit and usability considerations, meaning that static verification is apparently not yet in a stage where its cost clearly outweighs its benefits.


%% static verification
% example?
These limitations not only affect programmers trying to statically verify their program.
Most general purpose programming languages (C/C++, C\#, Java, ...), usually driven by cost-benefit and usability considerations, haven't adopted this level of static analysis in the first place.

%% grad verification
The purpose of gradual verification is to weaken if not remove some of these limitations at the cost of turning some static checks into runtime checks, whenever inevitable.
We will present a procedure of turning a static verification into a gradual one.
% more detail about static limitations and how runtime circumvents them?

%% static typing weakened
This idea is not new at all and actually common practice in type systems:
In C\# or Java, explicit type casts are assertions about the actual type of a value.
This actual type (usually a subtype of the statically known type) could not be deduced by the static type system due to its limitations.
Such an assertion/cast allows subsequent static reasoning about the value assuming its new type at the cost of an additional runtime check, ensuring the validity of the cast.
Note that such deviations from a “purely” static type system (one where there is no need for runtime checks) do not affect type safety:
It is still guaranteed that execution does not enter an invalid state (one where runtime types are incompatible with statically annotated types) by simply interrupting execution whenever a runtime type check fails.
This is usually implemented by throwing an exception.
% mention that runtime cost reasonable, ...


%% dyn typing
At the other end of the spectrum are dynamically typed languages.
In scenarios where the limitations of a static type system would clutter up the source code, they allow expressing the same logic with less syntactic overhead, but at the cost of less static guarantees and early bug detection.

%% PLs are on the dynamic end of the verification spectrum
In terms of program verification, most general purpose languages are on the dynamic end of the spectrum.
If they exist as designated syntax, assertions are usually implemented as runtime checks and often even dropped entirely for “release” builds (the Java compiler drops them by default).
It is common practice to implement 
% research Eiffel!
% Design-by-Contract!!! Eiffel!
% D even has both

% this is more of a consequence of the “deep roots” of dynamic verification!!!
%But even preconditions at expression level are implemented as runtime checks, reflected all the way down at instruction architecture level.
%Examples:
%\begin{description}
%    \item[Division by zero]~\\
%    Integer division performs a dynamic check...
%    
%\end{description}

%% grad typing
A gradual type system is more flexible, as it provides the full continuum between static and dynamic typing, letting the programmer decide ... %TODO.
It can be seen as an extension  “unknown” type 


This work will also show that gradual verification ... other angle!

- 
What is the thesis about?
Why is it relevant or important?
What are the issues or problems?
What is the proposed solution or approach?
What can one expect in the rest of the thesis?

“Static verification checks that properties are always true, but it can be difficult and tedious to select a goal and to annotate programs for input to a static checker.” (http://www.sciencedirect.com/science/article/pii/S1571066104002567)


    \section{Motivational Examples}
    \label{sec:motivationexamples}
    
\subsection{As extension to unverified setting}
“Make dynamic setting more static”

\definecolor{cogreen}{RGB}{0,80,0}
\lstset{
    language=Java,
    basicstyle=\ttfamily,
    commentstyle=\ttfamily,
    frame=single,
    framesep=5pt,
    mathescape=true,
    %escapeinside={(*}{*)},
    keywordstyle=\color{blue}\ttfamily,
    stringstyle=\color{darkgray}\ttfamily,
    commentstyle=\color{cogreen}\ttfamily,
    morekeywords={requires, ensures}
    }

Motivating example:
\begin{lstlisting}
boolean hasLegalDriver(Car c)
{
    return c.driver.age >= 18;
}
\end{lstlisting}

Motivating example with potential leak:
\begin{lstlisting}
boolean hasLegalDriver(Car c)
{
    allocateSomething();
    boolean result = c.driver.age >= 18;
    releaseSomething();
    return result;
}
\end{lstlisting}

Motivating example with argument validation:
\begin{lstlisting}
boolean hasLegalDriver(Car c)
{
    if (!(c != null))
        throw new IllegalArgumentException("expected c != null");
    if (!(c.driver != null))
        throw new IllegalArgumentException("expected c.driver != null");
        
    // business logic (requires 'c.driver.age' to evaluate)
}
\end{lstlisting}

Motivating example with declarative approach (JML syntax):
\begin{lstlisting}
//@ requires c != null && c.driver != null;
boolean hasLegalDriver(Car c)
{
    // business logic (requires 'c.driver.age' to evaluate)
}
\end{lstlisting}

There are two basic ways to turn this annotation into a guarantee:
\begin{description}
    \item[Static Verification (run ESC/Java \cite{leino2000esc})]~\\
    In the unlikely event that the verifier can prove the precondition at all call sites, our problem is solved.
    Otherwise, we have to enhance the call sites in order to convince the verifier.
    Choices:
    \begin{itemize}
        \item 
        Add parameter validation, effectively duplicating the original runtime check across the program.
        \item
        Add further annotations, guiding the verifier towards a proof.
        This might not always work due to limitations of the verifier or decidability in general.
    \end{itemize}
    There are obvious limitations to this approach, static verification tends to be invasive.
    At least there is a performance benefit: 
    Runtime checks (originally part of every call) are now only necessary in places where verification would not succeed otherwise.
    
    \item[Runtime Assertion Checking (RAC, run JML4c, TODO: http://www.cs.utep.edu/cheon/download/jml4c/doc.php)]~\\
    This approach basically converts the annotation back into a runtime check equivalent to our manual argument validation.
    It is therefore less invasive, not requiring further changes to the code, but also lacks the advantages of static verification.
\end{description}

\subsection{As extension to fully verified setting}
“Make static setting more dynamic”

\begin{lstlisting}
int collatzIterations(int iter, int start)
    requires 0 < start;
    ensures  0 < result;
{
    // ...
}

int myRandom(int seed)
    requires 0 < seed   && seed   < 10000;
    ensures  0 < result && result < 4;     // not provable
{
    int result = collatzIterations(300, seed);
    // we know: result $\in$ { 1, 2, 4 }
    
    if (result == 4) result = 3;
    return result;
}
\end{lstlisting}

Non-solution:
\begin{lstlisting}
int collatzIterations(int iter, int start)
    requires 0 < start;
    ensures  0 < result;
{
    // ...
}

int myRandom(int seed)
    requires 0 < seed   && seed   < 10000;
    ensures  0 < result && result < 4;
{
    int result = collatzIterations(300, seed);
    // we know: result $\in$ { 1, 2, 4 }
    
    // "cast"
    if (!(result < 5))
        throw new IllegalStateException("expected result < 5");

    // verifier now knows:  0 < result && result < 5 
    
    if (result == 4) result = 3;
    return result;
}
\end{lstlisting}
This solution is not satisfying, 
- much to write, have to think about what to write (requires you to kind of thing from verifiers perspective)
- intuitively the problem is with the method's postcondition being too weak, i.e. we solved the problem at the wrong place!


\begin{lstlisting}
int collatzIterations(int iter, int start)
    requires 0 < start;
    ensures  0 < result && ?;
{
    // ...
}

int myRandom(int seed)
    requires 0 < seed   && seed   < 10000;
    ensures  0 < result && result < 4;
{
    int result = collatzIterations(300, seed);
    // we know: result $\in$ { 1, 2, 4 }
    
    // verifier allowed to
    //  assume 0 < result && result < 5
    //  from   0 < result && ?
    // (adding runtime check)
    
    if (result == 4) result = 3;
    return result;
}
\end{lstlisting}

% gradual:
% - not in a sense of GraVy
% - assertions are GUARANTEED not to be violated (analogous to type safety of gradually typed languages), meaning that
%    - as much as possible is verified statically
%    - if necessary, (ideally: as few as possible) dynamic checks kick in
% - full continuum between static and dynamic

% static checking
% dynamic checking
% combine static and dynamic checking
% views:
% - add designated "?" to statically checked language, making checking optional
% - introduce checking to unchecked language, making "?" the default value
% typing
% transition to verification
% - of particular interest, cause limitations due to syntax or decidability make full checking impossible,
%    possibly making use of static verification impossible and the program thus unverifiable
%    inevitable incompleteness of static verifiers!
%    GraVy: measure, only \textit{classifies} code into sections: correctness statically guaranteed, correctness statically disproved, no static guarantee so far

% very desirable in practice
% gradual version of a language is inherently a strict superset of original language
% 

% our approach allows extension of existing languages (without any verification) by adding transpilation step

% why would you want continuum?
% - move toward satically verified program without giving up guarantees!
% - while "barely typed" languages hardly make sense, programs with "little verification" are already highly useful
%    i.e. verification even makes sense as "rarely used feature"!
%    ArgumentException-example
%    -    performance benefit!
%    -    cleaner code!
%    -    makes more sense semantically
% having optional types 

% Future Work:
% implement on top of existing languages
% - tracking access via ThreadLocalStorage
% - transpilation (e.g. C# Roslyn compiler extension)



\chapter{Background}
\label{sec:categorization-of-existing}
Design-by-Contract, a term coined by Bertrand Meyer \cite{meyer2002design}, is a paradigm aiming for verifiable source code, e.g. by adding method contracts and tightly integrating them with the compiler and runtime.
Meyer realized this concept in his programming language Eiffel, providing compiler support for generating runtime checks required for dynamic verification (often called runtime verification).
Combining design-by-contract with static verification techniques to was investigated by \cite{crocker2004safe} as what they call “verified design-by-contract”.

Similar developments took place regarding Java and JML annotations.
Static verification using theorem provers was investigated by \cite{jacobs2001logic} and is implemented as part of ESC/Java \cite{nelson2004extended}.
Turning the annotations into runtime assertion checks (RAC) to drive dynamic verification was investigated by \cite{cheon2002runtime} and lead up to the development of JML4c \cite{sarcar2010new}.

A more recent programming language that comes with integrated support for specification and both static and dynamic verification is Spec\# \cite{the-spec-programming-system-an-overview}.
Its compiler facilitates theorem provers for static verification and emits runtime checks for dynamic verification.
It was developed further with current challenges of concurrent object-orientation in mind \cite{a-statically-verifiable-programming-model-for-concurrent-object-oriented-programs}.
The concepts found their way to main stream programming in the form of “Code Contracts” \cite{embedded-contract-languages}, a tool-set deeply integrated with the .NET framework and thus available in a variety of programming languages.

The limitations of both static and dynamic verification lead to a recent trend of using both approaches at the same time.
Static verification is meant as a best effort service and supplemented with dynamic verification to give the guarantee that static verification potentially failed to provide.
Recent work focuses on combining both approaches in a more meaningful and complementary way by focusing dynamic verification and testing efforts specifically to code areas where static verification had less success.
\cite{ChristakisMuellerWuestholz16} describe how programs can be annotated during static verification in order to prioritize certain tests over others or even prune the search space by aborting tests that lead to fully verified code.

Still static and dynamic verification concepts are treated as independent for the most part.
The same was once true for static and dynamic type systems, before advances in formalizing gradual type systems seamlessly bridged the gap.
Our goal is to achieve the same for program verification, i.e. static and dynamic verification are no longer to be treated as independent concepts (that are combines as smart as possible) but instead treated as complementary and tightly coupled.

Note that \cite{arlt2014gradual} mentions gradual verification, yet it is meant as the process of “gradually” increasing the coverage of static verification.
The work describes a metric for estimating this coverage, giving the developer feedback while annotating and closing in on fully static verification.
A similar metric implicitly arises from our notion of gradual verification: The amount of dynamic checks injected to ensure compliance with annotations is a direct indicator of locations where static verification failed so far.

\section{Gradual Typing}
\label{ssec:abstracting-gradual-typing}
As this work is based on the advances in gradual typing, it is helpful to understand the developments in that area.

Gradual type systems originated from efforts to overcome limitations and drawbacks of purely static or dynamic type systems.
Corresponding extensions were proposed for .NET by Meijer and Drayton \cite{meijer2004static}, for Java by Gray et al. \cite{gray2005fine} and for Scheme by Bres et al. \cite{bres2004compiling}.
Siek and Taha provided a type-theoretic foundation, formalizing gradual typing for functional programming \cite{siek2006gradual}.
They describe a $\lambda$-calculus with optional type annotations, which is sound w.r.t. the simply-typed $\lambda$-calculus for fully annotated terms.
Static and dynamic type checking are seamlessly combined by automatically inserting runtime checks (casts) where necessary.
They later adapted their approach to object-based languages \cite{siek2007gradual}.

Based on their work, Wolff et al. introduced “gradual typestate” \cite{wolff2011gradual}, circumventing the rigidity of static typestate checking.
Schwerter, Garcia and Tanter developed a theory of gradual effect systems \cite{banados2014theory}, making it possible to incrementally annotate and statically check effects by adding a notion of unknown effects.
An implementation for gradual effects in Scala was later given by Toro and Tanter \cite{toro2015customizable}.

\label{grad-guarantee}
Siek et al. recently formalized refined criteria for gradual typing, called the “gradual guarantee” \cite{siek2015refined}.
The gradual guarantee states that well typed programs will stay well typed when removing type annotations (the static part of the guarantee).
It furthermore states that well typed programs evaluating to a value will evaluate to the same value (in lockstep) when removing type annotations (the dynamic part of the guarantee).

With “Abstracting Gradual Typing” (AGT) \cite{garcia2016abstracting} Garcia, Clark and Tanter propose a new formal foundation for gradual typing.
Their approach draws on the principles of abstract interpretation, defining a gradual type system in terms of an existing static one.
The resulting system satisfies the gradual guarantee by construction.
Subsequent work by Garcia and Tanter demonstrates the flexibility of AGT by applying the concept to a security-typed language, yielding a gradual security language \cite{garcia2015deriving}, which in contrast to prior work does not require explicit security casts.
Furthermore Lehmann and Tanter \cite{nico} applied the approach to refinement types, resulting in a gradual language that is able to deal with imprecise logical information and dependent function types.

\section{Hoare Logic}
\label{sec:hoare-logic}

...for static semantics

\cite{hoare1969axiomatic}

\section{Implicit Dynamic Frames}
\label{ssec:implicit-dynamic-frames}
%% intro, aliasing problem
Reasoning about programs using shared mutable data structures (the default in object orientation) is not possible using traditional Hoare logic.
The following Hoare triple should not be verifiable using a sound Hoare logic due to potential aliasing.
\begin{displaymath}
\hoare
{\phiAnd{\phiEq{p1.age}{19}}{\phiEq{p2.age}{19}}}
{\ttt{p1.age++}}
{\phiAnd{\phiEq{p1.age}{20}}{\phiEq{p2.age}{19}}}
\end{displaymath}
The problem is that \ttt{p1} and \ttt{p2} might be aliases, meaning that they reference the same memory.
The increment operation would thus also affect \ttt{p2.age}, rendering the postcondition invalid.
As we will demonstrate gradual verification on a Java-like language in chapter \ref{ch:case-study--implicit}, we need a logic that is capable of dealing with mutable data structures.

%% SL
Separation logic \cite{reynolds2002separation} is an extension of Hoare logic that explicitly tracks mutable data structures (i.e. heap references) and adds a “separating conjunction” to the formula syntax.
In contrast to ordinary conjunction (\ttt{∧}),
separating conjunction (\ttt{*}) ensures that both sides of the conjunction reference disjoint areas of the heap.
The following Hoare triple would thus be verifiable:
\begin{displaymath}
\hoare
{\phiCons{\phiMapsTo{p1.age}{19}}{\phiMapsTo{p2.age}{19}}}
{\ttt{p1.age++}}
{\phiCons{\phiMapsTo{p1.age}{20}}{\phiMapsTo{p2.age}{19}}}
\end{displaymath}
Note also the changed syntax explicitly tracking the values of certain heap locations.

%% IDF
A drawback of separation logic is that formulas cannot contain heap-dependent expressions (e.g. $\ttt{p1.age > 19}$) as they are not directly expressible using the explicit syntax for heap references.
Implicit dynamic frames (IDF) \cite{smans2009implicit} addresses this issue by decoupling the concept of access to a certain heap location from assertions about its value.
It introduces an “accessibility predicate” \ttt{acc(\textit{loc})} that represents the permission to access \textit{loc}.
Above example can be rewritten in terms of IDF:
\begin{displaymath}
\hoare
{\phiCons {\phiCons{\phiAcc{p1}{age}}{\phiAcc{p2}{age}}} {\phiCons{\phiEq{p1.age}{19}}{\phiEq{p2.age}{19}}}}
{\ttt{p1.age++}}
{\phiCons {\phiCons{\phiAcc{p1}{age}}{\phiAcc{p2}{age}}} {\phiCons{\phiEq{p1.age}{20}}{\phiEq{p2.age}{19}}}}
\end{displaymath}
The separating conjunction makes sure that the accessibility predicates mention disjoint memory locations, whereas it has no further meaning for “ordinary” predicates like equality.




Reasoning about concurrent programs is not possible using traditional Hoare logic.
Consider the following Hoare triple:
\begin{displaymath}
\hoare{\ttt{person ≠ \enull}}{\sSeq{\sFieldAssign{person}{age}{18}}{\ttt{person.age++}}}{\ttt{person.age = 19}}
\end{displaymath}
In case the programming language supports parallelism or any form of preemptive context switching, this Hoare triple should be deducible by a sound Hoare logic.
Other code may have access to \ttt{person} and thus overwrite the \ttt{age}-field concurrently, resulting in arbitrary values of when execution reaches the end of above two statements.


in the way it treats heap references.


Hoare logic has been extended in order to enable reasoning concurrent programs.

Race-free Assertion language! => static verification tool able to reason soundly about concurrent programs

% separating conjuction (= Multiplicative conjunction ⊗, lollipop, ...), access is resource, cannot duplicate access, ...
% TODO: explain Frame Problem
% translation of separating conjuction to regular conjunction? would simplify later reasoning A LOT



\cite{leino2009verification}
Chalice, a verification methodology based on implicit dynamic frames

Chalice’s verification methodology centers around permissions and permission transfer. In particular, a memory location may be accessed by a thread only if that thread has permission to do so. Proper use of permissions allows Chalice to deduce upper bounds on the set of locations modifiable by a method and guarantees the absence of data races for concurrent programs. The lecture notes informally explain how Chalice works through various examples.

also: Viper (Verification Infrastructure for Permission-based Reasoning; is a suite of tools developed at ETH Zurich, providing an architecture on which new verification tools and prototypes can be developed simply and quickly.) has Chalice as front-end

\cite{summers2013formal}
In this paper, we provide both an isorecursive and an equirecursive formal
semantics for recursive definitions in the context of Chalice

\cite{parkinson2011relationship}
VERY IMPORTANT: chapter 2.2

Finally, we show that we can encode the separation
logic fragment of our logic into the implicit dynamic frames fragment, preserving
semantics. For the connectives typically supported by tools, this shows that separation
logic can be faithfully encoded in a first-order automatic verification tool (Chalice).

Although IDF was partially inspired by separation logic, there are many differences
between SL and IDF that make understanding their relationship difficult. SL does not
allow expressions that refer to the heap, while IDF does. SL is defined on partial heaps,
while IDF is defined using total heaps and permission masks. The semantics of IDF are only defined by its translation to first-order verification conditions, while SL has a direct
Kripke semantics for its assertions.






\chapter{Gradualization of a Statically Verified Language}
\label{ch:gradualization-of-a}
%% why static->gradual
As illustrated in section \ref{sec:motivationexamples} gradual verification can be seen as an extension of both static and dynamic verification.
Yet, the approach of “gradualization” (adapted from AGT) derives the gradual semantics in terms of static semantics.
In this chapter we will thus describe our approach of deriving a gradually verified language “{\gvl}” starting with a generic statically verified language “{\svl}”.
An informal description of how to tackle the opposite direction can be found in section \ref{sec:enhancing-an-unverified}.

%% structure of this chapter
Section \ref{sec:a-statically-verified} contains the description of “{\svl}” or rather the assumptions we make about it.
In section \ref{sec:gradual-formulas} we describe the syntax extensions necessary to give programmers the opportunity to deviate from purely static annotations.
We immediately give a meaning to the new “gradual” syntax, driven by the concepts of abstract interpretation.
In section \ref{sec:lifting-predicates-and} we explain “lifting”, a procedure adapting predicates and functions in order for them to deal with gradual parameters.
To guide the following efforts to determine gradual semantics of \gvl, we present gradual soundness in section \ref{ssec:gradual-soundness} and point out the associated challenges.

With the necessary tools for gradualization available, we apply them to the static semantics of \svl in section \ref{sec:abstracting-static-semantics}.
Finally, we develop gradual dynamic semantics in section \ref{sec:abstracting-dynamic-semantics} and show how gradual soundness is achievable.

% reference implementation chapter, explain difference (here: general, there: making use of specifics, optimality, minimal runtime overhead (0 if static, ...) ...)

\section{A Generic Statically Verified Language (\svl)}% REITERATE
\label{sec:a-statically-verified}
%% more about our language
We now intrude a very simple Java-like statically verified language “\svl” that uses Chalice/Eiffel/Spec\# %???
 sub-syntax to express method contracts.

% more about simplicity
% - decidable satisfiability & implication of formulas (will later investigate how to extend... ) % TODO
% - exactly one method arg, return type

\section{Gradual Formulas}
\label{sec:gradual-formulas}
% It is one of our design goals to make any program of the static system a valid program of the 
% the resulting gradual system compatible with the old one (otherwise it would trivially not  satisfy the gradual guarantee \ref{grad-guarantee}).

We introduce the concepts of gradual verification by introducing a wildcard formula $\qm$ into the formula syntax, resulting in a new set of gradual formulas $\setGFormula$.
There are different ways to introduce the wildcard, we will describe two common options in the following sections.

%% COMMON KNOWLEDGE
Note that we want to strictly extend the existing formula syntax ($\setFormula \subset \setGFormula$) in order to maintain compatibility with the static system.
This design goal ensures that any program considered syntactically valid by the static system will still be syntactically valid in the gradual system (motivated by gradual guarantee \ref{grad-guarantee}).

We decorate formulas $\grad{\phi} \in \setGFormula$ to distinguish them from formulas drawn from $\setFormula$.
Using the concept of abstract interpretation, we want to reason about gradual formulas by mapping them back to a set of static formulas (called “concretization”) and then applying static reasoning to that set.

% An example of that approach ... in section \ref{sec:lifting-predicates-and}

Without knowing specifics of the syntax extension, we can already formalize this approach for static formulas:
\begin{definition}[Concretization]~\\
    Let $\gamma : \setGFormula \rightarrow \PP(\setFormula)$ be defined as follows:
    \begin{align*}
    &\gamma(\phi) = \{~ \phi ~\}     \quad\quad \forall \phi \in \setFormula\\
    &\textit{other cases to be defined when extending the syntax}
    \end{align*}
\end{definition}

% more detail about singleton mapping and why it makes sense?

We illustrate two typical ways of extending the formula syntax.
    \subsection{Dedicated Wildcard Formula}
    \label{ssec:dedicated-wildcard-formula}
    
The most straight forward way to extend the syntax is by simply adding \qm as a dedicated formula:
\begin{displaymath}
\grad{\phi} ::= \phi ~|~ \qm
\end{displaymath}
This is analogous to how most gradually typed languages are realized (e.g. \ttt{dynamic}-type in C\# 4.0 and upward).

Since \qm is supposed to be a placeholder for an arbitrary formula, its concretization is defined as.
\begin{displaymath}
\gamma(\qm) = \setFormulaA
\end{displaymath}

% EXPLAIN why \setFormulaA

This approach is limited since programmers cannot express any additional static knowledge they might have.
For example, a programmer might resort to using the wildcard lacking some knowledge about variable \ex{x} (or being unable to express it), whereas he could give a static formula for \ex{y}, say \phiEq{y}{3}.
Yet, there is no way to express this information as soon as the wildcard is used.
    
    \subsection{Wildcard with Upper Bound}
    \label{ssec:wildcard-with-upper}
    
To allow combining wildcards with static knowledge, we might view \qm merely as an unknown conjunctive term within a formula:
\begin{displaymath}
\grad{\phi} ::= \phi ~|~ \withqmGen{\phi}
\end{displaymath}

We pose $\qm \defeq \withqmGen{\phiTrue}$

We expect $\withqmGen{\phi}$ to be a placeholder for a formula that also “contains” $\phi$.
There are two ways to express this containment, resulting in different concretizations.
\begin{description}
    \item[Syntactic]\quad
    $\gamma_1(\withqmGen{\phi}) := \{~ \phi \wedge \phi' ~|~ \phi' \in \setFormulaA ~\}$
    \item[Semantic]\quad
    $\gamma_2(\withqmGen{\phi}) := \{~ \phi' \in \setFormulaA ~|~ \phi' \implies \phi ~\}$
\end{description}

\begin{lemma} 
    $\forall \grad{\phi} \in \setGFormula.~ \gamma_1(\grad{\phi}) \subseteq \gamma_2(\grad{\phi})$
\end{lemma}
\begin{lemma} 
    $\forall \grad{\phi} \in \setGFormula.~ \gamma_1(\grad{\phi}) = \gamma_2(\grad{\phi})$ modulo equivalence
\end{lemma}



    \subsection{Precision}
    \label{ssec:precision}
    Comparing gradual formulas (e.g. $\ttt{x = 3}$, $\withqmGen{\ttt{x = 3}}$, $\qm$) gives rise to a notion of “precision”.
Intuitively, $\ttt{x = 3}$ is more precise than $\withqmGen{\ttt{x = 3}}$ which is more precise than $\qm$.
Using concretization, we can formalize this intuition.
\begin{definition}[Formula Precision]\label{def:mpt}
    $$\grad{\phi_a} \sqsubseteq \grad{\phi_b}  \quad\iff\quad  \gamma(\grad{\phi_a}) \subseteq \gamma(\grad{\phi_b})$$
    Read: Formula $\grad{\phi_a}$ is “at least as precise as” $\grad{\phi_b}$.
\end{definition}
The strict version $\sqsubset$ is defined accordingly. 
    
    % TODO: define static() helper function somewhere
    
    \subsection{Gradual Statements}
    \label{sec:gradual-statements}
    %% intro
Formulas play a role for some statements, extending their syntax may thus also affect the syntax of statements.
A common example are assertion statements $\sAssert{$\phi$}$, which can now be extended to $\sAssert{$\grad{\phi}$}$.
However, having a gradual formula syntax available does not necessarily mean that all statements have to adopt it.
% Examples can be found in chapter \ref{ch:case-study--implicit} which derives a gradual language \gvlidf that .

%% call
A more complex example affected by gradualization of formulas is a call statement $\ttt{$m$();}$ in presence of method contracts.
Although not directly visible, this statement's semantics (static and dynamic) is affected by the contract of $m$, consisting of pre- and postcondition.
One can think of $m$ as a reference to some method definition including method contract.
Note that in practice such method definitions usually reside in a “program context” that the static and dynamic semantics are parameterized with.
As the full meaning of such a statement is unknown without context, it is hard to reason about it abstractly.
W.l.o.g. we will thus think of $m$ as syntactic sugar for 
\begin{align*}
&\ttt{assert $\phi_{m_{pre}}$;}\\
&\ttt{// body of $m$}\\
&\ttt{assume $\phi_{m_{post}}$;}
\end{align*}

%%% contract
As one of the main goals of gradual verification is to allow for gradual method contracts, it makes sense to extend the syntax accordingly.
This means that the syntax of the desugared call statement is affected:
\begin{align*}
&\ttt{assert $\grad{\phi_{m_{pre}}}$;}\\
&\ttt{// body of $m$}\\
&\ttt{assume $\grad{\phi_{m_{post}}}$;}
\end{align*}

%% general
In general, statement syntax is extended, resulting in a superset $\setGStmt \supseteq \setStmt$ of gradual statements.
Note that the superset is induced merely by allowing $\setGFormula$ instead of $\setFormula$ in certain places (chosen by the gradual language designer).
We give meaning to gradual statements using a concretization function. % MORE detail?
\begin{definition}[Concretization of Gradual Statements]
    Let $\gamma_s : \setGStmt \rightarrow \PP(\setStmt)$ be defined as
    \begin{displaymath}
    \gamma_s(\grad{s}) = \{~ s \in \setStmt ~|~ \textit{$s$ is $\grad{s}$ with all gradual formulas replaced by some concretizations} ~\}
    \end{displaymath}
\end{definition}
\begin{definition}[Precision of Gradual Statement]
    Let $\mpts \subseteq \setGStmt \times \setGStmt$ be a predicate defined as
    $$\grad{s_a} \mpts \grad{s_b}  \quad\iff\quad  \gamma_s(\grad{s_a}) \subseteq \gamma_s(\grad{s_b})$$
\end{definition}

%% outlook
The notion of gradual statements will become important for the gradualized semantics of \gvl.


    \subsection{Gradual Program State}
    \label{sec:gradual-program-state}
    Recall that program state has a notion of continuation, see section \ref{sec:a-statically-verified} for examples.
As the set of possible statements has been augmented from $\setStmt$ to $\setGStmt$, this notion might have to be augmented as well in order to allow encoding the additional statements.

This augmentation leads to a superset $\setGProgramState \supseteq \setProgramState$ of gradual program states.
\begin{example}{Gradual Program State}
\label{ex:grad-ps}
$$\setProgramState ~  =~ (\setVar \rightharpoonup \mathbb{Z}) ~\times~ \setStmt$$
is extended to
$$\setGProgramState ~ =~ (\setVar \rightharpoonup \mathbb{Z}) ~\times~ \setGStmt$$
\end{example}

Concretization and precision are defined accordingly, drawing on concretization of gradual statements.

\begin{comment}
Consequence:
\begin{displaymath}
\forall \grad{\pi_{\grad{s}}} \in \setGProgramState_{\grad{s}}, \pi \in \gamma_{\pi}(\grad{\pi_{\grad{s}}}).~ \exists s \in \gamma_s(\grad{s}).~ \pi \in \setProgramState_s
\end{displaymath}
\end{comment}

\begin{lemma}[Gradual Program State Does Not Affect Formula Semantics]
    \label{lemma:gradPS-form-sem}~\\
    We demand that formula semantics are not affected by gradualization of the program state:
    \begin{displaymath}
    \forall \phi \in \setFormula, \grad{\pi} \in \setGProgramState, \pi \in \gamma_{\pi}(\grad{\pi}).~~ \evalphiGen{\grad{\pi}}{\phi} \iff \evalphiGen{\pi}{\phi}
    \end{displaymath}
    
    This is trivially the case if evaluation does not depend on the (now gradual) continuation in the first place.
\end{lemma}



\section{Lifting Predicates and Functions}
\label{sec:lifting-predicates-and}
The Hoare logic of \svl is a ternary predicate $\thoare{~}{\cdot}{\cdot}{\cdot} \subseteq \setFormula \times \setStmt \times \setFormula$.
% TODO: ensure consistency, setFormulaA vs setFormula
Since \gvl contains gradual formulas and gradual statements, the gradualized Hoare logic is expected to have signature $\gthoare{~}{\cdot}{\cdot}{\cdot} \subseteq \setGFormula \times \setGStmt \times \setGFormula$.
Similarly, the gradualized small-step semantics is expected to have signature $\gsssem : \setGProgramState \rightharpoonup \setGProgramState$ instead of $\sssem : \setProgramState \rightharpoonup \setProgramState$.
Usually semantics are defined inductively, meaning that they are defined in terms of further predicates or functions (e.g. implication between formulas).
These functions will have new signatures as well in order to deal with the extended syntax of \gvl.
This section will present a procedure called “lifting”, which formalizes this adaptation of predicates and functions.

\begin{definition}[Gradual Lifting]
    The procedure of extending an existing predicate/function in order to deal with gradual formulas.
    The resulting predicate/function has the same signature as the original one, with occurrences of \setFormula, \setStmt and \setProgramState replaced by \setGFormula, \setGStmt, \setGProgramState.
\end{definition}

Our rules for lifting rely merely on the existence of a concretization function and a notion of precision.
We will thus restrict our formalizations and explanations to (gradual) formulas, whereas they are directly applicable to other gradualized sets.

    \subsection{Gradual Guarantee of Verification}
    \label{ssec:gfconclusion}
    Since lifted predicates and functions directly affect the gradual semantics of \gvl, they must adhere to certain rules in order to be sound.
What soundness means is a direct consequence of the gradual guarantee for gradual verification systems, which we derive from the gradual guarantee for gradual type systems by Siek et al. \cite{siek2015refined}.

For simplicity we will simply call programs “acceptable” if they are successfully verifiable by the gradual verifier.

\begin{definition}[Gradual Guarantee (Static Semantics)]
    \label{grad-guarantee-static}~\\
    Acceptable programs remain acceptable when reducing precision of any formula.
\end{definition}

\begin{definition}[Gradual Guarantee (Dynamic Semantics)]
    \label{grad-guarantee-dynamic}~\\
    Acceptable programs with a particular observational behavior (termination, values of variables, output, etc.) will have the same observational behavior after reducing precision of any formula.
\end{definition}

\begin{comment}


% "static" function?
% - here: always part of the concretization, ... later we will see counterexample


PROBABLY UNNECESSARY:\\
Because of its generality, we will pursue the approach introduced in section \ref{ssec:wildcard-with-upper} for the remainder of this chapter.
As concretization we chose the semantic version, as it is more flexible than the syntactic one in practice.
For reference, the full definitions:
\begin{align*} 
&\text{Syntax:}\\
&\grad{\phi} ::= \phi ~|~ \withqmGen{\phi}\\
\\
&\text{Concretization:}\\
&\gamma(\phi) = \{~ \phi ~\}     \quad\quad \forall \phi \in \setFormulaA\\
&\gamma(\withqmGen{\phi}) = \{~ \phi' \in \setFormulaA ~|~ \phiImplies{\phi'}{\phi} ~\}\\
&\gamma(\grad{\phi}) = \emptyset    \quad\textit{otherwise}
\end{align*}
\end{comment}


    \subsection{Lifting Predicates}
    \label{ssec:lifting-predicates}
    
% TODO: we dissect the gradual lifting notion of AGT
In this section, we assume that we are dealing with a binary predicate $P \subseteq \setFormula \times \setFormula$.
The concepts are directly applicable to predicates with different arity or with additional non-formula parameters.
The lifted version we are targeting has signature $\grad{P} \subseteq \setGFormula \times \setGFormula$.
W.l.o.g. we further assume that $P$ appears unnegated in the axiomatic semantics (otherwise we simply regard the negation of that predicate as $P$).

Restrictions imposed by the gradual guarantee:
\begin{description}
    \item[Introduction of Gradual System]~\\
    Having source code that is considered valid by the static verification system, the same source code must be considered valid by the gradual verification system.
    In other words, switching to the gradual system may never “break the code”.
    This means that arguments satisfying $P$ must satisfy $\grad{P}$:
    \begin{displaymath}
    \forall \phi_1, \phi_2 \in \setFormula.~ P(\phi_1, \phi_2) \implies \grad{P}(\phi_1, \phi_2)
    \end{displaymath}
    or equivalently, using set notation
    \begin{displaymath}
    P \subseteq \grad{P}
    \end{displaymath}
    
    \item[Reducing Precision/Monotonicity???]~\\
    A central point of a gradual verification system is enabling programmers to specify contracts with less precision.
    Source code that is rejected by the verifier might get accepted after reducing precision.
    If the opposite would happen, though, that would be highly counter-intuitive and ...??? workflow.
    To prevent such behavior, we expect satisfied predicates to still be satisfied after reducing the precision of arguments:
    
    \begin{displaymath}
    \forall \grad{\phi_1}, \grad{\phi_2}, \grad{\phi_1'}, \grad{\phi_2'} \in \setGFormula.~ \grad{\phi_1} \sqsubseteq \grad{\phi_1'} \wedge \grad{\phi_2} \sqsubseteq \grad{\phi_2'} \wedge \grad{P}(\grad{\phi_1}, \grad{\phi_2}) \implies \grad{P}(\grad{\phi_1'}, \grad{\phi_2'})
    \end{displaymath}
    or equivalently, thinking of predicates as boolean functions
    \begin{displaymath}
    \grad{P}  \text{ is monotonic w.r.t. $\sqsubseteq$}
    \end{displaymath}
    or something with set terminology!???
    \begin{displaymath}
    \grad{P}  \text{ is somewhat closed over weakening}
    \end{displaymath}
\end{description}

\begin{definition}[Soundness of Lifted Predicate]
    A lifted predicate is \textbf{sound} if it satisfies both ?????
\end{definition}

For sound rules this holds:
\begin{displaymath}
\forall \grad{\phi_1}, \grad{\phi_2} \in \setGFormula.~ (\exists \phi_1 \in \gamma(\grad{\phi_1}), \phi_2 \in \gamma(\grad{\phi_2}).~ P(\phi_1, \phi_2)) \implies \grad{P}(\grad{\phi_1}, \grad{\phi_2})
\end{displaymath}

Note that there is a “smallest” set/predicate??? ... induced by the rules.

\begin{definition}[Optimality of Lifted Predicate]
    A lifted predicate is \textbf{optimal} if there exists no ???
\end{definition}

...same as “consistent lifting” in AGT!

% TODO: delve into what is important, what is optional, etc.
\begin{displaymath}
\forall \grad{\phi_1}, \grad{\phi_2} \in \setGFormula.~ (\exists \phi_1 \in \gamma(\grad{\phi_1}), \phi_2 \in \gamma(\grad{\phi_2}).~ P(\phi_1, \phi_2)) \iff \grad{P}(\grad{\phi_1}, \grad{\phi_2})
\end{displaymath}



%TODO: restate that in terms of deduction stuff?



    
        \subsubsection{Examples}
        \label{sssec:examples-lift-predicates}
        %% phiImplies
\begin{lemma}[Optimal Lifting of Implication]~\\
    Let $\setGFormula$ be extended using the “total unknown” approach (see section \ref{ssec:dedicated-wildcard-formula}).
    Let $~~\gphiImplies{\cdot}{\cdot}~ \subseteq \setGFormula \times \setGFormula$ be defined inductively as
    \begin{mathpar}
        \inferrule* [Right=\gradT ImplStatic]
        {
            \phiImplies{\phi_1}{\phi_2}
        }
        {
            \gphiImplies{\phi_1}{\phi_2}
        }
    \end{mathpar}
    \begin{mathpar}
        \inferrule* [Right=\gradT ImplGrad1]
        {
            \phi \in \setFormulaA
        }
        {
            \gphiImplies{\qm}{\phi}
        }
    \end{mathpar}
    \begin{mathpar}
        \inferrule* [Right=\gradT ImplGrad2]
        {
            ~
        }
        {
            \gphiImplies{\grad{\phi}}{\qm}
        }
    \end{mathpar}
    Then $~\gphiImplies{\cdot}{\cdot}~$ is an optimal lifting of $~\phiImplies{\cdot}{\cdot}~$.
\end{lemma}
\begin{proof}~\\
    Goal:
    $$\gphiImplies{\grad{\phi_1}}{\grad{\phi_2}} \iff \exists \phi_1 \in \gamma(\grad{\phi_1}), \phi_2 \in \gamma(\grad{\phi_2}).~ \phiImplies{\phi_1}{\phi_2}$$
    
    \begin{description}
        \item[Case $\implies$]~\\
        \begin{description}
            \item[Case \tset{\gradT ImplStatic}]
            \begin{align*}
            &\gphiImplies{\phi_1}{\phi_2}\\
            \implies
            &\phiImplies{\phi_1}{\phi_2}\\
            \implies
            &(\exists \phi_1' \in \gamma(\phi_1), \phi_2' \in \gamma(\phi_2).~ \phiImplies{\phi_1'}{\phi_2'})
            \end{align*}
            
            \item[Case \tset{\gradT ImplGrad1}]
            \begin{align*}
            &\gphiImplies{\qm}{\phi}\\
            \implies
            &\phi \in \setFormulaA\\
            \implies
            &(\exists \phi_1 \in \setFormulaA.~ \phiImplies{\phi_1}{\phi})\\
            \implies
            &(\exists \phi_1 \in \gamma(\qm), \phi_2 \in \gamma(\phi).~ \phiImplies{\phi_1}{\phi_2})
            \end{align*}
            
            \item[Case \tset{\gradT ImplGrad2}]
            \begin{align*}
            &\gamma(\grad{\phi}) \neq \emptyset ~\wedge~ \phiTrue \in \gamma(\qm)\\
            \implies
            &(\exists \phi_1 \in \gamma(\grad{\phi}).~ \phiImplies{\phi_1}{\phiTrue}) ~\wedge~ \phiTrue \in \gamma(\qm)\\
            \implies
            &(\exists \phi_1 \in \gamma(\grad{\phi}), \phi_2 \in \gamma(\qm).~ \phiImplies{\phi_1}{\phi_2})
            \end{align*}
        \end{description}
        \item[Case $\impliedby$]~\\
        Given: $\phi_1 \in \gamma(\grad{\phi_1}) \wedge \phi_2 \in \gamma(\grad{\phi_2}) \wedge \phiImplies{\phi_1}{\phi_2}$
        \begin{description}
            \item[Case $\grad{\phi_2} = \qm$]~\\
            Apply $\tset{\gradT ImplGrad2}$.
            
            \item[Case $\grad{\phi_2} = \phi_2 \wedge \grad{\phi_1} = \qm$]
            \begin{align*}
            &\phi_1 \in \gamma(\qm) \wedge \phiImplies{\phi_1}{\phi_2}\\
            \implies
            &\phi_1 \in \setFormulaA \wedge \phiImplies{\phi_1}{\phi_2}\\
            \implies
            &\phi_2 \in \setFormulaA
            \end{align*}
            Apply $\tset{\gradT ImplGrad1}$.
            
            \item[Case $\grad{\phi_2} = \phi_2 \wedge \grad{\phi_1} = \phi_1$]~\\
            Apply $\tset{\gradT ImplStatic}$.
        \end{description}
    \end{description}
\end{proof}

%% evalphi
\begin{lemma}[Optimal Lifting of Evaluation]
    \label{ex:opt-lift-evalphi}~\\
    Let $\setGFormula$ be extended using the “bounded unknown” approach (see section \ref{ssec:wildcard-with-upper}).
    Let $~~\evalgphiGen{\cdot}{\cdot}~ \subseteq \setProgramState \times \setGFormula$ be defined inductively as
    \begin{mathpar}
        \inferrule* [Right=\gradT EvalStatic]
        {
            \evalphiGen{\pi}{\phi}
        }
        {
            \evalgphiGen{\pi}{\phi}
        }
    \end{mathpar}
    \begin{mathpar}
        \inferrule* [Right=\gradT EvalGrad]
        {
            \evalphiGen{\pi}{\phi}
        }
        {
            \evalgphiGen{\pi}{\withqmGen{\phi}}
        }
    \end{mathpar}
    
    Then $~\evalgphiGen{\cdot}{\cdot}~$ is an optimal lifting of $~\evalphiGen{\cdot}{\cdot}~$.
\end{lemma}
\begin{proof}~\\
    Goal:
    $$\evalgphiGen{\pi}{\grad{\phi}} \iff \exists \phi \in \gamma(\grad{\phi}).~ \evalphiGen{\pi}{\phi}$$
    
    \begin{description}
        \item[Case $\implies$]~\\
        \begin{description}
            \item[Case \tset{\gradT EvalStatic}]
            \begin{align*}
            &\evalgphiGen{\pi}{\phi}\\
            \implies
            &\evalphiGen{\pi}{\phi}\\
            \implies
            &(\exists \phi' \in \gamma(\phi).~ \evalphiGen{\pi}{\phi'})
            \end{align*}
            
            \item[Case \tset{\gradT EvalGrad}]
            \begin{align*}
            &\evalgphiGen{\pi}{\withqmGen{\phi}}\\
            \implies
            &\evalphiGen{\pi}{\phi}\\
            \implies
            &(\exists \phi' \in \gamma(\withqmGen{\phi}).~ \evalphiGen{\pi}{\phi'})
            \end{align*}
            
        \end{description}
        \item[Case $\impliedby$]~\\
        Given: $\phi' \in \gamma(\grad{\phi}) \wedge \evalphiGen{\pi}{\phi'}$
        \begin{description}
            \item[Case $\grad{\phi} = \withqmGen{\phi}$] ~\\
            It follows from definition \ref{def:gamma-bounded-unk} that $\phiImplies{\phi'}{\phi}$ and thus $\evalphiGen{\pi}{\phi}$.
            Apply $\tset{\gradT EvalGrad}$.
            \item[Case $\grad{\phi} = \phi$] ~\\
            It follows that $\phi = \phi'$.
            Apply $\tset{\gradT EvalStatic}$.
        \end{description}
    \end{description}
\end{proof}

Note that the definition of lifted evaluation was lifted only w.r.t. the second parameter.
There is no point in lifting evaluation w.r.t. the program state since gradual program state has no impact on evaluation (see lemma \ref{lemma:gradPS-form-sem}).

\begin{comment}
We define denotational semantics of gradual formulas analogous to the non-gradual variant (see definition \ref{def:frm-den-sem}):
\begin{definition}[Denotational Formula Semantics $\envs{\cdot}$ of Gradual Formulas]~\\
    \label{def:gfrm-den-sem}
    Let $\envs{\cdot} : \setGFormula \rightarrow \PP^{\setProgramState}$ be defined as
    \begin{displaymath}
    \envs{\grad{\phi}} \defeq \{~ \pi \in \setProgramState ~|~ \evalgphiGen{\pi}{\grad{\phi}} ~\}
    \end{displaymath}
\end{definition}
\end{comment}

%% composite
\begin{lemma}[Sound Lifting of Composite Predicate]
    \label{lemma:pred-lift-comp}~\\
    Let $P, Q \subseteq \setFormula \times \setFormula$ be arbitrary binary predicates.
    Let $(P \circ Q) \subseteq \setFormula \times \setFormula$ be defined as
    \begin{displaymath}
    (P \circ Q)(\phi_1, \phi_3) ~\defiff~ \exists \phi_2 \in \setFormula.~ P(\phi_1, \phi_2) \wedge Q(\phi_2, \phi_3)
    \end{displaymath}
    
    Let $\grad{(P \circ Q)} \subseteq \setGFormula \times \setGFormula$ be defined as
    \begin{displaymath}
    \grad{(P \circ Q)} ~\defeq~ \grad{P} \circ \grad{Q}
    \end{displaymath}
    with sound liftings $\grad{P}$ and $\grad{Q}$.
    
    Then $\grad{(P \circ Q)}$ is a sound lifting of $(P \circ Q)$, i.e. “piecewise” lifting of composite predicates is allowed.
    Optimality of $\grad{P}$ and $\grad{Q}$ does not imply optimality of $\grad{(P \circ Q)}$.
\end{lemma}
\begin{proof}
    \begin{description}
        \item[Introduction] 
        \begin{align*}
        &(P \circ Q)(\phi_1, \phi_3)\\
        \overset{Definition}{~\quad\quad\implies\quad\quad~}
        &(\exists \phi_2 \in \setFormula.~ P(\phi_1, \phi_2) \wedge Q(\phi_2, \phi_3))\\
        \overset{Introduction}{~\quad\quad\implies\quad\quad~}
        &(\exists \phi_2 \in \setFormula.~ \grad{P}(\phi_1, \phi_2) \wedge \grad{Q}(\phi_2, \phi_3))\\
        \overset{Definition}{~\quad\quad\implies\quad\quad~}
        &(\grad{P} \circ \grad{Q})(\phi_1, \phi_3)\\
        \overset{Definition}{~\quad\quad\implies\quad\quad~}
        &\grad{(P \circ Q)}(\phi_1, \phi_3)\\
        \end{align*}
        
        \item[Monotonicity] 
        \begin{align*}
        &\grad{(P \circ Q)}(\grad{\phi_1}, \grad{\phi_3}) ~\wedge~ \grad{\phi_1} \mpt \grad{\phi_1'} ~\wedge~ \grad{\phi_3} \mpt \grad{\phi_3'}\\
        \overset{Definition}{~\quad\quad\implies\quad\quad~}
        &(\grad{P} \circ \grad{Q})(\grad{\phi_1}, \grad{\phi_3}) ~\wedge~ \grad{\phi_1} \mpt \grad{\phi_1'} ~\wedge~ \grad{\phi_3} \mpt \grad{\phi_3'}\\
        \overset{Definition}{~\quad\quad\implies\quad\quad~}
        &(\exists \grad{\phi_2} \in \setGFormula.~ \grad{P}(\grad{\phi_1}, \grad{\phi_2}) \wedge \grad{Q}(\grad{\phi_2}, \grad{\phi_3})) ~\wedge~ \grad{\phi_1} \mpt \grad{\phi_1'} ~\wedge~ \grad{\phi_3} \mpt \grad{\phi_3'}\\
        \overset{Monotonicity}{~\quad\quad\implies\quad\quad~}
        &(\exists \grad{\phi_2} \in \setGFormula.~ \grad{P}(\grad{\phi_1'}, \grad{\phi_2}) \wedge \grad{Q}(\grad{\phi_2}, \grad{\phi_3'}))\\
        \overset{Definition}{~\quad\quad\implies\quad\quad~}
        &(\grad{P} \circ \grad{Q})(\grad{\phi_1'}, \grad{\phi_3'})\\
        \overset{Definition}{~\quad\quad\implies\quad\quad~}
        &\grad{(P \circ Q)}(\grad{\phi_1'}, \grad{\phi_3'})\\
        \end{align*}
    \end{description}
\end{proof}

%% conjunction
\begin{lemma}[Sound Lifting of Conjunctive Predicate]~\\
    Let $P, Q \subseteq \setFormula$ be arbitrary binary predicates.
    Let $(P \wedge Q) \subseteq \setFormula$ be defined as
    \begin{displaymath}
    (P \wedge Q)(\phi) ~\defiff~ P(\phi) \wedge Q(\phi)
    \end{displaymath}
    
    Let $\grad{(P \wedge Q)} \subseteq \setGFormula$ be defined as
    \begin{displaymath}
    \grad{(P \wedge Q)} ~\defeq~ \grad{P} \wedge \grad{Q}
    \end{displaymath}
    with sound liftings $\grad{P}$ and $\grad{Q}$.
    
    Then $\grad{(P \wedge Q)}$ is a sound lifting of $(P \wedge Q)$, i.e. term-wise lifting of disjunctive predicates is allowed.
    Optimality of $\grad{P}$ and $\grad{Q}$ does not imply optimality of $\grad{(P \wedge Q)}$.
\end{lemma}
\begin{proof}
    \begin{description}
        \item[Introduction] 
        \begin{align*}
        &(P \wedge Q)(\phi)\\
        \overset{Definition}{~\quad\quad\implies\quad\quad~}
        &P(\phi) ~\wedge~ Q(\phi)\\
        \overset{Introduction}{~\quad\quad\implies\quad\quad~}
        &\grad{P}(\phi) ~\wedge~ \grad{Q}(\phi)\\
        \overset{Definition}{~\quad\quad\implies\quad\quad~}
        &(\grad{P} ~\wedge~ \grad{Q})(\phi)\\
        \overset{Definition}{~\quad\quad\implies\quad\quad~}
        &\grad{(P \wedge Q)}(\phi)\\
        \end{align*}
        
        \item[Monotonicity] 
        \begin{align*}
        &(\grad{P} \wedge \grad{Q})(\grad{\phi}) ~\wedge~ \grad{\phi} \mpt \grad{\phi'}\\
        \overset{Definition}{~\quad\quad\implies\quad\quad~}
        &\grad{P}(\grad{\phi}) ~\wedge~ \grad{Q}(\grad{\phi}) ~\wedge~ \grad{\phi} \mpt \grad{\phi'}\\
        \overset{Monotonicity}{~\quad\quad\implies\quad\quad~}
        &\grad{P}(\grad{\phi'}) ~\wedge~ \grad{Q}(\grad{\phi'})\\
        \overset{Definition}{~\quad\quad\implies\quad\quad~}
        &(\grad{P} \wedge \grad{Q})(\grad{\phi'})\\
        \overset{Definition}{~\quad\quad\implies\quad\quad~}
        &\grad{(P \wedge Q)}(\grad{\phi'})\\
        \end{align*}
    \end{description}
\end{proof}

%% disjunction
\begin{lemma}[Optimal Lifting of Disjunctive Predicate]~\\
    Let $P, Q \subseteq \setFormula$ be arbitrary binary predicates.
    Let $(P \vee Q) \subseteq \setFormula$ be defined as
    \begin{displaymath}
    (P \vee Q)(\phi) ~\defiff~ P(\phi) \vee Q(\phi)
    \end{displaymath}
    
    Let $\grad{(P \vee Q)} \subseteq \setGFormula$ be defined as
    \begin{displaymath}
    \grad{(P \vee Q)} ~\defeq~ \grad{P} \vee \grad{Q}
    \end{displaymath}
    with sound liftings $\grad{P}$ and $\grad{Q}$.
    
    Then $\grad{(P \vee Q)}$ is a sound lifting of $(P \vee Q)$, i.e. term-wise lifting of disjunctive predicates is allowed.
    Optimality of $\grad{P}$ and $\grad{Q}$ does imply optimality of $\grad{(P \vee Q)}$.
\end{lemma}
\begin{proof}
    \begin{align*}
    &\grad{(P \vee Q)}(\grad{\phi})\\
    \overset{Definition}{~\quad\quad\iff\quad\quad~}
    &(\grad{P} \vee \grad{Q})(\grad{\phi})\\
    \overset{Definition}{~\quad\quad\iff\quad\quad~}
    &\grad{P}(\grad{\phi}) \vee \grad{Q}(\grad{\phi})\\
    \overset{AGT Def.}{~\quad\quad\iff\quad\quad~}
    &(\exists \phi \in \gamma(\grad{\phi}).~ P(\phi)) \vee (\exists \phi \in \gamma(\grad{\phi}).~ Q(\phi))\\
    \overset{}{~\quad\quad\iff\quad\quad~}
    &(\exists \phi \in \gamma(\grad{\phi}).~ P(\phi) ~\vee~ Q(\phi))\\
    \overset{Definition}{~\quad\quad\iff\quad\quad~}
    &(\exists \phi \in \gamma(\grad{\phi}).~ (P \vee Q)(\phi))
    \end{align*}
\end{proof}

\begin{comment}
We define $\setGFormulaA = \{~ \grad{\phi} \in \setGFormula ~|~ \exists \pi.~ \evalgphiGen {\pi} {\grad{\phi}} ~\}$ as the set of satisfiable gradual formulas.

\begin{lemma}[Restricted Domain of Concretization]~\\
    $\restr{\gamma}{\setGFormulaA}$ never returns the empty set.
\end{lemma}

\end{comment}

    
    \subsection{Lifting Functions}
    \label{ssec:lifting-functions}
    % assuming total for now (otherwise: split partial function into total function and definedness predicate)

In this section, we assume that we are dealing with a \emph{total} function $f : \setFormula \rightarrow \setFormula$.
Partial functions are dealt with in section \ref{sssec:lifting-partial-functions}.

The following concepts are directly applicable to functions with higher arity.

\begin{description}
    \item[Introduction]~\\
    With predicate lifting we made sure to design a gradual verification system that is “immune” to reduction of precision.
    Therefore, when replacing function $f$ with its gradual lifting $\grad{f}$, we expect the result to be the same or less precise.
    \begin{displaymath}
    \forall \phi \in \setFormula.~ f(\phi) \sqsubseteq \grad{f}(\phi)
    \end{displaymath}
    %Equivalently:
    %\begin{displaymath}
    %\forall \phi \in \setFormula.~ f(\phi) \in \gamma(\grad{f}(\phi))
    %\end{displaymath}
    
    \item[Monotonicity]~\\
    Reducing precision of a parameter may only result in a loss of precision of the result.
    In other words, the function must be monotonic w.r.t. $\sqsubseteq$.
    
    \begin{displaymath}
    \forall \grad{\phi_1}, \grad{\phi_2} \in \setGFormula.~ 
    \grad{\phi_1} \sqsubseteq \grad{\phi_2} 
    \implies 
    \grad{f}(\grad{\phi_1}) \sqsubseteq \grad{f}(\grad{\phi_2})
    \end{displaymath}
\end{description}

\begin{definition}[Sound Function Lifting]
    A lifted function is \textbf{sound} if it adheres to the above rules.
\end{definition}

Note that the rules for sound lifting only give a lower bound for the gradual return values.
Thus a function $\grad{f} : \setGFormula \rightarrow \setGFormula$ constantly returning $\qm$ is a sound lifting of any function $f : \setFormula \rightarrow \setFormula$.
This observation motivates an additional notion of optimality.

\begin{definition}[Optimal Function Lifting]
    A sound lifted function is \textbf{optimal} if its return values are at least as precise as the return values of any other sound lifted function.
\end{definition}

%% AGT
Again, definition of optimal function lifting coincides with the definition of “consistent function lifting” given by AGT.
\begin{lemma}[Equivalence with Consistent Function Lifting (AGT)]\label{lemma:consistent-func-lifting-direct}~\\
    \label{lemma:eq-fun-lift-agt}
    Let $\alpha : \PP(\setFormula) \rightharpoonup \setGFormula$ be a partial function such that $\langle \gamma, \alpha \rangle$ is a $\{ \overline{f} \}$-partial Galois connection (see appendix \ref{app:part-galois}, definition \ref{def:pgc}).\\ 
    Let $\grad{f} : \setGFormula \rightarrow \setGFormula$ be defined as
    \begin{displaymath}
    \grad{f}(\grad{\phi}) \defeq \alpha(\overline{f}(\gamma(\grad{\phi})))
    \end{displaymath}
    where $\overline{f}$ means that $f$ is applied to every element of the set.
    Then $\grad{f}$ is an optimal lifting of $f$.
\end{lemma}
\begin{proof}~
    \begin{description}
        \item[Adjoint Equation] 
        \begin{align*}
        \alpha(\overline{f}(\gamma(\phi))) = f(\phi)
        \end{align*}
        Proof:
        
        $\alpha(\overline{f}(\gamma(\phi)))$ defined, since $\{ \overline{f} \}$-partial Galois connection, i.e.
        \begin{align}
        \label{frm:pgc-ass}
        \alpha(\overline{f}(\gamma(\phi))) = \alpha(\{ f(\phi) \}) = \grad{\phi}
        \end{align}
        Applying rule 1 of partial Galois connections to \ref{frm:pgc-ass}
        \begin{align}
        \label{frm:pgc-ass1}
        &\{~ f(\phi) ~\} \subseteq \gamma(\grad{\phi})\\
        \end{align}
        Applying rule 2 of partial Galois connections to \ref{frm:pgc-ass}, using $\{~ f(\phi) ~\} \subseteq \gamma(f(\phi))$
        \begin{align}
        \label{frm:pgc-ass2}
        & \grad{\phi} \sqsubseteq f(\phi)
        \end{align}
        
        Combining \ref{frm:pgc-ass1} and \ref{frm:pgc-ass2}
        \begin{align*}
        &\{~ f(\phi) ~\} \subseteq \gamma(\grad{\phi}) \subseteq \gamma(f(\phi))\\
        \implies
        & \gamma(\grad{\phi}) = \{~ f(\phi) ~\}\\
        \implies
        & \grad{\phi} = f(\phi)
        \end{align*}
        
        \item[Soundness]~
        \begin{description}
            \item[Introduction] 
            \begin{align*}
            &\grad{f}(\phi)\\
            =~
            &\alpha(\overline{f}(\gamma(\phi)))\\
            =~ 
            &f(\phi)
            \end{align*}
            
            \item[Monotonicity]~\\ 
            We assume $\grad{\phi_1}, \grad{\phi_2} \in \setGFormula$ with $\grad{\phi_1} \mpt \grad{\phi_2}$
            \begin{align*}
            &\grad{\phi_1} \mpt \grad{\phi_2}\\
            \implies
            &\gamma(\grad{\phi_1}) \subseteq \gamma(\grad{\phi_2})\\
            \implies
            &\overline{f}(\gamma(\grad{\phi_1})) \subseteq \overline{f}(\gamma(\grad{\phi_2}))\\
            \implies % closure
            &\overline{f}(\gamma(\grad{\phi_1})) \subseteq \gamma(\alpha(\overline{f}(\gamma(\grad{\phi_2}))))\\
            \implies % rule 2
            &\alpha(\overline{f}(\gamma(\grad{\phi_1}))) \mpt \alpha(\overline{f}(\gamma(\grad{\phi_2})))\\
            \end{align*}
        \end{description}
        
        \item[Optimality]~\\
        
        Proof by contradiction.
        Assume there exists a sound lifting $\grad{f'}$ such that $\grad{f'}(\grad{\phi}) \sqsubset \grad{f}(\grad{\phi})$ for some $\grad{\phi} \in \setGFormula$.
        Using the introduction rule:
        \begin{align*}
        &\forall \phi \in \setFormula.~ f(\phi) \mpt \grad{f'}(\phi)
        \end{align*}
        Using the monotonicity rule:
        \begin{align*}
        &\forall \phi \in \gamma(\grad{\phi}).~ \grad{f'}(\phi) \mpt \grad{f'}(\grad{\phi})
        \end{align*}
        Transitivity:
        \begin{align*}
        &\forall \phi \in \gamma(\grad{\phi}).~ f(\phi) \mpt \grad{f'}(\grad{\phi})\\
        \implies
        &\forall \phi \in \gamma(\grad{\phi}).~ f(\phi) \in \gamma(\grad{f'}(\grad{\phi}))\\
        \implies
        &\overline{f}(\gamma(\grad{\phi})) \subseteq \gamma(\grad{f'}(\grad{\phi}))\\
        \end{align*}
        Using rule 2 of partial Galois connections
        \begin{align*}
        &\overline{f}(\gamma(\grad{\phi})) \subseteq \gamma(\grad{f'}(\grad{\phi}))\\
        \implies
        &\alpha(\overline{f}(\gamma(\grad{\phi}))) \mpt \grad{f'}(\grad{\phi})\\
        \implies
        &\grad{f}(\grad{\phi}) \mpt \grad{f'}(\grad{\phi})\\
        \end{align*}
        Contradiction.
    \end{description}       
\end{proof}


%\begin{displaymath}
%\alpha(\overline{\phi}) = \min_{\sqsubseteq} {\{~ \grad{\phi} ~|~ \overline{\phi} \subseteq \gamma(\grad{\phi}) ~\}}
%\end{displaymath}

    
        \subsubsection{Examples}
        \label{sssec:examples-lift-functions}
        % intro!? NOT the same gradual formula syntax as for predicate examples

%% lAND
\begin{lemma}[Optimal Lifting of And]~\\
    Let $\setGFormula$ be extended as “wildcard with upper bound” (see section \ref{ssec:wildcard-with-upper}).
    We assume that $\vee$ is part of the formula syntax such that $\envs{\phi_1 \vee \phi_2} = \envs{\phi_1} \cup \envs{\phi_2}$.
    $phiOr{\:}{\:}$ can be viewed as a binary function on formulas.
    Let $\gphiOr{\:}{\:} : \setGFormula \times \setGFormula \rightarrow \setGFormula$ be defined as
    \begin{flalign*}
    \gphiOr{$\phi_1$}{$\phi_2$} & \defeq \phiOr{$\phi_1$}{$\phi_2$} \\
    \gphiOr{$\phi_1$}{$(\withqmGen{\phi_2})$} & \defeq \\
    \gphiOr{$(\withqmGen{\phi_1})$}{$\phi_2$} & \defeq \\
    \gphiOr{$(\withqmGen{\phi_1})$}{$(\withqmGen{\phi_2})$} & \defeq \withqmGen{(\phiOr{$\phi_1$}{$\phi_2$})}
    \end{flalign*}
    Then $\gphiOr{\:}{\:}$ is an optimal lifting of $\phiOr{\:}{\:}$.
\end{lemma}
\begin{proof}
    Soundness
        Introduction
        \begin{align*}
        &\phiAnd{$\phi_1$}{$\phi_2$}\\
        =
        &\gphiOr{$\phi_1$}{$\phi_2$}
        \end{align*}
        
        Monotonicity
        Known: $\grad{\phi_1} \mpt \grad{\phi_1'} ~\wedge~ \grad{\phi_2} \mpt \grad{\phi_2'}$
        Case $\grad{\phi_1} = \phi_1 ~\wedge~ \grad{\phi_2} = \phi_2$
            Case $\grad{\phi_1'} = \phi_1 ~\wedge~ \grad{\phi_2'} = \phi_2$
                \begin{align*}
                &\gphiOr{$\grad{\phi_1}$}{$\grad{\phi_2}$} \\
                =
                &\gphiOr{$\phi_1$}{$\phi_2$} \\
                =
                &\gphiOr{$\grad{\phi_1'}$}{$\grad{\phi_2'}$} \\
                \end{align*}
            Case $\grad{\phi_1'} = \withqm{\phi_1'} ~\wedge~ \grad{\phi_2'} = \phi_2$
                \begin{align*}
                &\gphiOr{$\grad{\phi_1}$}{$\grad{\phi_2}$} \\
                =
                &\gphiOr{$\phi_1$}{$\phi_2$} \\
                =
                &\phiOr{$\phi_1$}{$\phi_2$} \\
                \overset{!}{\mpt}
                &\withqmGen{(\phiOr{$\phi_1'$}{$\phi_2$})} \\
                =
                &\gphiOr{$(\withqmGen{\phi_1'})$}{$\phi_2$} \\
                =
                &\gphiOr{$\grad{\phi_1'}$}{$\grad{\phi_2'}$} \\
                \end{align*}
                \begin{align*}
                &\gamma(\phiOr{$\phi_1$}{$\phi_2$}) \\
                =
                &\{~ \phiOr{$\phi_1$}{$\phi_2$} ~\} \\
                \overset{!}{\subseteq}
                &\{~ \phi \in \setFormulaA ~|~ \phiImplies{\phi}{\phiOr{$\phi_1'$}{$\phi_2$}} ~\} \\
                =
                &\gamma(\withqmGen{(\phiOr{$\phi_1'$}{$\phi_2$})})
                \end{align*}
                Note that $\phi_1$ (and thus $\phiOr{$\phi_1$}{$\phi_2$}$) is satisfiable due to $\grad{\phi_1} \mpt \grad{\phi_1'}$.
                
                \begin{align*}
                &\phi_1 \mpt \withqmGen{\phi_1'} \\
                \iff
                &\gamma(\phi_1) \subseteq \gamma(\withqmGen{\phi_1'}) \\
                \iff
                &\phi_1 \in \gamma(\withqmGen{\phi_1'}) \\
                \iff
                &\phi_1 \in \{~ \phi \in \setFormulaA ~|~ \phiImplies{\phi}{\phi_1'} ~\} \\
                \implies
                &\phiImplies{\phi_1}{\phi_1'} \\
                \iff
                &\envs{\phi_1} \subseteq \envs{\phi_1'} \\
                \implies
                &\envs{\phi_1} \cup \envs{\phi_2} \subseteq \envs{\phi_1'} \cup \envs{\phi_2} \\
                \iff
                &\envs{\phiOr{$\phi_1$}{$\phi_2$})} \subseteq \envs{(\phiOr{$\phi_1'$}{$\phi_2$})} \\
                \iff
                &\phiImplies{(\phiOr{$\phi_1$}{$\phi_2$})}{(\phiOr{$\phi_1'$}{$\phi_2$})} \\
                \iff
                &(\phiOr{$\phi_1$}{$\phi_2$}) \in \{~ \phi \in \setFormulaA ~|~ \phiImplies{\phi}{\phiOr{$\phi_1'$}{$\phi_2$}} ~\} \\
                \iff
                &\{~ \phiOr{$\phi_1$}{$\phi_2$} ~\} \subseteq \{~ \phi \in \setFormulaA ~|~ \phiImplies{\phi}{\phiOr{$\phi_1'$}{$\phi_2$}} ~\}
                \end{align*}
            Case $\grad{\phi_1'} = \phi_1 ~\wedge~ \grad{\phi_2'} = \withqm{\phi_2'}$
                Analogous.
            Case $\grad{\phi_1'} = \withqm{\phi_1'} ~\wedge~ \grad{\phi_2'} = \withqm{\phi_2'}$
                Analogous.
        Case $\grad{\phi_1} = \withqmGen{\phi_1} ~\wedge~ \grad{\phi_2} = \phi_2$
            It follows that $\grad{\phi_1'} = \withqmGen{\phi_1'}$ for some $\phi_1'$.
            Case $\grad{\phi_2'} = \phi_2$
                \begin{align*}
                &\gphiOr{$\grad{\phi_1}$}{$\grad{\phi_2}$} \\
                =
                &\gphiOr{$(\withqmGen{\phi_1})$}{$\phi_2$} \\
                =
                &\withqmGen{(\phiOr{$\phi_1$}{$\phi_2$})} \\
                \overset{!}{\mpt}
                &\withqmGen{(\phiOr{$\phi_1'$}{$\phi_2$})} \\
                =
                &\gphiOr{$(\withqmGen{\phi_1'})$}{$\phi_2$} \\
                =
                &\gphiOr{$\grad{\phi_1'}$}{$\grad{\phi_2'}$} \\
                \end{align*}
                \begin{align*}
                &\gamma(\withqmGen{(\phiOr{$\phi_1$}{$\phi_2$})}) \\
                =
                &\{~ \phi \in \setFormulaA ~|~ \phiImplies{\phi}{\phiOr{$\phi_1$}{$\phi_2$}} ~\} \\
                \overset{!}{\subseteq}
                &\{~ \phi \in \setFormulaA ~|~ \phiImplies{\phi}{\phiOr{$\phi_1'$}{$\phi_2$}} ~\} \\
                =
                &\gamma(\withqmGen{(\phiOr{$\phi_1'$}{$\phi_2$})})
                \end{align*}
                \begin{align*}
                &\withqmGen{\phi_1} \mpt \withqmGen{\phi_1'} \\
                \iff
                &\gamma(\withqmGen{\phi_1}) \subseteq \gamma(\withqmGen{\phi_1'}) \\
                \iff
                &\{~ \phi \in \setFormulaA ~|~ \phiImplies{\phi}{\phi_1} ~\}
                \subseteq 
                \{~ \phi \in \setFormulaA ~|~ \phiImplies{\phi}{\phi_1'} ~\}\\
                \iff
                &\{~ \phi \in \setFormulaA ~|~ \phiImplies{\phi}{\phi_1} ~\vee~ \phiImplies{\phi}{\phi_2} ~\}
                \subseteq 
                \{~ \phi \in \setFormulaA ~|~ \phiImplies{\phi}{\phi_1'} ~\vee~ \phiImplies{\phi}{\phi_2} ~\}\\
                \iff
                &\{~ \phi \in \setFormulaA ~|~ \phiImplies{\phi}{\phiOr{$\phi_1$}{$\phi_2$}} ~\}
                \subseteq 
                \{~ \phi \in \setFormulaA ~|~ \phiImplies{\phi}{\phiOr{$\phi_1'$}{$\phi_2$}} ~\}
                \end{align*}
            Case $\grad{\phi_2'} = \withqmGen{\phi_2'}$ for some $\phi_2'$.
                \begin{align*}
                &\gphiOr{$\grad{\phi_1}$}{$\grad{\phi_2}$} \\
                =
                &\gphiOr{$(\withqmGen{\phi_1})$}{$\phi_2$} \\
                =
                &\withqmGen{(\phiOr{$\phi_1$}{$\phi_2$})} \\
                \overset{!}{\mpt}
                &\withqmGen{(\phiOr{$\phi_1'$}{$\phi_2'$})} \\
                =
                &\gphiOr{$(\withqmGen{\phi_1'})$}{$(\withqmGen{\phi_2'})$} \\
                =
                &\gphiOr{$\grad{\phi_1'}$}{$\grad{\phi_2'}$} \\
                \end{align*}
                \begin{align*}
                &\gamma(\withqmGen{(\phiOr{$\phi_1$}{$\phi_2$})}) \\
                =
                &\{~ \phi \in \setFormulaA ~|~ \phiImplies{\phi}{\phiOr{$\phi_1$}{$\phi_2$}} ~\} \\
                \overset{!}{\subseteq}
                &\{~ \phi \in \setFormulaA ~|~ \phiImplies{\phi}{\phiOr{$\phi_1'$}{$\phi_2'$}} ~\} \\
                =
                &\gamma(\withqmGen{(\phiOr{$\phi_1'$}{$\phi_2'$})})
                \end{align*}
                \begin{align*}
                &(\withqmGen{\phi_1} \mpt \withqmGen{\phi_1'}) ~\wedge~ (\withqmGen{\phi_2} \mpt \withqmGen{\phi_2'}) \\
                \iff
                &\gamma(\withqmGen{\phi_1}) \subseteq \gamma(\withqmGen{\phi_1'}) ~\wedge~ \gamma(\withqmGen{\phi_2}) \subseteq \gamma(\withqmGen{\phi_2'}) \\
                \iff
                &\{~ \phi \in \setFormulaA ~|~ \phiImplies{\phi}{\phi_1} ~\} \subseteq \{~ \phi \in \setFormulaA ~|~ \phiImplies{\phi}{\phi_1'} ~\} ~\wedge~ \{~ \phi \in \setFormulaA ~|~ \phiImplies{\phi}{\phi_2} ~\} \subseteq \{~ \phi \in \setFormulaA ~|~ \phiImplies{\phi}{\phi_2'} ~\} \\
                \implies
                &\{~ \phi \in \setFormulaA ~|~ \phiImplies{\phi}{\phi_1} ~\} \cup \{~ \phi \in \setFormulaA ~|~ \phiImplies{\phi}{\phi_2} ~\}
                \subseteq 
                \{~ \phi \in \setFormulaA ~|~ \phiImplies{\phi}{\phi_1'} ~\} \cup \{~ \phi \in \setFormulaA ~|~ \phiImplies{\phi}{\phi_2'} ~\}\\
                \iff
                &\{~ \phi \in \setFormulaA ~|~ \phiImplies{\phi}{\phi_1} ~\vee~ \phiImplies{\phi}{\phi_2} ~\}
                \subseteq 
                \{~ \phi \in \setFormulaA ~|~ \phiImplies{\phi}{\phi_1'} ~\vee~ \phiImplies{\phi}{\phi_2'} ~\}\\
                \iff
                &\{~ \phi \in \setFormulaA ~|~ \phiImplies{\phi}{\phiOr{$\phi_1$}{$\phi_2$}} ~\}
                \subseteq 
                \{~ \phi \in \setFormulaA ~|~ \phiImplies{\phi}{\phiOr{$\phi_1'$}{$\phi_2'$}} ~\}
                \end{align*}
        Case $\grad{\phi_1} = \phi_1 ~\wedge~ \grad{\phi_2} = \withqmGen{\phi_2}$
            Analogous.
        Case $\grad{\phi_1} = \withqmGen{\phi_1} ~\wedge~ \grad{\phi_2} = \withqmGen{\phi_2}$
            It follows that $\grad{\phi_1'} = \withqmGen{\phi_1'}$ for some $\phi_1'$ 
                   and that $\grad{\phi_2'} = \withqmGen{\phi_2'}$ for some $\phi_2'$.
                   \begin{align*}
                   &\gphiOr{$\grad{\phi_1}$}{$\grad{\phi_2}$} \\
                   =
                   &\gphiOr{$(\withqmGen{\phi_1})$}{$\withqmGen{\phi_2}$} \\
                   =
                   &\withqmGen{(\phiOr{$\phi_1$}{$\phi_2$})} \\
                   \overset{!}{\mpt}
                   &\withqmGen{(\phiOr{$\phi_1'$}{$\phi_2'$})} \\
                   =
                   &\gphiOr{$(\withqmGen{\phi_1'})$}{$(\withqmGen{\phi_2'})$} \\
                   =
                   &\gphiOr{$\grad{\phi_1'}$}{$\grad{\phi_2'}$} \\
                   \end{align*}
            
    
    \begin{comment} does not work due to non-existence of galois connection - not even partial for TOTAL and function (partial would work)
    Goal:
    \begin{displaymath}
    \forall \grad{\phi_1}, \grad{\phi_2} \in \setGFormula.~ \gphiAnd{$\grad{\phi_1}$}{$\grad{\phi_2}$} = \alpha(\{~ \phiAnd{$\phi_1$}{$\phi_2$} ~|~ \phi_1 \in \gamma(\grad{\phi_1}),\, \phi_2 \in \gamma(\grad{\phi_2}) ~\})
    \end{displaymath}
    
    Case $\grad{\phi_1} = \phi_1' \wedge \grad{\phi_2} = \phi_2'$:
    \begin{align*}
    &\alpha(\{~ \phiAnd{$\phi_1$}{$\phi_2$} ~|~ \phi_1 \in \gamma(\phi_1'),\, \phi_2 \in \gamma(\phi_2') ~\})\\
    =
    &\alpha(\{~ \phiAnd{$\phi_1'$}{$\phi_2'$} ~\})\\
    =
    &\phiAnd{$\phi_1'$}{$\phi_2'$}\\
    =
    &\gphiAnd{$\phi_1'$}{$\phi_2'$}\\
    \end{align*}
    
    Case $\grad{\phi_1} = \withqm{\phi_1'} \wedge \grad{\phi_2} = \phi_2'$:
    \begin{align*}
    &\alpha(\{~ \phiAnd{$\phi_1$}{$\phi_2$} ~|~ \phi_1 \in \gamma(\withqm{\phi_1'}),\, \phi_2 \in \gamma(\phi_2') ~\})\\
    =
    &\alpha(\{~ \phiAnd{$\phi_1$}{$\phi_2'$} ~|~ \phi_1 \in \setFormulaA ~\wedge~ \phiImplies{\phi_1}{\phi_1'} ~\})\\
    =
    &\alpha(\{~ \phiAnd{$\phi_1'$}{$\phi_2'$} ~\})\\
    =
    &\phiAnd{$\phi_1'$}{$\phi_2'$}\\
    =
    &\gphiAnd{$\phi_1'$}{$\phi_2'$}\\
    \end{align*}
    \end{comment}
\end{proof}

%% composite
\begin{lemma}[Sound Lifting of Composed Function]~\\
    Let $g, f : \setFormula \rightarrow \setFormula$ be arbitrary functions.
    
    Let $\grad{(g \circ f)} : \setGFormula \rightarrow \setGFormula$ be defined as
    \begin{displaymath}
    \grad{(g \circ f)} ~\defeq~ \grad{g} \circ \grad{f}
    \end{displaymath}
    with sound liftings $\grad{g}$ and $\grad{f}$.
    
    Then $\grad{(g \circ f)}$ is a sound lifting of $(g \circ f)$, i.e. “piecewise” lifting of composed functions is allowed.
    Optimality of $\grad{g}$ and $\grad{f}$ does not imply optimality of $\grad{(g \circ f)}$.
\end{lemma}
\begin{proof}
    Introduction
    \begin{align*}
    &g(f(\phi))\\
    \overset{Introduction~\grad{g}}{\mpt}
    &\grad{g}(f(\phi))\\
    \overset{\substack{Introduction~\grad{f}\\\&\\Monotonicity~\grad{g}}}{\mpt}
    &\grad{g}(\grad{f}(\phi))\\
    =
    &\grad{g}(\grad{f}(\phi))\\
    =
    &(\grad{g} \circ \grad{f})(\phi)\\
    =
    &\grad{(g \circ f)}(\phi)
    \end{align*}
    
    Monotonicity
    \begin{align*}
    &\grad{\phi_1} \mpt \grad{\phi_2}\\
    \overset{Monotonicity~\grad{f}}{\implies}
    &\grad{f}(\grad{\phi_1}) \mpt \grad{f}(\grad{\phi_2})\\
    \overset{Monotonicity~\grad{g}}{\implies}
    &\grad{g}(\grad{f}(\grad{\phi_1})) \mpt \grad{g}(\grad{f}(\grad{\phi_2}))\\
    \overset{Definition}{\implies}
    &\grad{(g \circ f)}(\grad{\phi_1}) \mpt \grad{(g \circ f)}(\grad{\phi_2})
    \end{align*}
\end{proof}

\begin{comment}
\begin{align*}
\grad{f}(\grad{\phi_1}, \grad{\phi_2}) = \alpha(\{~ \phiAnd{$\phi_1$}{$\phi_2$} ~|~ \phi_1 \in \gamma(\grad{\phi_1}) \wedge \phi_2 \in \gamma(\grad{\phi_2}) ~\})
\end{align*}
\end{comment}




% mention alpha(...), galois connection (does not always exist, make example... so partial GC instead (reference)...)

% function composition (soundness, optimality?), ...
        
        \subsubsection{Lifting Partial Functions}
        \label{sssec:lifting-partial-functions}
        Semantics can be defined in terms of partial functions or even be a partial function as is the case for the small-step semantics of \svl.
We derive rules for lifting partial functions using the following decomposition:

\begin{lemma}[Partial Function Decomposition]
    Let $f : \setFormula \rightharpoonup \setFormula$ be a partial function.
    Then there exists a total function $f' : \setFormula \rightarrow \setFormula$ and a predicate $F \subseteq \setFormula$ such that
    \begin{flalign*}
    &f(\phi) = f'(\phi)  \quad\text{ if } F(\phi)\\
    &f ~\text{ undefined otherwise}
    \end{flalign*}
\end{lemma}

Composing $\grad{f}$ from the gradual liftings of $f$'s decomposition gives rise to the following rules for lifting partial functions.

\begin{description}
    \item[Introduction]~\\
    \begin{displaymath}
    \forall \phi \in \setFormula \cap \dom(f).~ f(\phi) \sqsubseteq \grad{f}(\phi)
    \end{displaymath}
    
    \item[Monotonicity]~\\
    \begin{displaymath}
    \forall \grad{\phi_1}, \grad{\phi_2} \in \setGFormula.~ 
    \grad{\phi_1} \sqsubseteq \grad{\phi_2} \wedge \grad{\phi_1} \in \dom(\grad{f}) \implies \grad{f}(\grad{\phi_1}) \sqsubseteq \grad{f}(\grad{\phi_2})
    \end{displaymath}
\end{description}

Soundness and optimality are defined as usual.

    
    \subsection{Generalized Lifting}
    \label{ssec:generalized-lifting}
    The previous sections describe how lifting is performed in order to deal with $\setGFormula$ instead of $\setFormula$.
In general, the same rules apply to any gradual extension of an existing set that comes with a concretization function, e.g. $\setGStmt$ or $\setGProgramState$.

For instance, the signature of Hoare rules contains $\setStmt$ and can therefore be lifted w.r.t. this parameter using the definitions in section \ref{sec:gradual-statements}.

\section{Gradual Soundness vs Gradual Guarantee}
\label{ssec:gradual-soundness}
Valid Hoare triples for gradual system
\begin{flalign*}
& \gtHoare {~} {\cdot} {\cdot} {\cdot} ~~~\subseteq~~~ \setGFormula \times \setGStmt \times \setGFormula                                                                                                                                                                          \\
& \gtHoare {~} {\grad{\phi_{pre}}} {\grad{s}} {\grad{\phi_{post}}} ~\defiff~ 
\forall \langle \grad{\pi_{pre}}, \grad{\pi_{post}} \rangle \in \gsssem^{\grad{s}}.~ \evalgphiGen{\grad{\pi_{pre}}}{\grad{\phi_{pre}}} \implies \evalgphiGen{\grad{\pi_{post}}}{\grad{\phi_{post}}}
\end{flalign*}
(Note: NOT A gradual LIFTING! Sound lifting would accept $\gtHoare{~}{\qm}{\sVarAssign{x}{3}}{\withqmGen{\phiEq{y}{4}}}$)


Soundness of gradual system:
\begin{mathpar}
    \inferrule* [Right=GProgress]
    {
        \grad{\pi} \in \grad{\wsp}(\grad{s_1}) \\ 
    }
    {
        \exists n \in \setNat,\, \grad{s_2} \in \setGStmt.~ \gsssem^n(\grad{\pi}) \in \setProgramState_{\grad{s_2}}
    }
\end{mathpar} 
\begin{mathpar}
    \inferrule* [Right=GPreservation]
    {
        \gthoare{~}{\grad{\phi_1}}{\grad{s}}{\grad{\phi_2}}
    }
    {
        \gtHoare{~}{\grad{\phi_1}}{\grad{s}}{\grad{\phi_2}}
    }
\end{mathpar}


Gradual guarantee:
Let $\gthoare{~}{\cdot}{\cdot}{\cdot}$ be gradual lifting of $\thoare{~}{\cdot}{\cdot}{\cdot}$.
Then:
\begin{align*}
&&\thoare {~} {\phiEq{x}{2}} {\sVarAssign{y}{3}} {\phiEq{x}{2} \wedge \phiEq{y}{3}}&\\
\overset{Introduction}{\implies}&&
\gthoare {~} {\phiEq{x}{2}} {\sVarAssign{y}{3}} {\phiEq{x}{2} \wedge \phiEq{y}{3}}&\\
\overset{Monotonicity}{\implies}&&
\gthoare {~} {\qm} {\sVarAssign{y}{3}} {\phiEq{x}{2} \wedge \phiEq{y}{3}}&
\end{align*}
Preservation is obviously not satisfied!

Reiteration:
\begin{mathpar}
    \inferrule* [Right=Preservation']
    {
        \thoare{~}{\phi_1}{s}{\phi_2}
    }
    {
        \tHoare{~}{\phi_1}{\sSeq{$s$}{\sAssert{$\phi_2$}}}{\phi_2}
    }
\end{mathpar}
\begin{mathpar}
    \inferrule* [Right=GPreservation']
    {
        \gthoare{~}{\grad{\phi_1}}{\grad{s}}{\grad{\phi_2}}
    }
    {
        \gtHoare{~}{\grad{\phi_1}}{\sSeq{$\grad{s}$}{\sAssert{$\grad{\phi_2}$}}}{\grad{\phi_2}}
    }
\end{mathpar}

TODO: more bla, like “there is fundamentally no way around this - the programmer \textit{can} specify postconditions that...”

\section{Abstracting Static Semantics}
\label{sec:abstracting-static-semantics}
With the rules for lifting set up we can apply them to the static verification predicate:
Lifting 
$$\thoare {~} {\cdot} {\cdot} {\cdot} ~~~\subseteq~~~ \setFormula \times \setStmt \times \setFormula$$
w.r.t. both pre- and postcondition results in a predicate
$$\gthoare {~} {\cdot} {\cdot} {\cdot} ~~~\subseteq~~~ \setGFormula \times \setStmt \times \setGFormula$$

% EXAMPLE with proof

%% what's wrong with this predicate
Recall what soundness means for a static verification system:
Assume that $\thoare {~} {\phi_{pre}} {\overline{s}} {\phi_{post}}$ holds in the static verification system.
Given a program state that satisfies $\phi_{pre}$, soundness guarantees us both that execution won't get blocked when executing $\overline{s}$ (progress) and that the program state satisfies $\phi_{post}$ afterwards (preservation).

Unfortunately, the gradual verification predicate severely violates the preservation property.
While we don't expect progress to still be guaranteed without changes to dynamic semantics, there is no way to “fix” preservation.
Consider a sound static verification system (with an assignment rule and a sequence rule) that allows verifying

\begin{mathpar}
\inferrule* [Right=Seq]
{
    {
        \inferrule* [right=Ass]
        { ~ }
        {
            \thoare {~} {\phiTrue} {\sVarAssign{x}{2}} {\phiEq{x}{2}}
        }
    }\\
    {
        \inferrule* [Right=Ass]
        { ~ }
        {
            \thoare {~} {\phiEq{x}{2}} {\sVarAssign{y}{3}} {\phiEq{x}{2} \wedge \phiEq{y}{3}}
        }
    }
}
{
    \thoare {~} {\phiTrue} {\sVarAssign{x}{2}~\sVarAssign{y}{3}} {\phiEq{x}{2} \wedge \phiEq{y}{3}}
}
\end{mathpar}
With the rules of lifting we can deduce
\begin{align*}
&&\thoare {~} {\phiEq{x}{42}} {\sAssert{\phiNeq{x}{0}}} {\phiEq{x}{42}}&\\
\implies&&
\gthoare {~} {\phiEq{x}{42}} {\sAssert{\phiNeq{x}{0}}} {\phiEq{x}{42}}&\\
\implies&&
\gthoare {~} {\qm} {\sAssert{\phiNeq{x}{0}}} {\phiEq{x}{42}}&
\end{align*}

Apparently the precondition does not restrict the program state before executing the assertion.
Let's assume that we 

In the static system, we are guaranteed that the runtime satisfies a postcondition, given that the precondition should execution
There is something fundamentally wrong with this 



% non-deterministic static hoare rules make “quality of lifting” reasoning hard! if even {a}...{true} is valid statically, how to express/measure “badness” of emitting {a}...{?}
% Hoare rules (in a sound language) GUARANTEE that intermediate formulas hold at runtime 
% non-deterministic gradual hoare rules do NOT guarantee that!

% take burdon of chosing “good” intermediate formulas off verifier:
\begin{verbatim}
{ ? }
x := 2;
{ ? } // too weak, not optimal
assert (x = 3);
\end{verbatim}
\begin{verbatim}
{ ? }
assert (x != 0);
{ (x = 42) } // too strong, “somewhat unsound!!!” (no guarantee of holding at runtime, assuming previsous formula held)

// BUT supported by instantiation!!!

{ (x = 42) }
assert (x != 0);
{ (x = 42) }
\end{verbatim}
% instead, make rules deterministic and inherently sound
% => they might still not be optimal (we will define measure), but at least verifier is not to blame
% => about non-optimality formalism is to blame, not some inference mechanism (conflict of interest! formulas as strong as possible, verifier as “good”/successful as possible)
% => whether “working” intermediate formula exists is generally not even decidable! (already follows from satisfiability itself not being decidable...)

\begin{align*}
\funHoare_s : \setFormulaB \rightharpoonup \setFormulaB\\
\funHoare_s = \funHoareC_s \circ \funHoareB_s \circ \funHoareA_s\\
\funHoare_s(\pb{\phi}) = \funHoareC_s(\pb{\phi}) \quad\text{ if $\funHoareApred(s) \wedge \pb{\phi} \implies \funHoareBimp(s)$} \\
\funHoareA_s : \setFormulaB \rightharpoonup \setFormulaB\\
\funHoareA_s(\pb{\phi}) = \pb{\phi} \quad\text{ if $\funHoareApred(s)$} \\
\funHoareB_s : \setFormulaB \rightharpoonup \setFormulaB\\
\funHoareB_s(\pb{\phi}) = \pb{\phi} \quad\text{ if $\pb{\phi} \implies \funHoareBimp(s)$} \\
\funHoareC_s : \setFormulaB \rightarrow \setFormulaB\\
\end{align*}

% example of determinified Hoare rule

% lifting... challenges
    
    \subsection{The Problem with a Predicate Lifting}
    \label{ssec:the-problem-with}
    
%% Intro
As seen in section \ref{ssec:gradual-soundness}, the lifted Hoare predicate in general requires an additional assertion to guarantee preservation.
% There is no way around this (the programmer simply \textit{is} able to specify postconditions that are not guaranteed to hold)
Yet, there is a more fundamental design issue connected to the gradual lifting approach which we will illustrate in this section.

%% Rule-wise approach
...rule-wise lifting yields overall lifting... neat.

%% problem
Problem: non-deterministic! Compiler has to find “good” intermediate formulas
\begin{description}
    \item[too weak] could always choose $\qm$
    \item[too strong] could choose stuff that is not guaranteed by runtime... (so: inject runtime assertions? yes: could be wrong! no: could enter method violating precondition)
\end{description}
% other problem: compositition and guessing
    
    \subsection{The Deterministic Approach}
    \label{ssec:the-deterministic-approach}
    % proof by example: optimal deterministic lifting does not induce optimal predicate lifting!
% not verifiable with consistent predicate
% { ? }
% release acc(x.f)
% { ? * acc(x.f) }
%
% but
% { ? }
% release acc(x.f)
% { ? }
% =>
% { ? * acc(x.f) }
%
%
% not verifiable with consistent predicate
% { ? }
% x := random();
% { (x = 0) }
%
% but
% { ? }
% x := random();
% { ? }
% =>
% { (x = 0) }

The approach we propose is based on the idea to treat the Hoare predicate as a (multivalued) function, mapping preconditions to the set of possible/verifiable postconditions.
We can obtain a lifted version of this hypothetical construct and demand certain properties similar to the ones defined in section \ref{text/SSEC-lifting-functions}:

\begin{definition}[Deterministic Lifting]
    Given a binary predicate $P \subseteq \setFormula \times \setFormula$ we call a partial function $\dgrad{P} : \setFormula \rightharpoonup \setFormula$ \textbf{deterministic lifting} of $P$ if the following conditions are met:
    \begin{description}
        \item[Introduction]~\\
        \begin{displaymath}
        \forall (\phi_1, \phi_2) \in P.~ \phi_1 \in \dom(\dgrad{P})
        \end{displaymath}
        
        \item[Preservation]~\\
        \begin{mathpar}
        \forall \phi_1 \in \setFormula, \grad{\phi_2} \in \setGFormula.~ 
        \dgrad{P}(\phi_1) = \grad{\phi_2}\\
        \implies\\
        \exists \phi_2 \in \setFormula.~ P(\phi_1, \phi_2) ~~\wedge~~ \gphiImplies{\phi_2}{\grad{\phi_2}}\\
        \wedge\\
        \forall \phi_2 \in \setFormula.~ P(\phi_1, \phi_2) \implies \gphiImplies{\grad{\phi_2}}{\phi_2}
        \end{mathpar} 
        
        \item[Monotonicity]~\\
        Note: Identical to monotonicity condition of lifted partial functions. % say why? P/f doesn't show up.
        \begin{displaymath}
        \forall \grad{\phi_1}, \grad{\phi_2} \in \setGFormula.~ \grad{\phi_1} \sqsubseteq \grad{\phi_2} \wedge \grad{\phi_1} \in \dom(\dgrad{P}) \implies \dgrad{P}(\grad{\phi_1}) \sqsubseteq \dgrad{P}(\grad{\phi_2})
        \end{displaymath}
    \end{description}
\end{definition}

...assume we have obtained deterministic lifting $\dgthoare {~} {\cdot} {\cdot} {\cdot}$ of our Hoare triple. % TODO: notation talk
This gradual partial function has desirable properties: % use lemmas only? PROOFS
\begin{description}
    \item[Obtaining Sound Gradual Lifting]
    \begin{lemma}[Obtaining Sound Gradual Lifting]~\\
        \begin{displaymath}
        \gthoare {~} {\grad{\phi_1}} {\overline{s}} {\grad{\phi_2}} ~~\defiff~~ \exists \grad{\phi_2'}.~ \dgthoare {~} {\grad{\phi_1}} {\overline{s}} {\grad{\phi_2'}} \wedge \gphiImplies {\grad{\phi_2'}} {\grad{\phi_2}}
        \end{displaymath}
        is a sound gradual lifting.
    \end{lemma}
    
    \item[Determinism]~\\
    The verifier has no more degrees of freedom as it is now dealing with a function instead of a predicate. % So also no more of the problems/dilemmas apply
    \item[Preservation]~\\
    A (gradual) postcondition returned by the lifted function is guaranteed to reflect the execution state after executing the statements in question (given that the precondition was met).
    \item[Composability]~\\
    Composing two ...
\end{description}


% EXAMPLE LIFTINGS

\section{Abstracting Dynamic Semantics}
\label{sec:abstracting-dynamic-semantics}
%% intro
Static verification provides guarantees about the dynamic behavior of a program without runtime overhead.
With the introduction of dynamic components to the assertion language, these guarantees can be impossible to provide without additional runtime checks.

%% example
Consider a (static) Hoare logic that can verify
\begin{displaymath}
\thoare{~}{\phiEq{x}{3}}{\sVarAssign{y}{4}}{\phiAnd{\phiEq{x}{3}}{\phiEq{y}{4}}}
\end{displaymath}
The corresponding gradual logic can thus verify
\begin{displaymath}
\gthoare{~}{\qm}{\sVarAssign{y}{4}}{\phiAnd{\phiEq{x}{3}}{\phiEq{y}{4}}}
\end{displaymath}
but without additional measures there is no guarantee that every program state satisfying the precondition (here: literally every program state) will satisfy the post condition after executing the assignment.
Adding a runtime assertion to ensure that $\phiEq{x}{3}$ holds (terminating exceptionally in case it does not) would be sufficient and minimal to reestablish soundness in the above example.

We will generalize and formalize this approach, resulting in gradual a dynamic semantics that reestablishes the soundness of the static verification system by adding minimal runtime checks.

%% first: changes in source code!!!
% making formula gradual 
% - cannot let execution fail 
% - cannot satisfy less formulas (“observational compatibility”})

\begin{align*}
\grad{\pi_1} \sqsubseteq_{\pi} \grad{\pi_2}
~\defiff~
&\forall \phi \in \setFormula.~ \evalphiGen{\grad{\pi_1}}{\phi} \implies \evalphiGen{\grad{\pi_2}}{\phi} \\
~~\wedge~~
&\sqsubseteq \textit{f.a. statements}\\
~~\vee~~
&\grad{\pi_1} \in \setGProgramStateEx
\end{align*}
Ex. concretization = emptyset??? could simplify things! :)

% TODO:
- alter $\sssem$ in order to make static semantics complete ($\forall s \in \setStmt, \pi_s \in \setProgramState_s.~ \pi_s \in \wsp(s) \iff \sssem^s(\pi_s) \textit{ not stuck}$) - these are runtime checks
- lift that (see below)
- tango time



\begin{mathpar}
    \inferrule* [Right=DGSoundness]
    {
        \inferrule*
        {
            \inferrule*
            {
                \dgthoare{~}{\grad{\phi_1}}{\sSeq{$\grad{s_1}$}{$\grad{s_2}$}}{\grad{\phi_3}}\\
                \grad{\pi_{\sSeq{$\grad{s_1}$}{$\grad{s_2}$}}} \,\in \setGProgramState_{\sSeq{$\grad{s_1}$}{$\grad{s_2}$}}\\
                \evalgphiGen{\grad{\pi_{\sSeq{$\grad{s_1}$}{$\grad{s_2}$}}}}{\grad{\phi_1}}
            }
            {
                ...
            }
        }
        {
            \evalgphiGen{\gsssem^{\grad{s_2}}(\gsssem^{\grad{s_1}}(\grad{\pi_{\sSeq{$\grad{s_1}$}{$\grad{s_2}$}}}))}{\grad{\phi_3}}
        }
    }
    {
        \evalgphiGen{\gsssem^{\sSeq{$\grad{s_1}$}{$\grad{s_2}$}}(\grad{\pi_{\sSeq{$\grad{s_1}$}{$\grad{s_2}$}}})}{\grad{\phi_3}}
    }
\end{mathpar}

\begin{description}
    \item $\grad{s} \in \setGStmt$
    \item $\grad{\phi_1}, \grad{\phi_2} \in \setGFormula$
    \item $\grad{\pi_{\grad{s}}} \,\in \setGProgramState_{\grad{s}}$
    \item[1 = PremiseA] $\dgthoare{~}{\grad{\phi_1}}{\grad{s}}{\grad{\phi_2}}$
    \item[2 = PremiseB] $\evalgphiGen{\grad{\pi_{\grad{s}}}}{\grad{\phi_1}}$
    \item[3 = Case] $\exists \pi_s \in \gamma(\grad{\pi_{\grad{s}}}).~ \pi_s \in \wsp(s)$
    \item[4 = 1 + concret] $\evalgphiGen{\pi_s}{\grad{\phi_1}}$
    \item[5 = 3 + wsp def] $\exists \phi_1', \phi' \in \setFormula.~ \evalphiGen{\pi_s}{\phi_1'} ~\wedge \thoare{~}{\phi_1'}{s}{\phi'}$
    \item[6 = 4 + 5 + rule42] $\exists \phi_1 \in \gamma(\grad{\phi_1}).~ \phiImplies{\phi_1}{\phi_1'} \wedge \evalphiGen{\pi_s}{\phi_1}$
    \item[7 = 5 + 6 + mono] $\exists \phi \in \setFormula.~ \thoare{~}{\phi_1}{s}{\phi}$
    \item[8 = 7 + intro] $\exists \grad{\phi} \in \setGFormula.~ \dgthoare{~}{\phi_1}{s}{\grad{\phi}}$
    \item[9 = 1 + 6 + 8 + mono_det_hoare] $\grad{\phi} \sqsubseteq \grad{\phi_2}$
    \item[10 = 8 + pres] $\exists \phi_2 \in \gamma(\grad{\phi}).~ \thoare{~}{\phi_1}{s}{\phi_2}$
    \item[11 = 6 + 10 + snd] $\evalphiGen{\sssem^s(\pi_s)}{\phi_2}$
    \item[12 = 11 + intro] $\evalphiGen{\gsssem^s(\pi_s)}{\phi_2}$
    \item[13 = 3 + 12 + mono] $\evalphiGen{\gsssem^{\grad{s}}(\grad{\pi_{\grad{s}}})}{\phi_2}$
    \item[14 = 13 + intro] $\evalgphiGen{\gsssem^{\grad{s}}(\grad{\pi_{\grad{s}}})}{\phi_2}$
    \item[15 = 10 + 14 + mono] $\evalgphiGen{\gsssem^{\grad{s}}(\grad{\pi_{\grad{s}}})}{\grad{\phi}}$
    \item[16 = 9 + 15 + mono] $\evalgphiGen{\gsssem^{\grad{s}}(\grad{\pi_{\grad{s}}})}{\grad{\phi_2}}$
\end{description}


\begin{description}
    \item $\grad{s} \in \setGStmt$
    \item $\grad{\phi_1}, \grad{\phi_2} \in \setGFormula$
    \item $\grad{\pi_{\grad{s}}} \,\in \setGProgramState_{\grad{s}}$
    \item[1 = PremiseA] $\dgthoare{~}{\grad{\phi_1}}{\grad{s}}{\grad{\phi_2}}$
    \item[2 = PremiseB] $\evalgphiGen{\grad{\pi_{\grad{s}}}}{\grad{\phi_1}}$
    \item[3 = Case] $\neg \exists \pi_s \in \gamma(\grad{\pi_{\grad{s}}}).~ \pi_s \in \wsp(s)$
    \item[4 = 3 + completeness] $\forall \pi_s \in \gamma(\grad{\pi_{\grad{s}}}).~ \sssem^s(\pi_s) \textit{ stuck}$
    \item[5 = 4 + def] $\gsssem^{\grad{s}}(\grad{\pi_{\grad{s}}}) = \pi_{EX}$
    \item[6 = 5 + precision] $\evalgphiGen{\gsssem^{\grad{s}}(\grad{\pi_{\grad{s}}})}{\grad{\phi_2}}$
\end{description} 

\begin{mathpar}
    \inferrule* [Right=GSoundness]
    {
        \gthoare{~}{\grad{\phi_1}}{\grad{s}}{\grad{\phi_2}} \\ 
        \grad{\pi_{\grad{s}}} \,\in \setGProgramState_{\grad{s}}\\
        \evalgphiGen{\grad{\pi_{\grad{s}}}}{\grad{\phi_1}}
    }
    {
        \evalgphiGen{\gsssem^{\grad{s}}(\grad{\pi_{\grad{s}}})}{\grad{\phi_2}}
    }
\end{mathpar}
    
    \subsection{Perfect Knowledge}
    \label{ssec:perfect-knowledge}
    
Choose ~$\gsssem : \setGProgramState \rightarrow \setGProgramState$~ as lifted version of ~$\sssem : \setProgramState \rightharpoonup \setProgramState$ with $\gsssem(\grad{\pi}) = \pi_{EX}$ if stuck for all concretizations.

\begin{align*}
\wsp : \setGStmt \rightarrow \setProgramState \\
\wsp(\grad{s}) ~\defeq~ \bigcup_{s \in \gamma_s(\grad{s})} \wsp(s)
\end{align*}

\begin{align*}
\forall \grad{s} \in \setGStmt.~&\\ 
\grad{\pi_{\grad{s}}} \in \setGProgramState_{\grad{s}}.~& \wsp(\grad{s}) \cap \gamma_{\pi}(\grad{\pi_{\grad{s}}}) = \emptyset
\implies
\gsssem^{\grad{s}}(\grad{\pi_{\grad{s}}}) = \pi_{EX}
\end{align*}
 
 
MINUS:
- need above knowledge...
- 
    
    \subsection{Partial Knowledge}
    \label{ssec:atomic--knowledge}
    %% intro
Completeness as defined in definition \ref{def:completeness} tightly couples Hoare logic and small-step semantics.
In practice this can be impossible to achieve due to decidability, i.e. not all valid small-step derivations can be modeled using Hoare logic.
Similarly, semi-optimality of the gradual semantics might be hard to ensure.
Semi-optimality of the gradual Hoare logic requires deciding the existence of a Hoare logic derivation which involves first order logic for composite statements like sequences (which was a problem we originally wanted to avoid with deterministic liftings).
In this section we motivate that a weaker notion of completeness and semi-optimality may be sufficient to prove \tset{\dgradT Soundness}.

%% 
The sequence operator \ttt{;} is key to defining composite statements in most programming languages.
Fortunately, \tset{\dgradT Soundness} can be proved for sequences inductively.
\begin{lemma}[\tset{\dgradT Soundness} for Sequences]
    \label{lemma:gdpres-seq}~\\
    If \tset{\dgradT Soundness} holds for statements $s_1$ and $s_2$ then it holds for $s_1;s_2$
\end{lemma}

In the case study we will use this lemma to prove \tset{\dgradT Soundness} of a gradually verified language, but also show that this inductive approach can be applied to other composite statements like method calls (section \ref{sec:gradual-soundness}).
As a result, it is sufficient to prove \tset{\dgradT Soundness} for “primitive” statements, e.g. by using the approach introduced in the previous section:
Note that the definitions of completeness and semi-optimality are universally quantified over the set of all (gradual) statements.
Instead, they can be weakened to quantify only over a limited set of primitive statements.
The resulting proof of \tset{\dgradT Soundness} (lemma \ref{thm:compl-and-so-to-gdpres}) will apply only to this set of statements.

Again, we want to point out that we only give examples of sufficient criteria to prove \tset{\dgradT Soundness}.
It is possible that the approaches do not work for a certain programming language, or even that it is entirely impossible to satisfy \tset{\dgradT Soundness}.
However, recall that \tset{\dgradT Soundness} is not necessary for \gvl to be sound (i.e. satisfy \tset{\gradT Soundness}).



\chapter{Case Study: Implicit Dynamic Frames}
\label{ch:case-study--implicit}
To show the flexibility of our approach, we apply it to a simple statically verified Java-like language \svlidf that uses implicit dynamic frames to enable safe reasoning about mutable state (see section \ref{ssec:implicit-dynamic-frames} for introduction and examples).
The usage of implicit dynamic frames poses a challenge as it introduces elements of linear logic into formula semantics.
However, despite the fact that we used classical logic for the examples throughout chapter \ref{ch:gradualization-of-a}, our approach never made an assumption about it.
The logic at hand is abstracted away behind formula semantics $\evalphiGen{\pi}{\phi}$.

This chapter roughly follows the structure of chapter \ref{ch:gradualization-of-a}, obtaining a gradually verified language \gvlidf from \svlidf.
It starts with a full definition of \svlidf in section \ref{sec:language}, instantiating the elements postulated in section \ref{sec:a-statically-verified}.
Section \ref{sec:cs-gradual-formulas} describes our decisions regarding the syntax of \gvlidf, defining \setGFormula, \setGStmt and \setGProgramState.
Using the concept of deterministic lifting introduced in section \ref{ssec:the-deterministic-approach} we will obtain gradual Hoare logic of \gvlidf in section \ref{sec:gradualize-hoare-rules}.
In section \ref{sec:gradual-dyn--semantics} we derive gradual small-step semantics in a way that makes the gradual verification system sound and also complies with the stronger notion of preservation postulated in section \ref{ssec:gradual-soundness}.

\section{Language}
\label{sec:language}

%% more about our language
We now introduce a simplified Java-like statically verified language \svlidf that uses Chalice-like sub-syntax to express method contracts.

% more about simplicity
% - decidable satisfiability & implication of formulas (will later investigate how to extend... )
% - exactly one method arg, return type
% - not only decidable, but even polytime static semantics - entire area of research to extend boundaries there

% somewhere: implication (refer to general language)

    \subsection{Syntax}
    \label{sec:syntax}
    Figure \ref{fig:idf-syntax} shows the full syntax of \svlidf.
\begin{figure}[h]
    \newcommand{\tempStmtA}{\sSkip
                    ~|~ \sDeclare {$T$} {$x$}
                    ~|~ \sFieldAssign {$x$} {$f$} {$y$} 
                    ~|~ \sVarAssign {$x$} {$e$}
                    ~|~ \sAlloc {$x$} {$C$} 
                    ~|~ \sCall {$x$} {$y$} {$m$} {$z$}}
\newcommand{\tempStmtB}{~~~ ~|~ \sReturn {$x$}  
                            ~|~ \sAssert {$\phi$} 
                            ~|~ \sRelease {$\phi$} 
                            ~|~ \sHold {$\phi$} {$s$}
                            ~|~ \sSeq {$s_1$} {$s_2$}}
\newcommand{\tempFrm}{  \phiTrue 
                    ~|~ \phiEq {$e$} {$e$} 
                    ~|~ \phiNeq {$e$} {$e$}
                    ~|~ \phiAcc {$e$} {$f$}
                    ~|~ \phiCons {$\phi$} {$\phi$}}
\newcommand{\tempExpr}{ \ev{$v$}
                    ~|~ \ex{$x$}
                    ~|~ \edot{$e$}{$f$}}

\begin{align*}
	program  & \in \setProgram    &  & ::= \ttt{$\overline{cls}$~$s$}                              \\
	cls      & \in \setClass      &  & ::= \class {$C$} {$\overline{field}$} {$\overline{method}$} \\
	field    & \in \setField      &  & ::= \field {$T$} {$f$}                                      \\
	method   & \in \setMethod     &  & ::= \method {$T$} {$m$} {$T$} {$x$} {$contract$} {$s$}      \\
	contract & \in \setContract   &  & ::= \contract{$\phi$}{$\phi$}                              \\
	T        & \in \setType       &  & ::= \Tint ~|~ C                                             \\
	s        & \in \setStmt       &  & ::= \tempStmtA                                              \\
	         &                    &  & \tempStmtB                                                  \\
	\phi     & \in \setFormula    &  & ::= \tempFrm                                                \\
	e        & \in \setExpr       &  & ::= \tempExpr                                               \\
	x, y, z  & \in \setVar        &  & ::= \ethis ~|~ \eresult ~|~ identifier                      \\
	v        & \in \setVal        &  & ::= o ~|~ n ~|~ \enull                                      \\
	o        & \in \setLoc        &  & \text{(infinite set of memory locations)}                   \\
	n        & \in \mathbb{Z}     &  &  \\
	C        & \in \setClassName  &  & ::= identifier                                              \\
	f        & \in \setFieldName  &  & ::= identifier                                              \\
	m        & \in \setMethodName &  & ::= identifier
\end{align*}
    \caption{\svlidf: Syntax}
    \label{fig:idf-syntax}
\end{figure}
% EXPLAIN what stuff (like hold, release) means!

%% parser rassoc, skip
We pose $\phiFalse \defeq \phiNeq{\enull}{\enull}$.
We define $\ttt{;}$ to be right-associative and assume that parsing a sequence of statements (e.g. method body) operates analogously, obviating the need for parenthesis.
Furthermore we assume that the parser terminates every sequence with $\sSkip$.
\begin{exmp}~
    \begin{lstlisting}
    ...
    {
        $s_1$;
        $s_2$;
        $s_3$;
    }
    \end{lstlisting}
    is parsed as
    
    \includegraphics[trim={3cm 3cm 3cm 3cm}, clip, width=6cm]{graphics/rightAssocSkip}
\end{exmp}
These assumptions highly simplify reasoning about statements.

%% helper methods
We define the following helper methods:
\begin{figure}[h]
    \begin{description}
    \item[Extraction]
    To extract elements from a given program $p \in \setProgram$ we define the following functions:
    \begin{flalign*}
    	 & \fieldType : \setClassName \times \setFieldName \rightharpoonup \setType                & ~ \\
    	 & \fieldType(C, f) = \text{type of field $f$ in class $C$ in $p$}                         &  \\
    	 & ~                                                                                       &  \\
    	 & \predicate{fields$_p$} : \setClassName \rightharpoonup \PP^{\setField}                  &  \\
    	 & \fields{C} = \text{field declarations of class $C$ in $p$}                              &  \\
    	 & ~                                                                                       &  \\
    	 & \predicate{method$_p$} : \setClassName \times \setMethodName \rightharpoonup \setMethod &  \\
    	 & \mmethod{C, m} = \text{declaration of method $m$ in class $C$ in $p$}                   &  \\
    	 & ~                                                                                       &  \\
    	 & \predicate{mpre$_p$} : \setClassName \times \setMethodName \rightharpoonup \setFormula  &  \\
    	 & \mpre{C, m} = \text{precondition of method $m$ in class $C$ in $p$}                    &  \\
    	 & ~                                                                                       &  \\
    	 & \predicate{mpost$_p$} : \setClassName \times \setMethodName \rightharpoonup \setFormula &  \\
    	 & \mpost{C, m} = \text{postcondition of method $m$ in class $C$ in $p$}                   &
    \end{flalign*}
    
    \item[Free Variables]~\\
    The semantics of \svlidf will sometimes reason about the free variables of expressions or formulas.
    
    Let $\FV : (\setExpr \cup \setFormula) \rightarrow \PP^{\setVar}$ be defined as
    \begin{alignat*}{3}
    	  & \FV(v)                            &  & = \emptyset                    & ~ \\
    	  & \FV(x)                            &  & = \{ x \}                      &  \\
    	  & \FV(\edot{$e$}{$f$})              &  & = \FV(e)                       &  \\
    	~ &  \\
    	  & \FV(\phiTrue)                     &  & = \emptyset                    &  \\
    	  & \FV(\phiEq{$e_1$}{$e_2$})         &  & = \FV(e_1) \cup \FV(e_2)       &  \\
    	  & \FV(\phiNeq{$e_1$}{$e_2$})        &  & = \FV(e_1) \cup \FV(e_2)       &  \\
    	  & \FV(\phiAcc{$e$}{$f$})            &  & = \FV(e)                       &  \\
    	  & \FV(\phiCons{$\phi_1$}{$\phi_1$}) &  & = \FV(\phi_1) \cup \FV(\phi_2) &
    \end{alignat*}
    
    \item[Default Value of Type]~\\
    \svlidf assigns default values to declared variables.
    
    Let $\predicate{defaultValue} : \setType \rightarrow \setVal$ be defined as
    \begin{alignat*}{3}
    	 & \defaultValue{\Tint} &  & = 0      & ~ \\
    	 & \defaultValue{$C$}   &  & = \enull &
    \end{alignat*}
    
    \item[Required Access]~\\
    Expressions mentioning fields are heap dependent and thus require access.
    To enable treating expressions in a uniform fashion, we define a pseudo accessibility-predicate which is also defined for expressions that do not mention fields.
    
    Let $\predicate{acc} : \setExpr \rightarrow \setFormula$ be defined as
    \begin{alignat*}{3}
    	 & \accFor{v}               &  & = \phiTrue          & ~ \\
    	 & \accFor{x}               &  & = \phiTrue          & ~ \\
    	 & \accFor{\edot{$e$}{$f$}} &  & = \phiAcc{$e$}{$f$} &
    \end{alignat*}
    
    \item[Preventing Writes]
    Under rare circumstances, overwriting a certain variable is not allowed in \svlidf.
    To reliably check whether a variable is written to by a statement, we define the following predicate.
    
    Let $\writesTo \subseteq \setStmt \times \setVar$ be defined inductively as
    %% Inductive writesTo
\begin{mathpar}
\inferrule* [Right=wtVarAssign]
{
    ~
}
{
    \writesTo({x}, {\sVarAssign {${x}$} {${e}$}})
}
\end{mathpar}

\begin{mathpar}
\inferrule* [Right=wtAlloc]
{
    ~
}
{
    \writesTo({x}, {\sAlloc {${x}$} {${C}$}})
}
\end{mathpar}

\begin{mathpar}
\inferrule* [Right=wtCall]
{
    ~
}
{
    \writesTo({x}, {\sCall {${x}$} {${y}$} {${m}$} {${z}$}})
}
\end{mathpar}

\begin{mathpar}
\inferrule* [Right=wtReturn]
{
    ~
}
{
    \writesTo({\eresult}, {\sReturn {${x}$}})
}
\end{mathpar}

\begin{mathpar}
\inferrule* [Right=wtDeclare]
{
    ~
}
{
    \writesTo({x}, {\sDeclare {${T}$} {${x}$}})
}
\end{mathpar}

\begin{mathpar}
\inferrule* [Right=wtHold]
{
    {s} \in {ss} \\
    \writesTo({x}, {s})
}
{
    \writesTo({x}, {\sHold {${p}$} {${ss}$}})
}
\end{mathpar}


    \begin{mathpar}
        \inferrule* [right=wtVarAssign]
        {
            ~
        }
        {
            \writesTo({x},\, {\sVarAssign {${x}$} {${e}$}})
        }
        
        \inferrule* [right=wtAlloc]
        {
            ~
        }
        {
            \writesTo({x},\, {\sAlloc {${x}$} {${C}$}})
        }
        
        \inferrule* [right=wtCall]
        {
            ~
        }
        {
            \writesTo({x},\, {\sCall {${x}$} {${y}$} {${m}$} {${z}$}})
        }
        
        \inferrule* [right=wtReturn]
        {
            ~
        }
        {
            \writesTo({\eresult},\, {\sReturn {${x}$}})
        }
        
        \inferrule* [right=wtDeclare]
        {
            ~
        }
        {
            \writesTo({x}, {\sDeclare {${T}$} {${x}$}})
        }
        
        \inferrule* [right=wtHold]
        {
            \writesTo({x},\, {s})
        }
        {
            \writesTo({x},\, {\sHold {${p}$} {${s}$}})
        }
        
        \inferrule* [right=wtSeq1]
        {
            \writesTo({x},\, {s_1})
        }
        {
            \writesTo({x},\, \sSeq{$s_1$}{$s_2$})
        }
        
        \inferrule* [right=wtSeq2]
        {
            \writesTo({x},\, {s_2})
        }
        {
            \writesTo({x},\, \sSeq{$s_1$}{$s_2$})
        }
    \end{mathpar}
\end{description}
    \caption{\svlidf: Helper Methods}
    \label{fig:idf-helpers}
\end{figure}



% Programs consist of classes and a main method, represented directly as the list of its instructions.
% TODO: introduce all the extraction predicates/functions!

    % if then else???
    
    \subsection{Program State}
    \label{ssec:program-state}
    The set of program states is defined as $\setProgramState = \setHeap \times \setStack$ with
\begin{align*}
	S & \in \setStack      &  & ::= E \cdot S ~|~ \nil                               \\
	E & \in \setStackEntry &  & =~~ \setVarEnv \times \setDFootprint \times \setStmt
\end{align*}

\begin{comment}
REQUIRED?
\begin{definition}[Topmost Stack Entry]
    Let $\topmost : \setStack \rightharpoonup \setStackEntry$ be defined as
    \begin{align*}
    &\topmost(E \cdot S) = E\\
    &\topmost(\nil) \quad\textit{ undefined}
    \end{align*}
\end{definition}
\end{comment}

%TODO: rethink where footprints are introduced... would make sense to know what it is prior to this figure...

\begin{comment}
Program states with scheduled statement $s$ are defined as
\begin{displaymath}
\setProgramState_s ~\defeq~ \setHeap ~\times~ \{~~ (\rho, A_d, s) \cdot S ~~|~~ \rho \in \setVarEnv,~ A_d \in \setDFootprint,~ S \in \setStack ~~\}
\end{displaymath}
\end{comment}
        
    \subsection{Formula Semantics}
    \label{ssec:formula-semantics}
    
\begin{figure}
    \boxed{\dynamicFP {H} {\rho} {\phi} = A_d}
    
\begin{align*}
	 & \dynamicFP {H} {\rho} {\phiTrue}                     &  & = \emptyset                                                          \\
	 & \dynamicFP {H} {\rho} {\phiEq{$e_1$}{$e_2$}}         &  & = \emptyset                                                          \\
	 & \dynamicFP {H} {\rho} {\phiNeq{$e_1$}{$e_2$}}        &  & = \emptyset                                                          \\
	 & \dynamicFP {H} {\rho} {\phiAcc{$x$}{$f$}}            &  & = \{(o,f)\} \text{ where } \evale x o                                \\
	 & \dynamicFP {H} {\rho} {\phiCons{$\phi_1$}{$\phi_2$}} &  & = \dynamicFP {H} {\rho} {\phi_1} \cup \dynamicFP {H} {\rho} {\phi_2}
\end{align*}

What about undefinedness of acc case? Guess: propagates to undefinedness of small-step rule => covered by soundness
    \caption{\svlidf: Dynamic Footprint}
\end{figure}

\begin{figure}
    \boxed{\evale {e} {v}}
    
\begin{mathpar}
    \inferrule* [right=EEVar]
    {~}
    {\evale {x} {\rho(x)}}
    
    \inferrule* [right=EEValue]
    {~}
    {\evale {v} {v}}
    
    \inferrule* [Right=EEAcc]
    {\evale {e} {o}\\
     H(o) = \langle C, l \rangle}
    {\evale {e.f} {l(f)}}
\end{mathpar}
    \caption{\svlidf: Evaluating Expressions}
\end{figure}

\begin{figure}
    \boxed{\evalphi \phi}
    % Inductive Semantics.evalphi'
\begin{mathpar}
\inferrule* [Right=EATrue]
{
    ~
}
{
    \evalphix {H} {\rho} {A} {\phiTrue}
}
\end{mathpar}

\begin{mathpar}
\inferrule* [Right=EAEqual]
{
    \evalex {H} {\rho} {e_1} {v_1} \\
    \evalex {H} {\rho} {e_2} {v_2} \\
    {v_1} = {v_2}
}
{
    \evalphix {H} {\rho} {A} {\phiEq {${e_1}$} {${e_2}$}}
}
\end{mathpar}

\begin{mathpar}
\inferrule* [Right=EANEqual]
{
    \evalex {H} {\rho} {e_1} {v_1} \\
    \evalex {H} {\rho} {e_2} {v_2} \\
    {v_1} \neq {v_2}
}
{
    \evalphix {H} {\rho} {A} {\phiNeq {${e_1}$} {${e_2}$}}
}
\end{mathpar}

\begin{mathpar}
\inferrule* [Right=EAAcc]
{
    \evalex {H} {\rho} {e} {{o}} \\
    \evalex {H} {\rho} {\edot{${e}$}{${f}$}} {v} \\
    {({o}, {f})} \in {A}
}
{
    \evalphix {H} {\rho} {A} {\phiAcc {${e}$} {${f}$}}
}
\end{mathpar}



\begin{mathpar}
    \inferrule* [Right=EASepOp]
    {
        A_1 = A \backslash A_2 \\
        \evalphix H \rho {A_1} {\phi_1} \\
        \evalphix H \rho {A_2} {\phi_2}
    }
    {\evalphi {\phiCons {$\phi_1$} {$\phi_2$}}}
\end{mathpar}
    \caption{\svlidf: Evaluating Formulas}
\end{figure}

\begin{figure}
    \boxed{\evalphiGen {\pi} {\phi}}
    \begin{mathpar}
        \inferrule* [Right=EvalFrm]
        {
            \evalphi {\phi}
        }
        {
            \evalgphiGen{(H, (\rho, A, s) \cdot S)}{\phi}
        }
    \end{mathpar}
    \caption{\svlidf: Evaluating Formulas}
\end{figure}
    
        \subsubsection{Framing}
        \label{sssec:framing}
        %% intro
The integral advantage of IDF is the ability to use heap-dependent predicates in formulas.
Accessibility-predicates are explicitly tracked as part of formulas and represent exclusive access to a certain field.
As illustrated in section \ref{ssec:implicit-dynamic-frames} sound reasoning is only possible if a formula contains access $\ttt{acc}$ for all the fields it mentions.
This property is called self-framing and will be formalized in this section.
%An IDF assertion is self-framing if:
%For any state in which the assertion is true,
%it remains true if we replace the heap with any that agrees on the locations to which it requires permissions

%% footprint
To decide whether a formula contains sufficient access to be self-framing, we need to extract from it the set of fields that it has access to.
This set is called footprint and has type $\setSFootprint \defeq \PP^{\setExpr \times \setFieldName}$.
A function $\staticFP{\cdot} : \setFormula \rightarrow \setSFootprint$ obtaining a formulas footprint is defined in figure \ref{fig:sfp}.
\begin{figure}[h]
    \boxed{\staticFP {\phi} = A_s}
    \begin{align*}
	 & \staticFP {\phiTrue}                       &  & = \emptyset                                  \\
	 & \staticFP {\phiEq {$e_1$} {$e_2$}}         &  & = \emptyset                                  \\
	 & \staticFP {\phiNeq {$e_1$} {$e_2$}}        &  & = \emptyset                                  \\
	 & \staticFP {\phiAcc {$e$} {$f$}}            &  & = \{\langle e,f \rangle\}                                  \\
	 & \staticFP {\phiCons {$\phi_1$} {$\phi_2$}} &  & = \staticFP {\phi_1} \cup \staticFP {\phi_2}
\end{align*}


    \caption{\svlidf: Static Footprint}
    \label{fig:sfp}
\end{figure}

\begin{example}{Footprints of Formulas}
\begin{flalign*}
  	 & \staticFP{\phiCons{\phiEq{x}{3}}{\phiNeq{p.age}{24}}} = \emptyset                                                                                                                       \\
  	 & \staticFP{\phiCons{\phiCons{\phiEq{x}{3}}{\phiAcc{p}{age}}}{\phiNeq{p.age}{24}}} = \{ \langle \ttt{p}, \ttt{age} \rangle \}                                                             \\
  	 & \staticFP{\phiCons{\phiCons{\phiAcc{p}{age}}{\phiAcc{p}{name}}}{\phiAcc{x}{f}}} = \{ \langle \ttt{p}, \ttt{age} \rangle, \langle \ttt{p}, \ttt{name} \rangle, \langle \ttt{x}, \ttt{f} \rangle \}
\end{flalign*}
\end{example}
% examples? more explanation?

With the notion of footprints, we can determine whether an expression is “framed” by a given footprint, i.e. whether the footprint contains all the fields the expression contains.
We formalize this check using a predicate $\sfrme \subseteq \setSFootprint \times \setExpr$, defined in figure \ref{fig:svl-frme}.
\begin{figure}
    \boxed{A_s \sfrme e}
    \input{data/svl-sem-stat-frme}
    \caption{\svlidf: Framing Expressions}
    \label{fig:svl-frme}
\end{figure}

At last, we can define a predicate $\sfrmphi \subseteq \setSFootprint \times \setFormula$ that checks whether given footprint is sufficient to frame an entire formula, i.e. all the expressions the formula contains.
Figure \ref{fig:svl-frmphi} formalizes the predicate.
\begin{figure}
    \boxed{A_s \sfrmphi \phi}
    %% Inductive Semantics.sfrmphi'
\begin{mathpar}
\inferrule* [Right=WFTrue]
{
    ~
}
{
    {A} \sfrmphi {\phiTrue}
}
\end{mathpar}

\begin{mathpar}
\inferrule* [Right=WFEqual]
{
    {A} \sfrme {e_1} \\
    {A} \sfrme {e_2}
}
{
    {A} \sfrmphi {\phiEq {${e_1}$} {${e_2}$}}
}
\end{mathpar}

\begin{mathpar}
\inferrule* [Right=WFNEqual]
{
    {A} \sfrme {e_1} \\
    {A} \sfrme {e_2}
}
{
    {A} \sfrmphi {\phiNeq {${e_1}$} {${e_2}$}}
}
\end{mathpar}

\begin{mathpar}
\inferrule* [Right=WFAcc]
{
    {A} \sfrme {e}
}
{
    {A} \sfrmphi {\phiAcc {${e}$} {${f}$}}
}
\end{mathpar}


% Inductive Semantics.sfrmphi'

\begin{mathpar}
    \inferrule* [right=SfrmTrue]
    {
        ~
    }
    {
        {A_s} \sfrmphi {\phiTrue}
    }
    
    \inferrule* [right=SfrmEqual]
    {
        {A_s} \sfrme {e_1} \\
        {A_s} \sfrme {e_2}
    }
    {
        {A_s} \sfrmphi {\phiEq {${e_1}$} {${e_2}$}}
    }
    
    \inferrule* [right=SfrmNEqual]
    {
        {A_s} \sfrme {e_1} \\
        {A_s} \sfrme {e_2}
    }
    {
        {A_s} \sfrmphi {\phiNeq {${e_1}$} {${e_2}$}}
    }
    
    \inferrule* [right=SfrmAcc]
    {
        {A_s} \sfrme {e}
    }
    {
        {A_s} \sfrmphi {\phiAcc {${e}$} {${f}$}}
    }
    
    \inferrule* [right=SfrmSepOp]
    {
        A_s \sfrmphi \phi_1 \\ 
        A_s \cup \staticFP {\phi_1} \sfrmphi \phi_2
    }
    {A_s \sfrmphi \phiCons{$\phi_1$}{$\phi_2$}}
\end{mathpar}

    \caption{\svlidf: Framing Formulas}
    \label{fig:svl-frmphi}
\end{figure}
Note how \tset{SfrmSepOp} augments the footprint the right sub-formula is checked against using the footprint of the left sub-formula.
This way, access predicates within a formula are able to frame expressions in the same formula, however only to the right.

\begin{example}{Framing Formulas}
    \begin{flalign*}
    	\emptyset                                & \sfrmphi \phiNeq{p.age}{24}                            && \text{does not hold} \\
    	\{ \langle \ttt{p}, \ttt{age} \rangle \} & \sfrmphi \phiNeq{p.age}{24}                            && \text{holds}         \\
    	\emptyset                                & \sfrmphi \phiCons{\phiAcc{p}{age}}{\phiNeq{p.age}{24}} && \text{holds}         \\
    	\emptyset                                & \sfrmphi \phiCons{\phiNeq{p.age}{24}}{\phiAcc{p}{age}} && \text{does not hold}
    \end{flalign*}
\end{example}

We omit the emptyset... 

\begin{definition}[Self-Framing Formula]
    A formula $\phi$ is \textbf{self-framing} iff
    \begin{displaymath}
    \sfrmphi \phi
    \end{displaymath}
    Let $\setFormulaB \subseteq \setFormulaA$ be the set of \textbf{self-framing and satisfiable} formulas.
\end{definition}


\svl will thus only consider method contracts using self-framing and satisfiable formulas well-formed (see section \ref{sec:well-formedness}).


%IS conservative approximation of formulas that are dynamically framed (not possible precisely anyway!)
        
        % RULES arising from this semantics!!! e.g. implication:
        % - a * b => a
        % - a !=> a * a
    
    \subsection{Static Semantics}
    \label{sec:static-semantics}
    
\begin{figure}[h]
    %% Inductive Semantics.hasStaticType
\begin{mathpar}
\inferrule* [Right=STValNum]
{
    ~
}
{
    \sType {\Gamma} {\ev{${{n}}$}} {\Tint}
}
\end{mathpar}

\begin{mathpar}
\inferrule* [Right=STValNull]
{
    ~
}
{
    \sType {\Gamma} {\ev{${\enull}$}} {{C}}
}
\end{mathpar}

\begin{mathpar}
\inferrule* [Right=STVar]
{
    {\Gamma(x)} = {{T}}
}
{
    \sType {\Gamma} {\ex{${x}$}} {T}
}
\end{mathpar}

\begin{mathpar}
\inferrule* [Right=STField]
{
    \sType {\Gamma} {e} {{C}} \\
    {\fieldType({C}, {f})} = {{T}}
}
{
    \sType {\Gamma} {\edot{${e}$}{${f}$}} {T}
}
\end{mathpar}


% Inductive Semantics.hasStaticType
\begin{mathpar}
\inferrule* [right=STValNum]
{
    ~
}
{
    \sType {\Gamma} {\ev{${{n}}$}} {\Tint}
}

\inferrule* [right=STValNull]
{
    ~
}
{
    \sType {\Gamma} {\ev{${\enull}$}} {{C}}
}

\inferrule* [right=STVar]
{
    {\Gamma(x)} = {{T}}
}
{
    \sType {\Gamma} {\ex{${x}$}} {T}
}

\inferrule* [right=STField]
{
    \sType {\Gamma} {e} {{C}} \\
    {\fieldType({C}, {f})} = {{T}}
}
{
    \sType {\Gamma} {\edot{${e}$}{${f}$}} {T}
}
\end{mathpar}


    \caption{Static Typing Rules}
\end{figure}

%% FRAMING
%% Inductive Semantics.sfrme
\begin{mathpar}
\inferrule* [right=FrmVar]
{
    ~
}
{
    {A_s} \sfrme {\ex{${x}$}}
}

\inferrule* [right=FrmVal]
{
    ~
}
{
    {A_s} \sfrme {\ev{${v}$}}
}

\inferrule* [right=FrmField]
{
    {({e}, {f})} \in {A} \\
    {A_s} \sfrme {e}
}
{
    {A_s} \sfrme {\edot{${e}$}{${f}$}}
}
\end{mathpar}


        
        \subsubsection{Typing}
        \label{sssec:typing}
        
\begin{figure}[h]
    \boxed{\sType{\Gamma}{e}{T}}
    %% Inductive Semantics.hasStaticType
\begin{mathpar}
\inferrule* [Right=STValNum]
{
    ~
}
{
    \sType {\Gamma} {\ev{${{n}}$}} {\Tint}
}
\end{mathpar}

\begin{mathpar}
\inferrule* [Right=STValNull]
{
    ~
}
{
    \sType {\Gamma} {\ev{${\enull}$}} {{C}}
}
\end{mathpar}

\begin{mathpar}
\inferrule* [Right=STVar]
{
    {\Gamma(x)} = {{T}}
}
{
    \sType {\Gamma} {\ex{${x}$}} {T}
}
\end{mathpar}

\begin{mathpar}
\inferrule* [Right=STField]
{
    \sType {\Gamma} {e} {{C}} \\
    {\fieldType({C}, {f})} = {{T}}
}
{
    \sType {\Gamma} {\edot{${e}$}{${f}$}} {T}
}
\end{mathpar}


% Inductive Semantics.hasStaticType
\begin{mathpar}
\inferrule* [right=STValNum]
{
    ~
}
{
    \sType {\Gamma} {\ev{${{n}}$}} {\Tint}
}

\inferrule* [right=STValNull]
{
    ~
}
{
    \sType {\Gamma} {\ev{${\enull}$}} {{C}}
}

\inferrule* [right=STVar]
{
    {\Gamma(x)} = {{T}}
}
{
    \sType {\Gamma} {\ex{${x}$}} {T}
}

\inferrule* [right=STField]
{
    \sType {\Gamma} {e} {{C}} \\
    {\fieldType({C}, {f})} = {{T}}
}
{
    \sType {\Gamma} {\edot{${e}$}{${f}$}} {T}
}
\end{mathpar}


    \caption{\svlidf: Static Typing of Expressions}
\end{figure}
    
        \subsubsection{Verification}
        \label{sssec:verification}
        
\begin{figure}[h!]
    \boxed{\thoare {\Gamma} {\phi_{pre}} {s} {\phi_{post}}}
    %% Inductive Semantics.hoare
\begin{mathpar}
\inferrule* [Right=HAlloc]
{
    {\phi} \implies {\phi'} \\
    {\emptyset} \sfrmphi {\phi'} \\
    {x} \not \in {FV(\phi')} \\
    {\Gamma} \vdash {\ex{${x}$}} : {{C}} \\
    {\fields({C})} = {{\overline{f}}}
}
{
    {\Gamma} \hoare {\phi} {{\sAlloc {${x}$} {${C}$}}} {\phiCons{${\phi'}$}{${\phiCons{${\phiNeq {${\ex{${x}$}}$} {${\ev{${\vnull}$}}$}}$}{${\overline{\acc({x}, f_i)}}$}}$}}
}
\end{mathpar}

\begin{mathpar}
\inferrule* [Right=HFieldAssign]
{
    {\phi} \implies {\phiCons{${\phiAcc {${\ex{${x}$}}$} {${f}$}}$}{${\phi'}$}} \\
    {\emptyset} \sfrmphi {\phi'} \\
    {\Gamma} \vdash {\ex{${x}$}} : {{C}} \\
    {\Gamma} \vdash {\ex{${y}$}} : {T} \\
    \vdash {C}.{f} : {T}
}
{
    {\Gamma} \hoare {\phi} {{\sFieldAssign {${x}$} {${f}$} {${y}$}}} {\phiCons{${\phi'}$}{${\phiCons{${\phiAcc {${\ex{${x}$}}$} {${f}$}}$}{${\phiCons{${\phiNeq {${\ex{${x}$}}$} {${\ev{${\vnull}$}}$}}$}{${\ensuremath{{\phiEq {${\edot{${\ex{${x}$}}$}{${f}$}}$} {${\ex{${y}$}}$}}}}$}}$}}$}}
}
\end{mathpar}

\begin{mathpar}
\inferrule* [Right=HVarAssign]
{
    {\phi} \implies {\phi'} \\
    {\emptyset} \sfrmphi {\phi'} \\
    {x} \not \in {FV(\phi')} \\
    {x} \not \in {FV({e})} \\
    {\Gamma} \vdash {\ex{${x}$}} : {T} \\
    {\Gamma} \vdash {e} : {T} \\
    {\accFor {{e}}} \subseteq {\phi'}
}
{
    {\Gamma} \hoare {\phi} {{\sVarAssign {${x}$} {${e}$}}} {\phiCons{${\phi'}$}{${\ensuremath{{\phiEq {${\ex{${x}$}}$} {${e}$}}}}$}}
}
\end{mathpar}

\begin{mathpar}
\inferrule* [Right=HReturn]
{
    {\phi} \implies {\phi'} \\
    {\emptyset} \sfrmphi {\phi'} \\
    {\xresult} \not \in {FV(\phi')} \\
    {\Gamma} \vdash {\ex{${x}$}} : {T} \\
    {\Gamma} \vdash {\ex{${\xresult}$}} : {T}
}
{
    {\Gamma} \hoare {\phi} {{\sReturn {${x}$}}} {\phiCons{${\phi'}$}{${\ensuremath{{\phiEq {${\ex{${\xresult}$}}$} {${\ex{${x}$}}$}}}}$}}
}
\end{mathpar}

\begin{mathpar}
\inferrule* [Right=HCall]
{
    {\Gamma} \vdash {\ex{${y}$}} : {{C}} \\
    {\mmethod({C}, {m})} = {{\method {${T_r}$} {${m}$} {${T_p}$} {${z}$} {${\requires {\phi_{pre}};~\ensures {\phi_{post}};}$} {${\_}$}}} \\
    {\Gamma} \vdash {\ex{${x}$}} : {T_r} \\
    {\Gamma} \vdash {\ex{${z'}$}} : {T_p} \\
    {\phi} \implies {\phiCons{${\phiNeq {${\ex{${y}$}}$} {${\ev{${\vnull}$}}$}}$}{${\phiCons{${\phi_p}$}{${\phi'}$}}$}} \\
    {\emptyset} \sfrmphi {\phi'} \\
    {x} \not \in {FV(\phi')} \\
    x \neq y \wedge x \neq z' \\
    {\phi_p} = {{\phi_{pre}}[{y}, {z'} / {\xthis}, {{z}}]} \\
    {\phi_q} = {{\phi_{post}}[{y}, {z'}, {x} / {\xthis}, {{z}}, {\xresult}]}
}
{
    {\Gamma} \hoare {\phi} {{\sCall {${x}$} {${y}$} {${m}$} {${z'}$}}} {\phiCons{${\phi'}$}{${\phi_q}$}}
}
\end{mathpar}

\begin{mathpar}
\inferrule* [Right=HAssert]
{
    {\phi} \implies {\phi'}
}
{
    {\Gamma} \hoare {\phi} {{\sAssert {${\phi'}$}}} {\phi}
}
\end{mathpar}

\begin{mathpar}
\inferrule* [Right=HRelease]
{
    {\phi} \implies {\phiCons{${\phi_r}$}{${\phi'}$}} \\
    {\emptyset} \sfrmphi {\phi'}
}
{
    {\Gamma} \hoare {\phi} {{\sRelease {${\phi_r}$}}} {\phi'}
}
\end{mathpar}

\begin{mathpar}
\inferrule* [Right=HDeclare]
{
    {x} \not\in \dom({\Gamma}) \\
    {{\Gamma}, {x} : {T}} \hoare {\phiCons{${\phiEq {${\ex{${x}$}}$} {${\ev{${\texttt{defaultValue}({T})}$}}$}}$}{${\phi}$}} {\overline{s}} {\phi'}
}
{
    {\Gamma} \hoare {\phi} {{\sDeclare {${T}$} {${x}$}} {\overline{s}}} {\phi'}
}
\end{mathpar}

\begin{mathpar}
\inferrule* [Right=HHold]
{
    {\phi_f} \implies {\phiCons{${\phi_r}$}{${\phi'}$}} \\
    {\phi'} \implies {\phi} \\
    {\Gamma} \hoare {\phi_r} {\overline{s}} {\phi_r'}
}
{
    {\Gamma} \hoare {\phi_f} {{\sHold {${\phi}$} {${\overline{s}}$}}} {\phiCons{${\phi_r'}$}{${\phi'}$}}
}
\end{mathpar}

\begin{mathpar}
\inferrule* [Right=HSeq]
{
    {\Gamma} \hoare {\phi_p} {\overline{s_1}} {\phi_q} \\
    {\Gamma} \hoare {\phi_q} {\overline{s_2}} {\phi_r}
}
{
    {\Gamma} \hoare {\phi_p} {{\overline{s_1}}\ttt{;} {\overline{s_2}}} {\phi_r}
}
\end{mathpar}


% Inductive Semantics.hoare
\begin{mathpar}
\inferrule* [right=HSkip]
{
    \phiImplies {\phi} {\phi'}
}
{
    \thoare {\Gamma} {\phi} {{\sSkip}} {\phi'}
}

\inferrule* [right=HAlloc]
{
    \phiImplies {\phi} {\phi'} \\
    \sfrmphi {\phi'} \\
    {x} \not \in {\FV(\phi')} \\
    \sType {\Gamma} {\ex{${x}$}} {{C}} \\
    {\fields{C}} = {{\overline{\field{$T$}{$f$}}}}
}
{
    \thoare {\Gamma} {\phi} {{\sAlloc {${x}$} {${C}$}}} {\phiCons{${\phi'}$}{${\phiCons{${\phiNeq {${\ex{${x}$}}$} {${\ev{${\enull}$}}$}}$}{${\overline{\phiCons{\phiAcc {$x$} {$f_i$}}{\phiEq {\edot {$x$} {$f_i$}} {\defaultValue {$T_i$}}}}}$}}$}}
}

\inferrule* [right=HFieldAssign]
{
    \phiImplies {\phi} {\phiCons{${\phiAcc {${\ex{${x}$}}$} {${f}$}}$}{${\phi'}$}} \\
    \sfrmphi {\phi'} \\
    \sType {\Gamma} {\ex{${x}$}} {{C}} \\
    \sType {\Gamma} {\ex{${y}$}} {T} \\
    \vdash {C}.{f} : {T}
}
{
    \thoare {\Gamma} {\phi} {{\sFieldAssign {${x}$} {${f}$} {${y}$}}} {\phiCons{${\phi'}$}{${\phiCons{${\phiAcc {${\ex{${x}$}}$} {${f}$}}$}{${\phiCons{${\phiNeq {${\ex{${x}$}}$} {${\ev{${\enull}$}}$}}$}{${\ensuremath{{\phiEq {${\edot{${\ex{${x}$}}$}{${f}$}}$} {${\ex{${y}$}}$}}}}$}}$}}$}}
}

\inferrule* [right=HVarAssign]
{
    \phiImplies {\phi} {\phi'} \\
    \phiImplies {\phi} {\accFor {{e}}} \\
    \sfrmphi {\phi'} \\
    {x} \not \in {\FV(\phi')} \\
    {x} \not \in {\FV({e})} \\
    \sType {\Gamma} {\ex{${x}$}} {T} \\
    \sType {\Gamma} {e} {T}
}
{
    \thoare {\Gamma} {\phi} {{\sVarAssign {${x}$} {${e}$}}} {\phiCons{${\phi'}$}{${\ensuremath{{\phiEq {${\ex{${x}$}}$} {${e}$}}}}$}}
}

\inferrule* [right=HReturn]
{
    \phiImplies {\phi} {\phi'} \\
    \sfrmphi {\phi'} \\
    {\eresult} \not \in {\FV(\phi')} \\
    \sType {\Gamma} {\ex{${x}$}} {T} \\
    \sType {\Gamma} {\ex{${\eresult}$}} {T}
}
{
    \thoare {\Gamma} {\phi} {{\sReturn {${x}$}}} {\phiCons{${\phi'}$}{${\ensuremath{{\phiEq {${\ex{${\eresult}$}}$} {${\ex{${x}$}}$}}}}$}}
}

\inferrule* [right=HCall]
{
    \sType {\Gamma} {\ex{${y}$}} {{C}} \\
    {\mmethod{{C}, {m}}} = {{\method {${T_r}$} {${m}$} {${T_p}$} {${z}$} {${\contract {${\phi_{pre}}$} {${\phi_{post}}$}}$} {${\_}$}}} \\
    \sType {\Gamma} {\ex{${x}$}} {T_r} \\
    \sType {\Gamma} {\ex{${z'}$}} {T_p} \\
    \phiImplies {\phi} {\phiCons{${\phiNeq {${\ex{${y}$}}$} {${\ev{${\enull}$}}$}}$}{${\phiCons{${\phi_p}$}{${\phi'}$}}$}} \\
    \sfrmphi {\phi'} \\
    {x} \not \in {\FV(\phi')} \\
    x \neq y \wedge x \neq z' \\
    {\phi_p} = {{\phi_{pre}}[{y}, {z'} / {\ethis}, {{z}}]} \\
    {\phi_q} = {{\phi_{post}}[{y}, {z'}, {x} / {\ethis}, {{z}}, {\eresult}]}
}
{
    \thoare {\Gamma} {\phi} {{\sCall {${x}$} {${y}$} {${m}$} {${z'}$}}} {\phiCons{${\phi'}$}{${\phi_q}$}}
}

\inferrule* [right=HAssert]
{
    \phiImplies {\phi} {\phi_a}
}
{
    \thoare {\Gamma} {\phi} {{\sAssert {${\phi_a}$}}} {\phi}
}

\inferrule* [right=HRelease]
{
    \phiImplies {\phi} {\phiCons{${\phi_r}$}{${\phi'}$}} \\
    \sfrmphi {\phi'}
}
{
    \thoare {\Gamma} {\phi} {{\sRelease {${\phi_r}$}}} {\phi'}
}

\inferrule* [right=HDeclare]
{
    {x} \not\in \dom{\Gamma} \\
    \thoare {{\Gamma}, {x} : {T}} {\phiCons{${\phiEq {${\ex{${x}$}}$} {${\ev{${\defaultValue{${T}$}}$}}$}}$}{${\phi}$}} {s} {\phi'}
}
{
    \thoare {\Gamma} {\phi} {\sSeq {\sDeclare {${T}$} {${x}$}} {s}} {\phi'}
}

\inferrule* [right=HHold]
{
    \sfrmphi {\phi} \\
    \phiImplies {\phi_f} {\phiCons{${\phi_r}$}{${\phi'}$}} \\
    \phiImplies {\phi'} {\phi} \\
    {\FV(\phi')} = {\FV(\phi)} \\
    \mods(s) \cap \FV(\phi) = \emptyset \\
    \thoare {\Gamma} {\phi_r} {s} {\phi_r'}
}
{
    \thoare {\Gamma} {\phi_f} {{\sHold {${\phi}$} {${s}$}}} {\phiCons{${\phi_r'}$}{${\phi'}$}}
}

\inferrule* [right=HSeq]
{
    \thoare {\Gamma} {\phi_p} {s_1} {\phi_q} \\
    \thoare {\Gamma} {\phi_q} {s_2} {\phi_r}
}
{
    \thoare {\Gamma} {\phi_p} {\sSeq{$s_1$}{$s_2$}} {\phi_r}
}
\end{mathpar}


    \caption{\svl: Hoare Logic} 
\end{figure}

    
    \subsection{Well-Formedness}
    \label{sec:well-formedness}
    With static semantics in place, we can define what makes programs well-formed.
Well-formedness is required to ...
The following predicates 

A program is well-formed if both its classes and main method are.
For the main method to be well-formed, it must satisfy our Hoare predicate, given no assumptions.

\begin{mathpar}
\inferrule* [Right=OkProgram]
{
\overline{cls_i \OK} \\
\thoare {~} {\phiTrue} {\overline{s}} {\phiTrue}
}
{(\overline{cls_i}~\overline{s}) \OK}
\end{mathpar}

% TODO: start with deterministic semantics!?

\begin{mathpar}
\inferrule* [Right=OkClass]
{
\text{unique $field$-names} \\
\text{unique $method$-names} \\
\overline{method_i \OKinC}
}
{(\class {$C$} {$\overline{field_i}$} {$\overline{method_i}$}) \OK}
\end{mathpar}

\begin{mathpar}
\inferrule* [Right=OkMethod]
{
FV(\phi_1) \subseteq \{ x, \ethis \} \\
FV(\phi_2) \subseteq \{ x, \ethis, \eresult \} \\
\thoare {x : T_x, \ethis : C, \eresult : T_m} {\phi_1} {\overline{s}} {\phi_2} \\
\phi_1, \phi_2 \in \setFormulaB \\
\overline{\neg \writesTo(s_i, x)}
}
{(\method {$T_m$} {$m$} {$T_x$} {$x$} {\contract {$\phi_1$} {$\phi_2$}} {$\overline{s}$}) \OKinC}
\end{mathpar}
    
    \subsection{Dynamic Semantics}
    \label{ssec:dynamic-semantics}
    % MENTION that most of that stuff is completely redundant if soundness holds - and only used to prove just that!
The small-step semantics $\sstep{\cdot}{\cdot} : \setProgramState \rightharpoonup \setProgramState$ of \svlidf are defined inductively in figure \ref{fig:svl-sem-dyn-sstep}.
Note that right-associativity of \ttt{;} and termination of sequences with $\sSkip$ (see section \ref{sec:syntax}) obviate the need for dedicated sequence rules.

Using inductive rules to define a partial function, we have to make sure that at most one result is deducible for every input.
\begin{lemma}[$\sstep{\cdot}{\cdot}$ Well-Defined]
    \label{lemma:ss-wd}~\\
    The small-step semantics of \svlidf is well-defined.
\end{lemma}

\begin{figure}
    \boxed{\sstep{\pi}{\pi}}
    %% Inductive Semantics.dynSem
\begin{mathpar}
\inferrule* [Right=ESSkip]
{
    ~
}
{
    \sstep {({H}, {{({{\rho}, {A}}, {\sSeq{${\sSkip}$}{${s}$}})} \cdot {S}})} {({H}, {{({{\rho}, {A}}, {s})} \cdot {S}})}
}
\end{mathpar}

\begin{mathpar}
\inferrule* [Right=ESFieldAssign]
{
    \evalex {H} {\rho} {\ex{${x}$}} {{o}} \\
    \evalex {H} {\rho} {\ex{${y}$}} {v_y} \\
    {({o}, {f})} \in {A} \\
    {H'} = {{H}[{o} \mapsto [{f} \mapsto {v_y}]]}
}
{
    \sstep {({H}, {{({{\rho}, {A}}, {\sSeq{${\sFieldAssign {${x}$} {${f}$} {${y}$}}$}{${s}$}})} \cdot {S}})} {({H'}, {{({{\rho}, {A}}, {s})} \cdot {S}})}
}
\end{mathpar}

\begin{mathpar}
\inferrule* [Right=ESVarAssign]
{
    \evalex {H} {\rho} {e} {v} \\
    {\rho'} = {{\rho}[{x} \mapsto {v}]}
}
{
    \sstep {({H}, {{({{\rho}, {A}}, {\sSeq{${\sVarAssign {${x}$} {${e}$}}$}{${s}$}})} \cdot {S}})} {({H}, {{({{\rho'}, {A}}, {s})} \cdot {S}})}
}
\end{mathpar}

\begin{mathpar}
\inferrule* [Right=ESAlloc]
{
    {o} \not\in \dom{{H}} \\
    {\fields{{C}}} = {{\overline{\field{$T$}{$f$}}}} \\
    {\rho'} = {{\rho}[{x} \mapsto {{o}}]} \\
    {A'} = {{A} \cup {\overline{({o}, f_i)}}} \\
    {H'} = {{H}[{o} \mapsto [\overline{f_i \mapsto \defaultValue{$T_i$}}]]}
}
{
    \sstep {({H}, {{({{\rho}, {A}}, {\sSeq{${\sAlloc {${x}$} {${C}$}}$}{${s}$}})} \cdot {S}})} {({H'}, {{({{\rho'}, {A'}}, {s})} \cdot {S}})}
}
\end{mathpar}

\begin{mathpar}
\inferrule* [Right=ESReturn]
{
    \evalex {H} {\rho} {\ex{${x}$}} {v_x} \\
    {\rho'} = {{\rho}[{\eresult} \mapsto {v_x}]}
}
{
    \sstep {({H}, {{({{\rho}, {A}}, {\sSeq{${\sReturn {${x}$}}$}{${s}$}})} \cdot {S}})} {({H}, {{({{\rho'}, {A}}, {s})} \cdot {S}})}
}
\end{mathpar}

\begin{mathpar}
\inferrule* [Right=ESCall]
{
    \evalex {H} {\rho} {\ex{${y}$}} {{o}} \\
    \evalex {H} {\rho} {\ex{${z}$}} {v} \\
    {H(o)} = {{({C}, {\usc})}} \\
    {\mmethod{{C}, {m}}} = {{\method {${T_r}$} {${m}$} {${T}$} {${w}$} {${\contract {${\phi}$} {${\usc}$}}$} {${r}$}}} \\
    {\rho'} = {[{\eresult} \mapsto {\defaultValue{${T_r}$}}, {\ethis} \mapsto {{o}}, {{w}} \mapsto {v}]} \\
    \evalphix {H} {\rho'} {A} {\phi} \\
    {A'} = {\dynamicFP {H} {\rho'} {\phi}}
}
{
    \sstep {({H}, {{({{\rho}, {A}}, {\sSeq{${\sCall {${x}$} {${y}$} {${m}$} {${z}$}}$}{${s}$}})} \cdot {S}})} {({H}, {{({{\rho'}, {A'}}, {r})} \cdot {{({{\rho}, {{A} \backslash {A'}}}, {\sSeq{${\sCall {${x}$} {${y}$} {${m}$} {${z}$}}$}{${s}$}})} \cdot {S}}})}
}
\end{mathpar}

\begin{mathpar}
\inferrule* [Right=ESCallFinish]
{
    \evalex {H} {\rho} {\ex{${y}$}} {{o}} \\
    {H(o)} = {{({C}, {\usc})}} \\
    {\mpost{{C}, {m}}} = {{\phi}} \\
    \evalphix {H} {\rho'} {A'} {\phi} \\
    {A''} = {\dynamicFP {H} {\rho'} {\phi}} \\
    \evalex {H} {\rho'} {\ex{${\eresult}$}} {v_r}
}
{
    \sstep {({H}, {{({{\rho'}, {A'}}, {\sSkip})} \cdot {{({{\rho}, {A}}, {\sSeq{${\sCall {${x}$} {${y}$} {${m}$} {${z}$}}$}{${s}$}})} \cdot {S}}})} {({H}, {{({{{\rho}[{x} \mapsto {v_r}]}, {{A} \cup {A''}}}, {s})} \cdot {S}})}
}
\end{mathpar}

\begin{mathpar}
\inferrule* [Right=ESAssert]
{
    \evalphix {H} {\rho} {A} {\phi}
}
{
    \sstep {({H}, {{({{\rho}, {A}}, {\sSeq{${\sAssert {${\phi}$}}$}{${s}$}})} \cdot {S}})} {({H}, {{({{\rho}, {A}}, {s})} \cdot {S}})}
}
\end{mathpar}

\begin{mathpar}
\inferrule* [Right=ESRelease]
{
    \evalphix {H} {\rho} {A} {\phi} \\
    {A'} = {{A} \backslash {\dynamicFP {H} {\rho} {\phi}}}
}
{
    \sstep {({H}, {{({{\rho}, {A}}, {\sSeq{${\sRelease {${\phi}$}}$}{${s}$}})} \cdot {S}})} {({H}, {{({{\rho}, {A'}}, {s})} \cdot {S}})}
}
\end{mathpar}

\begin{mathpar}
\inferrule* [Right=ESDeclare]
{
    {\rho'} = {{\rho}[{x} \mapsto {\defaultValue{${T}$}}]}
}
{
    \sstep {({H}, {{({{\rho}, {A}}, {\sSeq{${\sDeclare {${T}$} {${x}$}}$}{${s}$}})} \cdot {S}})} {({H}, {{({{\rho'}, {A}}, {s})} \cdot {S}})}
}
\end{mathpar}

\begin{mathpar}
\inferrule* [Right=ESHold]
{
    \evalphix {H} {\rho} {A} {\phi} \\
    {A'} = {\dynamicFP {H} {\rho} {\phi}}
}
{
    \sstep {({H}, {{({{\rho}, {A}}, {\sSeq{${\sHold {${\phi}$} {${s'}$}}$}{${s}$}})} \cdot {S}})} {({H}, {{({{\rho}, {{A} \backslash {A'}}}, {s'})} \cdot {{({{\rho}, {A'}}, {\sSeq{${\sHold {${\phi}$} {${s'}$}}$}{${s}$}})} \cdot {S}}})}
}
\end{mathpar}

\begin{mathpar}
\inferrule* [Right=ESHoldFinish]
{
    ~
}
{
    \sstep {({H}, {{({{\rho'}, {A'}}, {\sSkip})} \cdot {{({{\rho}, {A}}, {\sSeq{${\sHold {${\phi}$} {${s'}$}}$}{${s}$}})} \cdot {S}}})} {({H}, {{({{\rho'}, {{A} \cup {A'}}}, {s})} \cdot {S}})}
}
\end{mathpar}


% Inductive Semantics.dynSem
\begin{mathpar}
\inferrule* [Right=ESSkip]
{
    ~
}
{
    \sstep {\langle{H}, {{\langle{{\rho}, {A}}, {\sSeq{${\sSkip}$}{${s}$}}\rangle} \cdot {S}}\rangle} {\langle{H}, {{\langle{{\rho}, {A}}, {s}\rangle} \cdot {S}}\rangle}
}

\inferrule* [Right=ESFieldAssign]
{
    \evalex {H} {\rho} {\ex{${x}$}} {{o}} \\
    \evalex {H} {\rho} {\ex{${y}$}} {v_y} \\
    {\langle{o}, {f}\rangle} \in {A} \\
    {H'} = {{H}[{o} \mapsto [{f} \mapsto {v_y}]]}
}
{
    \sstep {\langle{H}, {{\langle{{\rho}, {A}}, {\sSeq{${\sFieldAssign {${x}$} {${f}$} {${y}$}}$}{${s}$}}\rangle} \cdot {S}}\rangle} {\langle{H'}, {{\langle{{\rho}, {A}}, {s}\rangle} \cdot {S}}\rangle}
}

\inferrule* [Right=ESVarAssign]
{
    \evalex {H} {\rho} {e} {v} \\
    {\rho'} = {{\rho}[{x} \mapsto {v}]}
}
{
    \sstep {\langle{H}, {{\langle{{\rho}, {A}}, {\sSeq{${\sVarAssign {${x}$} {${e}$}}$}{${s}$}}\rangle} \cdot {S}}\rangle} {\langle{H}, {{\langle{{\rho'}, {A}}, {s}\rangle} \cdot {S}}\rangle}
}

\inferrule* [Right=ESAlloc]
{
    {o} \not\in \dom{{H}} \\
    {\fields{{C}}} = {{\overline{\field{$T$}{$f$}}}} \\
    {\rho'} = {{\rho}[{x} \mapsto {{o}}]} \\
    {A'} = {{A} \cup {\overline{\langle{o}, f_i\rangle}}} \\
    {H'} = {{H}[{o} \mapsto [\overline{f_i \mapsto \defaultValue{$T_i$}}]]}
}
{
    \sstep {\langle{H}, {{\langle{{\rho}, {A}}, {\sSeq{${\sAlloc {${x}$} {${C}$}}$}{${s}$}}\rangle} \cdot {S}}\rangle} {\langle{H'}, {{\langle{{\rho'}, {A'}}, {s}\rangle} \cdot {S}}\rangle}
}

\inferrule* [Right=ESReturn]
{
    \evalex {H} {\rho} {\ex{${x}$}} {v_x} \\
    {\rho'} = {{\rho}[{\eresult} \mapsto {v_x}]}
}
{
    \sstep {\langle{H}, {{\langle{{\rho}, {A}}, {\sSeq{${\sReturn {${x}$}}$}{${s}$}}\rangle} \cdot {S}}\rangle} {\langle{H}, {{\langle{{\rho'}, {A}}, {s}\rangle} \cdot {S}}\rangle}
}

\inferrule* [Right=ESCall]
{
    \evalex {H} {\rho} {\ex{${y}$}} {{o}} \\
    \evalex {H} {\rho} {\ex{${z}$}} {v} \\
    {H(o)} = {{\langle{C}, {\usc}\rangle}} \\
    {\mmethod{{C}, {m}}} = {{\method {${T_r}$} {${m}$} {${T}$} {${w}$} {${\contract {${\phi}$} {${\usc}$}}$} {${r}$}}} \\
    {\rho'} = {[{\eresult} \mapsto {\defaultValue{${T_r}$}}, {\ethis} \mapsto {{o}}, {{w}} \mapsto {v}]} \\
    \evalphix {H} {\rho'} {A} {\phi} \\
    {A'} = {\dynamicFP {H} {\rho'} {\phi}}
}
{
    \sstep {\langle{H}, {{\langle{{\rho}, {A}}, {\sSeq{${\sCall {${x}$} {${y}$} {${m}$} {${z}$}}$}{${s}$}}\rangle} \cdot {S}}\rangle} {\langle{H}, {{\langle{{\rho'}, {A'}}, {r}\rangle} \cdot {{\langle{{\rho}, {{A} \backslash {A'}}}, {\sSeq{${\sCall {${x}$} {${y}$} {${m}$} {${z}$}}$}{${s}$}}\rangle} \cdot {S}}}\rangle}
}

\inferrule* [Right=ESCallFinish]
{
    \evalex {H} {\rho} {\ex{${y}$}} {{o}} \\
    {H(o)} = {{\langle{C}, {\usc}\rangle}} \\
    {\mpost{{C}, {m}}} = {{\phi}} \\
    \evalphix {H} {\rho'} {A'} {\phi} \\
    \evalex {H} {\rho'} {\ex{${\eresult}$}} {v_r}
}
{
    \sstep {\langle{H}, {{\langle{{\rho'}, {A'}}, {\sSkip}\rangle} \cdot {{\langle{{\rho}, {A}}, {\sSeq{${\sCall {${x}$} {${y}$} {${m}$} {${z}$}}$}{${s}$}}\rangle} \cdot {S}}}\rangle} {\langle{H}, {{\langle{{{\rho}[{x} \mapsto {v_r}]}, {{A} \cup {A'}}}, {s}\rangle} \cdot {S}}\rangle}
}

\inferrule* [Right=ESAssert]
{
    \evalphix {H} {\rho} {A} {\phi}
}
{
    \sstep {\langle{H}, {{\langle{{\rho}, {A}}, {\sSeq{${\sAssert {${\phi}$}}$}{${s}$}}\rangle} \cdot {S}}\rangle} {\langle{H}, {{\langle{{\rho}, {A}}, {s}\rangle} \cdot {S}}\rangle}
}

\inferrule* [Right=ESRelease]
{
    \evalphix {H} {\rho} {A} {\phi} \\
    {A'} = {{A} \backslash {\dynamicFP {H} {\rho} {\phi}}}
}
{
    \sstep {\langle{H}, {{\langle{{\rho}, {A}}, {\sSeq{${\sRelease {${\phi}$}}$}{${s}$}}\rangle} \cdot {S}}\rangle} {\langle{H}, {{\langle{{\rho}, {A'}}, {s}\rangle} \cdot {S}}\rangle}
}

\inferrule* [Right=ESDeclare]
{
    {\rho'} = {{\rho}[{x} \mapsto {\defaultValue{${T}$}}]}
}
{
    \sstep {\langle{H}, {{\langle{{\rho}, {A}}, {\sSeq{${\sDeclare {${T}$} {${x}$}}$}{${s}$}}\rangle} \cdot {S}}\rangle} {\langle{H}, {{\langle{{\rho'}, {A}}, {s}\rangle} \cdot {S}}\rangle}
}

\inferrule* [Right=ESHold]
{
    \evalphix {H} {\rho} {A} {\phi} \\
    {A'} = {\dynamicFP {H} {\rho} {\phi}}
}
{
    \sstep {\langle{H}, {{\langle{{\rho}, {A}}, {\sSeq{${\sHold {${\phi}$} {${s'}$}}$}{${s}$}}\rangle} \cdot {S}}\rangle} {\langle{H}, {{\langle{{\rho}, {{A} \backslash {A'}}}, {\overline{s'}}\rangle} \cdot {{\langle{{\rho}, {A'}}, {\sSeq{${\sHold {${\phi}$} {${s'}$}}$}{${s}$}}\rangle} \cdot {S}}}\rangle}
}

\inferrule* [Right=ESHoldFinish]
{
    ~
}
{
    \sstep {\langle{H}, {{\langle{{\rho'}, {A'}}, {\sSkip}\rangle} \cdot {{\langle{{\rho}, {A}}, {\sSeq{${\sHold {${\phi}$} {${s'}$}}$}{${s}$}}\rangle} \cdot {S}}}\rangle} {\langle{H}, {{\langle{{\rho'}, {{A} \cup {A'}}}, {s}\rangle} \cdot {S}}\rangle}
}
\end{mathpar}


    \caption{\svlidf: Small-Step Semantics}
    \label{fig:svl-sem-dyn-sstep}
\end{figure}


% Again, the semantics is implicitly parameterized over some program $p$.
    
    \subsection{Soundness}
    % invariants?

\section{Gradualization}
\label{sec:cs-gradual-formulas}
We will now follow along the procedure introduced in chapter \ref{ch:gradualization-of-a} to design a gradually verified language “\gvl” based on \svl.
% BLA

The path we take:
\begin{align*} 
&\text{Syntax:}\\
&\grad{\phi} \quad::=\quad \phi ~|~ \withqm{\phi}\\
\\
&\text{Concretization:}\\
&\gamma(\phi) = \{~ \phi ~\}     \quad\quad \forall \phi \in \setFormulaB\\
&\gamma(\withqm{\phi}) = \{~ \phi' \in \setFormulaB ~|~ \phiImplies{\phi'}{\phi} ~\}\\
&\gamma(\grad{\phi}) = \emptyset    \quad\textit{otherwise}
\end{align*}


    \subsection{Extension: Statements}
    \label{ssec:extension--statements}
    
%% method contract extension
In \gvl we want the programmer to specify gradual method contracts.
Therefore we extend their syntax as follows.
\begin{align*}
\grad{contract} & \in \setGContract   &  & ::= \ttt{requires $\grad{\phi}$;~ensures $\grad{\phi}$;}
\end{align*}

%%% propagation
This extension is propagated to method declarations (now accepting gradual contracts but not changing otherwise), yielding $\setGMethod$.
Carrying on with the same logic, we get an extended set of class definitions $\setGClass$ and finally an extended set of programs $\setGProgram$.
Again, note that the only syntactical difference is the acceptance of gradual formulas in method contracts.

%% statements
We see no motive to extend the syntax of statements themselves and define $\setGStmt = \setStmt$.
As postulated in section \ref{sec:gradual-statements}, the call statement hides away gradualized syntax by referencing a method with gradual contract.
This becomes obvious when looking at its static or dynamic semantics (see \tset{HCall} and \tset{ESCall???}/\tset{ESCallFinish}) where the method name is effectively dereferenced.
% SO we will remember that when lifting stuff...

    \subsection{Extension: Program State}
    $\setGProgramState = \setProgramState$

\section{Gradualize Hoare Rules}
\label{sec:gradualize-hoare-rules}

\begin{figure}[h!]
    \boxed{\thoare {\Gamma} {\grad{\phi_{pre}}} {s} {\grad{\phi_{post}}}}
    \begin{mathpar}
    \inferrule* [right=\dgradT HSkip]
    {
        ~
    }
    {
        \dgthoare {\Gamma} {\grad{\phi}} {{\sSkip}} {\grad{\phi}}
    }
    
    \inferrule* [Right=\dgradT HAlloc]
    {
        {\wo {\grad{\phi}} {x}} = \grad{\phi'}\\
        \sType {\Gamma} {\ex{${x}$}} {{C}} \\
        {\fields{C}} = {{\overline{\field{$T$}{$f$}}}}
    }
    {
        \dgthoare {\Gamma} {\grad{\phi}} {{\sAlloc {${x}$} {${C}$}}} {\gphiCons{$\grad{\phi'}$}{${\gphiCons{${\phiNeq {${\ex{${x}$}}$} {${\ev{${\enull}$}}$}}$}{${\overline{\gphiCons{\phiAcc {$x$} {$f_i$}}{\phiEq {\edot {$x$} {$f_i$}} {\defaultValue {$T_i$}} }}}$}}$}}
    }
    
    \inferrule* [Right=\dgradT HFieldAssign]
    {
        \wo {\grad{\phi}} {\phiAcc{$x$}{$f$}} = \grad{\phi'}\\
        \sType {\Gamma} {\ex{${x}$}} {{C}} \\
        \sType {\Gamma} {\ex{${y}$}} {T} \\
        \vdash {C}.{f} : {T}
    }
    {
        \dgthoare {\Gamma} {\grad{\phi}} {{\sFieldAssign {${x}$} {${f}$} {${y}$}}} {\gphiCons{$\grad{\phi'}$}{${\gphiCons{${\phiAcc {${\ex{${x}$}}$} {${f}$}}$}{${\gphiCons{${\phiNeq {${\ex{${x}$}}$} {${\ev{${\enull}$}}$}}$}{${\ensuremath{{\phiEq {${\edot{${\ex{${x}$}}$}{${f}$}}$} {${\ex{${y}$}}$}}}}$}}$}}$}}
    }
    
    \inferrule* [Right=\dgradT HVarAssign]
    {
        \gphiImplies{\grad{\phi}} {\accFor {{e}}}\\
        {\wo {\grad{\phi}} {x}} = \grad{\phi'}\\
        {x} \not \in {\FV({e})} \\
        \sType {\Gamma} {\ex{${x}$}} {T} \\
        \sType {\Gamma} {e} {T}
    }
    {
        \dgthoare {\Gamma} {\grad{\phi}} {{\sVarAssign {${x}$} {${e}$}}} {\gphiCons{$\grad{\phi'}$}{${\ensuremath{{\phiEq {${\ex{${x}$}}$} {${e}$}}}}$}}
    }
    
    \inferrule* [Right=\dgradT HReturn]
    {
        {\wo {\grad{\phi}} {\eresult}} = \grad{\phi'}\\
        \sType {\Gamma} {\ex{${x}$}} {T} \\
        \sType {\Gamma} {\ex{${\eresult}$}} {T}
    }
    {
        \dgthoare {\Gamma} {\grad{\phi}} {{\sReturn {${x}$}}} {\gphiCons{$\grad{\phi'}$}{${\ensuremath{{\phiEq {${\ex{${\eresult}$}}$} {${\ex{${x}$}}$}}}}$}}
    }
    
    \inferrule* [Right=\dgradT HCall]
    {
        \wo {\wo {\grad{\phi}} {x}} {\grad{\phi_p}} = \grad{\phi'}\\
        \sType {\Gamma} {\ex{${y}$}} {{C}} \\
        {\mmethod{{C}, {m}}} = {{\method {${T_r}$} {${m}$} {${T_p}$} {${z}$} {${\contract {$\grad{\phi_{pre}}$} {$\grad{\phi_{post}}$}}$} {${\usc}$}}} \\
        \sType {\Gamma} {\ex{${x}$}} {T_r} \\
        \sType {\Gamma} {\ex{${z'}$}} {T_p} \\
        \gphiImplies{\grad{\phi}}{\gphiCons{${\phiNeq {${\ex{${y}$}}$} {${\ev{${\enull}$}}$}}$}{$\grad{\phi_p}$}} \\
        x \neq y \wedge x \neq z' \\
        \grad{\phi_p} = {\grad{\phi_{pre}}[{y}, {z'} / {\ethis}, {{z}}]} \\
        \grad{\phi_q} = {\grad{\phi_{post}}[{y}, {z'}, {x} / {\ethis}, {{z}}, {\eresult}]}
    }
    {
        \dgthoare {\Gamma} {\grad{\phi}} {{\sCall {${x}$} {${y}$} {${m}$} {${z'}$}}} {\gphiCons{$\grad{\phi'}$}{$\grad{\phi_q}$}}
    }
    
    \inferrule* [right=\dgradT HAssert]
    {
        \gphiImpliesEv{\grad{\phi}}{\phi_a}{\grad{\phi'}}
    }
    {
        \dgthoare {\Gamma} {\grad{\phi}} {{\sAssert {${\phi_a}$}}} {\grad{\phi'}}
    }
    
    \inferrule* [Right=\dgradT HRelease]
    {
        \gphiImpliesEv{\grad{\phi}}{\phi_r}{\grad{\phi'}}\\
        {\wo {\grad{\phi'}} {\phi_r}} = \grad{\phi''}
    }
    {
        \dgthoare {\Gamma} {\grad{\phi}} {{\sRelease {${\phi_r}$}}} {\grad{\phi'}}
    }
    
    \inferrule* [Right=\dgradT HDeclare]
    {
        {x} \not\in \dom{\Gamma} \\
        \dgthoare {{\Gamma}, {x} : {T}} {\gphiCons{${\phiEq {${\ex{${x}$}}$} {${\ev{${\defaultValue{${T}$}}$}}$}}$}{$\grad{\phi}$}} {s} {\grad{\phi'}}
    }
    {
        \dgthoare {\Gamma} {\grad{\phi}} {\sSeq{\sDeclare {${T}$} {${x}$}}{$s$}} {\grad{\phi'}}
    }
    
    \inferrule* [Right=\dgradT HHold]
    {
        \sfrmphi {\phi} \\
        \gphiImpliesEv{\grad{\phi_f}}{\phi}{\grad{\phi_f'}}\\
        \wo {\grad{\phi_f'}} {\staticFP{\phi}} = \grad{\phi_r}\\
        \wo {\wo {\grad{\phi_f'}} {\static{$\grad{\phi_r}$}}}{(\FV(\grad{\phi_f'}) \backslash \FV(\phi))} = \grad{\phi'}\\
        \mods{s} \cap \FV(\phi) = \emptyset \\
        \dgthoare {\Gamma} {\grad{\phi_r}} {s} {\grad {\phi_r'}}
    }
    {
        \dgthoare {\Gamma} {\grad{\phi_f}} {{\sHold {${\phi}$} {$s$}}} {\gphiCons{$\grad{\phi_r'}$}{$\grad{\phi'}$}}
    }
    
    \inferrule* [Right=\dgradT HSeq]
    {
        \dgthoare {\Gamma} {\grad{\phi_p}} {s_1} {\grad{\phi_q}} \\
        \dgthoare {\Gamma} {\grad{\phi_q}} {s_2} {\grad{\phi_r}}
    }
    {
        \dgthoare {\Gamma} {\grad{\phi_p}} {\sSeq{$s_1$}{$s_2$}} {\grad{\phi_r}}
    }
\end{mathpar}





% Let $\gsc$ behave like $\hsc$ if first operand is static - otherwise its regular concatenation.




    \caption{\gvl: Gradual Hoare Logic} 
\end{figure}


\section{Gradual Dyn. Semantics}
% thing with HSec
% Tapp usw...

% THIS IS WHERE THE SIMPLIFICATION AND ALL ITS REASONING COMES IN!


\chapter{Evaluation/Analysis}
\label{ch:evaluation-analysis}

\begin{comment}
> E:
with gradual typestates the same problem happened: as soon as the potential for unknown annotations was accepted, there was a “baseline cost” just to maintain the necessary infrastructure.
With simple gradual types, it’s almost nothing. With gradual effects, we’ve shown that it can boil down to very little (a thread-local variable with little overhead, see OOPSLA’15). 
\end{comment}

% relationship LC and verification! which rules are derived by which, correspondence table

\section{Enhancing an Unverified Language}
\label{sec:enhancing-an-unverified}
%% intro
The approach we presented constructs a gradually verified language in terms of a statically verified one.
However, as the example in section \ref{ssec:argument-validation} motivates, gradual verification can also be seen as an extension of the dynamically verified setting.
The main drawbacks of dynamic verification (runtime overhead and potentially late error detection) would be counteracted by using static verification techniques where possible.
A gradual verifier will attempt to prove compliance with annotations (making runtime checks unnecessary) or may detect an inevitable violation of an annotation (detecting an error before the program is executed).
In this section, we will briefly describe how to turn a dynamically verified language into a gradually verified one.

%% dyn ver spectrum
To understand how to approach dynamically verified languages, we have to examine how they fit into the spectrum of a gradually verified one.
The static end is (by construction) the one where all formulas are static and the unknown formula “\qm” is never used.
In contrast, annotating $\qm$ everywhere corresponds to a system that solely relies on runtime checks.
One can further define that leaving out annotations (e.g. precondition, postcondition, or even both) corresponds to an annotation of $\qm$.
A dynamically verified language (say, Java) can be imitated using this process.
Given such a language, the steps are thus as follows:

%% steps
First one has to identify what the goal of a static verification system would be for the language at hand.
Common examples include race-condition avoidance or a no-throw guarantee.
One can then develop a sound Hoare logic that achieves that goal, including syntax extensions to the language that allow leveraging that Hoare logic.
Gradualization can be applied to that system, introducing $\qm$ as a “default formula”, which allows making all newly introduced syntax extensions optional again.
The syntax of the resulting gradually verified language should thus be a superset of the dynamically verified one.
Contracts that were before explicitly realized as runtime checks can now be removed and encoded using the extended syntax.


\chapter{Conclusion}
\label{ch:conclusion}

Recap, remind reader what big picture was.
Briefly outline your thesis, motivation, problem, and proposed solution.





\section{Conceptual Nugget: Comparison/Implication to AGT!}

\section{Limitations}
\label{sec:limitations}
\input{text/SEC-limitations}

\section{Future Work}
\label{sec:future-work}

% does it work with wlp semantics?
% wlp not an option since not well-defined for some rules:
\begin{displaymath}
\predicate{wlp}(\text{“\sVarAssign{x}{a.f}”}, \phiAcc{b}{f}) =
\begin{cases}
\phiCons {\phiAcc{b}{f}} {\phiAcc{a}{f}}\\
\phiCons {\phiAcc{b}{f}} {\phiEq{a}{b}}
\end{cases}
\end{displaymath}

\chapter{Appendix}


\chapter{UNSORTED}

\section{HoareMotivEx}
\label{sec:hoaremotivex}

% AGT: mention draft about refinement types?

Hoare Logic as formal setting

\begin{verbatim}
class Point
{
    int manhattenDistance(Point p)
        requires \phi_{pre};
        ensures  \phi_{post};
    {
        s1;
        s2;
        .
        .
        .
    }
}
\end{verbatim}

\begin{displaymath}
\thoare
    {\ethis : \type{Point}, \ex{p} : \type{Point}, \eresult : \Tint}
    {\phi_{pre}}
    {s1; s2; ...}
    {\phi_{post}}
\end{displaymath}

\section{NPC formula}
\label{sec:npc-formula}
%% reasonable: formulas evaluable at runtime
Checking a formula at runtime, i.e. performing a runtime assertion check, is the integral part of dynamic verification and thus plays a role in gradual verification.
Formally, a runtime assertion check corresponds to evaluating a closed formula since the environment provides an instantiation of the formula's free variables.
It is reasonable to demand that this check can be performed in a time polynomial, if not linear to the formula's length (the specifics are up to the language designer, of course).

%% impact on formula syntax: quantification
Such a requirement effectively restricts the formula syntax.
For example, a syntax containing universal quantification generally violates above runtime limitations:
A formula $\forall x_1 \in M, x_2 \in M, ..., x_n \in M. P(x_1, x_2, ..., x_n)$ would require $|M|^n$ steps to evaluate.
As a result, the execution time is already exponential if $M$ is finite -- and unbounded otherwise.

%% impact on formula syntax: still pretty little restriction
Putting quantification (and therefore the introduction of new variables) aside, there are little restrictions to formula syntax, essentially allowing any predicates or operations that can be evaluated in linear/polynomial time.
This includes equality/inequality relations, arithmetic and even own predicates that might be recursive to some extent.

%% nevertheless: static verification undec.
Nevertheless such “easily” evaluable formulas are also subject to higher order reasoning in the static verification rules, including checks like satisfiability of or implication between formulas.
Those judgments basically introduce quantification of the free variables, whereas evaluation works on a concrete instantiation.
This makes static verification NP-hard in general:
\begin{description}
    \item[NPC] One can easily encode SAT instances as formulas, either directly (if the syntax covers boolean variables, conjunction and disjunction) or using arithmetic (if the syntax covers addition and a comparison relation like “greater-than”). Note that although evaluating such formulas is trivial, checking for satisfiability is NP-complete. % TODO: reference???
    \item[Undecidability] ...Paeno-arithmetic % TODO
\end{description}

%% our syntax
We chose the formula syntax of ... specifically to ensure that even static semantics are decidable in polynomial time.
This allowed applying the procedures of AGT directly, as they are based on a decidable type system, i.e. decidable .

\subsection{Impact of NP-hard Verification Predicates}
Let's assume that our rules for static verification indeed contain an NP-hard predicate $P$. (NOTE: need positive occurrence for following reasoning!)
The immediate consequence is that any working verifier would have to realize a conservative approximation of the actual predicate.

Under-approximation: for static guarantees to hold, verifier must under-approximate $P$... blabla

Over-approximation: for (det.) gradual lifting to be ?sound?, it must over-approximate $P$... blabla