\begin{center}
\begin{minipage}{0.7\textwidth}
\vspace*{15\baselineskip}
\begin{center}
\textbf{Abstract}
\end{center}
% Purpose/Problem
%Methods of program verification can traditionally be categorized as either static or dynamic.
%Static verification uses formal methods like Hoare logic to guarantee conformance with a given specification without running the program.
%Dynamic verification uses runtime checks to ensure that deviations from the specification are detected during execution.
Both static and dynamic program verification approaches have disadvantages potentially disqualifying them as a single methodology to rely on.
% Proposal/Methods
Motivated by gradual type systems, which solve a very similar dilemma in the world of type systems, we propose \textit{gradual verification}, an approach that seamlessly combines static and dynamic verification.
Drawing on principles from abstract interpretation and recent work on \textit{abstracting gradual typing} by Garcia, Clark and Tanter, we formalize how to obtain a gradual verification system in terms of a static one.

% Results
This approach yields \textit{by construction} a verification system that is compatible with the original static system, but overcomes its rigidity by resorting to methods of dynamic verification if necessary.
% gradual guarantee
In a case study, we show the flexibility of our approach by applying it to a concurrent statically verified language that uses implicit dynamic frames to enable race-free reasoning.
\end{minipage}
\end{center}
