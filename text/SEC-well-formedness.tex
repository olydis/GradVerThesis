With static semantics in place, we can define what makes programs well-formed.
Well-formedness is required to ...
The following predicates 

A program is well-formed if both its classes and main method are.
For the main method to be well-formed, it must satisfy our Hoare predicate, given no assumptions.

\begin{mathpar}
\inferrule* [Right=OkProgram]
{
\overline{cls_i \OK} \\
\thoare {~} {\phiTrue} {\overline{s}} {\phiTrue}
}
{(\overline{cls_i}~\overline{s}) \OK}
\end{mathpar}

% TODO: start with deterministic semantics!?

\begin{mathpar}
\inferrule* [Right=OkClass]
{
\text{unique $field$-names} \\
\text{unique $method$-names} \\
\overline{method_i \OKinC}
}
{(\class {$C$} {$\overline{field_i}$} {$\overline{method_i}$}) \OK}
\end{mathpar}

\begin{mathpar}
\inferrule* [Right=OkMethod]
{
FV(\phi_1) \subseteq \{ x, \ethis \} \\
FV(\phi_2) \subseteq \{ x, \ethis, \eresult \} \\
\thoare {x : T_x, \ethis : C, \eresult : T_m} {\phi_1} {\overline{s}} {\phi_2} \\
\phi_1, \phi_2 \in \setFormulaB \\
\overline{\neg \writesTo(s_i, x)}
}
{(\method {$T_m$} {$m$} {$T_x$} {$x$} {\contract {$\phi_1$} {$\phi_2$}} {$\overline{s}$}) \OKinC}
\end{mathpar}