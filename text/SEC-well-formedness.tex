Apart from checking method contracts, a verifier or compiler may enforce further rules before accepting a program as “well formed”.
For \svlidf we give the following rules.

\begin{figure}[h]
    
\begin{mathpar}
\inferrule* [Right=OkProgram]
{
\overline{cls_i \OK} \\
\thoare {~} {\phiTrue} {s} {\phi}
}
{(\overline{cls_i}~s) \OK}
\end{mathpar}

\begin{mathpar}
\inferrule* [Right=OkClass]
{
\text{unique $field$-names} \\
\text{unique $method$-names} \\
\overline{method_i \OKinC}
}
{(\class {$C$} {$\overline{field_i}$} {$\overline{method_i}$}) \OK}
\end{mathpar}

\begin{mathpar}
\inferrule* [Right=OkMethod]
{
    \thoare {x : T_x, \ethis : C, \eresult : T_m} {\phi_1} {s} {\phi_2} \\\\
    \FV(\phi_1) \subseteq \{ x, \ethis \} \\
    \FV(\phi_2) \subseteq \{ x, \ethis, \eresult \} \\\\
    \sfrmphi \phi_1 \\
    \sfrmphi \phi_2 \\
    x, \ethis \not \in \mods(s)
}
{(\method {$T_m$} {$m$} {$T_x$} {$x$} {\contract {$\phi_1$} {$\phi_2$}} {$s$}) \OKinC}
\end{mathpar}
    \caption{\svlidf: Well-Formedness}
    \label{fig:idf-wf}
\end{figure}

%% OkMethod
The premises of $\tset{OkMethod}$ make sure that reasoning about calls is sound.
As expected, the method contract is checked, while also making sure that it contains self-framing formulas.
Furthermore, the free variables are restricted to those occurring in the method signature.

%% example
\begin{example}{Leaking Postcondition}
\begin{lstlisting}
int identity(int a)
    requires true;
    ensures  (b = 3);
{
    int b;
    b = 3;
    return a;
}
\end{lstlisting}
While the method passes static verification, it could lead to unsound proofs.
Note how \tset{HCall} forwards the postcondition after replacing known variables with their counterparts.
\phiEq{b}{3} is unaffected by this replacement, ending up in the postcondition of the call statement.
\end{example}