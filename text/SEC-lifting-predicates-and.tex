
The Hoare logic of our language are defined in terms of predicates and functions that operate on formulas.
Examples:
% TODO:
% - implication, e.g. in assertion statement
% - adding equality, e.g. in assignment statement (or look in static semantics for better examples...)
% - Hoare logic itself!

After introducing and giving meaning to gradual formulas, we will now describe how to redefine existing predicates and functions in order for them to deal with gradual formulas.

\begin{definition}[Gradual Lifting]
    The process of extending an existing predicate/function in order to deal with gradual formulas.
    The resulting predicate/function has the same signature as the original one, with occurrences of \setFormula replaced by \setGFormula.
\end{definition}

Lifted predicates and functions are not allowed to deal with gradual formulas arbitrarily but must do so in a sound way.
What soundness means is a direct consequence of the gradual guarantee (definition \ref{grad-guarantee-def}), i.e. an unsound predicate/function may cause the gradual verification system to break the gradual guarantee.
This idea is formalized in the following sections.