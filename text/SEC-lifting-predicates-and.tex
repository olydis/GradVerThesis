
The axiomatic semantics of our language are (PROBABLY!?) defined in terms of predicates and functions that operate on formulas.
Examples:
% TODO:
% - implication, e.g. in assertion statement
% - adding equality, e.g. in assignment statement (or look in static semantics for better examples...)
% - axiomatic semantics itself!

After introducing and giving meaning to gradual formulas, we will now describe how to redefine existing predicates and functions in order for them to deal with gradual formulas.

\begin{definition}[Gradual Lifting]
    The process of creating a “gradual version”/“lifted version” of a predicate/function.
    The resulting predicate/function has the same signature as the original one, with occurrences of \setFormula replaced by \setGFormula.
\end{definition}

The more important question is of course how to define such lifted versions in a consistent way.
What consistency means is a direct consequence of the gradual guarantee (definition \ref{grad-guarantee-def}), i.e. an inconsistently lifted predicate/function may cause the gradual verification system to break the gradual guarantee.
The specifics are described in the following sections.