% MENTION that most of that stuff is completely redundant if soundness holds - and only used to prove just that!

We define the small-step semantics $\sssem$ of \svlidf in terms of an inductively defined function $\sstep{\cdot}{\cdot} : \setProgramState \rightharpoonup \setProgramState$:
\begin{displaymath}
\sssem(\pi) = \pi' \defiff \sstep{\pi}{\pi'}
\end{displaymath}

Note that again, the semantics is implicitly parameterized over some program $p$.
Figure \ref{fig:svl-sem-dyn-sstep} shows the inductive rules.
\begin{figure}
    \boxed{\sstep{\pi}{\pi}}
    %% Inductive Semantics.dynSem
\begin{mathpar}
\inferrule* [Right=ESSkip]
{
    ~
}
{
    \sstep {({H}, {{({{\rho}, {A}}, {\sSeq{${\sSkip}$}{${s}$}})} \cdot {S}})} {({H}, {{({{\rho}, {A}}, {s})} \cdot {S}})}
}
\end{mathpar}

\begin{mathpar}
\inferrule* [Right=ESFieldAssign]
{
    \evalex {H} {\rho} {\ex{${x}$}} {{o}} \\
    \evalex {H} {\rho} {\ex{${y}$}} {v_y} \\
    {({o}, {f})} \in {A} \\
    {H'} = {{H}[{o} \mapsto [{f} \mapsto {v_y}]]}
}
{
    \sstep {({H}, {{({{\rho}, {A}}, {\sSeq{${\sFieldAssign {${x}$} {${f}$} {${y}$}}$}{${s}$}})} \cdot {S}})} {({H'}, {{({{\rho}, {A}}, {s})} \cdot {S}})}
}
\end{mathpar}

\begin{mathpar}
\inferrule* [Right=ESVarAssign]
{
    \evalex {H} {\rho} {e} {v} \\
    {\rho'} = {{\rho}[{x} \mapsto {v}]}
}
{
    \sstep {({H}, {{({{\rho}, {A}}, {\sSeq{${\sVarAssign {${x}$} {${e}$}}$}{${s}$}})} \cdot {S}})} {({H}, {{({{\rho'}, {A}}, {s})} \cdot {S}})}
}
\end{mathpar}

\begin{mathpar}
\inferrule* [Right=ESAlloc]
{
    {o} \not\in \dom{{H}} \\
    {\fields{{C}}} = {{\overline{\field{$T$}{$f$}}}} \\
    {\rho'} = {{\rho}[{x} \mapsto {{o}}]} \\
    {A'} = {{A} \cup {\overline{({o}, f_i)}}} \\
    {H'} = {{H}[{o} \mapsto [\overline{f_i \mapsto \defaultValue{$T_i$}}]]}
}
{
    \sstep {({H}, {{({{\rho}, {A}}, {\sSeq{${\sAlloc {${x}$} {${C}$}}$}{${s}$}})} \cdot {S}})} {({H'}, {{({{\rho'}, {A'}}, {s})} \cdot {S}})}
}
\end{mathpar}

\begin{mathpar}
\inferrule* [Right=ESReturn]
{
    \evalex {H} {\rho} {\ex{${x}$}} {v_x} \\
    {\rho'} = {{\rho}[{\eresult} \mapsto {v_x}]}
}
{
    \sstep {({H}, {{({{\rho}, {A}}, {\sSeq{${\sReturn {${x}$}}$}{${s}$}})} \cdot {S}})} {({H}, {{({{\rho'}, {A}}, {s})} \cdot {S}})}
}
\end{mathpar}

\begin{mathpar}
\inferrule* [Right=ESCall]
{
    \evalex {H} {\rho} {\ex{${y}$}} {{o}} \\
    \evalex {H} {\rho} {\ex{${z}$}} {v} \\
    {H(o)} = {{({C}, {\usc})}} \\
    {\mmethod{{C}, {m}}} = {{\method {${T_r}$} {${m}$} {${T}$} {${w}$} {${\contract {${\phi}$} {${\usc}$}}$} {${r}$}}} \\
    {\rho'} = {[{\eresult} \mapsto {\defaultValue{${T_r}$}}, {\ethis} \mapsto {{o}}, {{w}} \mapsto {v}]} \\
    \evalphix {H} {\rho'} {A} {\phi} \\
    {A'} = {\dynamicFP {H} {\rho'} {\phi}}
}
{
    \sstep {({H}, {{({{\rho}, {A}}, {\sSeq{${\sCall {${x}$} {${y}$} {${m}$} {${z}$}}$}{${s}$}})} \cdot {S}})} {({H}, {{({{\rho'}, {A'}}, {r})} \cdot {{({{\rho}, {{A} \backslash {A'}}}, {\sSeq{${\sCall {${x}$} {${y}$} {${m}$} {${z}$}}$}{${s}$}})} \cdot {S}}})}
}
\end{mathpar}

\begin{mathpar}
\inferrule* [Right=ESCallFinish]
{
    \evalex {H} {\rho} {\ex{${y}$}} {{o}} \\
    {H(o)} = {{({C}, {\usc})}} \\
    {\mpost{{C}, {m}}} = {{\phi}} \\
    \evalphix {H} {\rho'} {A'} {\phi} \\
    {A''} = {\dynamicFP {H} {\rho'} {\phi}} \\
    \evalex {H} {\rho'} {\ex{${\eresult}$}} {v_r}
}
{
    \sstep {({H}, {{({{\rho'}, {A'}}, {\emptyset})} \cdot {{({{\rho}, {A}}, {\sSeq{${\sCall {${x}$} {${y}$} {${m}$} {${z}$}}$}{${s}$}})} \cdot {S}}})} {({H}, {{({{{\rho}[{x} \mapsto {v_r}]}, {{A} \cup {A''}}}, {s})} \cdot {S}})}
}
\end{mathpar}

\begin{mathpar}
\inferrule* [Right=ESAssert]
{
    \evalphix {H} {\rho} {A} {\phi}
}
{
    \sstep {({H}, {{({{\rho}, {A}}, {\sSeq{${\sAssert {${\phi}$}}$}{${s}$}})} \cdot {S}})} {({H}, {{({{\rho}, {A}}, {s})} \cdot {S}})}
}
\end{mathpar}

\begin{mathpar}
\inferrule* [Right=ESRelease]
{
    \evalphix {H} {\rho} {A} {\phi} \\
    {A'} = {{A} \backslash {\dynamicFP {H} {\rho} {\phi}}}
}
{
    \sstep {({H}, {{({{\rho}, {A}}, {\sSeq{${\sRelease {${\phi}$}}$}{${s}$}})} \cdot {S}})} {({H}, {{({{\rho}, {A'}}, {s})} \cdot {S}})}
}
\end{mathpar}

\begin{mathpar}
\inferrule* [Right=ESDeclare]
{
    {\rho'} = {{\rho}[{x} \mapsto {\defaultValue{${T}$}}]}
}
{
    \sstep {({H}, {{({{\rho}, {A}}, {\sSeq{${\sDeclare {${T}$} {${x}$}}$}{${s}$}})} \cdot {S}})} {({H}, {{({{\rho'}, {A}}, {s})} \cdot {S}})}
}
\end{mathpar}

\begin{mathpar}
\inferrule* [Right=ESHold]
{
    \evalphix {H} {\rho} {A} {\phi} \\
    {A'} = {\dynamicFP {H} {\rho} {\phi}}
}
{
    \sstep {({H}, {{({{\rho}, {A}}, {\sSeq{${\sHold {${\phi}$} {${s'}$}}$}{${s}$}})} \cdot {S}})} {({H}, {{({{\rho}, {{A} \backslash {A'}}}, {s'})} \cdot {{({{\rho}, {A'}}, {\sSeq{${\sHold {${\phi}$} {${s'}$}}$}{${s}$}})} \cdot {S}}})}
}
\end{mathpar}

\begin{mathpar}
\inferrule* [Right=ESHoldFinish]
{
    ~
}
{
    \sstep {({H}, {{({{\rho'}, {A'}}, {\emptyset})} \cdot {{({{\rho}, {A}}, {\sSeq{${\sHold {${\phi}$} {${s'}$}}$}{${s}$}})} \cdot {S}}})} {({H}, {{({{\rho'}, {{A} \cup {A'}}}, {s})} \cdot {S}})}
}
\end{mathpar}


% Inductive Semantics.dynSem
\begin{mathpar}
\inferrule* [Right=ESSkip]
{
    ~
}
{
    \sstep {\langle{H}, {{\langle{{\rho}, {A}}, {\sSeq{${\sSkip}$}{${s}$}}\rangle} \cdot {S}}\rangle} {\langle{H}, {{\langle{{\rho}, {A}}, {s}\rangle} \cdot {S}}\rangle}
}

\inferrule* [Right=ESFieldAssign]
{
    \evalex {H} {\rho} {\ex{${x}$}} {{o}} \\
    \evalex {H} {\rho} {\ex{${y}$}} {v_y} \\
    {\langle{o}, {f}\rangle} \in {A} \\
    {H'} = {{H}[{o} \mapsto [{f} \mapsto {v_y}]]}
}
{
    \sstep {\langle{H}, {{\langle{{\rho}, {A}}, {\sSeq{${\sFieldAssign {${x}$} {${f}$} {${y}$}}$}{${s}$}}\rangle} \cdot {S}}\rangle} {\langle{H'}, {{\langle{{\rho}, {A}}, {s}\rangle} \cdot {S}}\rangle}
}

\inferrule* [Right=ESVarAssign]
{
    \evalex {H} {\rho} {e} {v} \\
    {\rho'} = {{\rho}[{x} \mapsto {v}]}
}
{
    \sstep {\langle{H}, {{\langle{{\rho}, {A}}, {\sSeq{${\sVarAssign {${x}$} {${e}$}}$}{${s}$}}\rangle} \cdot {S}}\rangle} {\langle{H}, {{\langle{{\rho'}, {A}}, {s}\rangle} \cdot {S}}\rangle}
}

\inferrule* [Right=ESAlloc]
{
    {o} \not\in \dom{{H}} \\
    {\fields{{C}}} = {{\overline{\field{$T$}{$f$}}}} \\
    {\rho'} = {{\rho}[{x} \mapsto {{o}}]} \\
    {A'} = {{A} \cup {\overline{\langle{o}, f_i\rangle}}} \\
    {H'} = {{H}[{o} \mapsto [\overline{f_i \mapsto \defaultValue{$T_i$}}]]}
}
{
    \sstep {\langle{H}, {{\langle{{\rho}, {A}}, {\sSeq{${\sAlloc {${x}$} {${C}$}}$}{${s}$}}\rangle} \cdot {S}}\rangle} {\langle{H'}, {{\langle{{\rho'}, {A'}}, {s}\rangle} \cdot {S}}\rangle}
}

\inferrule* [Right=ESReturn]
{
    \evalex {H} {\rho} {\ex{${x}$}} {v_x} \\
    {\rho'} = {{\rho}[{\eresult} \mapsto {v_x}]}
}
{
    \sstep {\langle{H}, {{\langle{{\rho}, {A}}, {\sSeq{${\sReturn {${x}$}}$}{${s}$}}\rangle} \cdot {S}}\rangle} {\langle{H}, {{\langle{{\rho'}, {A}}, {s}\rangle} \cdot {S}}\rangle}
}

\inferrule* [Right=ESCall]
{
    \evalex {H} {\rho} {\ex{${y}$}} {{o}} \\
    \evalex {H} {\rho} {\ex{${z}$}} {v} \\
    {H(o)} = {{\langle{C}, {\usc}\rangle}} \\
    {\mmethod{{C}, {m}}} = {{\method {${T_r}$} {${m}$} {${T}$} {${w}$} {${\contract {${\phi}$} {${\usc}$}}$} {${r}$}}} \\
    {\rho'} = {[{\eresult} \mapsto {\defaultValue{${T_r}$}}, {\ethis} \mapsto {{o}}, {{w}} \mapsto {v}]} \\
    \evalphix {H} {\rho'} {A} {\phi} \\
    {A'} = {\dynamicFP {H} {\rho'} {\phi}}
}
{
    \sstep {\langle{H}, {{\langle{{\rho}, {A}}, {\sSeq{${\sCall {${x}$} {${y}$} {${m}$} {${z}$}}$}{${s}$}}\rangle} \cdot {S}}\rangle} {\langle{H}, {{\langle{{\rho'}, {A'}}, {r}\rangle} \cdot {{\langle{{\rho}, {{A} \backslash {A'}}}, {\sSeq{${\sCall {${x}$} {${y}$} {${m}$} {${z}$}}$}{${s}$}}\rangle} \cdot {S}}}\rangle}
}

\inferrule* [Right=ESCallFinish]
{
    \evalex {H} {\rho} {\ex{${y}$}} {{o}} \\
    {H(o)} = {{\langle{C}, {\usc}\rangle}} \\
    {\mpost{{C}, {m}}} = {{\phi}} \\
    \evalphix {H} {\rho'} {A'} {\phi} \\
    {A''} = {\dynamicFP {H} {\rho'} {\phi}} \\
    \evalex {H} {\rho'} {\ex{${\eresult}$}} {v_r}
}
{
    \sstep {\langle{H}, {{\langle{{\rho'}, {A'}}, {\sSkip}\rangle} \cdot {{\langle{{\rho}, {A}}, {\sSeq{${\sCall {${x}$} {${y}$} {${m}$} {${z}$}}$}{${s}$}}\rangle} \cdot {S}}}\rangle} {\langle{H}, {{\langle{{{\rho}[{x} \mapsto {v_r}]}, {{A} \cup {A''}}}, {s}\rangle} \cdot {S}}\rangle}
}

\inferrule* [Right=ESAssert]
{
    \evalphix {H} {\rho} {A} {\phi}
}
{
    \sstep {\langle{H}, {{\langle{{\rho}, {A}}, {\sSeq{${\sAssert {${\phi}$}}$}{${s}$}}\rangle} \cdot {S}}\rangle} {\langle{H}, {{\langle{{\rho}, {A}}, {s}\rangle} \cdot {S}}\rangle}
}

\inferrule* [Right=ESRelease]
{
    \evalphix {H} {\rho} {A} {\phi} \\
    {A'} = {{A} \backslash {\dynamicFP {H} {\rho} {\phi}}}
}
{
    \sstep {\langle{H}, {{\langle{{\rho}, {A}}, {\sSeq{${\sRelease {${\phi}$}}$}{${s}$}}\rangle} \cdot {S}}\rangle} {\langle{H}, {{\langle{{\rho}, {A'}}, {s}\rangle} \cdot {S}}\rangle}
}

\inferrule* [Right=ESDeclare]
{
    {\rho'} = {{\rho}[{x} \mapsto {\defaultValue{${T}$}}]}
}
{
    \sstep {\langle{H}, {{\langle{{\rho}, {A}}, {\sSeq{${\sDeclare {${T}$} {${x}$}}$}{${s}$}}\rangle} \cdot {S}}\rangle} {\langle{H}, {{\langle{{\rho'}, {A}}, {s}\rangle} \cdot {S}}\rangle}
}

\inferrule* [Right=ESHold]
{
    \evalphix {H} {\rho} {A} {\phi} \\
    {A'} = {\dynamicFP {H} {\rho} {\phi}}
}
{
    \sstep {\langle{H}, {{\langle{{\rho}, {A}}, {\sSeq{${\sHold {${\phi}$} {${s'}$}}$}{${s}$}}\rangle} \cdot {S}}\rangle} {\langle{H}, {{\langle{{\rho}, {{A} \backslash {A'}}}, {\overline{s'}}\rangle} \cdot {{\langle{{\rho}, {A'}}, {\sSeq{${\sHold {${\phi}$} {${s'}$}}$}{${s}$}}\rangle} \cdot {S}}}\rangle}
}

\inferrule* [Right=ESHoldFinish]
{
    ~
}
{
    \sstep {\langle{H}, {{\langle{{\rho'}, {A'}}, {\sSkip}\rangle} \cdot {{\langle{{\rho}, {A}}, {\sSeq{${\sHold {${\phi}$} {${s'}$}}$}{${s}$}}\rangle} \cdot {S}}}\rangle} {\langle{H}, {{\langle{{\rho'}, {{A} \cup {A'}}}, {s}\rangle} \cdot {S}}\rangle}
}
\end{mathpar}


    \caption{\svlidf: Small-Step Semantics}
    \label{fig:svl-sem-dyn-sstep}
\end{figure}
Note that right-associativity of \ttt{;} and termination of sequences with $\sSkip$ (see section \ref{sec:syntax}) obviates the need for dedicated sequence rules.

Using inductive rules to define partial function may not result in more than one distinct deducible results given any parameters.
\begin{lemma}[$\sssem$ Well-Defined]
    The small-step semantics of \svlidf is well-defined.
\end{lemma}