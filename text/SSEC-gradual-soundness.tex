With the notion of sound gradual lifting we have the tools to gradualize the semantics of \svl, resulting in gradual semantics of \gvl.
More specifically, predicate lifting is applied to the Hoare logic of \svl, resulting in a gradual Hoare logic $$\gthoare{~}{\cdot}{\cdot}{\cdot} ~\subseteq~ \setGFormula \times \setGStmt \times \setGFormula$$ (see section \ref{sec:abstracting-static-semantics}).
Furthermore, function lifting is applied to the small-step semantics of \svl, resulting in gradual small-step semantics $$\gsstep{\cdot}{\cdot} : \setGProgramState \rightharpoonup \setGProgramState$$ (see section \ref{sec:abstracting-dynamic-semantics}).
These semantics will by construction be compatible with the semantics of \svl and comply with the gradual guarantee.

Note however that there is an additional requirement concerning the correct interplay between Hoare logic and small-step semantics, namely soundness.
The following definitions attempt to define soundness of \gvl analogous to soundness of \svl (see definition \ref{def:valid-hoare-triple} and \ref{def:snd}).

%% valid gradual Hoare triples
\begin{definition}[Semantical Validity of Gradual Hoare Triples]~\\
    \label{def:valid-ghoare-triple}
    A Hoare triple $\hoare{\grad{\phi_{pre}}} {\grad{s}} {\grad{\phi_{post}}}$ is \textbf{valid}, written
    $\gtHoare{}{\grad{\phi_{pre}}} {\grad{s}} {\grad{\phi_{post}}}$
    iff
    \begin{flalign*}
    \forall \grad{\pi_{pre}}, \grad{\pi_{post}} \in \setGProgramState.~ \gsstepConsume{\grad{s}}{\grad{\pi_{pre}}}{\grad{\pi_{post}}} \implies (\evalgphiGen{\grad{\pi_{pre}}}{\grad{\phi_{pre}}} \implies \evalgphiGen{\grad{\pi_{post}}}{\grad{\phi_{post}}})
    \end{flalign*}
\end{definition}
\begin{lemma}[Compositional Validity of Gradual Hoare Triples]
    \label{lem:comp-gtHoare}
    \begin{align*}
    \forall \grad{\phi_1}, \grad{\phi_2}, \grad{\phi_3} \in \setGFormula,\, \grad{s_1}, \grad{s_2} \in \setGStmt.~ 
    &\gtHoare{}{\grad{\phi_1}} {\grad{s_1}} {\grad{\phi_2}} ~\wedge~ \gtHoare{}{\grad{\phi_2}} {\grad{s_2}} {\grad{\phi_3}} \\
    \implies
    &\tHoare{}{\grad{\phi_1}} {\sSeq{$\grad{s_1}$}{$\grad{s_2}$}} {\grad{\phi_3}}
    \end{align*}
\end{lemma}

\begin{definition}[\gradT Soundness (First Iteration)]\label{def:gsnd-i}~\\
    Let $\grad{\phi_1}, \grad{\phi_2} \in \setGFormula$ and $\grad{s} \in \setGStmt$.\\
    If $\gthoare{}{\grad{\phi_1}}{\grad{s}}{\grad{\phi_2}}$ then $\gtHoare{}{\grad{\phi_1}}{\grad{s}}{\grad{\phi_2}}$
\end{definition}

%% ...have nothing to do with gradual lifting
Note that $\gtHoare {~} {\cdot} {\cdot} {\cdot}$ is not a sound gradual lifting of $\tHoare{~}{\cdot}{\cdot}{\cdot}$.
A gradual lifting would declare the Hoare triple $\hoare{\qm}{\sVarAssign{x}{3}}{\phiEq{y}{4}}$ valid due to the existence of a valid concretization, e.g. $\hoare{\phiEq{y}{4}}{\sVarAssign{x}{3}}{\phiEq{y}{4}}$).
However, this triple is clearly not valid semantically since the postcondition is not guaranteed for all executions satisfying the precondition ($\qm$ is always satisfied).
Recall that gradual lifting was introduced in order to comply with the expectations a programmer would have when using a gradual verification system.
The validity predicate $\gtHoare {} {\cdot} {\cdot} {\cdot}$ on the other hand is defined semantically and therefore not affected by the reasoning behind gradual lifting.

%% clash
Unfortunately, the different concepts collide in gradual soundness as defined in definition \ref{def:gsnd-i}.
On the one hand gradual Hoare logic must comply with the gradual guarantee and thus verify $\gthoare{~}{\qm}{\sVarAssign{x}{3}}{\phiEq{y}{4}}$.
On the other hand $\gtHoare{~}{\qm}{\sVarAssign{x}{3}}{\phiEq{y}{4}}$ does not hold since the Hoare triple is not semantically valid.
Gradual soundness is therefore unsatisfiable if formalized as above.

%% moral: we need dynamic checks
This conflict is nothing but a reminder that gradualization is not for free, but may require runtime checks in order to be sound.
With this in mind we can revise our definition of soundness.

We extend the set of final states $\setGProgramStateFin$ with a designated exceptional state $\pi_{EX}$, representing failure of a runtime check.
Furthermore we introduce an assertion statement $\sAssert{$\grad{\phi}$}$ (if not already available) that throws an exception, i.e. steps to $\pi_{EX}$ should the condition not hold.
\begin{example}{Gradual Small-Step Semantics for Runtime Assertion}~\\
    \label{ex:ss-ra}
    We assume that $\setGProgramState = (\setVar \rightharpoonup \mathbb{Z}) \times \setGStmt$ as introduced in example \ref{ex:grad-ps}. %\ref{ex:ps-primitive}.
    Furthermore we assume that there exists a no-operation statement $\sSkip$.
    \begin{mathpar}
        \inferrule* [right=\gradT SsAssert~~~~]
        {
            \evalgphiGen{\langle \sigma, \sAssert{$\grad{\phi}$} \rangle}{\grad{\phi}}
        }
        {
            \gsstep{\langle \sigma, \sAssert{$\grad{\phi}$} \rangle}{\langle \sigma, \sSkip \rangle}
        }
        \inferrule* [right=\gradT SsAssertEx]
        {
            \neg~ \evalgphiGen{\langle \sigma, \sAssert{$\grad{\phi}$} \rangle}{\grad{\phi}}
        }
        {
            \gsstep{\langle \sigma, \sAssert{$\grad{\phi}$} \rangle}{\pi_{EX}}
        }
    \end{mathpar}
    
    See lemma \ref{ex:opt-lift-evalphi} for definition of $\evalgphiGen{\cdot}{\cdot}$.
\end{example}

Soundness can then be redefined as
\begin{definition}[\gradT Soundness]\label{def:gsnd}~\\
    Let $\grad{\phi_1}, \grad{\phi_2} \in \setGFormula$ and $\grad{s} \in \setGStmt$.\\
    If $\gthoare{}{\grad{\phi_1}}{\grad{s}}{\grad{\phi_2}}$ then $\gtHoare{}{\grad{\phi_1}}{\sSeq{$\grad{s}$}{\sAssert{$\grad{\phi_2}$}}}{\grad{\phi_2}}$
\end{definition}
Intuitively, this definition states that an assertion is inserted during verification, whenever gradual Hoare logic is used.
Such a system can be viewed as a “safe baseline” and corresponds to a strong dynamic type system: 
Program states are only checked against their expectations as lately/lazily as possible (like Scheme or Python check values for their runtime type if a corresponding function is called).
It may be desirable to check for inconsistencies more eagerly in order to terminate programs with unjustifiable behavior earlier or even as early as possible, like the “evidence” approach introduced by AGT \cite{garcia2016abstracting}.
In section \ref{ssec:the-deterministic-approach} we will derive an approach that ultimately allows us to move towards more eager checking.

\begin{lemma}[\tset{\gradT Soundness} Tautology]
    \label{lemma:tauto}~\\
    \tset{\gradT Soundness} is a tautology. 
\end{lemma}

That \tset{\gradT Soundness} is a tautology makes sense, realizing that (without demanding any degree of optimality) $\gthoare{}{\cdot}{\cdot}{\cdot}$ carries zero information.
We will address this issue later (section \ref{ssec:the-deterministic-approach}), also rethinking the way we obtain $\gthoare{}{\cdot}{\cdot}{\cdot}$ in the first place.

Note that there is room for an optimized implementation of the injected assertions.
\begin{example}{Non-Optimal Runtime Check Insertion}\label{ex:non-opt-rac-injection}~\\
    Deducing 
    $$\gthoare {} {\qm} {\sVarAssign{y}{2}} {\phiAnd{\phiEq{x}{3}}{\phiEq{y}{2}}}$$
    would lead to the injection of
    $$\sAssert{\phiAnd{\phiEq{x}{3}}{\phiEq{y}{2}}}$$
    in order to make
    $$\gtHoare {} {\qm} {\sSeq{\sVarAssign{y}{2}}{\sAssert{\phiAnd{\phiEq{x}{3}}{\phiEq{y}{2}}}}} {\phiAnd{\phiEq{x}{3}}{\phiEq{y}{2}}}$$
    hold.
    However, 
    $$\gtHoare {} {\qm} {\sSeq{\sVarAssign{y}{2}}{\sAssert{\phiEq{x}{3}}}} {\phiAnd{\phiEq{x}{3}}{\phiEq{y}{2}}}$$
    holds as well since $\phiEq{y}{2}$ is implied by the assignment.
\end{example}
Sometimes it is possible to reduce (or even remove) the assertion and therefore the runtime overhead associated with it.
We will address this observation later.

% address fact that checking that is expensive => would be cool to have static guaranteed knowledge to reduce checking => deterministic lifting...