% TODO: at some point define what statically verified (verification as part of static semantics, no compilation without verification success) vs dynamically verified (no static measures, only (implicit) runtime cecks) means!

%% why start with static language
As illustrated earlier %MAKE SURE!
gradual verification can be seen as an extension of both static and dynamic verification.
% Both can be seen as the endpoints of the continuum...?
Yet, our approach of “gradualization” formalizes the introduction of the dynamic aspect into a fully static system.
% This chapter describes this approach 
Later %TODO: ref
we will show how a programming language without static verification features can be approached.

%% structure of this chapter
% TODO

% TODO: overall approach (lift Hoare logic, then think about the places where lifted semantics reflect back into the system, e.g. function call)
% reference implementation chapter, explain difference (here: general, there: making use of specifics, optimality, minimal runtime overhead (0 if static, ...) ...)