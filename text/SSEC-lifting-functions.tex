% assuming total for now (otherwise: split partial function into total function and definedness predicate)

Static verification rules may contain functions manipulating formulas. % EX
We can also derive rules for lifting such functions from the gradual guarantee.
In this section, we assume that we are dealing with a function $f : \setFormula \rightarrow \setFormula$.
Again, the concepts are directly applicable to functions with higher arity.

Restrictions imposed by the gradual guarantee:
\begin{description}
    \item[Introduction of Gradual System]~\\
    We ensure that our verification system is “immune” to reduction of precision.
    Thus, when passing a static formula $\phi$ to $\grad{f}$, the result must be the same or less precise than $f(\phi)$.
    \begin{displaymath}
    \forall \phi \in \setFormula.~ f(\phi) \sqsubseteq \grad{f}(\phi)
    \end{displaymath}
    %Equivalently:
    %\begin{displaymath}
    %\forall \phi \in \setFormula.~ f(\phi) \in \gamma(\grad{f}(\phi))
    %\end{displaymath}
    
    \item[Monotonicity]~\\
    Reducing precision of a parameter, may only result in a loss of precision of the result.
    In other words, the function must be monotonic w.r.t. $\sqsubseteq$ (in every argument).
    
    \begin{displaymath}
    \forall \grad{\phi_1}, \grad{\phi_2} \in \setGFormula.~ \grad{\phi_1} \sqsubseteq \grad{\phi_2} \implies \grad{f}(\grad{\phi_1}) \sqsubseteq \grad{f}(\grad{\phi_2})
    \end{displaymath}
\end{description}


Holds (but prob. unimportant...)
\begin{align*}
\forall \phi \in \setFormula.~ &f(\phi) \sqsubseteq \grad{f}(\phi)\\
~\wedge~
\forall \grad{\phi_1}, \grad{\phi_2} \in \setGFormula.~ &\grad{\phi_1} \sqsubseteq \grad{\phi_2} \implies \grad{f}(\grad{\phi_1}) \sqsubseteq \grad{f}(\grad{\phi_2})\\
\implies\\
\forall \grad{\phi} \in \setGFormula, \phi \in \setFormula.~ &\phi \sqsubseteq \grad{\phi} \implies f(\phi) \sqsubseteq \grad{f}(\grad{\phi})\\
\end{align*}


% TODO: Partiality same as predicates + monotonicity in terms of \sqsubseteq???
PARTIAL:

\begin{displaymath}
\forall \phi \in \setFormula \cap \dom(f).~ f(\phi) \sqsubseteq \grad{f}(\phi)
\end{displaymath}

\begin{displaymath}
\forall \grad{\phi_1}, \grad{\phi_2} \in \setGFormula.~ \grad{\phi_1} \sqsubseteq \grad{\phi_2} \wedge \grad{\phi_1} \in \dom(\grad{f}) \implies \grad{f}(\grad{\phi_1}) \sqsubseteq \grad{f}(\grad{\phi_2})
\end{displaymath}


% mention alpha(...), galois connection (does not always exist, make example... so partial GC instead (reference)...)

% function composition (soundness, optimality?), ...