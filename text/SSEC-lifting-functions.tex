% assuming total for now (otherwise: split partial function into total function and definedness predicate)

Static verification rules may contain functions manipulating formulas. % EX
We can also derive rules for lifting such functions from the gradual guarantee.
In this section, we assume that we are dealing with a function $f : \setFormula \rightarrow \setFormula$.
Again, the concepts are directly applicable to functions with higher arity.

Restrictions imposed by the gradual guarantee:
\begin{description}
    \item[Introduction]~\\
    We ensure that our verification system is “immune” to reduction of precision.
    Thus, when passing a static formula $\phi$ to $\grad{f}$, the result must be the same or less precise than $f(\phi)$.
    \begin{displaymath}
    \forall \phi \in \setFormula.~ f(\phi) \sqsubseteq \grad{f}(\phi)
    \end{displaymath}
    %Equivalently:
    %\begin{displaymath}
    %\forall \phi \in \setFormula.~ f(\phi) \in \gamma(\grad{f}(\phi))
    %\end{displaymath}
    
    \item[Monotonicity]~\\
    Reducing precision of a parameter may only result in a loss of precision of the result.
    In other words, the function must be monotonic w.r.t. $\sqsubseteq$ (in every argument).
    
    \begin{displaymath}
    \forall \grad{\phi_1}, \grad{\phi_2} \in \setGFormula.~ 
    \grad{\phi_1} \sqsubseteq \grad{\phi_2} 
    \implies 
    \grad{f}(\grad{\phi_1}) \sqsubseteq \grad{f}(\grad{\phi_2})
    \end{displaymath}
\end{description}

\begin{definition}[Sound Function Lifting]
    A lifted function is \textbf{sound/valid} if it adheres to the above rules.
\end{definition}

Note that the rules for sound lifting only give a lower bound for the gradual return values.
Thus a function $\grad{f} : \setGFormula \rightarrow \setGFormula$ constantly returning $\qm$ is a sound lifting of any function $f : \setFormula \rightarrow \setFormula$.

\begin{definition}[Optimal Function Lifting]
    A sound lifted function is \textbf{optimal} if its return values are as precise as possible.
    %TODO: formally?
\end{definition}

This definition coincides with the definition of consistent function lifting in AGT:

\begin{lemma}[Optimal Function Lifting (Direct Definition)]\label{lemma:consistent-func-lifting-direct}~\\
    Let $\alpha : \PP(\setFormulaA) \rightharpoonup \setGFormula$ be a partial function such that $\langle \gamma, \alpha \rangle$ is a $\{ f \}$-partial Galois connection.\\ 
    Let $\grad{f} : \setGFormula \rightarrow \setGFormula$ be defined as
    \begin{displaymath}
    \grad{f}(\grad{\phi}) \defeq \alpha(\overline{f}(\gamma(\grad{\phi})))
    \end{displaymath}
    Then $\grad{f}$ is an optimal lifting of $f$.
\end{lemma} %PROOF




\textbf{Examples}

\begin{displaymath}
\alpha(\overline{\phi}) = \min_{\sqsubseteq} {\{~ \grad{\phi} ~|~ \overline{\phi} \subseteq \gamma(\grad{\phi}) ~\}}
\end{displaymath}

The logical and operator $\phiAnd{$\cdot$}{$\cdot$}$ of our formula syntax can be viewed as a binary function on formulas.
\begin{align*}
\grad{f}(\grad{\phi_1}, \grad{\phi_2}) = \alpha(\{~ \phiAnd{$\phi_1$}{$\phi_2$} ~|~ \phi_1 \in \gamma(\grad{\phi_1}) \wedge \phi_2 \in \gamma(\grad{\phi_2}) ~\})
\end{align*}



%Holds (but prob. unimportant...)
%\begin{align*}
%\forall \phi \in \setFormula.~ &f(\phi) \sqsubseteq \grad{f}(\phi)\\
%~\wedge~
%\forall \grad{\phi_1}, \grad{\phi_2} \in \setGFormula.~ &\grad{\phi_1} \sqsubseteq \grad{\phi_2} \implies \grad{f}(\grad{\phi_1}) \sqsubseteq \grad{f}(\grad{\phi_2})\\
%\implies\\
%\forall \grad{\phi} \in \setGFormula, \phi \in \setFormula.~ &\phi \sqsubseteq \grad{\phi} \implies f(\phi) \sqsubseteq \grad{f}(\grad{\phi})\\
%\end{align*}


% TODO: Partiality same as predicates + monotonicity in terms of \sqsubseteq???
PARTIAL:

\begin{displaymath}
\forall \phi \in \setFormula \cap \dom(f).~ f(\phi) \sqsubseteq \grad{f}(\phi)
\end{displaymath}

\begin{displaymath}
\forall \grad{\phi_1}, \grad{\phi_2} \in \setGFormula.~ \grad{\phi_1} \sqsubseteq \grad{\phi_2} \wedge \grad{\phi_1} \in \dom(\grad{f}) \implies \grad{f}(\grad{\phi_1}) \sqsubseteq \grad{f}(\grad{\phi_2})
\end{displaymath}


% mention alpha(...), galois connection (does not always exist, make example... so partial GC instead (reference)...)

% function composition (soundness, optimality?), ...