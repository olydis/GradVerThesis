% assuming total for now (otherwise: split partial function into total function and definedness predicate)

In this section, we assume that we are dealing with a \emph{total} function $f : \setFormula \rightarrow \setFormula$.
Partial functions are dealt with in section \ref{sssec:lifting-partial-functions}.

The following concepts are directly applicable to functions with higher arity.

\begin{description}
    \item[Introduction]~\\
    With predicate lifting we made sure to design a gradual verification system that is “immune” to reduction of precision.
    Therefore, when replacing function $f$ with its gradual lifting $\grad{f}$, we expect the result to be the same or less precise.
    \begin{displaymath}
    \forall \phi \in \setFormula.~ f(\phi) \sqsubseteq \grad{f}(\phi)
    \end{displaymath}
    %Equivalently:
    %\begin{displaymath}
    %\forall \phi \in \setFormula.~ f(\phi) \in \gamma(\grad{f}(\phi))
    %\end{displaymath}
    
    \item[Monotonicity]~\\
    Reducing precision of a parameter may only result in a loss of precision of the result.
    In other words, the function must be monotonic w.r.t. $\sqsubseteq$.
    
    \begin{displaymath}
    \forall \grad{\phi_1}, \grad{\phi_2} \in \setGFormula.~ 
    \grad{\phi_1} \sqsubseteq \grad{\phi_2} 
    \implies 
    \grad{f}(\grad{\phi_1}) \sqsubseteq \grad{f}(\grad{\phi_2})
    \end{displaymath}
\end{description}

\begin{definition}[Sound Function Lifting]
    A lifted function is \textbf{sound} if it adheres to the above rules.
\end{definition}

Note that the rules for sound lifting only give a lower bound for the gradual return values.
Thus a function $\grad{f} : \setGFormula \rightarrow \setGFormula$ constantly returning $\qm$ is a sound lifting of any function $f : \setFormula \rightarrow \setFormula$.
This observation motivates an additional notion of optimality.

\begin{definition}[Optimal Function Lifting]
    A sound lifted function is \textbf{optimal} if its return values are at least as precise as the return values of any other sound lifted function.
\end{definition}

%% AGT
Again, definition of optimal function lifting coincides with the definition of “consistent function lifting” given by AGT.
\begin{lemma}[Equivalence with Consistent Function Lifting (AGT)]\label{lemma:consistent-func-lifting-direct}~\\
    \label{lemma:eq-fun-lift-agt}
    Let $\alpha : \PP(\setFormula) \rightharpoonup \setGFormula$ be a partial function such that $\langle \gamma, \alpha \rangle$ is a $\{ \overline{f} \}$-partial Galois connection (see appendix \ref{app:part-galois}, definition \ref{def:pgc}).\\ 
    Let $\grad{f} : \setGFormula \rightarrow \setGFormula$ be defined as
    \begin{displaymath}
    \grad{f}(\grad{\phi}) \defeq \alpha(\overline{f}(\gamma(\grad{\phi})))
    \end{displaymath}
    where $\overline{f}$ means that $f$ is applied to every element of the set.
    Then $\grad{f}$ is an optimal lifting of $f$.
\end{lemma}
\begin{proof}~
    \begin{description}
        \item[Adjoint Equation] 
        \begin{align*}
        \alpha(\overline{f}(\gamma(\phi))) = f(\phi)
        \end{align*}
        Proof:
        
        $\alpha(\overline{f}(\gamma(\phi)))$ defined, since $\{ \overline{f} \}$-partial Galois connection, i.e.
        \begin{align}
        \label{frm:pgc-ass}
        \alpha(\overline{f}(\gamma(\phi))) = \alpha(\{ f(\phi) \}) = \grad{\phi}
        \end{align}
        Applying rule 1 of partial Galois connections to \ref{frm:pgc-ass}
        \begin{align}
        \label{frm:pgc-ass1}
        &\{~ f(\phi) ~\} \subseteq \gamma(\grad{\phi})\\
        \end{align}
        Applying rule 2 of partial Galois connections to \ref{frm:pgc-ass}, using $\{~ f(\phi) ~\} \subseteq \gamma(f(\phi))$
        \begin{align}
        \label{frm:pgc-ass2}
        & \grad{\phi} \sqsubseteq f(\phi)
        \end{align}
        
        Combining \ref{frm:pgc-ass1} and \ref{frm:pgc-ass2}
        \begin{align*}
        &\{~ f(\phi) ~\} \subseteq \gamma(\grad{\phi}) \subseteq \gamma(f(\phi))\\
        \implies
        & \gamma(\grad{\phi}) = \{~ f(\phi) ~\}\\
        \implies
        & \grad{\phi} = f(\phi)
        \end{align*}
        
        \item[Soundness]~
        \begin{description}
            \item[Introduction] 
            \begin{align*}
            &\grad{f}(\phi)\\
            =~
            &\alpha(\overline{f}(\gamma(\phi)))\\
            =~ 
            &f(\phi)
            \end{align*}
            
            \item[Monotonicity]~\\ 
            We assume $\grad{\phi_1}, \grad{\phi_2} \in \setGFormula$ with $\grad{\phi_1} \mpt \grad{\phi_2}$
            \begin{align*}
            &\grad{\phi_1} \mpt \grad{\phi_2}\\
            \implies
            &\gamma(\grad{\phi_1}) \subseteq \gamma(\grad{\phi_2})\\
            \implies
            &\overline{f}(\gamma(\grad{\phi_1})) \subseteq \overline{f}(\gamma(\grad{\phi_2}))\\
            \implies % closure
            &\overline{f}(\gamma(\grad{\phi_1})) \subseteq \gamma(\alpha(\overline{f}(\gamma(\grad{\phi_2}))))\\
            \implies % rule 2
            &\alpha(\overline{f}(\gamma(\grad{\phi_1}))) \mpt \alpha(\overline{f}(\gamma(\grad{\phi_2})))\\
            \end{align*}
        \end{description}
        
        \item[Optimality]~\\
        
        Proof by contradiction.
        Assume there exists a sound lifting $\grad{f'}$ such that $\grad{f'}(\grad{\phi}) \sqsubset \grad{f}(\grad{\phi})$ for some $\grad{\phi} \in \setGFormula$.
        Using the introduction rule:
        \begin{align*}
        &\forall \phi \in \setFormula.~ f(\phi) \mpt \grad{f'}(\phi)
        \end{align*}
        Using the monotonicity rule:
        \begin{align*}
        &\forall \phi \in \gamma(\grad{\phi}).~ \grad{f'}(\phi) \mpt \grad{f'}(\grad{\phi})
        \end{align*}
        Transitivity:
        \begin{align*}
        &\forall \phi \in \gamma(\grad{\phi}).~ f(\phi) \mpt \grad{f'}(\grad{\phi})\\
        \implies
        &\forall \phi \in \gamma(\grad{\phi}).~ f(\phi) \in \gamma(\grad{f'}(\grad{\phi}))\\
        \implies
        &\overline{f}(\gamma(\grad{\phi})) \subseteq \gamma(\grad{f'}(\grad{\phi}))\\
        \end{align*}
        Using rule 2 of partial Galois connections
        \begin{align*}
        &\overline{f}(\gamma(\grad{\phi})) \subseteq \gamma(\grad{f'}(\grad{\phi}))\\
        \implies
        &\alpha(\overline{f}(\gamma(\grad{\phi}))) \mpt \grad{f'}(\grad{\phi})\\
        \implies
        &\grad{f}(\grad{\phi}) \mpt \grad{f'}(\grad{\phi})\\
        \end{align*}
        Contradiction.
    \end{description}       
\end{proof}


%\begin{displaymath}
%\alpha(\overline{\phi}) = \min_{\sqsubseteq} {\{~ \grad{\phi} ~|~ \overline{\phi} \subseteq \gamma(\grad{\phi}) ~\}}
%\end{displaymath}
