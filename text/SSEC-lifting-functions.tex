% assuming total for now (otherwise: split partial function into total function and definedness predicate)

In this section, we assume that we are dealing with a total function $f : \setFormula \rightarrow \setFormula$.
The concepts are directly applicable to functions with higher arity.

\begin{description}
    \item[Introduction]~\\
    By making sure to comply with the gradual guarantee, we made design a gradual verification system immune to reduction of precision at any stage.
    Therefore, when replacing function $f$ with its gradual lifting $\grad{f}$, we expect the result to be the same or less precise.
    \begin{displaymath}
    \forall \phi \in \setFormula.~ f(\phi) \sqsubseteq \grad{f}(\phi)
    \end{displaymath}
    %Equivalently:
    %\begin{displaymath}
    %\forall \phi \in \setFormula.~ f(\phi) \in \gamma(\grad{f}(\phi))
    %\end{displaymath}
    
    \item[Monotonicity]~\\
    Reducing precision of a parameter may only result in a loss of precision of the result.
    In other words, the function must be monotonic w.r.t. $\sqsubseteq$.
    
    \begin{displaymath}
    \forall \grad{\phi_1}, \grad{\phi_2} \in \setGFormula.~ 
    \grad{\phi_1} \sqsubseteq \grad{\phi_2} 
    \implies 
    \grad{f}(\grad{\phi_1}) \sqsubseteq \grad{f}(\grad{\phi_2})
    \end{displaymath}
\end{description}

\begin{definition}[Sound Function Lifting]
    A lifted function is \textbf{sound/valid} if it adheres to the above rules.
\end{definition}

Note that the rules for sound lifting only give a lower bound for the gradual return values.
Thus a function $\grad{f} : \setGFormula \rightarrow \setGFormula$ constantly returning $\qm$ is a sound lifting of any function $f : \setFormula \rightarrow \setFormula$.
This observation motivates an additional notion of optimality.

\begin{definition}[Optimal Function Lifting]
    A sound lifted function is \textbf{optimal} if its return values are at least as precise as the return values of any other sound lifted function.
\end{definition}

%% AGT
Again, definition of optimal function lifting coincides with the definition of “consistent function lifting” given by AGT.
\begin{lemma}[Equivalence with Consistent Function Lifting (AGT)]\label{lemma:consistent-func-lifting-direct}~\\
    Let $\alpha : \PP(\setFormulaA) \rightharpoonup \setGFormula$ be a partial function such that $\langle \gamma, \alpha \rangle$ is a $\{ f \}$-partial Galois connection.\\ 
    Let $\grad{f} : \setGFormula \rightarrow \setGFormula$ be defined as
    \begin{displaymath}
    \grad{f}(\grad{\phi}) \defeq \alpha(\overline{f}(\gamma(\grad{\phi})))
    \end{displaymath}
    where $\overline{f}$ means that $f$ is applied to every element of the set.
    Then $\grad{f}$ is an optimal lifting of $f$.
\end{lemma} %PROOF (hard part by contradiction! :-)

%\begin{displaymath}
%\alpha(\overline{\phi}) = \min_{\sqsubseteq} {\{~ \grad{\phi} ~|~ \overline{\phi} \subseteq \gamma(\grad{\phi}) ~\}}
%\end{displaymath}
