In lemma \ref{lemma:consistent-func-lifting-direct} we mentioned partial Galois connections.
In this section, we give a formal definition and motivate why we used them instead of regular Galois connections.

%% def
The following definition originated from Miné \cite{mine2004weakly}, ... ??? nico, adapted to our setting.
\begin{definition}[Partial Galois connection]\label{def:pgc} ~\\
    Let $(\PP^{\setFormula}, \subseteq)$ and $(\setGFormula, \mpt)$ be two posets, $\mathcal{F}$ a set of operators on $\PP^{\setFormula}$, $\alpha : \PP^{\setFormula} \rightharpoonup \setGFormula$ a partial function and $\gamma : \setGFormula \rightarrow \PP^{\setFormula}$ a total function. 
    
    The pair $\langle \alpha, \gamma \rangle$ is an $\mathcal{F}$-partial Galois connection if and only if:
    \begin{enumerate}
        \item If $\alpha(\overline{\phi})$ is defined, then $\overline{\phi} \subseteq \gamma(\alpha(\overline{\phi}))$
        \item If $\alpha(\overline{\phi})$ is defined, then $\overline{\phi} \subseteq \gamma(\grad{\phi})$ implies $\alpha(\overline{\phi}) \mpt \grad{\phi}$
        \item For all $F \in \mathcal{F}$ and $\grad{\phi} \in \setGFormula$, $\alpha(F(\gamma(\grad{\phi})))$
    \end{enumerate}
This definition can be generalized for a set $\mathcal{F}$ of arbitrary $n$-ary operators.
\end{definition}

%% motivation
In our work we use partial Galois connections because a traditional Galois connection between $\setGFormula$ and $\PP^{\setFormula}$ may not always exist (and does not exist in case of \gvlidf, as introduced in chapter \ref{ch:case-study--implicit}).
\begin{lemma}[Non-Existence of Galois Connection in \gvlidf]~\\
    \label{lemma:gc-nonex}
    Assume that $\setFormula$, $\setGFormula$ and $\gamma$ are defined as in \svlidf/\gvlidf (see chapter \ref{ch:case-study--implicit}).
    Then there exists no $\alpha$ such that $\langle \alpha, \gamma \rangle$ is a Galois connection.
\end{lemma}

Partial Galois connections are more flexible by also giving a (sufficiently strong) meaning to partial abstraction functions $\alpha$.
%In our setting, it requires that $\alpha$ is defined only for sets of formulas that emerged from concretization and subsequent static reasoning.



