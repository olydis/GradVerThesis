%% intro
Formulas play a role for some statements, extending their syntax may thus also affect the syntax of statements.
%% assert
A common example where formulas are even part of the syntax is the assertion statement $\sAssert{$\phi$}$.
Having a gradual formula syntax available does not necessary mean that all statements have to adopt it.
In case of the assertion statement there might be little benefit in allowing gradual formulas.
% TODO? say why, say that it is nonsense for the syntax extensions we gave but NOT for the one in the implementation

%% call
A more complex example affected by gradualization of formulas is a call statement $\ttt{$m$();}$ in presence of method contracts.
Although not directly visible, this statement's semantics (static and dynamic) is affected by the contract of $m$, consisting of pre- and postcondition.
One can think of $m$ as a reference to some method definition including method contract.
Note that in practice such method definitions usually reside in some “program context” that is then passed to static and dynamic semantics.
As the full meaning of such a statement is unknown without context, it is hard to reason about it abstractly.
W.l.o.g. we will thus think of $m$ as syntactic sugar for 
\begin{align*}
&\ttt{assert $\phi_{m_{pre}}$;}\\
&\ttt{// body of $m$}\\
&\ttt{assume $\phi_{m_{post}}$;}
\end{align*}

%%% contract
As one of the main goals of gradual verification is to allow for gradual method contracts, it makes sense to extend the syntax accordingly.
This means that the syntax of our desugared call statement is affected:
\begin{align*}
&\ttt{assert $\grad{\phi_{m_{pre}}}$;}\\
&\ttt{// body of $m$}\\
&\ttt{assume $\grad{\phi_{m_{post}}}$;}
\end{align*}

%% general
In general, statement syntax is extended, resulting in a superset $\setGStmt \supseteq \setStmt$ of gradual statements.
Note that the superset is induced merely by allowing $\setGFormula$ instead of $\setFormula$ in certain places (chosen freely by the gradual language designer).
We give meaning to gradual statements using a concretization function. % MORE detail?
\begin{definition}[Concretization of Gradual Statements]
    Let $\gamma_s : \setGStmt \rightarrow \PP(\setStmt)$ be defined as
    \begin{displaymath}
    \gamma_s(\grad{s}) = \{~ s \in \setStmt ~|~ \textit{$s$ is $\grad{s}$ with all gradual formulas replaced by some concretizations} ~\}
    \end{displaymath}
\end{definition}
\begin{definition}[Precision of Gradual Statement]
    Let $\mpts \subseteq \setGStmt \times \setGStmt$ be a predicate defined as
    $$\grad{s_a} \mpts \grad{s_b}  \quad\iff\quad  \gamma_s(\grad{s_a}) \subseteq \gamma_s(\grad{s_b})$$
\end{definition}

%% outlook
The notion of gradual statements will become important for gradual static and dynamic semantics.
