%% intro
Formulas play a role in some statements, so extending their syntax may also affect the syntax of statements.
A common example are assertion statements $\sAssert{$\phi$}$, which can now be extended to $\sAssert{$\grad{\phi}$}$.
Note however, that having a gradual formula syntax available does not necessarily mean that all statements \emph{have} to adopt it.
% Examples can be found in chapter \ref{ch:case-study--implicit} which derives a gradual language \gvlidf that .

%% call
A more complex example affected by gradualization of formulas is a call statement $\ttt{$m$()}$ in presence of method contracts.
\begin{example}{Gradual Call Statement}~\\
    Although not directly visible, the (static and dynamic) semantics of a call statement is affected by the contract of $m$, consisting of pre- and postcondition.
    One can think of $m$ as a reference to or syntactic sugar for some method definition including a method contract.
    Note that in practice such method definitions usually reside in a “program context” that the static and dynamic semantics are parameterized with.
    
    As the full meaning of such a statement is unknown without context, it is hard to reason about it abstractly.
    W.l.o.g. we will thus think of $m$ as syntactic sugar for 
    \begin{align*}
    &\ttt{assert $\phi_{m_{pre}}$;}\\
    &\ttt{// body of $m$}\\
    &\ttt{assume $\phi_{m_{post}}$;}
    \end{align*}
    
    %%% contract
    As one of the main goals of gradual verification is to allow for gradual method contracts, it makes sense to extend the syntax accordingly.
    This means that the syntax of the desugared call statement is affected:
    \begin{align*}
    &\ttt{assert $\grad{\phi_{m_{pre}}}$;}\\
    &\ttt{// body of $m$}\\
    &\ttt{assume $\grad{\phi_{m_{post}}}$;}
    \end{align*}
\end{example}

%% general
In general, the statement syntax is extended, resulting in a superset $\setGStmt \supseteq \setStmt$ of gradual statements.
Note that the superset is induced merely by allowing $\setGFormula$ instead of $\setFormula$ in certain places (chosen by the gradual language designer).
We give meaning to gradual statements using a concretization function. % MORE detail?
\begin{definition}[Concretization of Gradual Statements]
    Let $\gamma_s : \setGStmt \rightarrow \PP(\setStmt)$ be defined as
    \begin{displaymath}
    \gamma_s(\grad{s}) = \{~ s \in \setStmt ~|~ \textit{$s$ is $\grad{s}$ with all gradual formulas replaced by some concretizations} ~\}
    \end{displaymath}
\end{definition}
\begin{definition}[Precision of Gradual Statement]
    Let $\cdot \mpts \cdot \subseteq \setGStmt \times \setGStmt$ be a predicate defined as
    $$\grad{s_a} \mpts \grad{s_b}  \quad\iff\quad  \gamma_s(\grad{s_a}) \subseteq \gamma_s(\grad{s_b})$$
\end{definition}

%% outlook
The notion of gradual statements will become important for the gradualized semantics of \gvl.
