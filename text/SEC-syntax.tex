Figure \ref{fig:idf-syntax} shows the full syntax of \svlidf.
\begin{figure}[h]
    \newcommand{\tempStmtA}{\sSkip
                    ~|~ \sDeclare {$T$} {$x$}
                    ~|~ \sFieldAssign {$x$} {$f$} {$y$} 
                    ~|~ \sVarAssign {$x$} {$e$}
                    ~|~ \sAlloc {$x$} {$C$} 
                    ~|~ \sCall {$x$} {$y$} {$m$} {$z$}}
\newcommand{\tempStmtB}{~~~ ~|~ \sReturn {$x$}  
                            ~|~ \sAssert {$\phi$} 
                            ~|~ \sRelease {$\phi$} 
                            ~|~ \sHold {$\phi$} {$s$}
                            ~|~ \sSeq {$s_1$} {$s_2$}}
\newcommand{\tempFrm}{  \phiTrue 
                    ~|~ \phiEq {$e$} {$e$} 
                    ~|~ \phiNeq {$e$} {$e$}
                    ~|~ \phiAcc {$e$} {$f$}
                    ~|~ \phiCons {$\phi$} {$\phi$}}
\newcommand{\tempExpr}{ \ev{$v$}
                    ~|~ \ex{$x$}
                    ~|~ \edot{$e$}{$f$}}

\begin{align*}
	program  & \in \setProgram    &  & ::= \ttt{$\overline{cls}$~$s$}                              \\
	cls      & \in \setClass      &  & ::= \class {$C$} {$\overline{field}$} {$\overline{method}$} \\
	field    & \in \setField      &  & ::= \field {$T$} {$f$}                                      \\
	method   & \in \setMethod     &  & ::= \method {$T$} {$m$} {$T$} {$x$} {$contract$} {$s$}      \\
	contract & \in \setContract   &  & ::= \contract{$\phi$}{$\phi$}                              \\
	T        & \in \setType       &  & ::= \Tint ~|~ C                                             \\
	s        & \in \setStmt       &  & ::= \tempStmtA                                              \\
	         &                    &  & \tempStmtB                                                  \\
	\phi     & \in \setFormula    &  & ::= \tempFrm                                                \\
	e        & \in \setExpr       &  & ::= \tempExpr                                               \\
	x, y, z  & \in \setVar        &  & ::= \ethis ~|~ \eresult ~|~ identifier                      \\
	v        & \in \setVal        &  & ::= o ~|~ n ~|~ \enull                                      \\
	o        & \in \setLoc        &  & \text{(infinite set of memory locations)}                   \\
	n        & \in \mathbb{Z}     &  &  \\
	C        & \in \setClassName  &  & ::= identifier                                              \\
	f        & \in \setFieldName  &  & ::= identifier                                              \\
	m        & \in \setMethodName &  & ::= identifier
\end{align*}
    \caption{\svlidf: Syntax}
    \label{fig:idf-syntax}
\end{figure}
% EXPLAIN what stuff (like hold, release) means!

%% parser rassoc, skip
We pose $\phiFalse \defeq \phiNeq{\enull}{\enull}$.
We define $\ttt{;}$ to be right-associative and assume that parsing a sequence of statements (e.g. method body) operates analogously, obviating the need for parenthesis.
Furthermore we assume that the parser terminates every sequence with $\sSkip$.
\begin{exmp}~
    \begin{lstlisting}
    ...
    {
        $s_1$;
        $s_2$;
        $s_3$;
    }
    \end{lstlisting}
    is parsed as
    
    \includegraphics[trim={3cm 3cm 3cm 3cm}, clip, width=6cm]{graphics/rightAssocSkip}
\end{exmp}
These assumptions highly simplify reasoning about statements.

%% helper methods
We define the following helper methods:
\begin{figure}[h]
    \begin{description}
    \item[Extraction]
    To extract elements from a given program $p \in \setProgram$ we define the following functions:
    \begin{flalign*}
    	 & \fieldType : \setClassName \times \setFieldName \rightharpoonup \setType                & ~ \\
    	 & \fieldType(C, f) = \text{type of field $f$ in class $C$ in $p$}                         &  \\
    	 & ~                                                                                       &  \\
    	 & \predicate{fields$_p$} : \setClassName \rightharpoonup \PP^{\setField}                  &  \\
    	 & \fields{C} = \text{field declarations of class $C$ in $p$}                              &  \\
    	 & ~                                                                                       &  \\
    	 & \predicate{method$_p$} : \setClassName \times \setMethodName \rightharpoonup \setMethod &  \\
    	 & \mmethod{C, m} = \text{declaration of method $m$ in class $C$ in $p$}                   &  \\
    	 & ~                                                                                       &  \\
    	 & \predicate{mpre$_p$} : \setClassName \times \setMethodName \rightharpoonup \setFormula  &  \\
    	 & \mpre{C, m} = \text{precondition of method $m$ in class $C$ in $p$}                    &  \\
    	 & ~                                                                                       &  \\
    	 & \predicate{mpost$_p$} : \setClassName \times \setMethodName \rightharpoonup \setFormula &  \\
    	 & \mpost{C, m} = \text{postcondition of method $m$ in class $C$ in $p$}                   &
    \end{flalign*}
    
    \item[Free Variables]~\\
    The semantics of \svlidf will sometimes reason about the free variables of expressions or formulas.
    
    Let $\FV : (\setExpr \cup \setFormula) \rightarrow \PP^{\setVar}$ be defined as
    \begin{alignat*}{3}
    	  & \FV(v)                            &  & = \emptyset                    & ~ \\
    	  & \FV(x)                            &  & = \{ x \}                      &  \\
    	  & \FV(\edot{$e$}{$f$})              &  & = \FV(e)                       &  \\
    	~ &  \\
    	  & \FV(\phiTrue)                     &  & = \emptyset                    &  \\
    	  & \FV(\phiEq{$e_1$}{$e_2$})         &  & = \FV(e_1) \cup \FV(e_2)       &  \\
    	  & \FV(\phiNeq{$e_1$}{$e_2$})        &  & = \FV(e_1) \cup \FV(e_2)       &  \\
    	  & \FV(\phiAcc{$e$}{$f$})            &  & = \FV(e)                       &  \\
    	  & \FV(\phiCons{$\phi_1$}{$\phi_1$}) &  & = \FV(\phi_1) \cup \FV(\phi_2) &
    \end{alignat*}
    
    \item[Default Value of Type]~\\
    \svlidf assigns default values to declared variables.
    
    Let $\predicate{defaultValue} : \setType \rightarrow \setVal$ be defined as
    \begin{alignat*}{3}
    	 & \defaultValue{\Tint} &  & = 0      & ~ \\
    	 & \defaultValue{$C$}   &  & = \enull &
    \end{alignat*}
    
    \item[Required Access]~\\
    Expressions mentioning fields are heap dependent and thus require access.
    To enable treating expressions in a uniform fashion, we define a pseudo accessibility-predicate which is also defined for expressions that do not mention fields.
    
    Let $\predicate{acc} : \setExpr \rightarrow \setFormula$ be defined as
    \begin{alignat*}{3}
    	 & \accFor{v}               &  & = \phiTrue          & ~ \\
    	 & \accFor{x}               &  & = \phiTrue          & ~ \\
    	 & \accFor{\edot{$e$}{$f$}} &  & = \phiAcc{$e$}{$f$} &
    \end{alignat*}
    
    \item[Preventing Writes]
    Under rare circumstances, overwriting a certain variable is not allowed in \svlidf.
    To reliably check whether a variable is written to by a statement, we define the following predicate.
    
    Let $\writesTo \subseteq \setStmt \times \setVar$ be defined inductively as
    %% Inductive writesTo
\begin{mathpar}
\inferrule* [Right=wtVarAssign]
{
    ~
}
{
    \writesTo({x}, {\sVarAssign {${x}$} {${e}$}})
}
\end{mathpar}

\begin{mathpar}
\inferrule* [Right=wtAlloc]
{
    ~
}
{
    \writesTo({x}, {\sAlloc {${x}$} {${C}$}})
}
\end{mathpar}

\begin{mathpar}
\inferrule* [Right=wtCall]
{
    ~
}
{
    \writesTo({x}, {\sCall {${x}$} {${y}$} {${m}$} {${z}$}})
}
\end{mathpar}

\begin{mathpar}
\inferrule* [Right=wtReturn]
{
    ~
}
{
    \writesTo({\eresult}, {\sReturn {${x}$}})
}
\end{mathpar}

\begin{mathpar}
\inferrule* [Right=wtDeclare]
{
    ~
}
{
    \writesTo({x}, {\sDeclare {${T}$} {${x}$}})
}
\end{mathpar}

\begin{mathpar}
\inferrule* [Right=wtHold]
{
    {s} \in {ss} \\
    \writesTo({x}, {s})
}
{
    \writesTo({x}, {\sHold {${p}$} {${ss}$}})
}
\end{mathpar}


    \begin{mathpar}
        \inferrule* [right=wtVarAssign]
        {
            ~
        }
        {
            \writesTo({x},\, {\sVarAssign {${x}$} {${e}$}})
        }
        
        \inferrule* [right=wtAlloc]
        {
            ~
        }
        {
            \writesTo({x},\, {\sAlloc {${x}$} {${C}$}})
        }
        
        \inferrule* [right=wtCall]
        {
            ~
        }
        {
            \writesTo({x},\, {\sCall {${x}$} {${y}$} {${m}$} {${z}$}})
        }
        
        \inferrule* [right=wtReturn]
        {
            ~
        }
        {
            \writesTo({\eresult},\, {\sReturn {${x}$}})
        }
        
        \inferrule* [right=wtDeclare]
        {
            ~
        }
        {
            \writesTo({x}, {\sDeclare {${T}$} {${x}$}})
        }
        
        \inferrule* [right=wtHold]
        {
            \writesTo({x},\, {s})
        }
        {
            \writesTo({x},\, {\sHold {${p}$} {${s}$}})
        }
        
        \inferrule* [right=wtSeq1]
        {
            \writesTo({x},\, {s_1})
        }
        {
            \writesTo({x},\, \sSeq{$s_1$}{$s_2$})
        }
        
        \inferrule* [right=wtSeq2]
        {
            \writesTo({x},\, {s_2})
        }
        {
            \writesTo({x},\, \sSeq{$s_1$}{$s_2$})
        }
    \end{mathpar}
\end{description}
    \caption{\svlidf: Helper Methods}
    \label{fig:idf-helpers}
\end{figure}



% Programs consist of classes and a main method, represented directly as the list of its instructions.
% TODO: introduce all the extraction predicates/functions!
