Proposed gradual Hoare logic is sound w.r.t. the small-step semantics if runtime checks are injected as proposed in section \ref{ssec:gradual-soundness} (see lemma \ref{lemma:tauto} for proof).
% BY construction, what about progress? \gsssem exceptional? THEOREM?

However, \gvlidf also complies with the stronger notion of soundness \tset{GDPreservation} introduced in section \ref{ssec:gradual-soundness}.
We will show how the small-step semantics $\sstep{\cdot}{\cdot}$ and $\gsstep{\cdot}{\cdot}$ realize the approach introduced in section \ref{ssec:atomic--knowledge}.

Specifically, we show that the semantics are complete and perfect w.r.t. all statements except declarations, sequences, method calls and hold.
For those remaining statements we prove by induction that \tset{GDPreservation} still holds.
The overall proof is a structural induction on statement $s$.

\begin{lemma}[Partial Completeness of \svlidf]
    \label{lemma:pc-idf}
    The semantics of \svlidf is complete (see definition \ref{def:completeness}) w.r.t. 
    \begin{itemize}
        \item $\sSkip$
        \item $\sFieldAssign {$x$} {$f$} {$y$} $
        \item $\sVarAssign {$x$} {$e$}$
        \item $\sAlloc {$x$} {$C$}$
        \item $\sReturn {$x$}$
        \item $\sAssert {$\phi$}$
        \item $\sRelease {$\phi$}$
    \end{itemize}
\end{lemma}
\begin{proof}
    Goal:
    \begin{flalign*}
    \forall \pi_1, \pi_2 \in \setProgramState.&~ \sstepConsume{s}{\pi_1}{\pi_2} \\
    \implies \exists \Gamma \in \setTypeEnv,\, \phi_1, \phi_2 \in \setFormula.&~ \thoare{\Gamma}{\phi_1}{s}{\phi_2} \wedge \evalphiGen{\pi_1}{\phi_1}
    \end{flalign*}
    for the statements $s$ mentioned in the lemma.
    
    \begin{description}
        \item[$\sSkip$]~\\
        $$\thoare{\Gamma}{\phi_1}{s}{\phi_2} \wedge \evalphiGen{\pi_1}{\phi_1}$$
        trivially holds for $\phi_1 = \phi_2 = \phiTrue$ and any $\Gamma$
        
        \item[$\sFieldAssign {$x$} {$f$} {$y$} $]~\\
        Known: $\sstepConsume{\sFieldAssign {$x$} {$f$} {$y$}}{\pi_1}{\pi_2}$ for some $x, y \in \setVar,\, f \in \setFieldName$ and $\pi_1, \pi_2 \in \setProgramState$.
        
        We chose 
        \begin{align*}
        &\phi_1 = \phiAcc{$x$}{$f$}\\
        &\phi_2 = \phiCons{\phiCons{\phiTrue}{\phiAcc{$x$}{$f$}}}{\phiCons{\phiNeq{x}{\enull}}{\phiEq{\edot{x}{f}}{y}}}
        \end{align*}
        Then $\thoare{\Gamma}{\phi_1}{\sFieldAssign {$x$} {$f$} {$y$}}{\phi_2}$ can be deduced using \tset{HFieldAssign} (for some $\Gamma$).
        Also $\evalphiGen{\pi_1}{\phi_1}$ holds by inversion of \tset{\dgradT SsFieldAssign}.
        
        \item[$\sVarAssign {$x$} {$e$}$]~\\
        \item[$\sAlloc {$x$} {$C$}$]~\\
        \item[$\sReturn {$x$}$]~\\
        \item[$\sAssert {$\phi$}$]~\\
        \item[$\sRelease {$\phi$}$]~\\
    \end{description}
\end{proof}

\begin{lemma}[Partial Semi-Optimality of \gvlidf]
    \label{lemma:pso-idf}
    The semantics of \gvlidf is semi-optimal (see definitions \ref{def:perf-stat} and \ref{def:perf-dyn}) w.r.t.
    \begin{itemize}
        \item $\sSkip$
        \item $\sFieldAssign {$x$} {$f$} {$y$} $
        \item $\sVarAssign {$x$} {$e$}$
        \item $\sAlloc {$x$} {$C$}$
        \item $\sReturn {$x$}$
        \item $\sAssert {$\phi$}$
        \item $\sRelease {$\phi$}$
    \end{itemize}
\end{lemma}

\begin{lemma}[\tset{GDPreservation} Induction Step for Declaration]
    \label{lemma:is-decl}
    Assume \tset{GDPreservation} holds for $s \in \setGStmt$.
    Then \tset{GDPreservation} holds for $\sSeq{\sDeclare{$T$}{$x$}}{$s$}$ (for all $T \in \setType,\, x \in \setVar$).
\end{lemma}
\begin{proof}
    Assumption:
    \begin{align}
    \dgthoare{\Gamma}{\grad{\phi}} {\sSeq {\sDeclare {${T}$} {${x}$}} {$s$}} {\grad{\phi'}}
    \end{align}
    Using inversion of rule $\tset{\dgradT HDeclare}$ we know that
    \begin{gather}
    \label{frm:tmpp1}
    {x} \not\in \dom{\Gamma} \\
    \label{frm:tmpp2}
    {x} \not \in {\FV(\grad{\phi})} \\
    \dgthoare {{\Gamma}, {x} : {T}} {\gphiCons{${\phiEq {${\ex{${x}$}}$} {${\ev{${\defaultValue{${T}$}}$}}$}}$}{$\grad{\phi}$}} {s} {\grad{\phi'}}
    \end{gather}
    Using the induction hypothesis we may apply \tset{GDPreservation} to $s$, deducing
    \begin{gather}
    \label{frm:tmpp4}
    \gtHoare {} {\gphiCons{${\phiEq {${\ex{${x}$}}$} {${\ev{${\defaultValue{${T}$}}$}}$}}$}{$\grad{\phi}$}} {s} {\grad{\phi'}}
    \end{gather}
    Furthermore we can show that
    \begin{gather}
    \label{frm:tmpp5}
    \gtHoare {} {\grad{\phi}} {\sDeclare {${T}$} {${x}$}} {\gphiCons{${\phiEq {${\ex{${x}$}}$} {${\ev{${\defaultValue{${T}$}}$}}$}}$}{$\grad{\phi}$}}
    \end{gather}
    holds (using \ref{frm:tmpp2}, find proof below).
    Using lemma ??? we can combine \ref{frm:tmpp4} and \ref{frm:tmpp5} to derive the goal:
    \begin{align}
    \gtHoare{}{\grad{\phi}} {\sSeq {\sDeclare {${T}$} {${x}$}} {$s$}} {\grad{\phi'}}
    \end{align}
    
    Proof for \ref{frm:tmpp5}:
    Assumptions:
    \begin{gather}
    \label{frm:tmpp7}
    \gsstepConsume{\sDeclare {${T}$} {${x}$}}{\pi_1}{\pi_2}\\
    \label{frm:tmpp8}
    \evalgphiGen {\pi_1} {\grad{\phi}}
    \end{gather}
    for some $\pi_1, \pi_2 \in \setGProgramState$.
    Goal:
    \begin{gather}
    \evalgphiGen {\pi_2} {\gphiCons{${\phiEq {${\ex{${x}$}}$} {${\ev{${\defaultValue{${T}$}}$}}$}}$}{$\grad{\phi}$}}
    \end{gather}
    Applying rule inversion (\tset{SsDeclare}) to \ref{frm:tmpp7} we can see that $\pi_2$ is like $\pi_1$, but with $x$ set to $\defaultValue{${T}$}$ (and the assertion consumed).
    It follows that
    \begin{gather}
    \evalgphiGen {\pi_2} {\phiEq {${\ex{${x}$}}$} {${\ev{${\defaultValue{${T}$}}$}}$}}
    \end{gather}
    Since we know that $\grad{\phi}$ does not contain $x$ (see \ref{frm:tmpp2}), we can deduce that
    \begin{gather}
    \evalgphiGen {\pi_2} {\grad{\phi}}
    \end{gather}
    The goal follows (the separating conjunction acts like classical conjunction since ${\phiEq {${\ex{${x}$}}$} {${\ev{${\defaultValue{${T}$}}$}}$}}$ does not mention the heap).
\end{proof}

\begin{lemma}[\tset{GDPreservation} Induction Step for Calls]
    \label{lemma:is-call}
    Assume \tset{GDPreservation} holds for the method body of method $\edot{y}{m}$.
    Then \tset{GDPreservation} holds for $\sCall {$x$} {$y$} {$m$} {$z$}}$ (for all $m \in \setMethodName,\, x, y, z \in \setVar$).
\end{lemma}
\begin{proof}
    Assumption:
    \begin{align}
    \dgthoare {\Gamma} {\grad{\phi}} {{\sCall {${x}$} {${y}$} {${m}$} {${z'}$}}} {\gphiCons{$\grad{\phi'}$}{$\grad{\phi_q}$}}
    \end{align}
    Using inversion of rule $\tset{\dgradT HCall}$ we know that
    \begin{gather}
    \label{frm:tmpp1a}
    \wo {\wo {\grad{\phi}} {x}} {\grad{\phi_p}} = \grad{\phi'}\\
    {\mmethod{{C}, {m}}} = {{\method {${T_r}$} {${m}$} {${T_p}$} {${z}$} {${\contract {$\grad{\phi_{pre}}$} {$\grad{\phi_{post}}$}}$} {$s$}}} \\
    \gphiImplies{\grad{\phi}}{\gphiCons{${\phiNeq {${\ex{${y}$}}$} {${\ev{${\enull}$}}$}}$}{$\grad{\phi_p}$}} \\
    x \neq y \wedge x \neq z' \\
    \grad{\phi_p} = {\grad{\phi_{pre}}[{y}, {z'} / {\ethis}, {{z}}]} \\
    \label{frm:tmpp5a}
    \grad{\phi_q} = {\grad{\phi_{post}}[{y}, {z'}, {x} / {\ethis}, {{z}}, {\eresult}]}
    \end{gather}
    From gradual well-formedness (section \ref{ssec:well-formedness}) it follows that
    \begin{gather}
    \gthoare {z : T_p, \ethis : C, \eresult : T_r} {\grad{\phi_{pre}}} {s} {\grad{\phi_{post}}} \\
    \FV(\grad{\phi_{pre}}) \subseteq \{ z, \ethis \} \\
    \label{frm:tmpp8a}
    \FV(\grad{\phi_{post}}) \subseteq \{ z, \ethis, \eresult \} \\
    \sfrmgphi \grad{\phi_{pre}} \\
    \sfrmgphi \grad{\phi_{post}} \\
    \label{frm:tmpp11a}
    z, \ethis \not \in \mods(s)
    \end{gather}
    \begin{comment}
    Using the induction hypothesis we may assume that
    \begin{gather}
    \label{frm:tmpp100a}
    \gtHoare {} {\grad{\phi_{pre}}} {\textit{<method body>}} {\grad{\phi_{post}}}
    \end{gather}
    (note that the method body is $s$, combined with necessary measures to ensure the postcondition).
    \end{comment}
    As shown below, it follows that 
    \begin{align}
    \label{frm:tmpp101a}
    \gtHoare {} {\grad{\phi}} {{\sCall {${x}$} {${y}$} {${m}$} {${z'}$}}} {\gphiCons{$\grad{\phi'}$}{$\grad{\phi_q}$}}
    \end{align}
    
    Proof for \ref{frm:tmpp101a}:
    Assumptions:
    \begin{gather}
    \label{frm:tmpp1000a}
    \gsstepConsume{\sCall {${x}$} {${y}$} {${m}$} {${z'}$}}{\pi_1}{\pi_2}\\
    \label{frm:tmpp1001a}
    \evalgphiGen {\pi_1} {\grad{\phi}}
    \end{gather}
    for some $\pi_1, \pi_2 \in \setGProgramState$.
    Goal:
    \begin{gather}
    \evalgphiGen {\pi_2} {\gphiCons{$\grad{\phi'}$}{$\grad{\phi_q}$}}
    \end{gather}
    Applying rule inversion to \ref{frm:tmpp1000a} we derive that there exists $\pi_{pre}, \pi_{post} \in \setGProgramState$ such that
    \begin{gather}
    \label{frm:tmpp1010a}
    \gsstep{\pi_1}{\pi_{pre}}\\
    \label{frm:tmpp1011a}
    \gsstepConsume{s}{\pi_{pre}}{\pi_{post}}\\
    \label{frm:tmpp1012a}
    \gsstep{\pi_{post}}{\pi_2}
    \end{gather}
    where \ref{frm:tmpp1010a} is determined by \tset{\gradT SsCall} and \ref{frm:tmpp1012a} by \tset{\gradT SsCallFinish}.
    Applying rule inversion to \ref{frm:tmpp1012a} it follows that
    \begin{gather}
    \evalgphiGen {\pi_{post}} {\grad{\phi_{post}}}
    \end{gather}
    Combining \ref{frm:tmpp5a} and \ref{frm:tmpp8a} we can show that $\grad{\phi_{q}}$ may only contain $y, z', x$.
    \tset{\gradT SsCallFinish} assigns $\eresult$ to $x$, which reflects the substitution in \ref{frm:tmpp5a}.
    Because of \ref{frm:tmpp11a} $\grad{\phi_{q}}$ does not contain false information about $y$ and $z'$ (as described in section \ref{sec:well-formedness}).
    It follows that
    \begin{gather}
    \label{frm:tmpp2010a}
    \evalgphiGen {\pi_2} {\grad{\phi_{q}}}
    \end{gather}
    Since the variables and fields mentioned in $\grad{\phi'}$ are guaranteed to be disjoint from those in $\grad{\phi_p}$ and due to \ref{frm:tmpp1a}, those memory locations cannot be changed by the call.
    Furthermore $x$ is not in $\grad{\phi'}$ such that there cannot be a contradiction to the knowledge in $\grad{\phi_{q}}$ (which may contain information of $x$, provided by the call).
    It follows that
    \begin{gather}
    \label{frm:tmpp2011a}
    \evalgphiGen {\pi_2} {\grad{\phi'}}
    \end{gather}
    The goal follows from \ref{frm:tmpp2010a} and \ref{frm:tmpp2011a} using the fact that the formulas contain disjoint memory locations.
\end{proof}

\begin{lemma}[\tset{GDPreservation} Induction Step for Hold]
    \label{lemma:is-hold}
    Assume \tset{GDPreservation} holds for $s \in \setGStmt$.
    Then \tset{GDPreservation} holds for $\sHold {$\phi$} {$s$}$ (for all $\phi \in \setFormula$).
\end{lemma}
\begin{proof}
    Omitted.
\end{proof}

\begin{lemma}[\tset{GDPreservation} Induction Step for Sequences]
    \label{lemma:is-seq}
    Assume \tset{GDPreservation} holds for $s_1, s_2 \in \setGStmt$.
    Then \tset{GDPreservation} holds for $\sSeq {$s_1$} {$s_2$}$.
\end{lemma}
\begin{proof}
    Analogous to lemma \ref{lemma:gdpres-seq}.
\end{proof}

\begin{theorem}[\gvlidf Satisfies \tset{GDPreservation}]
    The semantics of \gvlidf comply with \tset{GDPreservation}.
\end{theorem}
\begin{proof}
    Structural induction over statement $s$.
    Most cases are handled by theorem \ref{thm:compl-and-so-to-gdpres} and lemmas \ref{lemma:pc-idf} and \ref{lemma:pso-idf}.
    The remaining cases are covered by lemmas 
    \ref{lemma:is-decl},
    \ref{lemma:is-call},
    \ref{lemma:is-hold} and
    \ref{lemma:is-seq}.
\end{proof}
