Proposed gradual Hoare logic is sound w.r.t. the small-step semantics if runtime checks are injected as proposed in section \ref{ssec:gradual-soundness} (see lemma \ref{lemma:tauto} for proof).
% BY construction, what about progress? \gsssem exceptional? THEOREM?

However, \gvlidf also complies with the stronger notion of soundness \tset{\dgradT Soundness} introduced in section \ref{ssec:gradual-soundness}.
To show this, we follow the approach introduced in section \ref{ssec:atomic--knowledge}.

Specifically, we show that the semantics are complete and semi-optimal w.r.t. all statements except declarations, sequences, method calls and hold.
For those remaining statements we prove by induction that \tset{\dgradT Soundness} still holds.
The overall proof is a structural induction on statement $s$.

\begin{lemma}[Partial Completeness of \svlidf]
    \label{lemma:pc-idf}~\\
    The semantics of \svlidf is complete (see definition \ref{def:completeness}) w.r.t. \\
    $\sSkip, \sFieldAssign {$x$} {$f$} {$y$}, \sVarAssign {$x$} {$e$}, \sAlloc {$x$} {$C$}, \sReturn {$x$}, \sAssert {$\phi$}, \sRelease {$\phi$}$
    \begin{comment}
        \item $\sSkip$
        \item $\sFieldAssign {$x$} {$f$} {$y$} $
        \item $\sVarAssign {$x$} {$e$}$
        \item $\sAlloc {$x$} {$C$}$
        \item $\sReturn {$x$}$
        \item $\sAssert {$\phi$}$
        \item $\sRelease {$\phi$}$
    \end{comment}
\end{lemma}

\begin{lemma}[Partial Semi-Optimality of \gvlidf]
    \label{lemma:pso-idf}~\\
    The semantics of \gvlidf is semi-optimal (see definitions \ref{def:perf-stat} and \ref{def:perf-dyn}) w.r.t.\\
    $\sSkip, \sFieldAssign {$x$} {$f$} {$y$}, \sVarAssign {$x$} {$e$}, \sAlloc {$x$} {$C$}, \sReturn {$x$}, \sAssert {$\phi$}, \sRelease {$\phi$}$
    \begin{comment}
        \item $\sSkip$
        \item $\sFieldAssign {$x$} {$f$} {$y$} $
        \item $\sVarAssign {$x$} {$e$}$
        \item $\sAlloc {$x$} {$C$}$
        \item $\sReturn {$x$}$
        \item $\sAssert {$\phi$}$
        \item $\sRelease {$\phi$}$
    \end{comment}
\end{lemma}

\begin{lemma}[\tset{\dgradT Soundness} Induction Step for Declaration]
    \label{lemma:is-decl}~\\
    Assume \tset{\dgradT Soundness} holds for $s \in \setGStmt$.\\
    Then \tset{\dgradT Soundness} holds for $\sSeq{\sDeclare{$T$}{$x$}}{$s$}$ (for all $T \in \setType,\, x \in \setVar$).
\end{lemma}

\begin{lemma}[\tset{\dgradT Soundness} Induction Step for Calls]
    \label{lemma:is-call}~\\
    Assume \tset{\dgradT Soundness} holds for the method body of method $\edot{$y$}{$m$}$.\\
    Then \tset{\dgradT Soundness} holds for $\sCall {$x$} {$y$} {$m$} {$z$}$ (for all $m \in \setMethodName,\, x, y, z \in \setVar$).
\end{lemma}

\begin{lemma}[\tset{\dgradT Soundness} Induction Step for Hold]
    \label{lemma:is-hold}~\\
    Assume \tset{\dgradT Soundness} holds for $s \in \setGStmt$.\\
    Then \tset{\dgradT Soundness} holds for $\sHold {$\phi$} {$s$}$ (for all $\phi \in \setFormula$).
\end{lemma}

\begin{lemma}[\tset{\dgradT Soundness} Induction Step for Sequences]
    \label{lemma:is-seq}~\\
    Assume \tset{\dgradT Soundness} holds for $s_1, s_2 \in \setGStmt$.\\
    Then \tset{\dgradT Soundness} holds for $\sSeq {$s_1$} {$s_2$}$.
\end{lemma}

\begin{theorem}[\gvlidf satisfies \tset{\dgradT Soundness}]\label{thm:gvlidf-dgsnd}~\\
    The semantics of \gvlidf satisfy \tset{\dgradT Soundness}.
\end{theorem}
