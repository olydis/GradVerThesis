Formula semantics of \svlidf only depend on the heap and on the top most stack frame of a program state (more specifically: the local variable environment and the accessible fields, but not the continuation statement).
To drastically simplify notation, we will therefore define $\vDash \subseteq \setProgramState \times \setFormula$ (as postulated in section \ref{sec:a-statically-verified}) indirectly:
\begin{mathpar}
    \inferrule* [Right=EvalFrm]
    {
        \evalphi {\phi}
    }
    {
        \evalphiGen{(H, (\rho, A, s) \cdot S)}{\phi}
    }
\end{mathpar}

Figure \ref{fig:svl-evalphi} defines the actual semantics as a predicate $\vDash \subseteq \setHeap \times \setVarEnv \times \setDFootprint \times \setFormula$.
\begin{figure}
    \boxed{\evalphi \phi}
    % Inductive Semantics.evalphi'
\begin{mathpar}
\inferrule* [Right=EATrue]
{
    ~
}
{
    \evalphix {H} {\rho} {A} {\phiTrue}
}
\end{mathpar}

\begin{mathpar}
\inferrule* [Right=EAEqual]
{
    \evalex {H} {\rho} {e_1} {v_1} \\
    \evalex {H} {\rho} {e_2} {v_2} \\
    {v_1} = {v_2}
}
{
    \evalphix {H} {\rho} {A} {\phiEq {${e_1}$} {${e_2}$}}
}
\end{mathpar}

\begin{mathpar}
\inferrule* [Right=EANEqual]
{
    \evalex {H} {\rho} {e_1} {v_1} \\
    \evalex {H} {\rho} {e_2} {v_2} \\
    {v_1} \neq {v_2}
}
{
    \evalphix {H} {\rho} {A} {\phiNeq {${e_1}$} {${e_2}$}}
}
\end{mathpar}

\begin{mathpar}
\inferrule* [Right=EAAcc]
{
    \evalex {H} {\rho} {e} {{o}} \\
    \evalex {H} {\rho} {\edot{${e}$}{${f}$}} {v} \\
    {({o}, {f})} \in {A}
}
{
    \evalphix {H} {\rho} {A} {\phiAcc {${e}$} {${f}$}}
}
\end{mathpar}



\begin{mathpar}
    \inferrule* [Right=EASepOp]
    {
        A_1 = A \backslash A_2 \\
        \evalphix H \rho {A_1} {\phi_1} \\
        \evalphix H \rho {A_2} {\phi_2}
    }
    {\evalphi {\phiCons {$\phi_1$} {$\phi_2$}}}
\end{mathpar}
    \caption{\svlidf: Evaluating Formulas}
    \label{fig:svl-evalphi}
\end{figure}

% TODO: subtelties of EAAcc?
% SEPOp, making aliasing acc false, identical acc thus unsatisfiable

We define implication analogous to the generic definition postulated in section \ref{sec:a-statically-verified}.
\begin{definition}[\svlidf: Formula Implication]
    \begin{displaymath}
    \phiImplies{\phi_1}{\phi_2} ~\defiff~ 
    \forall H \in \setHeap,\, \rho \in \setVarEnv,\, A \in \setDFootprint.~ 
    \evalphi{\phi_1} \implies \evalphi{\phi_2}
    \end{displaymath}
\end{definition}

A number of rules arise naturally from this semantical definition of implication.
\begin{example}{\svlidf: Derivable Rules}
    \begin{description}
        \item[Decomposition]
        \begin{mathpar}
            \inferrule* [Right=FrmDecomp1]
            {
                ~
            }
            {
                \phiImplies
                {\phiCons{$\phi_1$}{$\phi_2$}}
                {\phi_1}
            }
        \end{mathpar}
        \begin{mathpar}
            \inferrule* [Right=FrmDecomp2]
            {
                ~
            }
            {
                \phiImplies
                {\phiCons{$\phi_1$}{$\phi_2$}}
                {\phi_2}
            }
        \end{mathpar}
        
        \item[Transitivity]
        \begin{mathpar}
            \inferrule* [Right=FrmTrans]
            {
                \phiImplies
                {\phi_a}
                {\phi_b}\\
                \phiImplies
                {\phi_b}
                {\phi_c}
            }
            {
                \phiImplies
                {\phi_a}
                {\phi_c}
            }
        \end{mathpar}
        \begin{mathpar}
        \inferrule* [Right=FrmTransEq]
            {
                ~
            }
            {
                \phiImplies
                {\phiCons{\phiEq{$a$}{$b$}}{\phiEq{$b$}{$c$}}}
                {\phiEq{$a$}{$c$}}
            }
        \end{mathpar}
        
        \item[Aliasing Prevention]
        \begin{mathpar}
            \inferrule* [Right=FrmAlias]
            {
                ~
            }
            {
                \phiImplies
                {\phiCons{\phiAcc{$e_1$}{$f$}}{\phiAcc{$e_2$}{$f$}}}
                {\phiNeq{$e_1$}{$e_2$}}
            }
        \end{mathpar}
        
        \item[Conjunction]
        \begin{mathpar}
            \inferrule* [Right=FrmComp]
            {
                \phiImplies
                {\phi}
                {\phi_a}\\
                \phiImplies
                {\phi}
                {\phi_b}\\
                \forall H \in \setHeap,\, \rho \in \setVarEnv.~ \dynamicFP{H}{\rho}{\phi_a} \cap \dynamicFP{H}{\rho}{\phi_b} = \emptyset
            }
            {
                \phiImplies
                {\phi}
                {\phiCons{$\phi_a$}{$\phi_b$}}
            }
        \end{mathpar}
    \end{description}
\end{example}