Formula semantics of \svlidf only depend on the heap and on the top most stack frame of a program state (more specifically: the local variable environment and the accessible fields, but not the continuation statement).
To drastically simplify notation, we will therefore define $\vDash \subseteq \setProgramState \times \setFormula$ (as postulated in section \ref{sec:a-statically-verified}) indirectly:
\begin{mathpar}
    \inferrule* [Right=EvalFrm]
    {
        \evalphi {\phi}
    }
    {
        \evalphiGen{(H, (\rho, A, s) \cdot S)}{\phi}
    }
\end{mathpar}

Figure \ref{fig:svl-evalphi} defines the actual semantics as a predicate $\vDash \subseteq \setHeap \times \setVarEnv \times \setDFootprint \times \setFormula$.
\begin{figure}
    \boxed{\evalphi \phi}
    % Inductive Semantics.evalphi'
\begin{mathpar}
\inferrule* [Right=EATrue]
{
    ~
}
{
    \evalphix {H} {\rho} {A} {\phiTrue}
}
\end{mathpar}

\begin{mathpar}
\inferrule* [Right=EAEqual]
{
    \evalex {H} {\rho} {e_1} {v_1} \\
    \evalex {H} {\rho} {e_2} {v_2} \\
    {v_1} = {v_2}
}
{
    \evalphix {H} {\rho} {A} {\phiEq {${e_1}$} {${e_2}$}}
}
\end{mathpar}

\begin{mathpar}
\inferrule* [Right=EANEqual]
{
    \evalex {H} {\rho} {e_1} {v_1} \\
    \evalex {H} {\rho} {e_2} {v_2} \\
    {v_1} \neq {v_2}
}
{
    \evalphix {H} {\rho} {A} {\phiNeq {${e_1}$} {${e_2}$}}
}
\end{mathpar}

\begin{mathpar}
\inferrule* [Right=EAAcc]
{
    \evalex {H} {\rho} {e} {{o}} \\
    \evalex {H} {\rho} {\edot{${e}$}{${f}$}} {v} \\
    {({o}, {f})} \in {A}
}
{
    \evalphix {H} {\rho} {A} {\phiAcc {${e}$} {${f}$}}
}
\end{mathpar}



\begin{mathpar}
    \inferrule* [Right=EASepOp]
    {
        A_1 = A \backslash A_2 \\
        \evalphix H \rho {A_1} {\phi_1} \\
        \evalphix H \rho {A_2} {\phi_2}
    }
    {\evalphi {\phiCons {$\phi_1$} {$\phi_2$}}}
\end{mathpar}
    \caption{\svlidf: Evaluating Formulas}
    \label{fig:svl-evalphi}
\end{figure}

% TODO: subtelties of EAAcc?
% SEPOp



% RULES arising from this semantics!!! e.g. implication:
% - a * b => a
% - a !=> a * a