Recall that program state has a notion of remaining work, see section \ref{sec:a-statically-verified} for examples.
As the set of possible statements has been augmented from $\setStmt$ to $\setGStmt$, the notion of remaining work might have to be augmented as well in order to allow encoding the additional statements.

This augmentation leads to a superset $\setGProgramState \supseteq \setProgramState$ of gradual program states.
\begin{example}{Gradual Program State}
$$\setProgramState ~  =~ (\setVar \rightharpoonup \mathbb{Z}) ~\times~ \setStmt$$
is extended to
$$\setGProgramState ~ =~ (\setVar \rightharpoonup \mathbb{Z}) ~\times~ \setGStmt$$
\end{example}

We give meaning to gradual program states using concretization.
\begin{definition}[Concretization of Gradual Program States]
    Let $\gamma_{\pi} : \setGProgramState \rightarrow \PP(\setProgramState)$ be defined as
    \begin{displaymath}
    \gamma_{\pi}(\grad{\pi}) = \{~ \pi \in \setProgramState ~|~ \textit{$\pi$ is $\grad{\pi}$ with all continuations replaced by a concretization} ~\}
    \end{displaymath} 
\end{definition}
\begin{definition}[Precision of Gradual Program States]
    Let $\mptpi \subseteq \setGProgramState \times \setGProgramState$ be a predicate defined as
    $$\grad{\pi_a} \mptpi \grad{\pi_b}  \quad\iff\quad  \gamma_{\pi}(\grad{\pi_a}) \subseteq \gamma_{\pi}(\grad{\pi_b})$$
    % TODO: observational instead... otherwise gradual release or call would not work
\end{definition}

\begin{comment}
Consequence:
\begin{displaymath}
\forall \grad{\pi_{\grad{s}}} \in \setGProgramState_{\grad{s}}, \pi \in \gamma_{\pi}(\grad{\pi_{\grad{s}}}).~ \exists s \in \gamma_s(\grad{s}).~ \pi \in \setProgramState_s
\end{displaymath}
\end{comment}

\begin{lemma}[Gradual Program State Does Not Affect Formula Semantics]
    \label{lemma:gradPS-form-sem}
    We demand that formula semantics are not affected by gradualization of the program state:
    \begin{displaymath}
    \forall \phi \in \setFormula, \grad{\pi} \in \setGProgramState, \pi \in \gamma_{\pi}(\grad{\pi}).~~ \evalphiGen{\grad{\pi}}{\phi} \iff \evalphiGen{\pi}{\phi}
    \end{displaymath}
    
    This is trivially the case if evaluation does not depend on the (now gradual) continuation in the first place.
\end{lemma}

