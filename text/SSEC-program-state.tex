The set of program states is defined as $\setProgramState = \setHeap \times \setStack$ where
\begin{align*}
	S & \in \setStack      &  & ::= E \cdot S ~|~ \nil                               \\
	E & \in \setStackEntry &  & =~~ \setVarEnv \times \setDFootprint \times \setStmt
\end{align*}

% A program state of \svlidf consists of a single heap and a stack.
Each stack frame has an environment $\setVarEnv$ for local variables, tracks a set of accessible fields $\setDFootprint$ and stores a continuation $\setStmt$.
Stack frames will be introduced by method calls, but also by dedicated scopes as used by the $\ttt{hold}$ statement.

\begin{comment}
REQUIRED?
\begin{definition}[Topmost Stack Entry]
    Let $\topmost : \setStack \rightharpoonup \setStackEntry$ be defined as
    \begin{align*}
    &\topmost(E \cdot S) = E\\
    &\topmost(\nil) \quad\textit{ undefined}
    \end{align*}
\end{definition}

Program states with scheduled statement $s$ are defined as
\begin{displaymath}
\setProgramState_s ~\defeq~ \setHeap ~\times~ \{~~ (\rho, A_d, s) \cdot S ~~|~~ \rho \in \setVarEnv,~ A_d \in \setDFootprint,~ S \in \setStack ~~\}
\end{displaymath}
\end{comment}