\begin{lemma}[Composability of Deterministic Lifting]~\\
    Let $\dgrad{P_1}, \dgrad{P_2}$ be sound deterministic liftings of predicates $P_1, P_2$.
    Then
    \begin{displaymath}
    \dgrad{P_3} ~~\defeq~~ \dgrad{P_2} \circ \dgrad{P_1}
    \end{displaymath}
    is a sound deterministic lifting of $P_3(\phi_1, \phi_3) = \exists \phi_2.~ P_1(\phi_1, \phi_2) \wedge P_2(\phi_2, \phi_3)$.
\end{lemma}
In other words, deterministic liftings of composite predicates can be obtained by composing piecewise deterministic liftings.
Optimality of the piecewise liftings does not guarantee optimality of the overall lifting.

We further demonstrate deterministic lifting by means of the following Hoare rules:
\begin{mathpar}
    \inferrule* [right=HSeq]
    {
        \phiImplies{\phi_{q1}} {\phi_{q2}} \\\\
        \thoare {} {\phi_p} {{s_1}} {\phi_{q1}} \\
        \thoare {} {\phi_{q2}} {{s_2}} {\phi_r}
    }
    {
        \thoare {} {\phi_p} {\sSeq{$s_1$}{$s_2$}} {\phi_r}
    }
    
    \inferrule* [Right=HAssign]
    {
        ~
    }
    {
        \thoare {} {\phi[e/x]} {\sVarAssign{$x$}{$e$}} {\phi}
    }
\end{mathpar}

Note that gradual liftings of these rules can be found in section \ref{ssec:the-problem-with}, where we demonstrated the problems that ultimately lead to deterministic liftings.

We start by lifting implication, which is used in \tset{HSeq}.
\begin{lemma}[Optimal Deterministic Lifting of Implication]~\\
    Let $\id : \setGFormula \rightarrow \setGFormula$ be the identity function.
    Then $\id$ is an optimal deterministic lifting of $\phiImplies{\cdot}{\cdot} \subseteq \setFormula \times \setFormula$.
\end{lemma}

Using lemma ??? and lemma ???, the deterministic lifting of \tset{HSeq} reduces to
\begin{mathpar}
    \inferrule* [right=DGHSeq]
    {
        \dgthoare {} {\grad{\phi_p}} {\grad{s_1}} {\grad{\phi_q}} \\
        \dgthoare {} {\grad{\phi_q}} {\grad{s_2}} {\grad{\phi_r}}
    }
    {
        \dgthoare {} {\grad{\phi_p}} {\sSeq{$\grad{s_1}$}{$\grad{s_2}$}} {\grad{\phi_r}}
    }
\end{mathpar}
Note that we used inductive notation to define a function -- there are no free variables.

For the assignment rule we can derive
\begin{mathpar}
    \inferrule* [right=DGHAssign1]
    {
        x \not \in \FV(\phi) \\
        x \not \in \FV(e)
    }
    {
        \dgthoare {} {\phi} {\sVarAssign{$x$}{$e$}} {\phiAnd{$\phi$}{\phiEq{$x$}{$e$}}}
    }
    
    \inferrule* [right=DGHAssign2]
    {
        \tset{DGHAssign1} \textit{ does not apply}
    }
    {
        \dgthoare {} {\grad{\phi}} {\sVarAssign{$x$}{$e$}} {\qm}
    }
\end{mathpar}
as a sound deterministic lifting.
% MORE expl?