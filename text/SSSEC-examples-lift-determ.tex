\begin{lemma}[Composability of Deterministic Lifting]
    \label{lemma:det-lift-comp}~\\
    Let $\dgrad{P_1}, \dgrad{P_2}$ be sound deterministic liftings of predicates $P_1, P_2$.\\
    Let $P_2$ be monotonic w.r.t. implication in its first parameter.
    Then
    \begin{displaymath}
    \dgrad{P_3} ~~\defeq~~ \dgrad{P_2} \circ \dgrad{P_1}
    \end{displaymath}
    is a sound deterministic lifting of $P_3(\phi_1, \phi_3) = \exists \phi_2.~ P_1(\phi_1, \phi_2) \wedge P_2(\phi_2, \phi_3)$.
\end{lemma}
In other words, deterministic liftings of composite predicates can be obtained by composing piecewise deterministic liftings.
Optimality of the piecewise liftings does not guarantee optimality of the overall lifting. % double release

We further demonstrate deterministic lifting by means of the following Hoare rules:
\begin{mathpar}
    \inferrule* [right=HSeq]
    {
        \phiImplies{\phi_{q1}} {\phi_{q2}} \\\\
        \thoare {} {\phi_p} {{s_1}} {\phi_{q1}} \\
        \thoare {} {\phi_{q2}} {{s_2}} {\phi_r}
    }
    {
        \thoare {} {\phi_p} {\sSeq{$s_1$}{$s_2$}} {\phi_r}
    }
    
    \inferrule* [Right=HAssign]
    {
        ~
    }
    {
        \thoare {} {\phi[e/x]} {\sVarAssign{$x$}{$e$}} {\phi}
    }
\end{mathpar}

Note that gradual liftings of these rules can be found in section \ref{ssec:the-problem-with}, where we demonstrated the problems that ultimately lead to deterministic liftings.

We start by lifting implication, which is used in \tset{HSeq}.
\begin{lemma}[Optimal Deterministic Lifting of Implication]
    \label{lemma:opt-lift-impl}~\\
    Let $\id : \setGFormula \rightarrow \setGFormula$ be the identity function.\\
    Then $\id$ is an optimal deterministic lifting of $\phiImplies{\cdot}{\cdot} \subseteq \setFormula \times \setFormula$.
\end{lemma}

Using lemma \ref{lemma:det-lift-comp} and lemma \ref{lemma:opt-lift-impl}, the deterministic lifting of \tset{HSeq} reduces to
\begin{mathpar}
    \inferrule* [right=\dgradT HSeq]
    {
        \dgthoare {} {\grad{\phi_p}} {\grad{s_1}} {\grad{\phi_q}} \\
        \dgthoare {} {\grad{\phi_q}} {\grad{s_2}} {\grad{\phi_r}}
    }
    {
        \dgthoare {} {\grad{\phi_p}} {\sSeq{$\grad{s_1}$}{$\grad{s_2}$}} {\grad{\phi_r}}
    }
\end{mathpar}
Note that we used inductive notation to define a function -- there are no free variables, i.e. the rule is syntax-directed.

For the assignment rule we can derive
\begin{mathpar}
    \inferrule* [right=\dgradT HAssign1]
    {
        x \not \in \FV(\phi) \\
        x \not \in \FV(e)
    }
    {
        \dgthoare {} {\phi} {\sVarAssign{$x$}{$e$}} {\phiAnd{$\phi$}{\phiEq{$x$}{$e$}}}
    }
    
    \inferrule* [right=\dgradT HAssign2]
    {
        \tset{\dgradT HAssign1} \textit{ does not apply}
    }
    {
        \dgthoare {} {\grad{\phi}} {\sVarAssign{$x$}{$e$}} {\qm}
    }
\end{mathpar}
as a sound deterministic lifting.
Note that this lifting is far from optimal, but rather a heuristic.
A more optimal lifting requires more sophisticated reasoning.
% MORE expl?