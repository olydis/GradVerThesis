\begin{lemma}[Composability of Deterministic Lifting]~\\
    \label{lemma:det-lift-comp}
    Let $\dgrad{P_1}, \dgrad{P_2}$ be sound deterministic liftings of predicates $P_1, P_2$.
    Let $P_2$ be monotonic w.r.t. implication in its first parameter.
    Then
    \begin{displaymath}
    \dgrad{P_3} ~~\defeq~~ \dgrad{P_2} \circ \dgrad{P_1}
    \end{displaymath}
    is a sound deterministic lifting of $P_3(\phi_1, \phi_3) = \exists \phi_2.~ P_1(\phi_1, \phi_2) \wedge P_2(\phi_2, \phi_3)$.
\end{lemma}
\begin{proof}
    Introduction
    \begin{align*}
    P_3(\phi_1, \phi_3)
    & \overset{}{~\quad\quad\implies\quad\quad~} 
    \exists \phi_2.~ P_1(\phi_1, \phi_2) \wedge P_2(\phi_2, \phi_3)\\
    & \overset{Introduction}{~\quad\quad\implies\quad\quad~} 
    \exists \phi_2.~ P_1(\phi_1, \phi_2) \wedge P_2(\phi_2, \phi_3)
    \wedge (\exists \grad{\phi_2}.~ \dgrad{P_1}(\phi_1) = \grad{\phi_2})\\
    & \overset{Strength}{~\quad\quad\implies\quad\quad~} 
    \exists \phi_2.~ P_1(\phi_1, \phi_2) \wedge P_2(\phi_2, \phi_3)
    \wedge (\exists \grad{\phi_2}.~ \dgrad{P_1}(\phi_1) = \grad{\phi_2} \wedge 
           (\exists \phi_2' \in \gamma(\grad{\phi_2}).~ \phiImplies{\phi_2'}{\phi_2}))\\
    & \overset{Mono~P_2}{~\quad\quad\implies\quad\quad~} 
           \exists \phi_2.~ P_1(\phi_1, \phi_2) \wedge P_2(\phi_2, \phi_3)
           \wedge (\exists \grad{\phi_2}.~ \dgrad{P_1}(\phi_1) = \grad{\phi_2} \wedge 
           (\exists \phi_2' \in \gamma(\grad{\phi_2}), \phi_3'.~ \phiImplies{\phi_2'}{\phi_2} \wedge P_2(\phi_2', \phi_3')))\\
    & \overset{}{~\quad\quad\implies\quad\quad~} 
           \exists \grad{\phi_2}.~ \dgrad{P_1}(\phi_1) = \grad{\phi_2} \wedge 
           (\exists \phi_2' \in \gamma(\grad{\phi_2}), \phi_3'.~ P_2(\phi_2', \phi_3'))\\
    & \overset{Introduction}{~\quad\quad\implies\quad\quad~} 
           \exists \grad{\phi_2}.~ \dgrad{P_1}(\phi_1) = \grad{\phi_2} \wedge 
           (\exists \phi_2' \in \gamma(\grad{\phi_2}), \grad{\phi_3'}.~ \dgrad{P_2}(\phi_2') = \grad{\phi_3'})\\
    & \overset{Monotonicity}{~\quad\quad\implies\quad\quad~} 
           \exists \grad{\phi_2}.~ \dgrad{P_1}(\phi_1) = \grad{\phi_2} \wedge 
           (\exists \grad{\phi_3}.~ \dgrad{P_2}(\grad{\phi_2}) = \grad{\phi_3})\\
    & \overset{}{~\quad\quad\implies\quad\quad~} 
           \exists \grad{\phi_3}.~ \dgrad{P_2}(\dgrad{P_1}(\phi_1)) = \grad{\phi_3}\\
    & \overset{}{~\quad\quad\implies\quad\quad~} 
           \exists \grad{\phi_3}.~ \dgrad{P_3}(\phi_1) = \grad{\phi_3}
    \end{align*}
    
    Strength
    \begin{align*}
    & \dgrad{P_3}(\grad{\phi_1}) = \grad{\phi_3} \wedge \phi_1 \in \gamma(\grad{\phi_1}) \wedge P_3(\phi_1, \phi)\\
    & \overset{Defenitions}{~\quad\quad\implies\quad\quad~} 
        \exists \grad{\phi_2}, \phi'.~ 
        \dgrad{P_1}(\grad{\phi_1}) = \grad{\phi_2} \wedge \dgrad{P_2}(\grad{\phi_2}) = \grad{\phi_3} \wedge 
        \phi_1 \in \gamma(\grad{\phi_1}) \wedge P_1(\phi_1, \phi') \wedge P_2(\phi', \phi)\\
    & \overset{Strength}{~\quad\quad\implies\quad\quad~} 
        \exists \grad{\phi_2}, \phi'.~ 
        \dgrad{P_1}(\grad{\phi_1}) = \grad{\phi_2} \wedge \dgrad{P_2}(\grad{\phi_2}) = \grad{\phi_3} \wedge 
        \phi_1 \in \gamma(\grad{\phi_1}) \wedge P_1(\phi_1, \phi') \wedge P_2(\phi', \phi)
        \wedge (\exists \phi_2 \in \gamma(\grad{\phi_2}).~ P_1(\phi_1, \phi_2) \wedge \phiImplies{\phi_2}{\phi'}) \\
    & \overset{}{~\quad\quad\implies\quad\quad~} 
        \exists \grad{\phi_2}, \phi', \phi_2 \in \gamma(\grad{\phi_2}).~ 
        \dgrad{P_1}(\grad{\phi_1}) = \grad{\phi_2} \wedge \dgrad{P_2}(\grad{\phi_2}) = \grad{\phi_3} \wedge 
        \phi_1 \in \gamma(\grad{\phi_1}) \wedge P_1(\phi_1, \phi_2) \wedge P_2(\phi', \phi)
        \wedge \phiImplies{\phi_2}{\phi'} \\
    & \overset{}{~\quad\quad\implies\quad\quad~} 
        \exists \grad{\phi_2}, \phi', \phi_2 \in \gamma(\grad{\phi_2}).~ 
        \dgrad{P_2}(\grad{\phi_2}) = \grad{\phi_3} \wedge 
        P_1(\phi_1, \phi_2) \wedge P_2(\phi', \phi)
        \wedge \phiImplies{\phi_2}{\phi'} \\
    & \overset{Mono~P_2}{~\quad\quad\implies\quad\quad~} 
        \exists \grad{\phi_2}, \phi_2 \in \gamma(\grad{\phi_2}), \phi''.~ 
        \dgrad{P_2}(\grad{\phi_2}) = \grad{\phi_3} \wedge 
        P_1(\phi_1, \phi_2) \wedge P_2(\phi_2, \phi'')
        \wedge \phiImplies{\phi''}{\phi} \\
    & \overset{Strength}{~\quad\quad\implies\quad\quad~} 
        \exists \grad{\phi_2}, \phi_2 \in \gamma(\grad{\phi_2}), \phi''.~ 
        \dgrad{P_2}(\grad{\phi_2}) = \grad{\phi_3} \wedge 
        P_1(\phi_1, \phi_2) \wedge P_2(\phi_2, \phi'') 
        \wedge \phiImplies{\phi''}{\phi} \wedge
        (\exists \phi_3 \in \gamma(\grad{\phi_3}).~ P_2(\phi_2, \phi_3) \wedge \phiImplies{\phi_3}{\phi''}) \\
    & \overset{}{~\quad\quad\implies\quad\quad~} 
        \exists \phi_2 \in \gamma(\grad{\phi_2}), \phi_3 \in \gamma(\grad{\phi_3}).~ 
        P_1(\phi_1, \phi_2) \wedge P_2(\phi_2, \phi_3)
        \wedge \phiImplies{\phi_3}{\phi} \\
    & \overset{Defenition}{~\quad\quad\implies\quad\quad~} 
        \exists \phi_3 \in \gamma(\grad{\phi_3}).~ 
        P_3(\phi_1, \phi_3)
        \wedge \phiImplies{\phi_3}{\phi} 
    \end{align*}
    
    Monotonicity
    holds by transitivity
\end{proof}
In other words, deterministic liftings of composite predicates can be obtained by composing piecewise deterministic liftings.
Optimality of the piecewise liftings does not guarantee optimality of the overall lifting.

We further demonstrate deterministic lifting by means of the following Hoare rules:
\begin{mathpar}
    \inferrule* [right=HSeq]
    {
        \phiImplies{\phi_{q1}} {\phi_{q2}} \\\\
        \thoare {} {\phi_p} {{s_1}} {\phi_{q1}} \\
        \thoare {} {\phi_{q2}} {{s_2}} {\phi_r}
    }
    {
        \thoare {} {\phi_p} {\sSeq{$s_1$}{$s_2$}} {\phi_r}
    }
    
    \inferrule* [Right=HAssign]
    {
        ~
    }
    {
        \thoare {} {\phi[e/x]} {\sVarAssign{$x$}{$e$}} {\phi}
    }
\end{mathpar}

Note that gradual liftings of these rules can be found in section \ref{ssec:the-problem-with}, where we demonstrated the problems that ultimately lead to deterministic liftings.

We start by lifting implication, which is used in \tset{HSeq}.
\begin{lemma}[Optimal Deterministic Lifting of Implication]~\\
    \label{lemma:opt-lift-impl}
    Let $\id : \setGFormula \rightarrow \setGFormula$ be the identity function.
    Then $\id$ is an optimal deterministic lifting of $\phiImplies{\cdot}{\cdot} \subseteq \setFormula \times \setFormula$.
\end{lemma}
\begin{proof}
    Soundness
        Introduction
        Identity function is total.
        
        Strength
        Known:
        $$\id(\grad{\phi_1}) = \grad{\phi_2} \wedge \phi_1 \in \gamma(\grad{\phi_1}) \wedge \phiImplies{\phi_1}{\phi}$$
        Goal:
        $$\exists \phi_2 \in \gamma(\grad{\phi_2}).~ \phiImplies{\phi_1}{\phi_2} \wedge \phiImplies{\phi_2}{\phi}$$
        The goal is satisfied when choosing $\phi_2 = \phi_1$

        Monotonicity
        Trivial.
\end{proof}

Using lemma \ref{lemma:det-lift-comp} and lemma \ref{lemma:opt-lift-impl}, the deterministic lifting of \tset{HSeq} reduces to
\begin{mathpar}
    \inferrule* [right=DGHSeq]
    {
        \dgthoare {} {\grad{\phi_p}} {\grad{s_1}} {\grad{\phi_q}} \\
        \dgthoare {} {\grad{\phi_q}} {\grad{s_2}} {\grad{\phi_r}}
    }
    {
        \dgthoare {} {\grad{\phi_p}} {\sSeq{$\grad{s_1}$}{$\grad{s_2}$}} {\grad{\phi_r}}
    }
\end{mathpar}
Note that we used inductive notation to define a function -- there are no free variables, i.e. the rule is syntax-directed.

For the assignment rule we can derive
\begin{mathpar}
    \inferrule* [right=DGHAssign1]
    {
        x \not \in \FV(\phi) \\
        x \not \in \FV(e)
    }
    {
        \dgthoare {} {\phi} {\sVarAssign{$x$}{$e$}} {\phiAnd{$\phi$}{\phiEq{$x$}{$e$}}}
    }
    
    \inferrule* [right=DGHAssign2]
    {
        \tset{DGHAssign1} \textit{ does not apply}
    }
    {
        \dgthoare {} {\grad{\phi}} {\sVarAssign{$x$}{$e$}} {\qm}
    }
\end{mathpar}
as a sound deterministic lifting.
Note that this lifting is far from optimal, but rather a heuristics.
A stronger lifting requires more sophisticated reasoning.
% MORE expl?