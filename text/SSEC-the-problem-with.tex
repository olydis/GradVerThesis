%% Intro
%As seen in section \ref{ssec:gradual-soundness}, the lifted Hoare predicate in general requires an additional runtime assertion to guarantee soundness.
In case the Hoare rules of \svl are given inductively, we can make use of the rules for composite, disjunctive and conjunctive predicate lifting (see section \ref{sssec:examples-lift-predicates}).
The rules allow us to soundly lift each individual inductive rule in order to end up with a sound lifting of the overall predicate.

%% example
\begin{example}{Rule-wise Hoare Logic Lifting}
    Given:
    \begin{mathpar}
        \inferrule* [right=HSeq]
        {
            \phiImplies{\phi_{q1}} {\phi_{q2}} \\\\
            \thoare {} {\phi_p} {{s_1}} {\phi_{q1}} \\
            \thoare {} {\phi_{q2}} {{s_2}} {\phi_r}
        }
        {
            \thoare {} {\phi_p} {\sSeq{$s_1$}{$s_2$}} {\phi_r}
        }
        
        \inferrule* [Right=HAssign]
        {
            ~
        }
        {
            \thoare {} {\phi[e/x]} {\sVarAssign{$x$}{$e$}} {\phi}
        }
    \end{mathpar}
    Rule-wise lifting:
    \begin{mathpar}
        \inferrule* [right=\gradT HSeq]
        {
            \gphiImplies{\grad{\phi_{q1}}} {\grad{\phi_{q2}}} \\\\
            \gthoare {} {\grad{\phi_p}} {{s_1}} {\grad{\phi_{q1}}} \\
            \gthoare {} {\grad{\phi_{q2}}} {{s_2}} {\grad{\phi_r}}
        }
        {
            \gthoare {} {\grad{\phi_p}} {\sSeq{$s_1$}{$s_2$}} {\grad{\phi_r}}
        }
        
        \inferrule* [right=\gradT HAssign1]
        {
            ~
        }
        {
            \gthoare {} {\phi[e/x]} {\sVarAssign{$x$}{$e$}} {\phi}
        }
        
        \inferrule* [right=\gradT HAssign2]
        {
            ~
        }
        {
            \gthoare {} {\qm} {\sVarAssign{$x$}{$e$}} {\grad{\phi}}
        }
        
        \inferrule* [right=\gradT HAssign3]
        {
            ~
        }
        {
            \gthoare {} {\grad{\phi}} {\sVarAssign{$x$}{$e$}} {\qm}
        }
    \end{mathpar}
    
    Note the usage of composite predicate lifting (lemma \ref{lemma:pred-lift-comp}) for \tset{\gradT HSeq}.
\end{example}

%% example descr
% The overall Hoare predicate $\thoare {} {\cdot} {\cdot} {\cdot}$ can be thought of as a disjunction of all inductive rules.
% It follows that $\gthoare {} {\cdot} {\cdot} {\cdot}$, being the disjunction of above lifted rules is also a sound lifting.

%% theory vs verifier practice (many choices)
Unfortunately, a gradual verifier using $\gthoare {} {\cdot} {\cdot} {\cdot}$ gets into a practical dilemma.
Consider the Hoare triple
\begin{displaymath}
\hoare{\qm}{\sSeq{\sVarAssign{y}{2}}{\sVarAssign{x}{3}}}{\phiAnd{\phiEq{x}{3}}{\phiEq{y}{2}}}
\end{displaymath}
It is the job of the gradual verifier to prove the triple using above gradual inductive rules.
Using rule inversion (rule \tset{\gradT HSeq}) it can deduce that 
\begin{gather*}
\gphiImplies{\grad{\phi_{q1}}} {\grad{\phi_{q2}}} \\
\gthoare {} {\qm} {\sVarAssign{y}{2}} {\grad{\phi_{q1}}} \\
\gthoare {} {\grad{\phi_{q2}}} {\sVarAssign{x}{3}} {\phiAnd{\phiEq{x}{3}}{\phiEq{y}{2}}}
\end{gather*}
has to hold for some $\grad{\phi_{q1}}, \grad{\phi_{q2}} \in \setGFormula$.
There are a variety of valid instantiations for both variables:
\begin{description}
    \item[Good: $\grad{\phi_{q1}} = \phiEq{y}{2},~ \grad{\phi_{q2}} = \phiEq{y}{2}$]~\\
    This instantiation aims to use static formulas as early as possible.
    The implication trivially holds.
        
    \item[Too weak: $\grad{\phi_{q1}} = \qm,~ \grad{\phi_{q2}} = \qm$]~\\
    Whenever there exists a valid instantiation at all, the wildcard is also a valid one, so always choosing it would be an easy way of implementing a gradual verifier.
    Note however, that the knowledge about the first statement is lost which results in the necessity of dynamic checks to ensure $\phiEq{x}{3}$ after $\sVarAssign{y}{2}$.
    Before, this check could have been optimized away.
    In general, choosing $\qm$ as intermediate gradual formulas allows verifying arbitrary inconsistent judgments (a manifestation of the lack of optimality of $\gthoare {} {\cdot} {\cdot} {\cdot}$).
    
    \begin{example}{Inconsistent Gradual Verification}
        \begin{displaymath}
        \gthoare{}{\phiTrue}{\sSeq{\sVarAssign{y}{2}}{\sVarAssign{x}{3}}}{\phiEq{y}{100}}
        \end{displaymath}
        is verifiable if the gradual verifier chooses $\qm$ as intermediate gradual formulas.
    \end{example}
    
    Consequently, the verifier may try to be “as precise as possible” in order to detect inconsistencies at compile time.
        
    \item[Too strict: $\grad{\phi_{q1}} = \phiAnd{\phiEq{x}{42}}{\phiEq{y}{2}},~ \grad{\phi_{q2}} = \phiEq{y}{2}$]~\\
    This instantiation is fully static (i.e. precise) and valid according to the rules:
    The implication holds (the knowledge about \ttt{x} is dropped), and $$\gthoare {} {\qm} {\sVarAssign{y}{2}} {\phiAnd{\phiEq{x}{42}}{\phiEq{y}{2}}}$$ holds since $$\thoare {} {\phiAnd{\phiEq{x}{42}}{\phiEq{2}{2}}} {\sVarAssign{y}{2}} {\phiAnd{\phiEq{x}{42}}{\phiEq{y}{2}}}$$ does.
    
    However, the instantiation is stricter than necessary, meaning that the Hoare triple 
    $$\hoare {\qm} {\sVarAssign{y}{2}} {\phiAnd{\phiEq{x}{42}}{\phiEq{y}{2}}}$$
    is not semantically valid.
    
    While in general it is possible to derive semantically invalid Hoare triples (this is the motivation for runtime checks, see section \ref{ssec:gradual-soundness}), a gradual verifier should not deliberately produce any.
    Invalid Hoare triples that are based on annotations represent the programmers \emph{intentions} and are thus ensured using runtime checks (see definition \ref{def:gsnd}).
    On the other hand, invalid Hoare triples that are merely caused by an instantiation by the gradual verifier cannot be backed using runtime checks, as this would alter runtime behavior (in this example, a runtime check for $\phiEq{x}{42}$ is clearly unjustified).
    
    As a result, such instantiations are not “credible” (neither semantically correct nor ensured using checks) and thus cannot be used for further reasoning (like branch prediction or simplifications based on the supposed value of \ttt{x}).
    
    \begin{comment}
    Programmers may provide annotations corresponding to invalid Hoare triples, since they may be unsure how or unable to express a valid one.
    
    however this is motivated by the possibility that programmers may have context information that render the triple valid after all.
    For instance, the code in question may only be executed under circumstances that indeed cause the postcondition to be satisfied.
    
    On the other hand  invalid Hoare triples:
    The instantiation involved (here: $\phiAnd{\phiEq{x}{3}}{\phiEq{y}{2}}$) may not be used for further analysis like branch prediction, since it does not represent semantical knowledge ($\phiEq{x}{3}$ may be a false assumption).
    \end{comment}
\end{description}

%% comparison: static verifier
Note that above observations do not apply to a static verifier as the former two cases rely on the existence of the wildcard:
The weak instantiation is syntactically impossible, the too strict instantiation relied on the fact that the precondition (of the overall judgment) is a wildcard.

%% dilemma, consequence
On the other hand, decisions of the gradual verifier have the power to change the runtime behavior or let obvious inconsistencies go unnoticed.
We propose a different approach, resulting in a deterministic gradual Hoare logic which also obviates the need of injecting runtime assertions.



