Gradual verification aims to overcome drawbacks of static or dynamic verification by seamlessly combining both approaches.
As a result, programmers can freely choose the amount of static annotations to provide and thus have control over the degree of static verification used.

In this work we have developed a procedure of deriving an imperative gradually verified programming language from a statically verified one.
We have done so by adapting the work on “Abstracting Gradual Typing” (AGT, \cite{garcia2016abstracting}) by Garcia, Clark and Tanter to the setting of program verification.
Therefore, we used the concepts of abstract interpretation to give a meaning to gradual formulas.
Furthermore we generalized their notion of gradual lifting and applied it to the semantics of the statically verified language.
We also developed the notion of deterministic lifting in order to overcome a number of practical issues related to predicate lifting and transitivity.
Note that we did not make use of their way of providing gradual dynamic semantics using “evidence”, see section \ref{sec:soundness-using-evidence} for a discussion.

Finally, we illustrated our approach by gradualizing an imperative statically verified language using implicit dynamic frames to enable safe reasoning about shared mutable data structures.

%Recap, remind reader what big picture was.
%Briefly outline your thesis, motivation, problem, and proposed solution.



