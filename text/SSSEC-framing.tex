

\svl uses the concepts of implicit dynamic frames to ensure that a statement can only access memory locations (more specifically: fields) which it is guaranteed to have exclusive access to.
This is achieved by explicitly tracking access tokens $\phiAcc{\textit{<expression>}}{\textit{<field>}}$ as part of formulas throughout the entire program during verification.
    
The axiomatic semantics of \svl also make sure that access is never duplicated within or across stack frames, effectively ruling out concurrent access to any field during runtime.

Implicit dynamic frames also allows static reasoning about the values of fields during verification, i.e. as part of verification formulas.
In order to guarantee that such formulas always reflect the program state (preservation), formulas mentioning a certain field must also contain the access token to that very field:
\begin{definition}[Self-Framing]
    A formula is \textbf{self-framing/self-framed} if it contains access to all fields it mentions.
\end{definition}

% EXAMPLE of verification without self-framing.

% MAYBE We will introduce two subsets \setFormulaA and \setFormulaB ...


\begin{figure}
    \boxed{A_s \sfrme e}
    
% Inductive Semantics.sfrme
\begin{mathpar}
\inferrule* [Right=WFVar]
{
    ~
}
{
    {A} \sfrme {\ex{${x}$}}
}
\end{mathpar}

\begin{mathpar}
\inferrule* [Right=WFValue]
{
    ~
}
{
    {A} \sfrme {\ev{${v}$}}
}
\end{mathpar}

\begin{mathpar}
\inferrule* [Right=WFField]
{
    {({e}, {f})} \in {A} \\
    {A} \sfrme {e}
}
{
    {A} \sfrme {\edot{${e}$}{${f}$}}
}
\end{mathpar}


    \caption{\svl: Framing Expressions}
\end{figure}

\begin{figure}
    \boxed{A_s \sfrmphi \phi}
    
% Inductive Semantics.sfrmphi'
\begin{mathpar}
\inferrule* [right=WFTrue]
{
    ~
}
{
    {A_s} \sfrmphi {\phiTrue}
}

\inferrule* [right=WFEqual]
{
    {A_s} \sfrme {e_1} \\
    {A_s} \sfrme {e_2}
}
{
    {A_s} \sfrmphi {\phiEq {${e_1}$} {${e_2}$}}
}

\inferrule* [right=WFNEqual]
{
    {A_s} \sfrme {e_1} \\
    {A_s} \sfrme {e_2}
}
{
    {A_s} \sfrmphi {\phiNeq {${e_1}$} {${e_2}$}}
}

\inferrule* [right=WFAcc]
{
    {A_s} \sfrme {e}
}
{
    {A_s} \sfrmphi {\phiAcc {${e}$} {${f}$}}
}

\inferrule* [right=WFSepOp]
{
    A_s \sfrmphi \phi_1 \\ 
    A_s \cup \staticFP {\phi_1} \sfrmphi \phi_2
}
{A_s \sfrmphi \phi_1 * \phi_2}
\end{mathpar}

    \caption{\svl: Framing Formulas}
\end{figure}

We omit the emptyset... bla self-framing

\begin{figure}
    \boxed{\staticFP {\phi} = A_s}
    \begin{align*}
	 & \staticFP {\phiTrue}                       &  & = \emptyset                                  \\
	 & \staticFP {\phiEq {$e_1$} {$e_2$}}         &  & = \emptyset                                  \\
	 & \staticFP {\phiNeq {$e_1$} {$e_2$}}        &  & = \emptyset                                  \\
	 & \staticFP {\phiAcc {$e$} {$f$}}            &  & = \{\langle e,f \rangle\}                                  \\
	 & \staticFP {\phiCons {$\phi_1$} {$\phi_2$}} &  & = \staticFP {\phi_1} \cup \staticFP {\phi_2}
\end{align*}


    \caption{\svl: Static Footprint}
\end{figure}


%IS conservative approximation of formulas that are dynamically framed (not possible precisely anyway!)