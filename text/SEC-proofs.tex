This section contains proofs for most lemmas and theorems mentioned in this work.
Further automated proofs can be found in \cite{olydisGitHub}.

\begin{proofatend}[Proof of Lemma \ref{lemma:po-form}]~\\
    Follows from definitions \ref{def:form-implication}, \ref{def:form-eq} and the fact that $\subseteq$ induces a partial order.
\end{proofatend}

\begin{proofatend}[Proof of Lemma \ref{lem:comp-tHoare}]~\\
    We are given 
    \begin{displaymath}
    \evalphiGen{\pi_1}{\phi_1}
    \end{displaymath}
    and
    \begin{displaymath}
    \sstepConsume{\sSeq{$s_1$}{$s_2$}}{\pi_1}{\pi_3}
    \end{displaymath}
    for some $\pi_1, \pi_3 \in \setProgramState$.
    
    According to assumption \ref{frm:assume-seq-ss} (see section \ref{sec:a-statically-verified}) this implies
    \begin{displaymath}
    \sstepConsume{s_1}{\pi_1}{\pi_2} \wedge \sstepConsume{s_2}{\pi_2}{\pi_3}
    \end{displaymath}
    for some $\pi_2 \in \setProgramState$.
    Using the assumptions
    \begin{displaymath}
    \tHoare{}{\phi_1} {s_1} {\phi_2} ~\wedge~ \tHoare{}{\phi_2} {s_2} {\phi_3}
    \end{displaymath}
    we can deduce $\evalphiGen{\pi_3}{\phi_3}$.
\end{proofatend}

\begin{proofatend}[Proof of Lemma \ref{lemma:ol-impl}]~\\
    Goal:
    $$\gphiImplies{\grad{\phi_1}}{\grad{\phi_2}} \iff \exists \phi_1 \in \gamma(\grad{\phi_1}), \phi_2 \in \gamma(\grad{\phi_2}).~ \phiImplies{\phi_1}{\phi_2}$$
    
    \begin{description}
        \item[Case $\implies$]~\\
        \begin{description}
            \item[Case \tset{\gradT ImplStatic}]
            \begin{align*}
            &\gphiImplies{\phi_1}{\phi_2}\\
            \implies
            &\phiImplies{\phi_1}{\phi_2}\\
            \implies
            &(\exists \phi_1' \in \gamma(\phi_1), \phi_2' \in \gamma(\phi_2).~ \phiImplies{\phi_1'}{\phi_2'})
            \end{align*}
            
            \item[Case \tset{\gradT ImplGrad1}]
            \begin{align*}
            &\gphiImplies{\qm}{\phi}\\
            \implies
            &\phi \in \setFormulaA\\
            \implies
            &(\exists \phi_1 \in \setFormulaA.~ \phiImplies{\phi_1}{\phi})\\
            \implies
            &(\exists \phi_1 \in \gamma(\qm), \phi_2 \in \gamma(\phi).~ \phiImplies{\phi_1}{\phi_2})
            \end{align*}
            
            \item[Case \tset{\gradT ImplGrad2}]
            \begin{align*}
            &\gamma(\grad{\phi}) \neq \emptyset ~\wedge~ \phiTrue \in \gamma(\qm)\\
            \implies
            &(\exists \phi_1 \in \gamma(\grad{\phi}).~ \phiImplies{\phi_1}{\phiTrue}) ~\wedge~ \phiTrue \in \gamma(\qm)\\
            \implies
            &(\exists \phi_1 \in \gamma(\grad{\phi}), \phi_2 \in \gamma(\qm).~ \phiImplies{\phi_1}{\phi_2})
            \end{align*}
        \end{description}
        \item[Case $\impliedby$]~\\
        Given: $\phi_1 \in \gamma(\grad{\phi_1}) \wedge \phi_2 \in \gamma(\grad{\phi_2}) \wedge \phiImplies{\phi_1}{\phi_2}$
        \begin{description}
            \item[Case $\grad{\phi_2} = \qm$]~\\
            Apply $\tset{\gradT ImplGrad2}$.
            
            \item[Case $\grad{\phi_2} = \phi_2 \wedge \grad{\phi_1} = \qm$]
            \begin{align*}
            &\phi_1 \in \gamma(\qm) \wedge \phiImplies{\phi_1}{\phi_2}\\
            \implies
            &\phi_1 \in \setFormulaA \wedge \phiImplies{\phi_1}{\phi_2}\\
            \implies
            &\phi_2 \in \setFormulaA
            \end{align*}
            Apply $\tset{\gradT ImplGrad1}$.
            
            \item[Case $\grad{\phi_2} = \phi_2 \wedge \grad{\phi_1} = \phi_1$]~\\
            Apply $\tset{\gradT ImplStatic}$.
        \end{description}
    \end{description}
\end{proofatend}

\begin{proofatend}[Proof of Lemma \ref{ex:opt-lift-evalphi}]~\\
    Goal:
    $$\evalgphiGen{\pi}{\grad{\phi}} \iff \exists \phi \in \gamma(\grad{\phi}).~ \evalphiGen{\pi}{\phi}$$
    
    \begin{description}
        \item[Case $\implies$]~\\
        \begin{description}
            \item[Case \tset{\gradT EvalStatic}]
            \begin{align*}
            &\evalgphiGen{\pi}{\phi}\\
            \implies
            &\evalphiGen{\pi}{\phi}\\
            \implies
            &(\exists \phi' \in \gamma(\phi).~ \evalphiGen{\pi}{\phi'})
            \end{align*}
            
            \item[Case \tset{\gradT EvalGrad}]
            \begin{align*}
            &\evalgphiGen{\pi}{\withqmGen{\phi}}\\
            \implies
            &\evalphiGen{\pi}{\phi}\\
            \implies
            &(\exists \phi' \in \gamma(\withqmGen{\phi}).~ \evalphiGen{\pi}{\phi'})
            \end{align*}
            
        \end{description}
        \item[Case $\impliedby$]~\\
        Given: $\phi' \in \gamma(\grad{\phi}) \wedge \evalphiGen{\pi}{\phi'}$
        \begin{description}
            \item[Case $\grad{\phi} = \withqmGen{\phi}$] ~\\
            It follows from definition \ref{def:gamma-bounded-unk} that $\phiImplies{\phi'}{\phi}$ and thus $\evalphiGen{\pi}{\phi}$.
            Apply $\tset{\gradT EvalGrad}$.
            \item[Case $\grad{\phi} = \phi$] ~\\
            It follows that $\phi = \phi'$.
            Apply $\tset{\gradT EvalStatic}$.
        \end{description}
    \end{description}
\end{proofatend}

\begin{proofatend}[Proof of Lemma \ref{lemma:pred-lift-comp}]~
    \begin{description}
        \item[Introduction] 
        \begin{align*}
        &(P \circ Q)(\phi_1, \phi_3)\\
        \overset{Definition}{~\quad\quad\implies\quad\quad~}
        &(\exists \phi_2 \in \setFormula.~ P(\phi_1, \phi_2) \wedge Q(\phi_2, \phi_3))\\
        \overset{Introduction}{~\quad\quad\implies\quad\quad~}
        &(\exists \phi_2 \in \setFormula.~ \grad{P}(\phi_1, \phi_2) \wedge \grad{Q}(\phi_2, \phi_3))\\
        \overset{Definition}{~\quad\quad\implies\quad\quad~}
        &(\grad{P} \circ \grad{Q})(\phi_1, \phi_3)\\
        \overset{Definition}{~\quad\quad\implies\quad\quad~}
        &\grad{(P \circ Q)}(\phi_1, \phi_3)\\
        \end{align*}
        
        \item[Monotonicity] 
        \begin{align*}
        &\grad{(P \circ Q)}(\grad{\phi_1}, \grad{\phi_3}) ~\wedge~ \grad{\phi_1} \mpt \grad{\phi_1'} ~\wedge~ \grad{\phi_3} \mpt \grad{\phi_3'}\\
        \overset{Definition}{~\quad\quad\implies\quad\quad~}
        &(\grad{P} \circ \grad{Q})(\grad{\phi_1}, \grad{\phi_3}) ~\wedge~ \grad{\phi_1} \mpt \grad{\phi_1'} ~\wedge~ \grad{\phi_3} \mpt \grad{\phi_3'}\\
        \overset{Definition}{~\quad\quad\implies\quad\quad~}
        &(\exists \grad{\phi_2} \in \setGFormula.~ \grad{P}(\grad{\phi_1}, \grad{\phi_2}) \wedge \grad{Q}(\grad{\phi_2}, \grad{\phi_3})) ~\wedge~ \grad{\phi_1} \mpt \grad{\phi_1'} ~\wedge~ \grad{\phi_3} \mpt \grad{\phi_3'}\\
        \overset{Monotonicity}{~\quad\quad\implies\quad\quad~}
        &(\exists \grad{\phi_2} \in \setGFormula.~ \grad{P}(\grad{\phi_1'}, \grad{\phi_2}) \wedge \grad{Q}(\grad{\phi_2}, \grad{\phi_3'}))\\
        \overset{Definition}{~\quad\quad\implies\quad\quad~}
        &(\grad{P} \circ \grad{Q})(\grad{\phi_1'}, \grad{\phi_3'})\\
        \overset{Definition}{~\quad\quad\implies\quad\quad~}
        &\grad{(P \circ Q)}(\grad{\phi_1'}, \grad{\phi_3'})\\
        \end{align*}
    \end{description}
\end{proofatend}

\begin{proofatend}[Proof of Lemma \ref{lemma:sl-conj-pred}]
    \begin{description}
        \item[Introduction] 
        \begin{align*}
        &(P \wedge Q)(\phi)\\
        \overset{Definition}{~\quad\quad\implies\quad\quad~}
        &P(\phi) ~\wedge~ Q(\phi)\\
        \overset{Introduction}{~\quad\quad\implies\quad\quad~}
        &\grad{P}(\phi) ~\wedge~ \grad{Q}(\phi)\\
        \overset{Definition}{~\quad\quad\implies\quad\quad~}
        &(\grad{P} ~\wedge~ \grad{Q})(\phi)\\
        \overset{Definition}{~\quad\quad\implies\quad\quad~}
        &\grad{(P \wedge Q)}(\phi)\\
        \end{align*}
        
        \item[Monotonicity] 
        \begin{align*}
        &(\grad{P} \wedge \grad{Q})(\grad{\phi}) ~\wedge~ \grad{\phi} \mpt \grad{\phi'}\\
        \overset{Definition}{~\quad\quad\implies\quad\quad~}
        &\grad{P}(\grad{\phi}) ~\wedge~ \grad{Q}(\grad{\phi}) ~\wedge~ \grad{\phi} \mpt \grad{\phi'}\\
        \overset{Monotonicity}{~\quad\quad\implies\quad\quad~}
        &\grad{P}(\grad{\phi'}) ~\wedge~ \grad{Q}(\grad{\phi'})\\
        \overset{Definition}{~\quad\quad\implies\quad\quad~}
        &(\grad{P} \wedge \grad{Q})(\grad{\phi'})\\
        \overset{Definition}{~\quad\quad\implies\quad\quad~}
        &\grad{(P \wedge Q)}(\grad{\phi'})\\
        \end{align*}
    \end{description}
\end{proofatend}

\begin{proofatend}[Proof of Lemma \ref{lemma:sl-disj-pred}]
    \begin{align*}
    &\grad{(P \vee Q)}(\grad{\phi})\\
    \overset{Definition}{~\quad\quad\iff\quad\quad~}
    &(\grad{P} \vee \grad{Q})(\grad{\phi})\\
    \overset{Definition}{~\quad\quad\iff\quad\quad~}
    &\grad{P}(\grad{\phi}) \vee \grad{Q}(\grad{\phi})\\
    \overset{AGT Def.}{~\quad\quad\iff\quad\quad~}
    &(\exists \phi \in \gamma(\grad{\phi}).~ P(\phi)) \vee (\exists \phi \in \gamma(\grad{\phi}).~ Q(\phi))\\
    \overset{}{~\quad\quad\iff\quad\quad~}
    &(\exists \phi \in \gamma(\grad{\phi}).~ P(\phi) ~\vee~ Q(\phi))\\
    \overset{Definition}{~\quad\quad\iff\quad\quad~}
    &(\exists \phi \in \gamma(\grad{\phi}).~ (P \vee Q)(\phi))
    \end{align*}
\end{proofatend}

\begin{proofatend}[Proof of Lemma \ref{lemma:eq-fun-lift-agt}]~
    \begin{description}
        \item[Adjoint Equation] 
        \begin{align*}
        \alpha(\overline{f}(\gamma(\phi))) = f(\phi)
        \end{align*}
        Proof:
        
        $\alpha(\overline{f}(\gamma(\phi)))$ defined, since $\{ \overline{f} \}$-partial Galois connection, i.e.
        \begin{align}
        \label{frm:pgc-ass}
        \alpha(\overline{f}(\gamma(\phi))) = \alpha(\{ f(\phi) \}) = \grad{\phi}
        \end{align}
        Applying rule 1 of partial Galois connections to \ref{frm:pgc-ass}
        \begin{align}
        \label{frm:pgc-ass1}
        &\{~ f(\phi) ~\} \subseteq \gamma(\grad{\phi})\\
        \end{align}
        Applying rule 2 of partial Galois connections to \ref{frm:pgc-ass}, using $\{~ f(\phi) ~\} \subseteq \gamma(f(\phi))$
        \begin{align}
        \label{frm:pgc-ass2}
        & \grad{\phi} \sqsubseteq f(\phi)
        \end{align}
        
        Combining \ref{frm:pgc-ass1} and \ref{frm:pgc-ass2}
        \begin{align*}
        &\{~ f(\phi) ~\} \subseteq \gamma(\grad{\phi}) \subseteq \gamma(f(\phi))\\
        \implies
        & \gamma(\grad{\phi}) = \{~ f(\phi) ~\}\\
        \implies
        & \grad{\phi} = f(\phi)
        \end{align*}
        
        \item[Soundness]~
        \begin{description}
            \item[Introduction] 
            \begin{align*}
            &\grad{f}(\phi)\\
            =~
            &\alpha(\overline{f}(\gamma(\phi)))\\
            =~ 
            &f(\phi)
            \end{align*}
            
            \item[Monotonicity]~\\ 
            We assume $\grad{\phi_1}, \grad{\phi_2} \in \setGFormula$ with $\grad{\phi_1} \mpt \grad{\phi_2}$
            \begin{align*}
            &\grad{\phi_1} \mpt \grad{\phi_2}\\
            \implies
            &\gamma(\grad{\phi_1}) \subseteq \gamma(\grad{\phi_2})\\
            \implies
            &\overline{f}(\gamma(\grad{\phi_1})) \subseteq \overline{f}(\gamma(\grad{\phi_2}))\\
            \implies % closure
            &\overline{f}(\gamma(\grad{\phi_1})) \subseteq \gamma(\alpha(\overline{f}(\gamma(\grad{\phi_2}))))\\
            \implies % rule 2
            &\alpha(\overline{f}(\gamma(\grad{\phi_1}))) \mpt \alpha(\overline{f}(\gamma(\grad{\phi_2})))\\
            \end{align*}
        \end{description}
        
        \item[Optimality]~\\
        
        Proof by contradiction.
        Assume there exists a sound lifting $\grad{f'}$ such that $\grad{f'}(\grad{\phi}) \sqsubset \grad{f}(\grad{\phi})$ for some $\grad{\phi} \in \setGFormula$.
        Using the introduction rule:
        \begin{align*}
        &\forall \phi \in \setFormula.~ f(\phi) \mpt \grad{f'}(\phi)
        \end{align*}
        Using the monotonicity rule:
        \begin{align*}
        &\forall \phi \in \gamma(\grad{\phi}).~ \grad{f'}(\phi) \mpt \grad{f'}(\grad{\phi})
        \end{align*}
        Transitivity:
        \begin{align*}
        &\forall \phi \in \gamma(\grad{\phi}).~ f(\phi) \mpt \grad{f'}(\grad{\phi})\\
        \implies
        &\forall \phi \in \gamma(\grad{\phi}).~ f(\phi) \in \gamma(\grad{f'}(\grad{\phi}))\\
        \implies
        &\overline{f}(\gamma(\grad{\phi})) \subseteq \gamma(\grad{f'}(\grad{\phi}))\\
        \end{align*}
        Using rule 2 of partial Galois connections
        \begin{align*}
        &\overline{f}(\gamma(\grad{\phi})) \subseteq \gamma(\grad{f'}(\grad{\phi}))\\
        \implies
        &\alpha(\overline{f}(\gamma(\grad{\phi}))) \mpt \grad{f'}(\grad{\phi})\\
        \implies
        &\grad{f}(\grad{\phi}) \mpt \grad{f'}(\grad{\phi})\\
        \end{align*}
        Contradiction.
    \end{description}       
\end{proofatend}

\begin{proofatend}[Proof of Lemma \ref{lemma:sl-comp}]~
    \begin{description}
        \item[Introduction]
        \begin{align*}
        &g(f(\phi))\\
        \overset{Introduction~\grad{g}}{\mpt}~
        &\grad{g}(f(\phi))\\
        \overset{\substack{Introduction~\grad{f}\\\&\\Monotonicity~\grad{g}}}{\mpt}~
        &\grad{g}(\grad{f}(\phi))\\
        =~
        &\grad{g}(\grad{f}(\phi))\\
        =~
        &(\grad{g} \circ \grad{f})(\phi)\\
        =~
        &\grad{(g \circ f)}(\phi)
        \end{align*}
        
        \item[Monotonicity]
        \begin{align*}
        &\grad{\phi_1} \mpt \grad{\phi_2}\\
        \overset{Monotonicity~\grad{f}}{\quad\quad\implies\quad\quad}
        &\grad{f}(\grad{\phi_1}) \mpt \grad{f}(\grad{\phi_2})\\
        \overset{Monotonicity~\grad{g}}{\quad\quad\implies\quad\quad}
        &\grad{g}(\grad{f}(\grad{\phi_1})) \mpt \grad{g}(\grad{f}(\grad{\phi_2}))\\
        \overset{Definition}{\quad\quad\implies\quad\quad}
        &\grad{(g \circ f)}(\grad{\phi_1}) \mpt \grad{(g \circ f)}(\grad{\phi_2})
        \end{align*}
    \end{description}
\end{proofatend}

\begin{proofatend}[Proof of Lemma \ref{lem:comp-gtHoare}]~\\
    Analogous to proof of lemma \ref{lem:comp-tHoare}.
\end{proofatend}

\begin{proofatend}[Proof of Lemma \ref{lemma:tauto}]~\\
    Goal: $\gtHoare{~}{\grad{\phi_1}}{\sSeq{$\grad{s}$}{\sAssert{$\grad{\phi_2}$}}}{\grad{\phi_2}}$.
    
    According to definition \ref{def:valid-ghoare-triple} we may assume
    \begin{align}
    \label{eq:ttemp1}
    \gsstepConsume{\sSeq{$\grad{s}$}{\sAssert{$\grad{\phi_2}$}}}{\grad{\pi_1}}{\grad{\pi_2}}
    \end{align}
    and
    \begin{align}
    \evalgphiGen{\grad{\pi_1}}{\grad{\phi_1}}
    \end{align}
    for some $\grad{\pi_1}, \grad{\pi_2} \in \setGProgramState$ and have to show
    \begin{align}
    \label{eq:ttemp1g}
    \evalgphiGen{\grad{\pi_2}}{\grad{\phi_2}}
    \end{align}
    
    It follows from \ref{eq:ttemp1} and lemma \ref{lem:comp-gtHoare} that there exists $\grad{\pi} \in \setGProgramState$ such that
    \begin{align}
    \label{eq:ttemp2}
    \gsstepConsume{\sAssert{$\grad{\phi_2}$}}{\grad{\pi}}{\grad{\pi_2}}
    \end{align}
    Then \ref{eq:ttemp1g} follows by definition of the assertion statement (see example \ref{ex:ss-ra} for an illustration).
\end{proofatend}

\begin{proofatend}[Proof of Lemma \ref{lem:det2grad}]~
    \begin{description}
        \item[Introduction] 
        \begin{align*}
        &P(\phi_1, \phi_2) \\
        \overset{Introduction}{~\quad\implies\quad~} 
        &\exists \grad{\phi_2}.~ \grad{P}(\phi_1) = \grad{\phi_2}\\
        \overset{Strength}{~\quad\implies\quad~} 
        &\exists \grad{\phi_2}.~ \grad{P}(\phi_1) = \grad{\phi_2} ~\wedge~ \exists \phi \in \gamma(\grad{\phi_2}).~ P(\phi_1, \phi) \wedge \phiImplies{\phi}{\phi_2} \\
        \overset{}{~\quad\implies\quad~} 
        &\exists \grad{\phi_2}.~ \grad{P}(\phi_1) = \grad{\phi_2} ~\wedge~ \exists \phi \in \gamma(\grad{\phi_2}).~ \phiImplies{\phi}{\phi_2} \\
        \overset{}{~\quad\implies\quad~} 
        &\exists \grad{\phi_2}.~ \grad{P}(\phi_1) = \grad{\phi_2} ~\wedge~  \gphiImplies{\grad{\phi_2}}{\phi_2} \\
        \overset{Definition}{~\quad\implies\quad~} 
        &\grad{P}(\phi_1, \phi_2)
        \end{align*}
        
        \item[Monotonicity] 
        \begin{align*}
        &\grad{P}(\grad{\phi_1}, \grad{\phi_2}) ~\wedge~ \grad{\phi_1} \mpt \grad{\phi_1'} ~\wedge~ \grad{\phi_2} \mpt \grad{\phi_2'}\\
        \overset{Definition}{~\quad\implies\quad~}
        &(\exists \grad{\phi}.~ \dgrad{P}(\grad{\phi_1}) = \grad{\phi} \wedge \gphiImplies {\grad{\phi}} {\grad{\phi_2}}) ~\wedge~ \grad{\phi_1} \mpt \grad{\phi_1'} ~\wedge~ \grad{\phi_2} \mpt \grad{\phi_2'}\\
        \overset{Monotonicity}{~\quad\implies\quad~}
        &(\exists \grad{\phi}, \grad{\phi'}.~ \dgrad{P}(\grad{\phi_1'}) = \grad{\phi'} \wedge \grad{\phi} \mpt \grad{\phi'} \wedge \gphiImplies {\grad{\phi}} {\grad{\phi_2}}) ~\wedge~ \grad{\phi_2} \mpt \grad{\phi_2'}\\
        \overset{}{~\quad\implies\quad~} 
        &(\exists \grad{\phi'}.~ \dgrad{P}(\grad{\phi_1'}) = \grad{\phi'} \wedge \gphiImplies {\grad{\phi'}} {\grad{\phi_2}}) ~\wedge~ \grad{\phi_2} \mpt \grad{\phi_2'}\\
        \overset{}{~\quad\implies\quad~} 
        &(\exists \grad{\phi'}.~ \dgrad{P}(\grad{\phi_1'}) = \grad{\phi'} \wedge \gphiImplies {\grad{\phi'}} {\grad{\phi_2'}})\\
        \overset{Definition}{~\quad\implies\quad~} 
        &\grad{P}(\phi_1', \phi_2')
        \end{align*}
    \end{description}
\end{proofatend}

\begin{proofatend}[Proof of Lemma \ref{lemma:runtime-checks-reduced}]~\\
    The restriction $\envs{\grad{\phi_2'}} \cap P \subseteq \envs{\grad{\phi_2}}$ makes sure that only program states that satisfy $\grad{\phi_2}$ reach the corresponding point during execution.
    $\grad{\phi_2'}$ is ensured by \tset{\dgradT Soundness}, $P$ is explicitly checked.
    
    On the other hand, $\envs{\grad{\phi_2'}} \cap \envs{\grad{\phi_2}} \subseteq \envs{\grad{\phi_2'}} \cap P$ ensures that $P$ is not stricter than necessary.
    Assume there exists a program state $\grad{\pi} \in (\envs{\grad{\phi_2'}} \cap \envs{\grad{\phi_2}})$ with $\grad{\pi} \not \in (\envs{\grad{\phi_2'}} \cap P)$.
    It follows that $\neg P(\grad{\pi})$, i.e. a runtime exception is thrown that indicates that a runtime check fails.
    However, $\grad{\pi}$ neither violates $\grad{\phi_2'}$ nor $\grad{\phi_2}$ and is thus not expected to violate any runtime checks.
\end{proofatend}

\begin{proofatend}[Proof of Corollary \ref{cor:runtime-checks-removed}]~\\
    The condition of lemma \ref{lemma:runtime-checks-reduced} is directly implied.
\end{proofatend}

\begin{proofatend}[Proof of Lemma \ref{lemma:det-lift-comp}]~
    \begin{description}
        \item[Introduction]
        \begin{align*}
        & \phantom{\overset{}{~\quad\quad\implies\quad\quad~}}~
        P_3(\phi_1, \phi_3)\\
        & \overset{}{~\quad\quad\implies\quad\quad~} 
        \exists \phi_2.~ P_1(\phi_1, \phi_2) \wedge P_2(\phi_2, \phi_3)\\
        & \overset{Introduction}{~\quad\quad\implies\quad\quad~} 
        \exists \phi_2.~ P_1(\phi_1, \phi_2) \wedge P_2(\phi_2, \phi_3)
        \wedge (\exists \grad{\phi_2}.~ \dgrad{P_1}(\phi_1) = \grad{\phi_2})\\
        & \overset{Strength}{~\quad\quad\implies\quad\quad~} 
        \exists \phi_2.~ P_1(\phi_1, \phi_2) \wedge P_2(\phi_2, \phi_3)
        \wedge (\exists \grad{\phi_2}.~ \dgrad{P_1}(\phi_1) = \grad{\phi_2} \\
        & \quad\quad\quad\quad\quad\quad\quad\quad\quad\quad\quad\quad\quad\quad
        \wedge 
        (\exists \phi_2' \in \gamma(\grad{\phi_2}).~ \phiImplies{\phi_2'}{\phi_2}))\\
        & \overset{Lemma~\ref{ass:hl-mono}}{~\quad\quad\implies\quad\quad~} 
        \exists \phi_2.~ P_1(\phi_1, \phi_2) \wedge P_2(\phi_2, \phi_3)
        \wedge (\exists \grad{\phi_2}.~ \dgrad{P_1}(\phi_1) = \grad{\phi_2} \\
        & \quad\quad\quad\quad\quad\quad\quad\quad\quad\quad\quad\quad\quad\quad
        \wedge 
        (\exists \phi_2' \in \gamma(\grad{\phi_2}), \phi_3'.~ \phiImplies{\phi_2'}{\phi_2} \wedge P_2(\phi_2', \phi_3')))\\
        & \overset{}{~\quad\quad\implies\quad\quad~} 
        \exists \grad{\phi_2}.~ \dgrad{P_1}(\phi_1) = \grad{\phi_2} \wedge 
        (\exists \phi_2' \in \gamma(\grad{\phi_2}), \phi_3'.~ P_2(\phi_2', \phi_3'))\\
        & \overset{Introduction}{~\quad\quad\implies\quad\quad~} 
        \exists \grad{\phi_2}.~ \dgrad{P_1}(\phi_1) = \grad{\phi_2} \wedge 
        (\exists \phi_2' \in \gamma(\grad{\phi_2}), \grad{\phi_3'}.~ \dgrad{P_2}(\phi_2') = \grad{\phi_3'})\\
        & \overset{Monotonicity}{~\quad\quad\implies\quad\quad~} 
        \exists \grad{\phi_2}.~ \dgrad{P_1}(\phi_1) = \grad{\phi_2} \wedge 
        (\exists \grad{\phi_3}.~ \dgrad{P_2}(\grad{\phi_2}) = \grad{\phi_3})\\
        & \overset{}{~\quad\quad\implies\quad\quad~} 
        \exists \grad{\phi_3}.~ \dgrad{P_2}(\dgrad{P_1}(\phi_1)) = \grad{\phi_3}\\
        & \overset{}{~\quad\quad\implies\quad\quad~} 
        \exists \grad{\phi_3}.~ \dgrad{P_3}(\phi_1) = \grad{\phi_3}
        \end{align*}
        
        \item[Strength]
        \begin{align*}
        & \phantom{\overset{}{~\quad\quad\implies\quad\quad~}}~
        \dgrad{P_3}(\grad{\phi_1}) = \grad{\phi_3} \wedge \phi_1 \in \gamma(\grad{\phi_1}) \wedge P_3(\phi_1, \phi)\\
        & \overset{Defenitions}{~\quad\quad\implies\quad\quad~} 
        \exists \grad{\phi_2}, \phi'.~ 
        \dgrad{P_1}(\grad{\phi_1}) = \grad{\phi_2} \wedge \dgrad{P_2}(\grad{\phi_2}) = \grad{\phi_3} \wedge 
        \phi_1 \in \gamma(\grad{\phi_1}) \\
        & \quad\quad\quad\quad\quad\quad\quad\quad\quad\quad
        \wedge P_1(\phi_1, \phi') \wedge P_2(\phi', \phi)\\
        & \overset{Strength}{~\quad\quad\implies\quad\quad~} 
        \exists \grad{\phi_2}, \phi'.~ 
        \dgrad{P_1}(\grad{\phi_1}) = \grad{\phi_2} \wedge \dgrad{P_2}(\grad{\phi_2}) = \grad{\phi_3} \wedge 
        \phi_1 \in \gamma(\grad{\phi_1}) \\
        & \quad\quad\quad\quad\quad\quad\quad\quad\quad\quad
        \wedge P_1(\phi_1, \phi') \wedge P_2(\phi', \phi)
        \wedge (\exists \phi_2 \in \gamma(\grad{\phi_2}).~ P_1(\phi_1, \phi_2) \wedge \phiImplies{\phi_2}{\phi'}) \\
        & \overset{}{~\quad\quad\implies\quad\quad~} 
        \exists \grad{\phi_2}, \phi', \phi_2 \in \gamma(\grad{\phi_2}).~ 
        \dgrad{P_1}(\grad{\phi_1}) = \grad{\phi_2} \wedge \dgrad{P_2}(\grad{\phi_2}) = \grad{\phi_3} \wedge 
        \phi_1 \in \gamma(\grad{\phi_1}) \\
        & \quad\quad\quad\quad\quad\quad\quad\quad\quad\quad
        \wedge P_1(\phi_1, \phi_2) \wedge P_2(\phi', \phi)
        \wedge \phiImplies{\phi_2}{\phi'} \\
        & \overset{}{~\quad\quad\implies\quad\quad~} 
        \exists \grad{\phi_2}, \phi', \phi_2 \in \gamma(\grad{\phi_2}).~ 
        \dgrad{P_2}(\grad{\phi_2}) = \grad{\phi_3}  \\
        & \quad\quad\quad\quad\quad\quad\quad\quad\quad\quad
        \wedge P_1(\phi_1, \phi_2) \wedge P_2(\phi', \phi)
        \wedge \phiImplies{\phi_2}{\phi'} \\
        & \overset{Lemma~\ref{ass:hl-mono}}{~\quad\quad\implies\quad\quad~} 
        \exists \grad{\phi_2}, \phi_2 \in \gamma(\grad{\phi_2}), \phi''.~ 
        \dgrad{P_2}(\grad{\phi_2}) = \grad{\phi_3}   \\
        & \quad\quad\quad\quad\quad\quad\quad\quad\quad\quad
        \wedge P_1(\phi_1, \phi_2) \wedge P_2(\phi_2, \phi'')
        \wedge \phiImplies{\phi''}{\phi} \\
        & \overset{Strength}{~\quad\quad\implies\quad\quad~} 
        \exists \grad{\phi_2}, \phi_2 \in \gamma(\grad{\phi_2}), \phi''.~ 
        \dgrad{P_2}(\grad{\phi_2}) = \grad{\phi_3}  \\
        & \quad\quad\quad\quad\quad\quad\quad\quad\quad\quad
        \wedge P_1(\phi_1, \phi_2) \wedge P_2(\phi_2, \phi'') 
        \wedge \phiImplies{\phi''}{\phi}  \\
        & \quad\quad\quad\quad\quad\quad\quad\quad\quad\quad
        \wedge (\exists \phi_3 \in \gamma(\grad{\phi_3}).~ P_2(\phi_2, \phi_3) \wedge \phiImplies{\phi_3}{\phi''}) \\
        & \overset{}{~\quad\quad\implies\quad\quad~} 
        \exists \phi_2 \in \gamma(\grad{\phi_2}), \phi_3 \in \gamma(\grad{\phi_3}).~ 
        P_1(\phi_1, \phi_2) \wedge P_2(\phi_2, \phi_3)
        \wedge \phiImplies{\phi_3}{\phi} \\
        & \overset{Defenition}{~\quad\quad\implies\quad\quad~} 
        \exists \phi_3 \in \gamma(\grad{\phi_3}).~ 
        P_3(\phi_1, \phi_3)
        \wedge \phiImplies{\phi_3}{\phi} 
        \end{align*}
        
        \item[Monotonicity]~\\
        Compositions of monotonic functions are monotonic.
    \end{description}
\end{proofatend}

\begin{proofatend}[Proof of Lemma \ref{lemma:opt-lift-impl}]~
    \begin{description}
        \item[Soundness]
        \begin{description}
            \item[Introduction] The identity function is total.
            
            \item[Strength]~\\
            Known:
            $$\id(\grad{\phi_1}) = \grad{\phi_2} \wedge \phi_1 \in \gamma(\grad{\phi_1}) \wedge \phiImplies{\phi_1}{\phi}$$
            Goal:
            $$\exists \phi_2 \in \gamma(\grad{\phi_2}).~ \phiImplies{\phi_1}{\phi_2} \wedge \phiImplies{\phi_2}{\phi}$$
            The goal is satisfied when choosing $\phi_2 = \phi_1$
            
            \item[Monotonicity] Trivial.
        \end{description}
        \item[Optimality]~\\
        Proof by contradiction.
        
        Assume there was another, more optimal deterministic lifting $\dgrad{f} : \setGFormula \rightarrow \setGFormula$.
        Note that $\dgrad{f}$ must be total as it would otherwise not satisfy the introduction rule.
        Let $\grad{\phi} \in \setGFormula$ be a chosen such that $\dgrad{f}(\grad{\phi}) \sqsubset \id(\grad{\phi})$ (exists, since $\dgrad{f}$ is more optimal).
        
        Then there exists $\phi \in (\gamma(\id(\grad{\phi})) \backslash \gamma(\dgrad{f}(\grad{\phi}))$, i.e. $\phi \in (\gamma(\grad{\phi}) \backslash \gamma(\dgrad{f}(\grad{\phi}))$.
        Applying the strength rule to $\dgrad{f}$ and $\phiImplies{\phi}{\phi}$ we can deduce
        $$\exists \phi' \in \gamma(\dgrad{f}(\grad{\phi})).~ \phiImplies{\phi}{\phi'} ~\wedge~ \phiImplies{\phi'}{\phi}$$
        This means that $\phi \in \gamma(\dgrad{f}(\grad{\phi}))$.
        Contradiction.
    \end{description}
\end{proofatend}

\begin{proofatend}[Proof of Lemma \ref{lemma:compl-sem}]~\\
    \svl only ever executes a statement $s$ if it was successfully verified.
    After extending the Hoare logic, derivable Hoare triples are still derivable.
    Successful verification implies that there exists some Hoare derivation $\thoare{}{\phi_1}{s}{\phi_2}$ for every statement $s$ executed by the program.
    Soundness of the Hoare logic implies that the small-step semantics do not get stuck while executing $s$.
    Thus, the restricted domain of $\sstep{\cdot}{\cdot}$ can never affect reachable executions.
\end{proofatend}

\begin{proofatend}[Proof of Theorem \ref{thm:compl-and-so-to-gdpres}]~\\
    To prove \tset{\dgradT Soundness} we may assume:
    \begin{align}
    \label{frm:tmp0}
    \dgthoare{}{\grad{\phi_1}}{\grad{s}}{\grad{\phi_2}}\\
    \label{frm:tmp1}
    \gsstepConsume{\grad{s}}{\grad{\pi_1}}{\grad{\pi_2}}\\
    \label{frm:tmp2}
    \evalgphiGen{\grad{\pi_1}}{\grad{\phi_1}}
    \end{align}
    Goal: 
    \begin{align}
    \label{frm:tmp3}
    \evalgphiGen{\grad{\pi_2}}{\grad{\phi_2}}
    \end{align}
    
    \begin{description}
        \item[Case:] $\grad{\phi_1} = \phi_1$ for some $\phi_1 \in \setFormula$~\\
        We can simplify \ref{frm:tmp2} as
        \begin{align}
        \label{frm:tmpx0}
        \evalphiGen{\grad{\pi_1}}{\phi_1}
        \end{align}
        From semi-optimality of \ref{frm:tmp0} (see definition \ref{def:perf-dyn}) we can derive
        \begin{align}
        \label{frm:tmpx1}
        \thoare{}{\phi_1}{s}{\phi}
        \quad\quad\text{for some } s \in \gamma(\grad{s}),\, \phi_2 \in \setFormula
        \end{align}
        Applying the strength rule for deterministic liftings we can derive from \ref{frm:tmpx1} and \ref{frm:tmp0} that
        \begin{align}
        \label{frm:tmpx2}
        \thoare{}{\phi_1}{s}{\phi_2} ~\wedge~ \phiImplies{\phi_2}{\phi}
        \quad\quad\text{for some } \phi_2 \in \gamma(\grad{\phi_2})
        \end{align}
        From soundness (progress) of the static Hoare logic we can deduce that executing $s$ from a program state satisfying $\phi_1$ (like $\grad{\pi_1}$) does not end up in a stuck state.
        Combining this knowledge with $\ref{frm:tmp1}$ we can conclude that executing $s$ must terminate (otherwise, $\gsstep{\cdot}{\cdot}$, being the gradual lifting, could not terminate for $\grad{s} \sqsupseteq s$).
        \begin{align}
        \label{frm:tmpx3}
        \sstepConsume{s}{\pi_1}{\pi_2}
        \quad\quad\text{for some } \pi_1 \in \gamma(\grad{\pi_1}),\, \pi_2 \in \setProgramState
        \end{align}
        From soundness of the static Hoare logic we can deduce
        \begin{align}
        \label{frm:tmpx4}
        \evalphiGen{\pi_2}{\phi_2}
        \end{align}
        Applying the introduction rule of gradually lifted functions to \ref{frm:tmpx3} we can derive
        \begin{align}
        \label{frm:tmpx5}
        \gsstepConsume{s}{\pi_1}{\grad{\pi}} ~\wedge~ \pi_2 \sqsubseteq \grad{\pi}
        \quad\quad\text{for some } \grad{\pi} \in \setGProgramState
        \end{align}
        Applying the monotonicity rule of gradually lifted functions to \ref{frm:tmpx5} and \ref{frm:tmp1} we can derive that
        \begin{align}
        \label{frm:tmpx7}
        \grad{\pi} \mpt \grad{\pi_2}
        \end{align}
        It follows from lemma \ref{lemma:gradPS-form-sem} that
        \begin{align}
        \label{frm:tmpx8}
        \evalphiGen{\grad{\pi_2}}{\phi_2}
        \end{align}
        We can generalize this using $\phi_2 \in \gamma(\grad{\phi_2})$ (see \ref{frm:tmpx2})
        \begin{align}
        \label{frm:tmpx9}
        \evalgphiGen{\grad{\pi_2}}{\grad{\phi_2}}
        \end{align}
        
        \item[Case:] $\grad{\phi_1}$ partially unknown~\\
        From semi-optimality of \ref{frm:tmp1} (see definition \ref{def:perf-dyn}) we can derive
        \begin{align}
        \label{frm:tmp6}
        \sstepConsume{s}{\pi_1}{\pi_2}
        \quad\quad\text{for some } \pi_1 \in \gamma(\grad{\pi_1}),\, s \in \gamma(\grad{s}),\, \pi_2 \in \setProgramState
        \end{align}
        Recall that formula evaluation is immune to concretization (REF), so from \ref{frm:tmp2} follows
        \begin{align}
        \label{frm:tmp5}
        \evalgphiGen{\pi_1}{\grad{\phi_1}}
        \text{\quad\quad or equivalently \quad\quad}
        \evalphiGen{\pi_1}{\phi_{1a}}
        \quad\quad\text{for some } \phi_{1a} \in \gamma(\grad{\phi_1}) 
        \end{align}
        Using the monotonicity of $\gsstep{\cdot}{\cdot}$ we can deduce from \ref{frm:tmp1} and \ref{frm:tmp6} that
        \begin{align}
        \label{frm:tmp6x}
        \pi_2 \in \gamma(\grad{\pi_2})
        \end{align}
        From completeness of the static system (see definition \ref{def:completeness}) it follows from \ref{frm:tmp6} that
        \begin{align}
        \label{frm:tmp7}
        \thoare{}{\phi_{1b}}{s}{\phi} ~\wedge~ \evalphiGen{\pi_1}{\phi_{1b}}
        \quad\quad\text{for some } \phi_{1b}, \phi \in \setFormula
        \end{align}
        Due to $\envs{\pi_1}$ being a filter we can derive from $\evalphiGen{\pi_1}{\phi_{1a}}$ (\ref{frm:tmp5}) and $\evalphiGen{\pi_1}{\phi_{1b}}$ (\ref{frm:tmp7}) that
        \begin{align}
        \label{frm:tmp8}
        \evalphiGen{\pi_1}{\phi_1} ~\wedge~ \phiImplies{\phi_1}{\phi_{1a}} ~\wedge~ \phiImplies{\phi_1}{\phi_{1b}}
        \quad\quad\text{for some } \phi_1 \in \setFormula
        \end{align}
        From monotonicity of $\thoare{}{\cdot}{\cdot}{\cdot}$ in its first argument (lemma \ref{ass:hl-mono}) we can deduce
        \begin{align}
        \label{frm:tmp9}
        \thoare{}{\phi_1}{s}{\phi'} ~\wedge~ \phiImplies{\phi'}{\phi}
        \quad\quad\text{for some } \phi_2 \in \setFormula
        \end{align}
        Applying the introduction rule for deterministic liftings we can deduce
        \begin{align}
        \label{frm:tmp10}
        \dgthoare{}{\phi_1}{s}{\grad{\phi}}
        \quad\quad\text{for some } \grad{\phi} \in \setGFormula
        \end{align}
        Applying the strength rule for deterministic liftings we can derive from \ref{frm:tmp9} and \ref{frm:tmp10} that
        \begin{align}
        \label{frm:tmp11}
        \thoare{}{\phi_1}{s}{\phi_2} ~\wedge~ \phiImplies{\phi_2}{\phi'}
        \quad\quad\text{for some } \phi_2 \in \gamma(\grad{\phi})
        \end{align}
        From soundness of the static system we can deduce (using \ref{frm:tmp10}, \ref{frm:tmp6}, \ref{frm:tmp8}) that
        \begin{align}
        \label{frm:tmp12}
        \evalphiGen{\pi_2}{\phi_2}
        \end{align}
        and therefore (using \ref{frm:tmp6x} and lemma \ref{lemma:gradPS-form-sem})
        \begin{align}
        \label{frm:tmp14}
        \evalgphiGen{\grad{\pi_2}}{\phi_2}
        \end{align}
        and therefore (using \ref{frm:tmp11})
        \begin{align}
        \label{frm:tmp13}
        \evalgphiGen{\grad{\pi_2}}{\grad{\phi}}
        \end{align}
        Now, using $\phi_1 \in \gamma(\grad{\phi_1})$ (implied from the fact that $\phi_{1a} \in \gamma(\grad{\phi_1})$, $\phiImplies{\phi_1}{\phi_{1a}}$, see see section \ref{ssec:generalization}) we can apply monotonicity of deterministic liftings to \ref{frm:tmp10}, we can derive
        \begin{align}
        \label{frm:tmp15}
        \grad{\phi} \mpt \grad{\phi_2}
        \end{align}
        and therefore
        \begin{align}
        \label{frm:tmp16}
        \evalgphiGen{\grad{\pi_2}}{\grad{\phi_2}}
        \end{align}
    \end{description}
\end{proofatend}

\begin{proofatend}[Proof of Lemma \ref{lemma:gdpres-seq}]
    \begin{mathpar}
        \inferrule* [Right=seq]
        {
            \inferrule* [Right=\dgradT Soundness]
            {
                \inferrule* [Right=inversion]
                {
                    \dgthoare{~}{\grad{\phi_1}}{\sSeq{$\grad{s_1}$}{$\grad{s_2}$}}{\grad{\phi_3}}\\
                }
                {
                    \dgthoare{~}{\grad{\phi_1}}{\grad{s_1}}{\grad{\phi_2}}\\
                    \dgthoare{~}{\grad{\phi_2}}{\grad{s_2}}{\grad{\phi_3}}
                }
            }
            {
                \gtHoare{~}{\grad{\phi_1}}{\grad{s_1}}{\grad{\phi_2}}\\
                \gtHoare{~}{\grad{\phi_2}}{\grad{s_2}}{\grad{\phi_3}}
            }
        }
        {
            \gtHoare{~}{\grad{\phi_1}}{\sSeq{$\grad{s_1}$}{$\grad{s_2}$}}{\grad{\phi_3}}
        }
    \end{mathpar}
\end{proofatend}

\begin{proofatend}[Proof of Lemma \ref{lemma:snd-svlidf}]
    Omitted (proof in \cite{olydisGitHub}).
\end{proofatend}

\begin{proofatend}[Proof of Lemma \ref{lemma:ss-wd}]~\\
    For $\sstep{\cdot}{\cdot}$ to be well-defined, at most one result can be deducible per input.
    The rules in figure \ref{fig:svl-sem-dyn-sstep} are syntax directed, so we can focus on individual rules when checking for determinism.
    This can be done by looking at the source of all variables used to construct return values (i.e. the variables used on the right hand side of $\sstep{}{}$ in the conclusion).
    All those variables must either be drawn directly from the input or be uniquely specified using premises.
    
    \tset{\gradT SsSkip}
    $H, \rho, A, s, S$ are directly drawn from the input.
    
    \tset{\gradT SsFieldAssign}
    $\rho, A, s, S$ are directly drawn from the input, $H'$ is defined using premises and depends on the uniqueness of $H, o, f, v_y$.
    These are drawn from input or are result of expression evaluation which is deterministic.
    
    The same approach can be used for all remaining rules.
\end{proofatend}

\begin{proofatend}[Proof of Lemma \ref{lemma:gvlidf-sdl-hl}]~
    \begin{description}
        \item[Well-definedness] The rules are syntax directed and only can deduce at most one result per input.
        \item[Deterministic lifting] Rule-wise.
        
        \tset{HSkip}:
        See lemma \ref{lemma:opt-lift-impl}.
        
        \tset{HAlloc}, \tset{HFieldAssign}, \tset{HReturn}, \tset{HAssert}:
        Composition of lifted components as defined in \ref{sec:gradual-liftings}.
        
        Remaining rules analogous.
    \end{description}
\end{proofatend}


\begin{proofatend}[Proof of Lemma \ref{lemma:gss-wd}]~\\
    For $\gsstep{\cdot}{\cdot}$ to be well-defined, the inductive rules may allow deducing at most one return value for every input.
    For the most part, this is the case due to $\sstep{\cdot}{\cdot}$ being well-defined (see lemma \ref{lemma:ss-wd}).
    We show that the same is true for adjustments made in figure \ref{fig:gvl-sem-dyn-sstep}.
    Note the adjustments do not change the fact that the inductive rules are syntax-directed.
    
    Rule \tset{\gradT SsCall} is deterministic:
    $H, \rho, A, x, y, m, z, s, S$ are forwarded from the input, $\rho', A', r$ are uniquely determined by the premises.
    
    Rule \tset{\gradT SsCallFinish} is deterministic:
    $H, \rho, x, A, A', s, S$ are forwarded from the input, $v_r$ is uniquely determined by the premises.
\end{proofatend}

\begin{proofatend}[Proof of Lemma \ref{lemma:gss-sl}]
    For the rules copied from $\sstep{\cdot}{\cdot}$, the rules are trivially satisfied as there is no difference between gradual and non-gradual statements.
    Precision is only meaningful for call statements:
    
    \begin{description}
        \item[\tset{\gradT SsCall}]~
        \begin{description}
            \item[Introduction]~\\
            For static precondition, \tset{\gradT SsCall} is identical to \tset{SsCall}.
            
            \item[Monotonicity]~\\
            Reducing the precision of the precondition results in all permissions being passed to the topmost stack frame.
            Having more permissions than before cannot introduce runtime failures.
            Succeeding executions will thus be observationally identical after reducing precision.
        \end{description}
        
        \item[\tset{\gradT SsCallFinish}]~
        \begin{description}
            \item[Introduction]~\\
            For static postcondition, \tset{\gradT SsCallFinish} is identical to \tset{SsCallFinish}.
            
            \item[Monotonicity]~\\
            The return value of \tset{\gradT SsCallFinish} is independent of the postcondition thus monotonic w.r.t. to it.
            In terms of definedness, reducing the precision of the postcondition cannot result in premises that were satisfied before to be unsatisfied.
        \end{description}
    \end{description}
\end{proofatend}

\begin{proofatend}[Proof of Lemma \ref{lemma:pc-idf}]~\\
    Goal:
    \begin{flalign*}
    \forall \pi_1, \pi_2 \in \setProgramState.&~ \sstepConsume{s}{\pi_1}{\pi_2} \\
    \implies \exists \Gamma \in \setTypeEnv,\, \phi_1, \phi_2 \in \setFormula.&~ \thoare{\Gamma}{\phi_1}{s}{\phi_2} \wedge \evalphiGen{\pi_1}{\phi_1}
    \end{flalign*}
    for statements $s$ mentioned in the lemma.
    
    \begin{description}
        \item[$\sSkip$]~\\
        $$\thoare{\Gamma}{\phi_1}{s}{\phi_2} \wedge \evalphiGen{\pi_1}{\phi_1}$$
        trivially holds for $\phi_1 = \phi_2 = \phiTrue$ and any $\Gamma$
        
        
        \item[$\sFieldAssign {$x$} {$f$} {$y$} $]~\\
        Known: $\sstepConsume{\sFieldAssign {$x$} {$f$} {$y$}}{\pi_1}{\pi_2}$ for some $x, y \in \setVar,\, f \in \setFieldName$ and $\pi_1, \pi_2 \in \setProgramState$.
        
        We choose 
        \begin{align*}
        &\phi_1 = \phiAcc{$x$}{$f$}\\
        &\phi_2 = \phiCons{\phiCons{\phiTrue}{\phiAcc{$x$}{$f$}}}{\phiCons{\phiNeq{x}{\enull}}{\phiEq{\edot{x}{f}}{y}}}
        \end{align*}
        Then $\thoare{\Gamma}{\phi_1}{\sFieldAssign {$x$} {$f$} {$y$}}{\phi_2}$ can be deduced using \tset{HFieldAssign} (for some $\Gamma$).
        Also $\evalphiGen{\pi_1}{\phi_1}$ holds by inversion of \tset{\dgradT SsFieldAssign}.
        
        
        \item[$\sVarAssign {$x$} {$e$}$]~\\
        Known: $\sstepConsume{\sVarAssign {$x$} {$e$}}{\pi_1}{\pi_2}$ for some $x \in \setVar,\, e \in \setExpr$ and $\pi_1, \pi_2 \in \setProgramState$.
        
        We choose 
        \begin{align*}
        &\phi_1 = \accFor{$e$}\\
        &\phi_2 = \phiCons{\phiTrue}{\phiEq{$x$}{$e$}}
        \end{align*}
        Then $\thoare{\Gamma}{\phi_1}{\sVarAssign {$x$} {$e$}}{\phi_2}$ can be deduced using \tset{HVarAssign} (for some $\Gamma$).
        The remaining premises as well as $\evalphiGen{\pi_1}{\phi_1}$ hold by inversion of \tset{\dgradT SsVarAssign}. 
        
        
        \item[$\sAlloc {$x$} {$C$}$]~\\
        $\thoare{\Gamma}{\phiTrue}{\sAlloc {$x$} {$C$}}{\phi_2}$ can be deduced using \tset{HAlloc} (for some $\Gamma$ and $\phi_2$).
        Furthermore $\evalphiGen{\pi_1}{\phiTrue}$ trivially holds. 
        
        
        \item[$\sReturn {$x$}$]~\\
        $\thoare{\Gamma}{\phiTrue}{\sReturn {$x$}}{\phi_2}$ can be deduced using \tset{HAlloc} (for some $\Gamma$ and $\phi_2$).
        Furthermore $\evalphiGen{\pi_1}{\phiTrue}$ trivially holds. 
        
        
        \item[$\sAssert {$\phi$}$]~\\
        $\thoare{\Gamma}{\phi}{\sAssert {$\phi$}}{\phi}$ can be deduced using \tset{HAlloc} (for some $\Gamma$).
        Furthermore $\evalphiGen{\pi_1}{\phi}$ holds by inversion of \tset{\dgradT SsAssert}.
        
        
        \item[$\sRelease {$\phi$}$]~\\
        $\thoare{\Gamma}{\phi}{\sRelease {$\phi$}}{\phiTrue}$ can be deduced using \tset{HRelease} (for some $\Gamma$).
        Furthermore $\evalphiGen{\pi_1}{\phi}$ holds by inversion of \tset{\dgradT SsRelease}.
    \end{description}
\end{proofatend}

\begin{proofatend}[Proof of Lemma \ref{lemma:pso-idf}]~\\
    Semi-optimality of the gradual small-step semantics is trivially the case due to being identical with the small-step semantics of \svlidf.
    
    Semi-optimality of the deterministic gradual Hoare logic:
    Goal:
    \begin{align*}
    \forall \Gamma \in \setTypeEnv,\, \phi_1 \in \setFormula,\, \grad{\phi_2} \in \setGFormula.&~ \dgthoare{\Gamma}{\phi_1}{\grad{s}}{\grad{\phi_2}}\\ \implies \exists s \in \gamma(\grad{s}),\, \phi_2 \in \setFormula.&~ \thoare{\Gamma}{\phi_1}{s}{\phi_2}
    \end{align*}
    for statements $s$ mentioned in the lemma.
    
    \begin{description}
        \item[$\sSkip$]~\\
        Known: $\dgthoare{\Gamma}{\phi_1}{\sSkip}{\grad{\phi_2}}$ for some $\phi_1 \in \setProgramState,\, \grad{\phi_2} \in \setGProgramState$.
        
        Then $\thoare{\Gamma}{\phi_1}{\sSkip}{\phi_1}$ holds (\tset{HSkip}).
        
        \item[$\sFieldAssign {$x$} {$f$} {$y$}$]~\\
        Known: $\dgthoare{\Gamma}{\phi_1}{\sFieldAssign {$x$} {$f$} {$y$}}{\grad{\phi_2}}$ for some $x, y \in \setVar,\, f \in \setFieldName$ and $\phi_1 \in \setProgramState,\, \grad{\phi_2} \in \setGProgramState$.
        Applying inversion (rule \tset{\dgradT HFieldAssign}) we can deduce:
        \begin{align*}
        \wo{\phi_1}{\phiAcc{x}{f}} = \grad{\phi'}
        \end{align*}
        for some $\grad{\phi'} \in \setGFormula$.
        Expanding the definition of $\wo{\cdot}{\cdot}$
        \begin{align*}
        \underset{\phiImplies{}{}}{\min} \{~ \phi_2 \in \setFormulaB ~|~ \phiImplies{\phi_1}{\phiCons{\phiAcc{x}{f}}{$\phi_2$}} ~\} = \grad{\phi'}
        \end{align*}
        It follows that $\phiImplies{\phi_1}{\phiAcc{x}{f}}$.
        
        Then $\thoare{\Gamma}{\phi_1}{\sFieldAssign {$x$} {$f$} {$y$}}{\phiCons{\phiCons{\phiTrue}{\phiAcc{$x$}{$f$}}}{\phiCons{\phiNeq{x}{\enull}}{\phiEq{\edot{x}{f}}{y}}}}$ holds (\tset{HFieldAssign}).
        
        
        \item[$\sVarAssign {$x$} {$e$}$]~\\
        Known: $\dgthoare{\Gamma}{\phi_1}{\sVarAssign {$x$} {$e$}}{\grad{\phi_2}}$ for some $x \in \setVar,\, e \in \setExpr$ and $\phi_1 \in \setProgramState,\, \grad{\phi_2} \in \setGProgramState$.
        Applying inversion (rule \tset{\dgradT HVarAssign}) we can deduce:
        \begin{gather*}
        \phiImplies{\phi_1}{\accFor{e}}\\
        x \not \in \FV(e)
        \end{gather*}
        
        Then $\thoare{\Gamma}{\phi_1}{\sVarAssign {$x$} {$e$}}{\phiCons{\phiTrue}{\phiEq{$x$}{$e$}}}$ holds (\tset{HVarAssign}).
        
        
        \item[$\sAlloc {$x$} {$C$}$]~\\
        Known: $\dgthoare{\Gamma}{\phi_1}{\sAlloc {$x$} {$C$}}{\grad{\phi_2}}$ for some $x \in \setVar,\, C \in \setClassName$ and $\phi_1 \in \setProgramState,\, \grad{\phi_2} \in \setGProgramState$.
        
        Then $\thoare{\Gamma}{\phi_1}{\sAlloc {$x$} {$C$}}{\phiCons{\phiCons{\phiTrue}{\phiNeq{$x$}{\enull}}}{...}}$ holds (\tset{HAlloc}).
        
        
        \item[$\sReturn {$x$}$]~\\
        Known: $\dgthoare{\Gamma}{\phi_1}{\sReturn {$x$}}{\grad{\phi_2}}$ for some $x \in \setVar$ and $\phi_1 \in \setProgramState,\, \grad{\phi_2} \in \setGProgramState$.
        
        Then $\thoare{\Gamma}{\phi_1}{\sReturn {$x$}}{\phiCons{\phiTrue}{\phiEq{\eresult}{$x$}}}$ holds (\tset{HReturn}).
        
        
        \item[$\sAssert {$\phi$}$]~\\
        Known: $\dgthoare{\Gamma}{\phi_1}{\sAssert {$\phi$}}{\grad{\phi_2}}$ for some $\phi \in \setFormula$ and $\phi_1 \in \setProgramState,\, \grad{\phi_2} \in \setGProgramState$.
        Applying inversion (rule \tset{\dgradT HAssert}) we can deduce:
        \begin{gather*}
        \gphiImpliesEv{\phi_1}{\phi}{\grad{\phi'}}
        \end{gather*}
        for some $\grad{\phi'} \in \setGFormula$.
        By definition of $\gphiImpliesEv{\cdot}{\cdot}{\cdot}$ it follows that $\phiImplies{\phi_1}{\phi}$.
        
        Then $\thoare{\Gamma}{\phi_1}{\sAssert {$\phi$}}{\phi_1}$ holds (\tset{HAssert}).
        
        
        \item[$\sRelease {$\phi$}$]~\\
        Known: $\dgthoare{\Gamma}{\phi_1}{\sRelease {$\phi$}}{\grad{\phi_2}}$ for some $\phi \in \setFormula$ and $\phi_1 \in \setProgramState,\, \grad{\phi_2} \in \setGProgramState$.
        Applying inversion (rule \tset{\dgradT HRelease}) we can deduce:
        \begin{gather*}
        \gphiImpliesEv{\phi_1}{\phi}{\grad{\phi'}}
        \end{gather*}
        for some $\grad{\phi'} \in \setGFormula$.
        By definition of $\gphiImpliesEv{\cdot}{\cdot}{\cdot}$ it follows that $\phiImplies{\phi_1}{\phi}$.
        
        Then $\thoare{\Gamma}{\phi_1}{\sRelease {$\phi$}}{\phiTrue}$ holds (\tset{HRelease}).
    \end{description}
\end{proofatend}


\begin{proofatend}[Proof of Lemma \ref{lemma:is-decl}]~\\
    Assumption:
    \begin{align}
    \dgthoare{\Gamma}{\grad{\phi}} {\sSeq {\sDeclare {${T}$} {${x}$}} {$s$}} {\grad{\phi'}}
    \end{align}
    Using inversion of rule $\tset{\dgradT HDeclare}$ we know that
    \begin{gather}
    \label{frm:tmpp1}
    {x} \not\in \dom{\Gamma} \\
    \label{frm:tmpp2}
    {x} \not \in {\FV(\grad{\phi})} \\
    \dgthoare {{\Gamma}, {x} : {T}} {\gphiCons{${\phiEq {${\ex{${x}$}}$} {${\ev{${\defaultValue{${T}$}}$}}$}}$}{$\grad{\phi}$}} {s} {\grad{\phi'}}
    \end{gather}
    Using the induction hypothesis we may apply \tset{\dgradT Soundness} to $s$, deducing
    \begin{gather}
    \label{frm:tmpp4}
    \gtHoare {} {\gphiCons{${\phiEq {${\ex{${x}$}}$} {${\ev{${\defaultValue{${T}$}}$}}$}}$}{$\grad{\phi}$}} {s} {\grad{\phi'}}
    \end{gather}
    Furthermore we can show that
    \begin{gather}
    \label{frm:tmpp5}
    \gtHoare {} {\grad{\phi}} {\sDeclare {${T}$} {${x}$}} {\gphiCons{${\phiEq {${\ex{${x}$}}$} {${\ev{${\defaultValue{${T}$}}$}}$}}$}{$\grad{\phi}$}}
    \end{gather}
    holds (using \ref{frm:tmpp2}, find proof below).
    Using lemma \ref{lem:comp-gtHoare} we can combine \ref{frm:tmpp4} and \ref{frm:tmpp5} to derive the goal:
    \begin{align}
    \gtHoare{}{\grad{\phi}} {\sSeq {\sDeclare {${T}$} {${x}$}} {$s$}} {\grad{\phi'}}
    \end{align}
    
    Proof for \ref{frm:tmpp5}:
    Assumptions:
    \begin{gather}
    \label{frm:tmpp7}
    \gsstepConsume{\sDeclare {${T}$} {${x}$}}{\pi_1}{\pi_2}\\
    \label{frm:tmpp8}
    \evalgphiGen {\pi_1} {\grad{\phi}}
    \end{gather}
    for some $\pi_1, \pi_2 \in \setGProgramState$.
    Goal:
    \begin{gather}
    \evalgphiGen {\pi_2} {\gphiCons{${\phiEq {${\ex{${x}$}}$} {${\ev{${\defaultValue{${T}$}}$}}$}}$}{$\grad{\phi}$}}
    \end{gather}
    Applying rule inversion (\tset{SsDeclare}) to \ref{frm:tmpp7} we can see that $\pi_2$ is like $\pi_1$, but with $x$ set to $\defaultValue{${T}$}$ (and the assertion consumed).
    It follows that
    \begin{gather}
    \evalgphiGen {\pi_2} {\phiEq {${\ex{${x}$}}$} {${\ev{${\defaultValue{${T}$}}$}}$}}
    \end{gather}
    Since we know that $\grad{\phi}$ does not contain $x$ (see \ref{frm:tmpp2}), we can deduce that
    \begin{gather}
    \evalgphiGen {\pi_2} {\grad{\phi}}
    \end{gather}
    The goal follows (the separating conjunction acts like classical conjunction since ${\phiEq {${\ex{${x}$}}$} {${\ev{${\defaultValue{${T}$}}$}}$}}$ does not mention the heap).
\end{proofatend}

\begin{proofatend}[Proof of Lemma \ref{lemma:is-call}]~\\
    Assumption:
    \begin{align}
    \dgthoare {\Gamma} {\grad{\phi}} {{\sCall {${x}$} {${y}$} {${m}$} {${z'}$}}} {\gphiCons{$\grad{\phi'}$}{$\grad{\phi_q}$}}
    \end{align}
    Using inversion of rule $\tset{\dgradT HCall}$ we know that
    \begin{gather}
    \label{frm:tmpp1a}
    \wo {\wo {\grad{\phi}} {x}} {\grad{\phi_p}} = \grad{\phi'}\\
    {\mmethod{{C}, {m}}} = {{\method {${T_r}$} {${m}$} {${T_p}$} {${z}$} {${\contract {$\grad{\phi_{pre}}$} {$\grad{\phi_{post}}$}}$} {$s$}}} \\
    \gphiImplies{\grad{\phi}}{\gphiCons{${\phiNeq {${\ex{${y}$}}$} {${\ev{${\enull}$}}$}}$}{$\grad{\phi_p}$}} \\
    x \neq y \wedge x \neq z' \\
    \grad{\phi_p} = {\grad{\phi_{pre}}[{y}, {z'} / {\ethis}, {{z}}]} \\
    \label{frm:tmpp5a}
    \grad{\phi_q} = {\grad{\phi_{post}}[{y}, {z'}, {x} / {\ethis}, {{z}}, {\eresult}]}
    \end{gather}
    From gradual well-formedness (section \ref{ssec:well-formedness}) it follows that
    \begin{gather}
    \gthoare {z : T_p, \ethis : C, \eresult : T_r} {\grad{\phi_{pre}}} {s} {\grad{\phi_{post}}} \\
    \FV(\grad{\phi_{pre}}) \subseteq \{ z, \ethis \} \\
    \label{frm:tmpp8a}
    \FV(\grad{\phi_{post}}) \subseteq \{ z, \ethis, \eresult \} \\
    \sfrmgphi \grad{\phi_{pre}} \\
    \sfrmgphi \grad{\phi_{post}} \\
    \label{frm:tmpp11a}
    z, \ethis \not \in \mods(s)
    \end{gather}
    %\begin{comment2}
    %Using the induction hypothesis we may assume that
    %\begin{gather}
    %\label{frm:tmpp100a}
    %\gtHoare {} {\grad{\phi_{pre}}} {\textit{<method body>}} {\grad{\phi_{post}}}
    %\end{gather}
    %(note that the method body is $s$, combined with necessary measures to ensure the postcondition).
    %\end{comment2}
    As shown below, it follows that 
    \begin{align}
    \label{frm:tmpp101a}
    \gtHoare {} {\grad{\phi}} {{\sCall {${x}$} {${y}$} {${m}$} {${z'}$}}} {\gphiCons{$\grad{\phi'}$}{$\grad{\phi_q}$}}
    \end{align}
    
    Proof for \ref{frm:tmpp101a}:
    Assumptions:
    \begin{gather}
    \label{frm:tmpp1000a}
    \gsstepConsume{\sCall {${x}$} {${y}$} {${m}$} {${z'}$}}{\pi_1}{\pi_2}\\
    \label{frm:tmpp1001a}
    \evalgphiGen {\pi_1} {\grad{\phi}}
    \end{gather}
    for some $\pi_1, \pi_2 \in \setGProgramState$.
    Goal:
    \begin{gather}
    \evalgphiGen {\pi_2} {\gphiCons{$\grad{\phi'}$}{$\grad{\phi_q}$}}
    \end{gather}
    Applying rule inversion to \ref{frm:tmpp1000a} we derive that there exists $\pi_{pre}, \pi_{post} \in \setGProgramState$ such that
    \begin{gather}
    \label{frm:tmpp1010a}
    \gsstep{\pi_1}{\pi_{pre}}\\
    \label{frm:tmpp1011a}
    \gsstepConsume{s}{\pi_{pre}}{\pi_{post}}\\
    \label{frm:tmpp1012a}
    \gsstep{\pi_{post}}{\pi_2}
    \end{gather}
    where \ref{frm:tmpp1010a} is determined by \tset{\gradT SsCall} and \ref{frm:tmpp1012a} by \tset{\gradT SsCallFinish}.
    Applying rule inversion to \ref{frm:tmpp1012a} it follows that
    \begin{gather}
    \evalgphiGen {\pi_{post}} {\grad{\phi_{post}}}
    \end{gather}
    Combining \ref{frm:tmpp5a} and \ref{frm:tmpp8a} we can show that $\grad{\phi_{q}}$ may only contain $y, z', x$.
    \tset{\gradT SsCallFinish} assigns $\eresult$ to $x$, which reflects the substitution in \ref{frm:tmpp5a}.
    Because of \ref{frm:tmpp11a} $\grad{\phi_{q}}$ does not contain false information about $y$ and $z'$ (as described in section \ref{sec:well-formedness}).
    It follows that
    \begin{gather}
    \label{frm:tmpp2010a}
    \evalgphiGen {\pi_2} {\grad{\phi_{q}}}
    \end{gather}
    Since the variables and fields mentioned in $\grad{\phi'}$ are guaranteed to be disjoint from those in $\grad{\phi_p}$ and due to \ref{frm:tmpp1a}, those memory locations cannot be changed by the call.
    Furthermore $x$ is not in $\grad{\phi'}$ such that there cannot be a contradiction to the knowledge in $\grad{\phi_{q}}$ (which may contain information of $x$, provided by the call).
    It follows that
    \begin{gather}
    \label{frm:tmpp2011a}
    \evalgphiGen {\pi_2} {\grad{\phi'}}
    \end{gather}
    The goal follows from \ref{frm:tmpp2010a} and \ref{frm:tmpp2011a} using the fact that the formulas contain disjoint memory locations.
\end{proofatend}

\begin{proofatend}[Proof of Lemma \ref{lemma:is-hold}]
    Analogous to previous proofs.
\end{proofatend}

\begin{proofatend}[Proof of Lemma \ref{lemma:is-seq}]
    Analogous to lemma \ref{lemma:gdpres-seq}.
\end{proofatend}

\begin{proofatend}[Proof of Theorem \ref{thm:gvlidf-dgsnd}]~\\
    Structural induction over statement $s$.
    Most cases are handled by theorem \ref{thm:compl-and-so-to-gdpres} and lemmas \ref{lemma:pc-idf} and \ref{lemma:pso-idf}.
    The remaining cases are covered by lemmas 
    \ref{lemma:is-decl},
    \ref{lemma:is-call},
    \ref{lemma:is-hold} and
    \ref{lemma:is-seq}.
\end{proofatend}

\begin{proofatend}[Proof of Lemma \ref{lemma:oh-free}]~\\
    We prove that 
    \begin{align}
    \begin{split}
    \label{frm:oh-free1}
    \forall \phi_1, \phi_{2b} \in \setFormula,\, s \in \setStmt.~
    &\thoare{}{\phi_1}{s}{\phi_{2b}} \\
    \implies 
    (\exists \phi_{2a} \in \setFormula.~ &\dgthoare{}{\phi_1}{s}{\phi_{2a}} \wedge \phiImplies{\phi_{2a}}{\phi_{2b}})
    \end{split}
    \end{align}
    As a consequence, all reasoning that allows the static system to be runtime check free must also apply to the gradual system (if all annotations are static).
    
    Proof for \ref{frm:oh-free1}:
    \begin{align*}
    \thoare{}{\phi_1}{s}{\phi_{2b}}
    \end{align*}
    for some $\phi_1, \phi_{2b} \in \setFormula,\, s \in \setStmt$.
    Using the introduction rule of deterministic liftings it follows that
    \begin{align*}
    \dgthoare{}{\phi_1}{s}{\grad{\phi_2}}
    \end{align*}
    for some $\grad{\phi_2} \in \setGFormula$.
    According to assumption \ref{lemma-frm:oh-free}, $\grad{\phi_2} = \phi_{2a}$ for some $\phi_{2a} \in \setFormula$.
    Using the strength rule of deterministic liftings it follows immediately that $\phiImplies{\phi_{2a}}{\phi_{2b}}$.
    
    \begin{comment}
    Proof for claim that optimality of $\dgthoare{}{\cdot}{\cdot}{\cdot}$ implies \ref{lemma-frm:oh-free} if \tset{\dgradT Soundness} holds.
    Let $\Phi \defeq \{~ \phi_2 ~|~ \thoare{}{\phi_1}{s}{\phi_2} ~\}$ for some  $\phi_1 \in \setFormula,\, s \in \setStmt$.
    $\Phi$ is the set of all deducible postconditions for given precondition and statement.
    If $\Phi$ is not empty, then the introduction rule of deterministic liftings implies that
    $\dgthoare{}{\phi_1}{s}{\grad{\phi_2}}$ for some $\grad{\phi_2} \in \setGFormula$.
    \begin{description}
    \item[Case 1: $\exists \pi_1, \pi_2.~ \sstepConsume{s}{\pi_1}{\pi_2} \wedge \evalphiGen{\pi_1}{\phi_1}$]~\\
    Due to soundness of the static system, it follows that
    \begin{displaymath}
    \forall \phi_2 \in \Phi.~ \evalphiGen{\pi_2}{\phi_2}
    \end{displaymath}
    \end{description}
    \end{comment}
\end{proofatend}

\begin{proofatend}[Proof of Lemma \ref{lemma:gc-nonex}]~\\
    It is essential that $\setFormula$ cannot express knowledge like 
    \begin{align}
    \label{ex-frm:appro}
    \ttt{x} \in \{1,2,3\}
    \end{align}
    since there is neither a logical disjunction (that would allow writing $\phiOr{\phiEq{x}{1}}{\phiOr{\phiEq{x}{2}}{\phiEq{x}{3}}}$) nor an inequality operator \ttt{<=} (that would allow writing $\phiCons{\ttt{(1 <= x)}}{\ttt{(x <= 3)}}$) nor any other sufficiently expressive syntax.
    
    Hence, this knowledge can only be approximated, e.g. as
    \begin{align*}
    \phiAnd{\phiAnd{\phiNeq{x}{0}}{\phiNeq{x}{4}}}{\phiNeq{x}{5}}
    \end{align*}
    Now, let $\overline{\phi}$ be the set of all approximation for $\ref{ex-frm:appro}$, i.e.
    \begin{align*}
    \label{ex-frm:appro-set}
    \overline{\phi} = \{~ \phi \in \setFormula ~|~ \envs{\phiEq{x}{1}} \cup \envs{\phiEq{x}{2}} \cup \envs{\phiEq{x}{3}} \subseteq \envs{\phi} ~\}
    \end{align*}
    Then there exists no least (most precise) element $\grad{\phi}$ that conservatively over-approximates $\overline{\phi}$.
    As a consequence $\alpha(\overline{\phi})$ cannot be defined.
    
    Proof by contradiction: \\
    Assume there exists a least element $\grad{\phi}$ that conservatively approximates $\overline{\phi}$, i.e. $\overline{\phi} \subseteq \gamma(\grad{\phi})$.
    Then $\grad{\phi}$ cannot contain an equality including \ttt{x}.
    As a result, $\grad{\phi}$ can only make statements about \ttt{x} using inequalities.
    Since $\grad{\phi}$ is finite, there must be an integer that has not yet been excluded from the set of possible values for \ttt{x}.
    Therefore, adding a corresponding inequality to $\grad{\phi}$ yields a gradual formula that is more precise than $\grad{\phi}$ and still conservatively approximates $\overline{\phi}$.\\
    This contradicts the assumption that $\grad{\phi}$ is the least element with that property.
\end{proofatend}

\begin{proofatend}[Proofs of Lemma \ref{ex:idf-opt-lift-evalphi}, \ref{ex:idf-opt-lift-sfrmphi}, \ref{lemma:impl-gimpl}, \ref{lemma:frm-extr}, \ref{lemma:var-extr}]
    Omitted (proofs in \cite{olydisGitHub}).
\end{proofatend}

\begin{proofatend}[Proof of Theorem \ref{thm:gnf}]~\\
    Omitted.
    
    Intuition behind the normal form:\\
    Recall that gradual formulas $\withqm{\phi_1}$ and $\withqm{\phi_2}$ are considered equal iff $\gamma(\withqm{\phi_1}) = \gamma(\withqm{\phi_1})$.
    The normal form makes use of the fact that concretizations of partially unknown formulas contain only self-framed formulas (see section \ref{sec:cs-gradual-formulas}).
    
    \begin{lemma}[Mentioning a Field Implies Access]\label{lemma:mention}~\\
        For any formula $\phi$ mentioning field $\edot{$x$}{$f$}$:
        $$\forall \phi' \in \gamma(\withqm{\phi}), \phiImplies{\phi'}{\phiAcc{$x$}{$f$}}$$
    \end{lemma}
    In other words: Merely mentioning a field will make sure that the concretization contains appropriate framing.
    We say access to a field is “restored” by the concretization.
    This is a helpful observation for justifying \emph{removal} of accessibility predicates from the static part.
    
    \begin{example}{Removal of Accessibility Predicate}\label{ex:rem-acc}~\\
        Using lemma \ref{lemma:mention} one can derive
        \begin{displaymath}
        \gamma(\withqm{\phiCons{\phiAcc{x}{f}}{\phiEq{x.f}{3}}}) = \gamma(\withqm{\phiEq{x.f}{3}})
        \end{displaymath}
        The accessibility predicate can be dropped since it is restored by concretization due to $\phiEq{x.f}{3}$ mentioning field \ttt{x.f}.
    \end{example}
    
    Note, however, that simply dropping access from the static part may not result in an equivalent gradual formula (like in example \ref{ex:rem-acc}) for two reasons:
    \begin{description}
        \item[1. No more mentions]~\\
        Dropping $\phiAcc{$x$}{$f$}$ might result in $\edot{$x$}{$f$}$ not being mentioned in the formula anymore, so there would be no more reason for the concretization to “restore” $\phiAcc{$x$}{$f$}$ in all concretizations.
        \begin{example}{}\label{ex:rem-acc-1}~\\
            \begin{displaymath}
            \gamma(\withqm{\phiCons{\phiAcc{x}{f}}{\phiEq{p}{3}}}) \neq \gamma(\withqm{\phiEq{p}{3}})
            \end{displaymath}
        \end{example}
        
        \item[2. Aliasing]~\\
        In general there are different ways in which access to multiple (syntactically different) fields can be restored.
        \begin{example}{Ambiguous Framing}\label{ex:rem-acc-1x}~\\
            Dropping all accessibility predicates from $$\phiCons{\phiCons{\phiAcc{a}{f}}{\phiAcc{b}{f}}}{\phiCons{\phiEq{a.f}{3}}{\phiEq{b.f}{x}}}$$
            results in
            $$\phiCons{\phiEq{a.f}{3}}{\phiEq{b.f}{x}}$$
            However, $\gamma(\withqm{\phiCons{\phiEq{a.f}{3}}{\phiEq{b.f}{x}}})$ contains
            $$\phiCons{\phiCons{\phiAcc{a}{f}}{\phiEq{a}{b}}}{\phiCons{\phiEq{a.f}{3}}{\phiEq{a.f}{x}}}$$
            whereas $\gamma(\withqm{\phiCons{\phiCons{\phiAcc{a}{f}}{\phiAcc{b}{f}}}{\phiCons{\phiEq{a.f}{3}}{\phiEq{b.f}{x}}}})$
            does not, i.e.
            \begin{displaymath}
            \gamma(\withqm{\phiCons{\phiCons{\phiAcc{a}{f}}{\phiAcc{b}{f}}}{\phiCons{\phiEq{a.f}{3}}{\phiEq{b.f}{x}}}}) \neq
            \gamma(\withqm{\phiCons{\phiEq{a.f}{3}}{\phiEq{b.f}{x}}})
            \end{displaymath}
        \end{example}
        %In other words, the possibility of aliasing may result in a variety of re-framed formulas that are not equivalent with the original one.
        % Elaborate in more detail why this is bad?
        % Also: this is where dominators play in... draw the line? How far?
    \end{description}
    
    Fortunately, we can prevent both problems from occurring by carefully preparing the static part before dropping all access, resulting in the following two-step approach:
    
    \begin{description}
        \item[1. Enhancement]~\\
        Enrich the static part to counteract above problems, i.e. to enforce that access is restored exactly the right way.
        This is achieved by appending certain implications of the access-terms as conjunctive terms:
        \begin{description}
            \item [$\phiAcc{$x$}{$f$} \implies \phiEq{\edot{$x$}{$f$}}{\edot{$x$}{$f$}}$]~\\
            Access to a field implicitly guarantees that it evaluates to some value (see formula semantics in figure \ref{fig:svl-evalphi}).
            The formula $\phiEq{\edot{$x$}{$f$}}{\edot{$x$}{$f$}}$ has the same effect, i.e. it can be appended as a conjunctive term without altering the formula semantics, but making sure that information is not lost when dropping the accessibility predicates.
            \begin{example}{Example \ref{ex:rem-acc-1} revised}\label{ex:rem-acc-2}~
                \begin{align*}
                &\gamma(\withqm{\phiCons{\phiAcc{x}{f}}{\phiEq{p}{3}}}) \\
                =~
                &\gamma(\withqm{\phiCons{\phiCons{\phiAcc{x}{f}}{\phiEq{p}{3}}}{\phiEq{x.f}{x.f}}})\\
                =~
                &\gamma(\withqm{\phiCons{\phiEq{p}{3}}{\phiEq{x.f}{x.f}}})
                \end{align*}
            \end{example}
            
            \item [$\phiCons{\phiAcc{$x$}{$f$}}{\phiAcc{$y$}{$f$}} \implies \phiNeq{$x$}{$y$}$]~\\
            Having access to the same field of different expressions actively prevents those expressions to alias due to the formula semantics.
            A corresponding inequality has the same effect, i.e. it can be appended as a conjunctive term without altering the formula semantics, but making sure that information is not lost when dropping the accessibility predicates.
            \begin{example}{Example \ref{ex:rem-acc-1x} revised}\label{ex:rem-accx-2}~
                \begin{align*}
                &\gamma(\withqm{\phiCons{\phiCons{\phiAcc{a}{f}}{\phiAcc{b}{f}}}{\phiCons{\phiEq{a.f}{3}}{\phiEq{b.f}{x}}}}) \\
                =~
                &\gamma(\withqm{\phiCons{\phiCons{\phiCons{\phiAcc{a}{f}}{\phiAcc{b}{f}}}{\phiCons{\phiEq{a.f}{3}}{\phiEq{b.f}{x}}}}{\phiNeq{a}{b}}})\\
                =~
                &\gamma(\withqm{\phiCons{\phiCons{\phiEq{a.f}{3}}{\phiEq{b.f}{x}}}{\phiNeq{a}{b}}})
                \end{align*}
            \end{example}
        \end{description}
        As illustrated above, the normal form $\snorm{\cdot}$ will append corresponding conjunctive terms in every possible way, i.e. accounting for all (pairs of) access-terms.
        It is worth noting that this process preserves equality of the formula as only terms are added that were implied by the original formula, anyway.
        
        \item[2. Delinearization]~\\
        All accessibility predicates are dropped.
        Due to above measures, they are restored by concretization correctly.
    \end{description}
\end{proofatend}

\begin{proofatend}[Proof of Lemma \ref{lemma:mention}]~\\
    Note that a formula mentioning the field $\edot{$x$}{$f$}$ cannot be implied by a formula that does not mention that field.
    This is a direct consequence of the formula semantics (see figure \ref{fig:svl-evalphi}).
    
    E.g. $\phiImplies{\phiTrue}{\phiEq{x.f}{x.f}}$ is not true, since $\phiEq{x.f}{x.f}$ actually ensures that \ttt{x.f} evaluates (to any value), whereas $\phiTrue$ does not (there, \ttt{x} may not be defined or field \ttt{f} may not exist on the value \ttt{x} evaluates to).
    
    Due to self-framing, all formulas in $\gamma(\withqm{\phi})$ must therefore contain $\phiAcc{x}{f}$.
\end{proofatend}