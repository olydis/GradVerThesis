
Optimality of the deterministic lifting does not imply optimality of the obtained gradual lifting (see \ref{cex:opt-det2grad}).




%% GNF

Recall that gradual formulas $\withqm{\phi_1}$ and $\withqm{\phi_2}$ are considered equal iff $\gamma(\withqm{\phi_1}) = \gamma(\withqm{\phi_1})$.
The normal form makes use of the fact that concretizations contain only self-framed formulas.

Lemma:
For any formula $\phi$ mentioning $\edot{$x$}{$f$}$:
$$\forall \hat{\phi} \in \gamma(\withqm{\phi}), \hat{\phi} \implies \phiAcc{$x$}{$f$}$$

In other words: Merely mentioning a field will make sure that the concretization contains appropriate framing.
This is a good starting point to justify removal of access-terms from the static part.
Note, however, that just dropping all access from the static part may not result in an equivalent gradual formula for two reasons:
\begin{description}
	\item[1. Mentioning]~\\
	Dropping $\phiAcc{$x$}{$f$}$ might result in $\edot{$x$}{$f$}$ not being mentioned in the formula anymore, so there would be no more reason for the access to be restored by concretization.
	
	\item[2. Aliasing]~\\
	In general there are different ways in which access to multiple fields can be restored (this is were linear logic plays in).
	Example: Dropping all access from $\phiCons{\phiCons{\phiAcc{a}{f}}{\phiAcc{b}{f}}}{\phiCons{\phiEq{a.f}{3}}{\phiEq{b.f}{x}}}$
	results in
	$\phiCons{\phiEq{a.f}{3}}{\phiEq{b.f}{x}}$
	which might be re-framed as
	$\phiCons{\phiCons{\phiAcc{a}{f}}{\phiEq{a}{b}}}{\phiCons{\phiEq{a.f}{3}}{\phiEq{a.f}{x}}}$.
	In other words, the possibility of aliasing may result in a variety of re-framed formulas that are not equivalent with the original one.
	% Elaborate in more detail why this is bad?
	% Also: this is where dominators play in... draw the line? How far?
\end{description}

Fortunately, we can prevent both problems from occurring by carefully preparing the static part before dropping all access, resulting in the following two-step approach:

\begin{description}
	\item[1. Enhancement]~\\
	Enrich the static part to counteract above problems, i.e. to enforce that access is restored exactly the right way.
	This is achieved by simply spelling out certain implications of the access-terms:
	\begin{description}
		\item [$\phiAcc{$x$}{$f$} \implies \phiEq{\edot{$x$}{$f$}}{\edot{$x$}{$f$}}$]~\\
		Access to a field implicitly guarantees that it actually evaluates to some (arbitrary) value.
		Note that $\phiEq{\edot{$x$}{$f$}}{\edot{$x$}{$f$}}$ is not a logical tautology (i.e. it is not implies by $\phiTrue$), since it indeed makes sure that $\edot{$x$}{$f$}$ evaluates, whereas $\phiTrue$ does not.
		The bottom line is that $\edot{$x$}{$f$}$ is being mentioned even after dropping $\phiAcc{$x$}{$f$}$, therefore solving the first problem. 
		\item [$\phiCons{\phiAcc{$x$}{$f$}}{\phiAcc{$y$}{$f$}} \implies \phiNeq{$x$}{$y$}$]~\\
		Having access to the same field of different expressions actively prevents those expressions to ever alias.
		Spelling out this restriction by adding the corresponding inequality also prevents re-framing in a way that relies on aliasing.
		The bottom line is that any valid re-framing must restore two distinct access-terms, therefore solving the second problem.
	\end{description}
	We enhance the non-linear part of our formula by spelling out above implications in every possible way, i.e. accounting for all (pairs of) access-terms.
	It is worth noting that this enhancement preserves equality of the formula as only terms are added that were implied by the original formula, anyway.
	
	\item[2. Delinearization]~\\
	All access-terms are dropped.
\end{description}