
% TODO: we dissect the gradual lifting notion of AGT
In this section, we assume that we are dealing with a binary predicate $P \subseteq \setFormula \times \setFormula$.
The concepts are directly applicable to predicates with different arity or with additional non-formula parameters.
The lifted version we are targeting has signature $\grad{P} \subseteq \setGFormula \times \setGFormula$.
W.l.o.g. we further assume that $P$ appears unnegated in the axiomatic semantics (otherwise we simply regard the negation of that predicate as $P$).

Restrictions imposed by the gradual guarantee:
\begin{description}
    \item[Introduction of Gradual System]~\\
    Having source code that is considered valid by the static verification system, the same source code must be considered valid by the gradual verification system.
    In other words, switching to the gradual system may never “break the code”.
    This means that arguments satisfying $P$ must satisfy $\grad{P}$:
    \begin{displaymath}
    \forall \phi_1, \phi_2 \in \setFormula.~ P(\phi_1, \phi_2) \implies \grad{P}(\phi_1, \phi_2)
    \end{displaymath}
    or equivalently, using set notation
    \begin{displaymath}
    P \subseteq \grad{P}
    \end{displaymath}
    
    \item[Reducing Precision/Monotonicity???]~\\
    A central point of a gradual verification system is enabling programmers to specify contracts with less precision.
    Source code that is rejected by the verifier might get accepted after reducing precision.
    If the opposite would happen, though, that would be highly counter-intuitive and ...??? workflow.
    To prevent such behavior, we expect satisfied predicates to still be satisfied after reducing the precision of arguments:
    
    \begin{displaymath}
    \forall \grad{\phi_1}, \grad{\phi_2}, \grad{\phi_1'}, \grad{\phi_2'} \in \setGFormula.~ \grad{\phi_1} \sqsubseteq \grad{\phi_1'} \wedge \grad{\phi_2} \sqsubseteq \grad{\phi_2'} \wedge \grad{P}(\grad{\phi_1}, \grad{\phi_2}) \implies \grad{P}(\grad{\phi_1'}, \grad{\phi_2'})
    \end{displaymath}
    or equivalently, thinking of predicates as boolean functions
    \begin{displaymath}
    \grad{P}  \text{ is monotonic w.r.t. $\sqsubseteq$}
    \end{displaymath}
    or something with set terminology!???
    \begin{displaymath}
    \grad{P}  \text{ is somewhat closed over weakening}
    \end{displaymath}
\end{description}

\begin{definition}[Soundness of Lifted Predicate]
    A lifted predicate is \textbf{sound} if it satisfies both ?????
\end{definition}

For sound rules this holds:
\begin{displaymath}
\forall \grad{\phi_1}, \grad{\phi_2} \in \setGFormula.~ (\exists \phi_1 \in \gamma(\grad{\phi_1}), \phi_2 \in \gamma(\grad{\phi_2}).~ P(\phi_1, \phi_2)) \implies \grad{P}(\grad{\phi_1}, \grad{\phi_2})
\end{displaymath}

Note that there is a “smallest” set/predicate??? ... induced by the rules.

\begin{definition}[Optimality of Lifted Predicate]
    A lifted predicate is \textbf{optimal} if there exists no ???
\end{definition}

...same as “consistent lifting” in AGT!

% TODO: delve into what is important, what is optional, etc.
\begin{displaymath}
\forall \grad{\phi_1}, \grad{\phi_2} \in \setGFormula.~ (\exists \phi_1 \in \gamma(\grad{\phi_1}), \phi_2 \in \gamma(\grad{\phi_2}).~ P(\phi_1, \phi_2)) \iff \grad{P}(\grad{\phi_1}, \grad{\phi_2})
\end{displaymath}



%TODO: restate that in terms of deduction stuff?


