In this section, we assume that we are dealing with a binary predicate $P \subseteq \setFormula \times \setFormula$ and want to obtain a lifted predicate $\grad{P} \subseteq \setGFormula \times \setGFormula$.
The concepts are directly applicable to predicates with different arity or with additional non-formula parameters.
% W.l.o.g. we further assume that $P$ appears unnegated in the axiomatic semantics (otherwise we simply regard the negation of that predicate as $P$).

We identify the following rules:
\begin{description}
    \item[Introduction]~\\
    We demand compatibility of the semantics of \gvl with the semantics of \svl.
    In other words, switching to the gradual system may never “break the code”.
    A predicate $\grad{P}$ that is part of the gradual semantics must thus satisfy:
    \begin{mathpar}
        \inferrule* [Right=GPredIntro]
        {
            P(\phi_1, \phi_2)
        }
        {
            \grad{P}(\phi_1, \phi_2)
        }
    \end{mathpar}
    
    Or equivalently, using set notation
    \begin{displaymath}
    P \subseteq \grad{P}
    \end{displaymath}
    
    \item[Monotonicity]~\\
    In order to satisfy the gradual guarantee, the semantics of \gvl must be immune to reduction of precision.
    A predicate $\grad{P}$ that is part of the gradual semantics must thus remain satisfied when reducing the precision of arguments
    \begin{mathpar}
        \inferrule* [Right=GPredMon]
        {
            \grad{P}(\grad{\phi_1}, \grad{\phi_2}) \\
            \grad{\phi_1} \sqsubseteq \grad{\phi_1'} \\
            \grad{\phi_2} \sqsubseteq \grad{\phi_2'}
        }
        {
            \grad{P}(\grad{\phi_1'}, \grad{\phi_2'})
        }
    \end{mathpar}
\end{description}

\begin{definition}[Soundness of Predicate Lifting]
    A lifted predicate is \textbf{sound} if it is closed under the above rules.
\end{definition}

Note that soundness only gives a lower bound for the predicate:
$$\grad{P} = \setGFormula \times \setGFormula$$ is a sound predicate lifting of any binary predicate $$P \subseteq \setFormula \times \setFormula$$
This observation motivates an additional notion of optimality.

\begin{definition}[Optimality of Predicate Lifting]
    A sound lifted predicate is \textbf{optimal} if it is the smallest set closed under the above rules.
\end{definition}

%% AGT
The definition of optimal predicate lifting coincides with the definition of “consistent predicate lifting” given by AGT \cite{garcia2016abstracting}.
\begin{lemma}[Equivalence with Consistent Predicate Lifting (AGT)]~\\
    \label{lemma:consistent-pred-lifting-direct}
    Let $\grad{P} \subseteq \setGFormula \times \setGFormula$ be defined as
    \begin{displaymath} 
    \grad{P}(\grad{\phi_1}, \grad{\phi_2}) ~\defiff~ \exists \phi_1 \in \gamma(\grad{\phi_1}), \phi_2 \in \gamma(\grad{\phi_2}).~ P(\phi_1, \phi_2)
    \end{displaymath}
    Then $\grad{P}$ is an optimal lifting of $P$.
\end{lemma} %PROOF
This is an intriguing observation since different approaches were used to end up with the same definition:
AGT immediately defines consistent predicate lifting as above, arguing that it reflects the intuition behind gradual formulas (as placeholders for plausible static formulas).
Therefore $\grad{P}(\grad{\phi_1}, \grad{\phi_2})$ is supposed to hold if it is plausible that there exists an instantiation satisfying the static predicate.
This intuition is directly formalized, using concretization to map from gradual formulas back to plausible static formulas.

\begin{comment}
We noticed early that this definition is too strong for gradual verification rules in general, due to the complexity of verification rules compared to typing rules:
A judgment $\thoare{}{\phi_1}{s}{\phi_2}$ (note that $s$ may be a sequence, contain method calls, etc.) 

In chapter \ref{sec:abstracting-static-semantics} we will see examples of gradual predicates which are not optimal and would thus not fit into AGT's model of consistent lifting.
\end{comment}

Realizing that consistent lifting is actually not necessary for ending up with a sound gradual verification system, we took a different, more general approach to define lifting.
Identifying the bare minimum of requirements (dictated by the gradual guarantee and compatibility with the static system) we ended up with our definition of sound lifting.
The optional notion of optimality bridges the gap between both approaches.

It is worth noting that optimality might not even be desirable under all circumstances as outlined in section \ref{sec:future-work}.

