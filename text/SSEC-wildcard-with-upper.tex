To allow combining wildcards with static knowledge, we might view $\qm$ merely as an unknown conjunctive term within a formula:
\begin{displaymath}
\grad{\phi} ::= \phi ~|~ \withqmGen{\phi}
\end{displaymath}

We pose $\qm \defeq \withqmGen{\phiTrue}$.

Intuitively, we expect $\withqmGen{\phi}$ to be a placeholder for a formula that implies $\phi$.
This intuition is formalized using concretization.
\begin{definition}[Concretization]~\\
    \label{def:gamma-bounded-unk}
    Let $\gamma : \setGFormula \rightarrow \PP(\setFormulaA)$ be defined as
    \begin{align*}
    &\gamma(\phi) = \{~ \phi ~\} \\
    &\gamma(\withqmGen{\phi}) = \{~ \phi' \in \setFormulaA ~|~ \phiImplies{\phi'}{\phi} ~\}
    \end{align*}
\end{definition}

Note that $\gamma(\qm) = \gamma(\withqmGen{\phiTrue}) = \{~ \phi' \in \setFormulaA ~|~ \phiImplies{\phi'}{\phiTrue} ~\} = \setFormulaA$.
The approach is thus compatible with and strictly superior to the previous one.

\begin{comment}
There are two ways to express this containment, resulting in different concretizations.
\begin{description}
    \item[Syntactic]\quad
    $\gamma_1(\withqmGen{\phi}) = \{~ \phi \wedge \phi' ~|~ \phi' \in \setFormulaA ~\}$
    \item[Semantic]\quad
\end{description}

\begin{l} 
    $\forall \grad{\phi} \in \setGFormula.~ \gamma_1(\grad{\phi}) \subseteq \gamma_2(\grad{\phi})$
\end{l}
\begin{l} 
    $\forall \grad{\phi} \in \setGFormula.~ \gamma_1(\grad{\phi}) = \gamma_2(\grad{\phi})$ modulo equivalence
\end{l}

Note that $\gamma_1(\qm) = \gamma_2(\qm) = \setFormulaA$, meaning that this approach of extending the formula syntax is compatible with (but superior to) the approach introduced in the previous section.
\end{comment}