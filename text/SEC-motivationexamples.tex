
\subsection{Parameter Validation}
\label{ssec:argument-validation}
The following Java example motivates the use of verification for parameter validation.
\begin{lstlisting}
// spec: callable only if (this.balance >= amount)
void withdrawCoins(int amount)
{
    // business logic
    this.balance -= amount;
}
\end{lstlisting}
It is supposed to be impossible for \ttt{withdrawCoins} to create a negative balance, hence the restriction.
To enforce this specification in Java, one may validate the parameter \ttt{amount} before entering the business logic.
\begin{lstlisting}
void withdrawCoins(int amount)
{
    if (!(this.balance >= amount))
        throw new IllegalArgEx("expected this.balance >= amount");
        
    // business logic
    this.balance -= amount;
}
\end{lstlisting}

Note that this runtime check dynamically verifies a method contract that has \ttt{this.balance >= amount} as precondition.
Naturally, the drawbacks of dynamic verification apply:
Violations of the method contract are only detected at runtime, possibly go unnoticed for a long time and impose a runtime overhead.

Java even has a dedicated assertion syntax, simplifying dynamic verification:
\begin{lstlisting}
void withdrawCoins(int amount)
{
    assert this.balance >= amount;
    
    // business logic
    this.balance -= amount;
}
\end{lstlisting}

Using additional tools, a more declarative approach is possible using JML syntax:
\begin{lstlisting}
//@ requires this.balance >= amount;
void withdrawCoins(int amount)
{
    // business logic
    this.balance -= amount;
}
\end{lstlisting}

Using JML4c (see \cite{sarcar2010new}) or similar tools, this annotation is compiled back into equivalent runtime assertions.
However, the declarative approach also enables the use of static verification tools like ESC/Java (see \cite{leino2000esc}) to ensure at compile time that every call to \ttt{hasLegalDriver} adheres to the method contract.
Static verification will only succeed if the precondition is provable at all call sites.

Consider the following example code:
\begin{lstlisting}
acc.balance = 100;
acc.withdrawCoins(50);
acc.withdrawCoins(30);
\end{lstlisting}

Using dynamic verification methods, the validity of both calls is only established at runtime, although the calls clearly adhere to the contract.
However, static verification most likely fails due to a lack of knowledge:
The method contract does not state how \ttt{this.balance} is changed during a call to \ttt{withdrawCoins}, meaning that the precondition for the second call cannot be proved (we assume that the static verifier exclusively relies on the contract while reasoning about a method call).
Annotating a postcondition $$\ttt{this.balance = old(this.balance) - amount}$$ would solve that problem, resulting in overall success of the verification.

Gradual verification would realize a combined, best-effort approach without the need for full annotation:
Static verification is used where possible (here: the first method call), dynamic verification where needed (here: the second method call).
Where static verification proves a violation of the method contract, the code is \emph{statically rejected}.
Where static verification proves compliance with the method contract, the code is \emph{statically accepted}, i.e. no runtime checks are emitted in order to enforce the contract.
In contrast to purely static verification, code that is provable neither to violate nor to comply with the contract will be \emph{optimistically accepted} and enhanced with runtime checks to guarantee compliance at runtime.

In above example, this behavior would be realized by treating the postcondition of \ttt{withdrawCoins} as unknown (“$\qm$”).
This will allow the gradual verifier to assume any plausible formula for subsequent reasoning, e.g. one stating that the balance is still high enough for the second method call.
Such treatment of $\qm$ is backed with a runtime check, making sure that all assumptions made by the gradual verifier were valid.
Bottom line, the example would pass gradual verification and a single runtime check (for \ttt{acc.balance >= 30}) prior to the second call would be inserted.

\subsection{Limitations of Static Verification}
\label{ssec:limitations-of-static}
The following example is written in a Java-like language with dedicated syntax for method contracts (similar to Eiffel \cite{meyer1988eiffel} and Spec\# \cite{the-spec-programming-system-an-overview}).
We assume that this language is statically verified, i.e. successful static verification is a requirement for compilation.

The example shows the limitations of static verification using the Collatz sequence as an algorithm too complex to describe and decide concisely in a method contract:
\begin{lstlisting}
int collatzIterations(int iter, int start)
    requires 1 <= start;
    ensures  1 <= result;
{
    // ...
}

int myRandom(int seed)
    requires 1 <= seed   && seed   <= 10000;
    ensures  1 <= result && result <= 3;     // not provable
{
    int result = collatzIterations(300, seed);
    // we know:        result $\in$ { 1, 2, 4 }
    // verifier knows: 1 <= result
    
    if (result == 4) result = 3;
    return result;
}
\end{lstlisting}
The first method \ttt{collatzIterations} iterates the Collatz sequence a given number of times \ttt{iter}, starting with given value \ttt{start}.
We assume that the only provable contract is that a positive start value results in a positive result.

The second method \ttt{myRandom} uses the Collatz sequence to generate a pseudo random number from given seed.
It is known to the programmer that start values up to 10000 result in convergence of the Collatz sequence after less than 300 iterations.
After mapping 4 to 3, we are thus given a number between 1 and 3, as described in the postcondition.

Unfortunately, the verifier cannot deduce this fact since the postcondition of \ttt{collatzIterations} only guarantees positive results, but no specific range of values.
We could resort to dynamic methods to aid verification, realizing a “cast” of knowledge:

\begin{lstlisting}
...
{
    int result = collatzIterations(300, seed);
    // we know:        result $\in$ { 1, 2, 4 }
    // verifier knows: 1 <= result
    
    // knowledge "cast"
    if (!(result <= 4))
        throw new IllegalStateException("expected result <= 4");

    // verifier knows: 1 <= result && result <= 4 
    
    if (result == 4) result = 3;
    return result;
}
\end{lstlisting}

This solution is not satisfying as it required additional work by the programmer to convince the static verifier.
Furthermore, the solution is in an unintuitive location:
The problem is caused by the weak postcondition of \ttt{collatzIterations}, yet it is addressed in \ttt{myRandom}.

Gradual verification allows enhancing the postcondition with “unknown” knowledge that can be interpreted by the verifier as an arbitrary plausible static formula.
Again, appropriate runtime checks will be injected to guarantee that the static treatment of $\qm$ is in fact valid:
\begin{lstlisting}
int collatzIterations(int iter, int start)
    requires 1 <= start;
    ensures  1 <= result && ?;
{
    // ...
}

int myRandom(int seed)
    requires 1 <= seed   && seed   <= 10000;
    ensures  1 <= result && result <= 3;
{
    int result = collatzIterations(300, seed);
    // we know: result $\in$ { 1, 2, 4 }
    
    // verifier allowed to
    //  assume 1 <= result && result <= 4
    //  from   1 <= result && ?
    // (adding runtime check)
    
    if (result == 4) result = 3;
    return result;
}
\end{lstlisting}
Note the \ttt{?} in the postcondition of \ttt{collatzIterations}.
As indicated in the comments, the verifier may now make assumptions in order to prove the postcondition of \ttt{myRandom}.

% gradual:
% - not in a sense of GraVy
% - assertions are GUARANTEED not to be violated (analogous to type safety of gradually typed languages), meaning that
%    - as much as possible is verified statically
%    - if necessary, (ideally: as few as possible) dynamic checks kick in
% - full continuum between static and dynamic

% static checking
% dynamic checking
% combine static and dynamic checking
% views:
% - add designated "?" to statically checked language, making checking optional
% - introduce checking to unchecked language, making "?" the default value
% typing
% transition to verification
% - of particular interest, cause limitations due to syntax or decidability make full checking impossible,
%    possibly making use of static verification impossible and the program thus unverifiable
%    inevitable incompleteness of static verifiers!
%    GraVy: measure, only \textit{classifies} code into sections: correctness statically guaranteed, correctness statically disproved, no static guarantee so far

% very desirable in practice
% gradual version of a language is inherently a strict superset of original language
% 

% our approach allows extension of existing languages (without any verification) by adding transpilation step

% why would you want continuum?
% - move toward satically verified program without giving up guarantees!
% - while "barely typed" languages hardly make sense, programs with "little verification" are already highly useful
%    i.e. verification even makes sense as "rarely used feature"!
%    ArgumentException-example
%    -    performance benefit!
%    -    cleaner code!
%    -    makes more sense semantically
% having optional types 

% Future Work:
% implement on top of existing languages
% - tracking access via ThreadLocalStorage
% - transpilation (e.g. C# Roslyn compiler extension)

\section{Overview}
Our approach is based on recent advances in formalizing gradual typing, specifically “Abstracting Gradual Typing” \cite{garcia2016abstracting} by Garcia, Clark and Tanter.
They describe a process called “gradualization” that uses the concept of abstract interpretation to define a gradual system in terms of a static one.
Gradual typing emerged from drawbacks of static and dynamic type systems that are very similar to the drawbacks of verification systems outlined and illustrated above.
This is no surprise from a theoretical perspective as type systems are a special case of program verification. 
These similarities motivated our idea of reinterpreting and adapting the gradual typing approach to the verification setting.

Chapter \ref{sec:categorization-of-existing} introduces the concepts motivating and driving our approach.
Furthermore it categorizes existing work that goes in a similar direction, pointing out how it differs from our work.
In chapter \ref{ch:gradualization-of-a} we describe our approach of gradualization in a generic way, meant to be used as a procedure or template for designing gradual verification systems.
We follow that procedure in form of a case study in chapter \ref{ch:case-study--implicit}, applying the approach to a statically verified language that uses implicit dynamic frames in order to enable safe static reasoning about shared mutable state.
We conclude with an evaluation of our approach (chapter \ref{ch:evaluation-analysis}) and a conclusion (chapter \ref{ch:conclusion}).
The appendix contains additional information like proofs for lemmas and theorems stated throughout this work. % TODO: is that all?
We have also implemented the gradually verified language developed in chapter \ref{ch:case-study--implicit} as an interactive web page (see section \ref{sec:implementation}).
