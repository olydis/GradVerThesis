
\cite{siek2006gradual}
Gradual Typing for Functional Languages

apply their gradual typing approach to other areas

\cite{siek2015refined}
Refined criteria for gradual typing
gradual guarantee:
     The gradual guarantee says that if a gradually typed program is
     well typed, then removing type annotations always produces a program that is still well typed.
     Further, if a gradually typed program evaluates to a value, then removing type annotations
     always produces a program that evaluates to an equivalent value.
     
\label{grad-guarantee}
% REDEFINE FOR VERIFICATION!

\cite{garcia2016abstracting}
AGT
In this paper, we propose a new formal foundation for gradual
typing, drawing on principles from abstract interpretation to
give gradual types a semantics in terms of preexisting static types.
Abstracting Gradual Typing (AGT for short) yields a formal account
of consistency—one of the cornerstones of the gradual typing
approach—that subsumes existing notions of consistency, which
were developed through intuition and ad hoc reasoning.

\cite{garcia2015deriving}
Abstracting Gradual Typing (AGT) is an approach to systematically
deriving gradual counterparts to static type disciplines (Garcia
et al. 2016). The approach consists of defining the semantics of
gradual types by interpreting them as sets of static types, and then
defining an optimal abstraction back to gradual types. These operations
are used to lift the static discipline to the gradual setting. The
runtime semantics of the gradual language then arises as reductions
on gradual typing derivations.
To demonstrate the flexibility of AGT, we gradualize
a prototypical security-typed language
with respect to only security labels rather than entire types, yielding
a type system that ranges gradually from simply-typed to securely typed.
We establish noninterference for our gradual language using Zdancewic’s logical relation proof method.
Whereas prior work presents gradual security cast languages,
which require explicit security casts, this work yields the first gradual
security source language, which requires no explicit casts.

prior to AGT
\cite{wolff2011gradual}
the language extends the notion of gradual typing to account for typestate: gradual typestate
checking seamlessly combines static and dynamic checking by automatically
inserting runtime checks into programs.

\cite{banados2014theory}
 develop a theory of gradual effect checking, which
 makes it possible to incrementally annotate and statically check
 effects, while still rejecting statically inconsistent programs. We
 extend the generic type-and-effect framework of Marino and Millstein
 with a notion of unknown effects, which turns out to be significantly
 more subtle than unknown types in traditional gradual
 typing. We appeal to abstract interpretation to develop and validate
 the concepts of gradual effect checking.

\cite{toro2015customizable}
Grad Effects in Scala, benchmarks on runtime impact!