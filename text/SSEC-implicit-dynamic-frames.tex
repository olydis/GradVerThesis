%% intro, aliasing problem
Reasoning about programs using shared mutable data structures (the default in object oriented programming languages) is not possible using traditional Hoare logic.

\begin{example}{Limitations of Hoare Logic}~\\
    The following Hoare triple is not verifiable using a sound Hoare logic due to potential aliasing.
    \begin{displaymath}
    \hoare
    {\phiAnd{\phiEq{p1.age}{19}}{\phiEq{p2.age}{19}}}
    {\ttt{p1.age++}}
    {\phiAnd{\phiEq{p1.age}{20}}{\phiEq{p2.age}{19}}}
    \end{displaymath}
    The problem is that \ttt{p1} and \ttt{p2} might be aliases, meaning that they reference the same memory.
    The increment operation could thus also affect \ttt{p2.age}, rendering the postcondition invalid.
\end{example}

We want to demonstrate gradual verification on a Java-like language in chapter \ref{ch:case-study--implicit}, so we need a logic that is capable of dealing with mutable data structures.

%% SL
Separation logic \cite{reynolds2002separation} is an extension of Hoare logic that explicitly tracks mutable data structures (i.e. heap references) and adds a “separating conjunction” to the formula syntax.
In contrast to ordinary conjunction (\ttt{∧}),
separating conjunction (\ttt{*}) ensures that both sides of the conjunction reference disjoint areas of the heap.
\begin{example}{Power of Separation Logic}~\\
    The following Hoare triple is verifiable using separation logic.
    \begin{displaymath}
    \hoare
    {\phiCons{\phiMapsTo{p1.age}{19}}{\phiMapsTo{p2.age}{19}}}
    {\ttt{p1.age++}}
    {\phiCons{\phiMapsTo{p1.age}{20}}{\phiMapsTo{p2.age}{19}}}
    \end{displaymath}
    The separating conjunction in the precondition guarantees that \ttt{p1.age} and \ttt{p2.age} reference disjoint memory locations.
    The operation is therefore guaranteed to leave \ttt{p2.age} untouched.
    
    Note that proving the “more powerful” precondition $\phiCons{\phiMapsTo{p1.age}{19}}{\phiMapsTo{p2.age}{19}}$ challenges the system to establish that \ttt{p1} and \ttt{p2} do not alias.
    In other words, more sophisticated reasoning and possibly stronger contracts annotated by the user are required to derive formulas containing the separating conjunction.
\end{example}

%% IDF
A drawback of separation logic is that formulas cannot contain heap-dependent expressions (e.g. $\ttt{p1.age <= 19}$) as they are not directly expressible using the explicit syntax (see above: values of memory locations are explicitly tracked).
The concept of implicit dynamic frames (IDF) \cite{smans2009implicit} addresses this issue by decoupling the permissions to access a certain heap location from assertions about its value.
It introduces an “accessibility predicate” \ttt{acc(\textit{loc})} that represents the permission to access heap location \textit{loc}.

Parkinson and Summers worked out the formal relationship between separation logic and IDF \cite{parkinson2011relationship}.
Specifically, they define a logic that syntactically subsumes and semantically agrees with both separation logic and implicit dynamic frames.
Finally, they show that the separation logic fragment of their logic can be encoded into the implicit dynamic frames fragment while preserving semantics.

\begin{example}{Power of Implicit Dynamic Frames}~\\
    \label{ex:pow-idf}
    The following Hoare triple is verifiable using implicit dynamic frames.
    \begin{alignat*}{1}
    &\{\phiCons {\phiCons{\phiAcc{p1}{age}}{\phiAcc{p2}{age}}} {\phiCons{\phiEq{p1.age}{19}}{\ttt{(p2.age <= 19)}}}\}\\
    &{\ttt{p1.age++}}\\
    &\{\phiCons {\phiCons{\phiAcc{p1}{age}}{\phiAcc{p2}{age}}} {\phiCons{\phiEq{p1.age}{20}}{\ttt{(p2.age <= 19)}}}\}
    \end{alignat*}
    The separating conjunction makes sure that the accessibility predicates $\phiAcc{p1}{age}$ and $\phiAcc{p2}{age}$ mention disjoint memory locations, whereas it has no further meaning for “traditional” predicates.
    Note how more complex predicates like $\ttt{<=}$ are now expressible.
\end{example}

As formulas can mention heap locations in arbitrary predicates, the verifier must ensure that a formula contains accessibility predicates to all heap locations mentioned.
This property of formulas is called “self-framing”.
The pre- and postcondition of example \ref{ex:pow-idf} are self-framed whereas the sub-formula $\phiEq{p1.age}{20}$ would not be.
% MENTION? It is essential that access cannot be duplicated, implying that and thus also not be shared between threads, allowing race-free reasoning about concurrent programs.

%% Chalice
Implicit dynamic frames was implemented as part of the Chalice verifier \cite{leino2009verification}.
Chalice is also the name of the underlying simple imperative programming language that has constructs for thread creation and thus relies on IDF for sound race-free reasoning.
Chalice was also implemented as a front-end of the Viper toolset \cite{MuellerSchwerhoffSummers16}. % developed at ETH Zürich.

%% formal semantics
The static semantics of our example language in chapter \ref{ch:case-study--implicit} is based on the Hoare logic for Chalice given by Summers and Drossopoulou \cite{summers2013formal}.


