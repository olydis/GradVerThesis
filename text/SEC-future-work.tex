\subsection{Optimality Revised}%% optimality
After realizing that optimality of gradual liftings (or “consistent lifting”, as called in AGT \cite{garcia2016abstracting}) is an unnecessary restriction in the verification setting, we focused mainly on soundness of liftings.
Furthermore, we showed that our approach of obtaining a gradual Hoare logic using deterministic liftings does not result in an optimally lifted Hoare logic, even if the deterministic lifting was optimal (see example \ref{cex:opt-det2grad}).
We have also shown that a more expressive gradual syntax can reduce that problem, yet we have not formally proved that relationship.

Furthermore, it is unclear whether optimality/consistency as defined by AGT is even desirable under all circumstances:
\begin{example}{Optimality and Decidability}
    Consider a statement $s$ which is very hard to reason about (for instance, think of 300 Collatz sequence iterations).
    With the static Hoare logic, we assume that
    \begin{displaymath}
    \thoare{}{\phiEq{x}{10000}}{s}{\phiAnd{\ttt{1 <= x}}{\ttt{x <= 4}}}
    \end{displaymath}
    can be verified and that no stronger postcondition could be verified (due to limitations of the verifier).
    However, we know that \ttt{x} will have value 4 afterwards, i.e. the following Hoare triple is valid:
    \begin{displaymath}
    \hoare{\phiEq{x}{10000}}{s}{\phiEq{x}{4}}
    \end{displaymath}
    
    Using gradual verification, we want to overcome the limitations imposed by decidability and be able to deduce 
    \begin{displaymath}
    \gthoare{}{\withqmGen{\phiEq{x}{10000}}}{s}{\withqmGen{\phiEq{x}{4}}}
    \end{displaymath}
    or similar.
    However, an optimal gradual lifting is not able to deduce this fact, since there exists no instantiation of the postcondition for which a corresponding static judgment holds (since $\phiAnd{\ttt{1 <= x}}{\ttt{x <= 4}} \not \in \gamma(\withqmGen{\phiEq{x}{4}})$).
   
    Note that our approach using a deterministic would work in this case:
    \begin{displaymath}
    \dgthoare{}{\withqmGen{\phiEq{x}{10000}}}{s}{\withqmGen{\phiAnd{\ttt{1 <= x}}{\ttt{x <= 4}}}}
    \end{displaymath}
    or a more imprecise postcondition must be deducible due to introduction, strength and monotonicity rules.
    Furthermore $\gphiImplies{\withqmGen{\phiAnd{\ttt{1 <= x}}{\ttt{x <= 4}}}}{\phiEq{x}{4}}$ holds (emitting a runtime check), such that 
    \begin{displaymath}
    \gthoare{}{\withqmGen{\phiEq{x}{10000}}}{s}{\phiEq{x}{4}}
    \end{displaymath}
    is deducible.
\end{example}

\begin{comment}
INTERESTING! Regarding “when does optimal det. lift. imply opt. pred. lift.”
For above example, the gradual formula syntax would have to provide a way of saying that there is NOT MORE knowledge about x hidden in ?.
This way, the gradual implication of x = 4 would (as expected) fail, as the optimal predicate would
\end{comment}

We conjecture that a notion of consistency that relies on \emph{valid} Hoare triples instead of \emph{deducible} Hoare triples (Hoare logic) would solve this problem, yet it is unclear what the implications of this definition would be.
Most importantly, it would severely complicate optimality proofs, as they suddenly rely on dynamic semantics (from which validity of Hoare triples arises) instead of static semantics (Hoare logic).

\subsection{Gradual Non-Termination}%% non-termination
The “plausibility” interpretation of unknown formulas motivated us to define $\gamma(\qm)$ as the set of all \emph{satisfiable} formulas \setFormulaA (see section \ref{sec:gradual-formulas}).
However, this definition implies that $\qm$ may not be usable as a postcondition of non-terminating statements (for which proving $\phiFalse$ as a postcondition would make sense).
As a consequence, programmers may not be able to be imprecise about termination of a statement.
Note that \svlidf does not contain non-terminating constructs, so we did not deal with this problem in chapter \ref{ch:case-study--implicit}.
It is not clear how this case should be approached since the current definition of $\gamma(\qm)$ is clearly justified (for terminating programs), as otherwise $\gphiImplies{\withqmGen{\phiEq{x}{3}}}{\phiEq{x}{4}}$ holds, regardless of whether the lifted implication is optimal or not.
Maybe the introduction of an additional wildcard $\qm_{\bot}$ with $\gamma(\qm_{\bot}) = \setFormula$ proves useful in order to be imprecise about termination.

\subsection{Leveraging Implicit Dynamic Frames}%% IDF
In the case study, we have not made full use of the capabilities of implicit dynamic frames, yet.
Tracking exclusive access to memory locations also allows race-free reasoning about concurrent programs (Summers and Drossopoulou \cite{summers2013formal} give a corresponding Hoare logic).
It would be interesting to see gradual verification applied to such a setting, as it reflects more closely the reality of modern programming languages.
Further potential extensions include the introduction of shared access or the addition of non-separating conjunction to the formula syntax.

\subsection{Evidence-Based Gradual Dynamic Semantics}\label{ssec:ev-based-gds}
Regarding gradual dynamic semantics (and soundness) of the gradual system, we have deviated from the approach presented by AGT.

%% AGT
AGT derives a direct runtime semantics for the gradual system that uses “evidence” to ensure that products of a runtime derivation (evaluating an application $t_1~t_2$) are still type safe.
Key to this approach is the observation that type safety proofs smoothly evolve as terms are reduced.
More specifically, a type safety proof for the reduced term can be computed from a type safety proof of the unreduced term (function application, according to rule “Tapp”).

In the static system, this process would always succeed (and is thus not necessary to ensure type safety).
However, in the gradual system, deriving a proof for a reduced term may fail due to inconsistent judgments in the proofs of the unreduced term.
For example, the unreduced term may judge $T_1~\grad{<}~\qm$ and $\qm~\grad{<}~T_2$, whereas the reduced term requires $T_1~\grad{<}~T_2$ to hold (a property called “consistent transitivity”).
Evidence is the minimum information necessary to detect those inconsistencies.
It is thus created while initially type checking a term (“initial evidence”, side product of gradual typing rule “\gradT Tapp”) and evolved during reduction as part of the runtime semantics.

In our setting, there are two candidate judgments that one could try to enforce using evidence (like type safety is enforced in AGT).
\begin{description}
    \item[Semantical]~\\
    When searching for a direct counterpart for type judgments $\vdash e : T$ in simply typed $\lambda$-calculus, one may come up with formula semantics $\evalgphiGen{\pi}{\phi}$.
    Formula semantics can be seen as a way to “type” program states using formulas (“$\vdash \pi : \phi$”).
    Just like type safety states that statically postulated type judgments hold during reduction of terms and expressions, it has to be ensured that annotated formulas are complied with during evolution of program states (which is what soundness states).
    
    However the formula semantics $\evalgphiGen{\pi}{\phi}$ is a semantical judgment, whereas the $\vdash e : T$ is syntactical.
    We will illustrate in section \ref{sec:soundness-using-evidence} why the evidence approach is not feasible when choosing formula semantics as the judgment to ensure.
    
    \item[Syntactical]
    Instead, one may find that Hoare logic $\thoare{}{\phi_1}{s}{\phi_2}$ acts like the conceptual counterpart of $\vdash e : T$.
    While the signature slightly differs, Hoare logic can be regarded as a way to “type” statements as a function from program states satisfying $\phi_1$ to program states satisfying $\phi_2$ (“$\vdash s : \phi_1 \rightarrow \phi_2$”).
    
    During execution of the program, evidence would have to make sure that the “remaining” work $s$ is consistently typeable using Hoare logic.
    \begin{example}{} \label{ex:ev-hl}~\\
        Let 
        \begin{align*}
        & \phi_1 = \phiAnd{\phiEq{x}{10}}{\phiEq{a}{5}}\\
        & \phi_3 = \phiAnd{\phiEq{x}{3}}{\phiEq{y}{4}}
        \end{align*}
        Assume that
        $$\thoare{}{\phi_1}{\sSeq{\sVarAssign{x}{3}}{\sVarAssign{y}{4}}}{\phi_3}$$
        was proved by the verifier.
        Starting execution of $\sSeq{\sVarAssign{x}{3}}{\sVarAssign{y}{4}}$ with a program state $\pi_1$ that satisfies $\phi_1$ one will reach a program state $\pi_2$ where $\sVarAssign{x}{3}$ has been executed but not yet $\sVarAssign{y}{4}$.
        Type safety as motivated above would have to ensure that there exists a formula $\phi_2$ with
        $$\evalphiGen{\pi_2}{\phi_2}$$
        and
        $$\thoare{}{\phi_2}{\sVarAssign{y}{4}}{\phi_3}$$
        In this example $\phi_2 = \phiAnd{\phiEq{x}{3}}{\phiEq{a}{5}}$ would be suitable.
        In other words, it is ensured that there exists a Hoare logic deduction justifying every intermediate state during execution.
        
        While such a formula always exists in the static setting (choose the one that was used as an intermediate formula when applying the sequence rule \tset{HSeq}), it may not exist in the gradual setting.
        It would be the job of evidence to ensure the existence of such a formula at runtime, without performing a Hoare logic derivation.
    \end{example}
    
    This approach requires a slightly stronger notion of soundness than the one we demanded in section \ref{sec:a-statically-verified}.
    Specifically, a statement has to be made about intermediate execution steps (as illustrated in example \ref{ex:ev-hl}) instead of merely the execution as a whole.
    
    As mentioned in section \ref{ssec:gradual-soundness} such an approach would detect inconsistencies more eagerly than the “safe baseline” we introduced with \tset{\gradT Soundness} and later improved with \tset{\dgradT Soundness}.
    Unfortunately, investigating this path was not in the scope of this work.
\end{description}

