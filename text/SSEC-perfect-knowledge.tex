\begin{definition}[Perfect Small-Step Semantics]~\\
    A small-step semantics $\sssem$ is \textbf{perfect} (w.r.t. a Hoare logic) if every execution can be supported with a matching derivation in Hoare logic:
    \begin{flalign*}
    & \forall s \in \setStmt,\, \pi_1, \pi_2 \in \setProgramState.~ \sssem^s(\pi_1, \pi_2) \implies \exists \phi_1, \phi_2 \in \setFormula.~ \thoare{}{\phi_1}{s}{\phi_2} \wedge \evalphiGen{\pi_1}{\phi_1} \wedge \evalphiGen{\pi_2}{\phi_2}
    \end{flalign*}
\end{definition}

Let ~$\sssem' : \setProgramState \rightharpoonup \setProgramState$~ be defined such that
\begin{flalign}
\sssem'^s(\pi_1, \pi_2) ~\defiff~ \sssem^s(\pi_1, \pi_2) ~\wedge~ \exists \phi_1, \phi_2 \in \setFormula.~ \thoare{}{\phi_1}{s}{\phi_2} \wedge \evalphiGen{\pi_1}{\phi_1} \wedge \evalphiGen{\pi_2}{\phi_2}
\label{frm:perf-def}
\end{flalign}
Intuitively, $\sssem'$ is the same as $\sssem$, but is artificially made undefined whenever there exists no Hoare rule that would be able to deduce static information about the execution.

\begin{lemma}[Restricted Domain of Small-Step Semantics]
    Replacing the small-step semantics $\sssem$ of \svl with $\sssem'$ as defined above would have no observable effects.
\end{lemma}
\begin{proof}
    \svl only ever executes a statement $s$ if it was successfully verified.
    This means that there exists some Hoare derivation $\thoare{}{\phi_1}{s}{\phi_2}$ and a guarantee (due to the soundness of the verification logic) that $s$ will only ever be executed starting with program states satisfying $\phi_1$.
    Thus, every for execution $\sssem^s(\pi_1, \pi_2)$, $\sssem'^s(\pi_1, \pi_2)$ must hold as well.
\end{proof}

Let ~$\gsssem : \setGProgramState \rightarrow \setGProgramState$ be a total gradual lifting of $\sssem'$ that throws exceptions whenever all static derivations would be stuck:
\begin{align}
\forall \grad{\pi} \in \setGProgramState.~ (\gamma(\grad{\pi}) \cap \dom{\sssem'} = \emptyset) \implies \gsssem(\grad{\pi}) \in \setProgramStateEx
\label{frm:perf-ex}
\end{align}

\begin{theorem}[]
    If the gradual small-step semantics $\gsssem$ of \gvl is defined as above, then \gvl satisfies \tset{GDPreservation}.
\end{theorem}
\begin{proof}
    To prove \tset{GDPreservation} we may assume:
    \begin{align}
    \dgthoare{}{\grad{\phi_1}}{\grad{s}}{\grad{\phi_2}}\\
    \label{frm:tmp0}
    \gsssem^{\grad{s}}(\grad{\pi_1}, \grad{\pi_2})\\
    \label{frm:tmp1}
    \evalgphiGen{\grad{\pi_1}}{\grad{\phi_1}}
    \label{frm:tmp2}
    \end{align}
    Goal: 
    \begin{align*}
    \evalgphiGen{\grad{\pi_2}}{\grad{\phi_2}}
    \label{frm:tmp3}
    \end{align*}
    
    Reversing formula \ref{frm:perf-ex} we can deduce that $\gamma(\grad{\pi_1}) \cap \dom{\sssem'} \neq \emptyset$, i.e.
    \begin{align}
    \exists \pi \in \setProgramState.~ \sssem'(\pi_1) = \pi 
    \quad\quad\text{for some } \pi_1 \in \gamma(\grad{\pi_1})
    \label{frm:tmp4}
    \end{align}
    Recall that formula evaluation is immune to concretization, so from \ref{frm:tmp2} follows
    \begin{align}
    \evalgphiGen{\pi_1}{\grad{\phi_1}}
    \text{\quad\quad or equivalently \quad\quad}
    \evalphiGen{\pi_1}{\phi_{1a}}
    \quad\quad\text{for some } \phi_{1a} \in \gamma(\grad{\phi_1}) 
    \label{frm:tmp5}
    \end{align}
    Iterating the reasoning that lead to \ref{frm:tmp4} we end up with
    \begin{displaymath}
    \sssem'^s(\pi_1, \pi)
    \quad\quad\text{for some } \pi \in \setProgramState,\, s \in \gamma(\grad{s}) 
    \label{frm:tmp6}
    \end{displaymath}
    Using the monotonicity of $\gsssem$ we can deduce from \ref{frm:tmp1} and \ref{frm:tmp6} that
    \begin{displaymath}
    \pi \in \gamma(\grad{\pi_2})
    \label{frm:tmp6x}
    \end{displaymath}
    Using \ref{frm:perf-def} it follows that
    \begin{align}
    \thoare{}{\phi_{1b}}{s}{\phi} ~\wedge~ \evalphiGen{\pi_1}{\phi_{1b}} ~\wedge~ \evalphiGen{\pi}{\phi}
    \quad\quad\text{for some } \phi_{1b}, \phi \in \setFormula
    \label{frm:tmp7}
    \end{align}
    Due to $\envs{\pi_1}$ being an ideal we can derive from $\evalphiGen{\pi_1}{\phi_{1a}}$ (\ref{frm:tmp5}) and $\evalphiGen{\pi_1}{\phi_{1b}}$ (\ref{frm:tmp7}) that
    \begin{align}
    \evalphiGen{\pi_1}{\phi_1} ~\wedge~ \phiImplies{\phi_1}{\phi_{1a}} ~\wedge~ \phiImplies{\phi_1}{\phi_{1b}}
    \quad\quad\text{for some } \phi_1 \in \setFormula
    \label{frm:tmp8}
    \end{align}
    From monotonicity of $\thoare{}{\cdot}{\cdot}{\cdot}$ in its first argument (% REF %) we can deduce
    \begin{align}
    \thoare{}{\phi_1}{s}{\phi'} ~\wedge~ \phiImplies{\phi'}{\phi}
    \quad\quad\text{for some } \phi_2 \in \setFormula
    \label{frm:tmp9}
    \end{align}
    Applying the introduction rule for deterministic liftings we can deduce
    \begin{align}
    \dgthoare{}{\phi_1}{s}{\grad{\phi}}
    \quad\quad\text{for some } \grad{\phi} \in \setGFormula
    \label{frm:tmp10}
    \end{align}
    Applying the strength rule for deterministic liftings we can derive from \ref{frm:tmp9} and \ref{frm:tmp10} that
    \begin{align}
    \thoare{}{\phi_1}{s}{\phi_2} ~\wedge~ \phiImplies{\phi_2}{\phi'}
    \quad\quad\text{for some } \phi_2 \in \gamma(\grad{\phi})
    \label{frm:tmp11}
    \end{align}
    From soundness of the static system we can deduce (using \ref{frm:tmp10}, \ref{frm:tmp6}, \ref{frm:tmp8}) that
    \begin{align}
    \evalphiGen{\pi}{\phi_2}
    \label{frm:tmp12}
    \end{align}
    and therefore (using \ref{frm:tmp11})
    \begin{align}
    \evalgphiGen{\pi}{\grad{\phi}}
    \label{frm:tmp13}
    \end{align}
    and therefore (using \ref{frm:tmp6x})
    \begin{align}
    \evalgphiGen{\grad{\pi_2}}{\grad{\phi}}
    \label{frm:tmp14}
    \end{align}
    Now, apply monotonicity of deterministic liftings to \ref{frm:tmp10}, we can use $\envs{\phi_1} \subseteq \envs{\grad{\phi_1}}$ to derive
    \begin{align}
    \envs{\grad{\phi}} \subseteq \envs{\grad{\phi_2}}
    \label{frm:tmp15}
    \end{align}
    and therefore
    \begin{align}
    \evalgphiGen{\grad{\pi_2}}{\grad{\phi_2}}
    \label{frm:tmp16}
    \end{align}
\end{proof}
 
\begin{comment} 
MINUS:
- need above knowledge...
- not always desirable
    \begin{verbatim}
    {i = 10000}
    n = collatzIterations(300, i);
    {1 <= n * n <= 4}
    {n = 4}
    staticAssert (n = 4);
    {n = 4}
    \end{verbatim}
    would throw exception

Proof:
\begin{description}
    \item $\grad{s} \in \setGStmt$
    \item $\grad{\phi_1}, \grad{\phi_2} \in \setGFormula$
    \item $\grad{\pi_1}, \grad{\pi_2} \,\in \setGProgramState$
    \item[1 = Premise] $\dgthoare{~}{\grad{\phi_1}}{\grad{s}}{\grad{\phi_2}}$
    \item[2 = HoareIntrosA] $\gsssem^{\grad{s}}(\grad{\pi_1}, \grad{\pi_2})$
    \item[3 = HoareIntrosB] $\evalgphiGen{\grad{\pi_1}}{\grad{\phi_1}}$
    \item[4 = Case] $\exists \pi_s \in \gamma(\grad{\pi_1}).~ \pi_s \in \wsp(s)$
    \item[5 = 4 + wsp def] $\exists \phi_1', \phi' \in \setFormula.~ \evalphiGen{\pi_s}{\phi_1'} ~\wedge \thoare{~}{\phi_1'}{s}{\phi'}$
    \item[6 = 4 + 5 + rule42] $\exists \phi_1 \in \gamma(\grad{\phi_1}).~ \phiImplies{\phi_1}{\phi_1'} \wedge \evalphiGen{\pi_s}{\phi_1}$
    \item[7 = 5 + 6 + mono] $\exists \phi \in \setFormula.~ \thoare{~}{\phi_1}{s}{\phi}$
    \item[8 = 7 + intro] $\exists \grad{\phi} \in \setGFormula.~ \dgthoare{~}{\phi_1}{s}{\grad{\phi}}$
    \item[9 = 1 + 6 + 8 + mono_det_hoare] $\grad{\phi} \sqsubseteq \grad{\phi_2}$
    \item[10 = 8 + pres] $\exists \phi_2 \in \gamma(\grad{\phi}).~ \thoare{~}{\phi_1}{s}{\phi_2}$
    \item[11 = 6 + 10 + snd] $\evalphiGen{\sssem^s(\pi_s)}{\phi_2}$
    \item[12 = 11 + intro] $\evalphiGen{\gsssem^s(\pi_s)}{\phi_2}$
    \item[13 = 3 + 12 + mono] $\evalphiGen{\gsssem^{\grad{s}}(\grad{\pi_{\grad{s}}})}{\phi_2}$
    \item[14 = 13 + intro] $\evalgphiGen{\gsssem^{\grad{s}}(\grad{\pi_{\grad{s}}})}{\phi_2}$
    \item[15 = 10 + 14 + mono] $\evalgphiGen{\gsssem^{\grad{s}}(\grad{\pi_{\grad{s}}})}{\grad{\phi}}$
    \item[16 = 9 + 15 + mono] $\evalgphiGen{\gsssem^{\grad{s}}(\grad{\pi_{\grad{s}}})}{\grad{\phi_2}}$
\end{description}


\begin{description}
    \item $\grad{s} \in \setGStmt$
    \item $\grad{\phi_1}, \grad{\phi_2} \in \setGFormula$
    \item $\grad{\pi_{\grad{s}}} \,\in \setGProgramState_{\grad{s}}$
    \item[1 = PremiseA] $\dgthoare{~}{\grad{\phi_1}}{\grad{s}}{\grad{\phi_2}}$
    \item[2 = PremiseB] $\evalgphiGen{\grad{\pi_{\grad{s}}}}{\grad{\phi_1}}$
    \item[3 = Case] $\neg \exists \pi_s \in \gamma(\grad{\pi_{\grad{s}}}).~ \pi_s \in \wsp(s)$
    \item[4 = 3 + completeness] $\forall \pi_s \in \gamma(\grad{\pi_{\grad{s}}}).~ \sssem^s(\pi_s) \textit{ stuck}$
    \item[5 = 4 + def] $\gsssem^{\grad{s}}(\grad{\pi_{\grad{s}}}) = \pi_{EX}$
    \item[6 = 5 + precision] $\evalgphiGen{\gsssem^{\grad{s}}(\grad{\pi_{\grad{s}}})}{\grad{\phi_2}}$
\end{description}

\end{comment}
