%% intro
When first introducing \tset{\dgradT Soundness} in section \ref{ssec:the-deterministic-approach}, we also gave an example (\ref{ex:cex-gdpres}) of a language violating \tset{\dgradT Soundness}.
The root of the problem seemed to be the existence of small-step derivations that are not verifiable using static Hoare logic.
A Hoare logic that can prove all small-step derivations correct is called “complete”.
However, as illustrated in example \ref{ex:cex-gdpres}, incompleteness of the Hoare logic may not necessarily indicate weakness of the Hoare logic, but may also be caused by optimized small-step semantics.

% EXAMPLES

%% completeness
As completeness seems to be a property of the entire semantics (instead of just the Hoare logic), we will define it as such.
\begin{definition}[Completeness]
    \label{def:completeness}~\\
    A semantics is \textbf{complete} if every execution of a statement $s$ can be supported with a matching Hoare logic derivation:
    \begin{flalign*}
    \forall s \in \setStmt,\, \pi_1, \pi_2 \in \setProgramState.&~ \sstepConsume{s}{\pi_1}{\pi_2} \\
    \implies \exists \phi_1, \phi_2 \in \setFormula.&~ \thoare{}{\phi_1}{s}{\phi_2} \wedge \evalphiGen{\pi_1}{\phi_1}
    \end{flalign*}
\end{definition}

(Note that $\evalphiGen{\pi_2}{\phi_2}$ is then implied by soundness of the static language and thus not stated in above definition.)

As indicated before, complete semantics may be derived by enhancing the existing Hoare logic but also by restricting the domain of the existing small-step semantics.
As long as soundness is not broken, those changes do not have observable effects on \svl programs, i.e. valid programs will behave identically before and after respective adjustments.
\begin{lemma}[Completion of Semantics]\label{lemma:compl-sem}~\\
    Deriving complete semantics by restricting the domain of the small-step semantics or extending the Hoare logic of \svl (without breaking soundness) does not have observable effects.
\end{lemma}

%% further stuff
Based on a complete semantics of \svl, the following requirements for \gvl are then sufficient to satisfy \tset{\dgradT Soundness}.
\begin{definition}[Semi-Optimal Gradual Deterministic Hoare Logic]
    \label{def:perf-stat}~\\
    We call a deterministically lifted Hoare logic $\dgthoare{}{\cdot}{\cdot}{\cdot}$ \textbf{semi-optimal} iff
    \begin{align*}
    \forall \phi_1 \in \setFormula,\, \grad{s} \in \setGStmt,\, \grad{\phi_2} \in \setGFormula.&~ \dgthoare{}{\phi_1}{\grad{s}}{\grad{\phi_2}}\\ \implies \exists s \in \gamma(\grad{s}),\, \phi_2 \in \setFormula.&~ \thoare{}{\phi_1}{s}{\phi_2}
    \end{align*}
    Intuitively, the deterministic lifting is only supposed to be defined for static preconditions if there exists some corresponding derivation of the static Hoare logic.
    
    As expected, optimal deterministic liftings are semi-optimal.
\end{definition}

\begin{comment}[Semi-Optimal Gradual Deterministic Assertion]\label{lemma:so-assert}~\\
    Let $\setGFormula$ be extended using the “bounded unknown” approach (see section \ref{ssec:wildcard-with-upper}).
    
    Assume there is a Hoare logic for assertion statements:
    \begin{mathpar}
        \inferrule* [Right=HAssert]
        {
            \phiImplies{\phi}{\phi_a}
        }
        {
            \thoare{~}{\phi}{\sAssert{$\phi_a$}}{\phi}
        }
    \end{mathpar}
    
    Then the following rules induce a semi-optimal deterministic lifting of that Hoare logic:
    \begin{mathpar}
        \inferrule* [Right=\dgradT HAssertStat1]
        {
            \phiImplies{\phi}{\phi_a}
        }
        {
            \dgthoare{~}{\phi}{\sAssert{$\phi_a$}}{\phi}
        }
    \end{mathpar}
    \begin{mathpar}
        \inferrule* [Right=\dgradT HAssertStat2]
        {
            \phiImplies{\phi}{\phi_a}
        }
        {
            \dgthoare{~}{\phi}{\sAssert{$\withqmGen{\phi_a}$}}{\phi}
        }
    \end{mathpar}
    \begin{mathpar}
        \inferrule* [Right=\dgradT HAssertStat2]
        {
            \phiImplies{\phi}{\phi_a}
        }
        {
            \dgthoare{~}{\phi}{\sAssert{$\withqmGen{\phi_a}$}}{\phi}
        }
    \end{mathpar}
\end{comment}
\begin{comment}%\ref{lemma:so-assert}
    \begin{description}
        \item[Semi-Optimality]~\\
        Assume that $\dgthoare{}{\phi_1}{\sVarAssign{$x$}{$e$}}{\grad{\phi_2}}$ holds for some $\phi_1 \in \setFormula,\, \grad{\phi_2} \in \setGFormula,\, x \in \setVar,\, e \in \setExpr$
        
        By inversion ()
    \end{description}
    
    
    
    \\ \implies \exists s \in \gamma(\grad{s}),\, \phi_2 \in \setFormula.&~ \thoare{}{\phi_1}{s}{\phi_2}
    \end{align*}
\end{comment}

\begin{definition}[Semi-Optimal Gradual Small-Step Semantics]
    \label{def:perf-dyn}~\\
    We call a total, lifted small-step semantics $\gsstep{\cdot}{\cdot}$ \textbf{semi-optimal} iff
    \begin{comment}
    %\label{frm:perf-ex}
    \forall \grad{\pi} \in \setGProgramState.~ (\forall \pi \in \gamma(\grad{\pi}).~ \sstepStuck{\pi}) \implies \gsstep{\grad{\pi}}{\pi_{EX}}
    \end{comment}
    \begin{align*}
    %\label{frm:perf-ex}
    \forall \grad{\pi_1}, \grad{\pi_2} \in \setGProgramState,\, \grad{s} \in \setGStmt.&~ \gsstepConsume{\grad{s}}{\grad{\pi_1}}{\grad{\pi_2}}\\ \implies \exists \pi_1 \in \gamma(\grad{\pi_1}),\, s \in \gamma(\grad{s}), \pi_2 \in \setProgramState.&~ \sstepConsume{s}{\pi_1}{\pi_2}
    \end{align*}
    Intuitively, the gradual small-step semantics is not supposed to define derivations if all concretizations would be stuck in the original small-step semantics.
    Note that the rules of gradual lifting (of partial functions) allow arbitrary behavior in case the small-step semantics are stuck (i.e. the function is undefined).
\end{definition}

We call the semantics of \gvl semi-optimal if both Hoare logic and small-step semantics are semi-optimal as defined above.

\begin{theorem}[Completeness and Semi-Optimality imply \tset{\dgradT Soundness}]
    \label{thm:compl-and-so-to-gdpres}~\\
    If the semantics of \svl are complete and the semantics of \gvl are semi-optimal, then \tset{\dgradT Soundness} is satisfied.
\end{theorem}
