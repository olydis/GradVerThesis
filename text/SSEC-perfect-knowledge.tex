%% intro
When first introducing \tset{GDPreservation} in section \ref{ssec:the-deterministic-approach}, we also gave an example (\ref{ex:cex-gdpres}) of a language violating \tset{GDPreservation}.
The root of the problem seemed to be the existence of small-step derivations that are not verifiable using static Hoare logic.
A Hoare logic that can prove all small-step derivations correct is called “complete”.
However, as illustrated in example \ref{ex:cex-gdpres}, incompleteness of the Hoare logic may not necessarily indicate weakness of the Hoare logic, but may also be due to optimized small-step semantics.

% EXAMPLES

%% completeness
As completeness seems to be a property of the entire semantics instead of just the Hoare logic, we will define it as such.
\begin{definition}[Completeness]~\\
    \label{def:completeness}
    A semantics is \textbf{complete} if every execution of a statement $s$ can be supported with a matching Hoare logic derivation:
    \begin{flalign*}
    & \forall s \in \setStmt,\, \pi_1, \pi_2 \in \setProgramState.~ \sstepConsume{s}{\pi_1}{\pi_2} \implies \exists \phi_1, \phi_2 \in \setFormula.~ \thoare{}{\phi_1}{s}{\phi_2} \wedge \evalphiGen{\pi_1}{\phi_1}
    \end{flalign*}
\end{definition}

As indicated before, complete semantics may be derived by enhancing the existing Hoare logic but also by restricting the domain of the existing small-step semantics.
As long as soundness is not broken, those changes do not have observable effects on \svl programs, i.e. valid programs will behave identically before and after respective adjustments.
\begin{lemma}[Completion of Semantics]
    Deriving sound, complete semantics by restricting the domain of the small-step semantics or extending the Hoare logic of \svl does not have observable effects.
\end{lemma}
\begin{proof}
    \svl only ever executes a statement $s$ if it was successfully verified.
    After extending the Hoare logic, derivable Hoare triples are still derivable.
    Successful verification implies that there exists some Hoare derivation $\thoare{}{\phi_1}{s}{\phi_2}$ for every statement $s$ executed by the program.
    Soundness of the Hoare logic implies that the small-step semantics do not get stuck while executing $s$.
    Thus, the restricted domain of $\sstep{\cdot}{\cdot}$ can never affect reachable executions.
\end{proof}

%% further stuff
Based on a complete semantics of \svl, the following requirements for \gvl are sufficient to satisfy \tset{GDPreservation}.
\begin{definition}[Perfect Gradual Hoare Logic]
    \label{def:perf-stat}
    We call a deterministically lifted Hoare logic $\dgthoare{}{\cdot}{\cdot}{\cdot}$ perfect iff
    \begin{displaymath}
    \forall \phi_1 \in \setFormula,\, \grad{s} \in \setGStmt,\, \grad{\phi_2} \in \setGFormula.~ \dgthoare{}{\phi_1}{\grad{s}}{\grad{\phi_2}} \implies \exists s \in \gamma(\grad{s}),\, \phi_2 \in \setFormula.~ \thoare{}{\phi_1}{s}{\phi_2}
    \end{displaymath}
    Intuitively, the deterministic lifting is only supposed to be defined for static preconditions if there exists some corresponding derivation of the static Hoare logic.
    Note that optimal deterministic liftings are perfect.
    Perfection can be viewed as a very weak optimality measure.
\end{definition}
\begin{definition}[Perfect Gradual Small-Step Semantics]
    \label{def:perf-dyn}
    We call a total, lifted small-step semantics $\gsstep{\cdot}{\cdot}$ perfect iff
    \begin{comment}
    %\label{frm:perf-ex}
    \forall \grad{\pi} \in \setGProgramState.~ (\forall \pi \in \gamma(\grad{\pi}).~ \sstepStuck{\pi}) \implies \gsstep{\grad{\pi}}{\pi_{EX}}
    \end{comment}
    \begin{displaymath}
    %\label{frm:perf-ex}
    \forall \grad{\pi_1}, \grad{\pi_2} \in \setGProgramState,\, \grad{s} \in \setGStmt.~ \gsstepConsume{\grad{s}}{\grad{\pi_1}}{\grad{\pi_2}} \implies \exists \pi_1 \in \gamma(\grad{\pi_1}),\, s \in \gamma(\grad{s}), \pi_2 \in \setFormula.~ \sstepConsume{s}{\pi_1}{\pi_2}
    \end{displaymath}
    Intuitively, the gradual small-step semantics is not supposed to define derivations if all concretizations would be stuck in the original small-step semantics.
    Note that the rules of gradual lifting (of partial functions) allow arbitrary behavior in case the small-step semantics are stuck (i.e. the function is undefined).
    Optimal deterministic liftings are perfect.
    Perfection can be viewed as a very weak optimality measure.
\end{definition}

We call the semantics of \gvl perfect if both Hoare logic and small-step semantics are perfect as defined above.

\begin{theorem}[]
    If the semantics of \svl are complete and the semantics of \gvl are perfect, then \tset{GDPreservation} is satisfied.
\end{theorem}
\begin{proof}
    To prove \tset{GDPreservation} we may assume:
    \begin{align}
    \label{frm:tmp0}
    \dgthoare{}{\grad{\phi_1}}{\grad{s}}{\grad{\phi_2}}\\
    \label{frm:tmp1}
    \gsstepConsume{\grad{s}}{\grad{\pi_1}}{\grad{\pi_2}}\\
    \label{frm:tmp2}
    \evalgphiGen{\grad{\pi_1}}{\grad{\phi_1}}
    \end{align}
    Goal: 
    \begin{align}
    \label{frm:tmp3}
    \evalgphiGen{\grad{\pi_2}}{\grad{\phi_2}}
    \end{align}
    
    Case $\grad{\phi_1} = \phi_1$ for some $\phi_1 \in \setFormula$:
        We can simplify \ref{frm:tmp2} as
        \begin{align}
        \label{frm:tmpx0}
        \evalphiGen{\grad{\pi_1}}{\phi_1}
        \end{align}
        From perfection of \ref{frm:tmp0} (see definition \ref{def:perf-dyn}) we can derive
        \begin{align}
        \label{frm:tmpx1}
        \thoare{}{\phi_1}{s}{\phi}
        \quad\quad\text{for some } s \in \gamma(\grad{s}),\, \phi_2 \in \setFormula
        \end{align}
        Applying the strength rule for deterministic liftings we can derive from \ref{frm:tmpx1} and \ref{frm:tmp0} that
        \begin{align}
        \label{frm:tmpx2}
        \thoare{}{\phi_1}{s}{\phi_2} ~\wedge~ \phiImplies{\phi_2}{\phi}
        \quad\quad\text{for some } \phi_2 \in \gamma(\grad{\phi_2})
        \end{align}
        From soundness (progress) of the static Hoare logic we can deduce that executing $s$ from a program state satisfying $\phi_1$ (like $\grad{\pi_1}$) does not end up in a stuck state.
        Combining this knowledge with $\ref{frm:tmp1}$ we can conclude that executing $s$ must terminate (otherwise, $\gsstep{\cdot}{\cdot}$, being the gradual lifting, could not terminate for $\grad{s} \sqsupseteq s$).
        \begin{align}
        \label{frm:tmpx3}
        \sstepConsume{s}{\pi_1}{\pi_2}
        \quad\quad\text{for some } \pi_1 \in \gamma(\grad{\pi_1}),\, \pi_2 \in \setProgramState
        \end{align}
        From soundness (preservation) of the static Hoare logic we can deduce
        \begin{align}
        \label{frm:tmpx4}
        \evalphiGen{\pi_2}{\phi_2}
        \end{align}
        Applying the introduction rule of gradually lifted functions to \ref{frm:tmpx3} we can derive
        \begin{align}
        \label{frm:tmpx5}
        \gsstepConsume{s}{\pi_1}{\grad{\pi}} ~\wedge~ \pi_2 \sqsubseteq \grad{\pi}
        \quad\quad\text{for some } \grad{\pi} \in \setGProgramState
        \end{align}
        It follows that
        \begin{align}
        \label{frm:tmpx6}
        \evalphiGen{\grad{\pi}}{\phi_2}
        \end{align}
        Applying the monotonicity rule of gradually lifted functions to \ref{frm:tmpx5} and \ref{frm:tmp1} we can derive that
        \begin{align}
        \label{frm:tmpx7}
        \grad{\pi} \mpt \grad{\pi_2}
        \end{align}
        and thus
        \begin{align}
        \label{frm:tmpx8}
        \evalphiGen{\grad{\pi_2}}{\phi_2}
        \end{align}
        We can generalize this using $\phi_2 \in \gamma(\grad{\phi_2})$ (see \ref{frm:tmpx2})
        \begin{align}
        \label{frm:tmpx9}
        \evalgphiGen{\grad{\pi_2}}{\grad{\phi_2}}
        \end{align}
    
    Case $\grad{\phi_1}$ partially unknown, i.e. $\gamma(\grad{\phi_1})$ closed over implication:
        From perfection of \ref{frm:tmp1} (see definition \ref{def:perf-dyn}) we can derive
        \begin{align}
        \label{frm:tmp6}
        \sstepConsume{s}{\pi_1}{\pi_2}
        \quad\quad\text{for some } \pi_1 \in \gamma(\grad{\pi_1}),\, s \in \gamma(\grad{s}),\, \pi_2 \in \setProgramState
        \end{align}
        Recall that formula evaluation is immune to concretization (REF), so from \ref{frm:tmp2} follows
        \begin{align}
        \label{frm:tmp5}
        \evalgphiGen{\pi_1}{\grad{\phi_1}}
        \text{\quad\quad or equivalently \quad\quad}
        \evalphiGen{\pi_1}{\phi_{1a}}
        \quad\quad\text{for some } \phi_{1a} \in \gamma(\grad{\phi_1}) 
        \end{align}
        Using the monotonicity of $\gsstep{\cdot}{\cdot}$ we can deduce from \ref{frm:tmp1} and \ref{frm:tmp6} that
        \begin{align}
        \label{frm:tmp6x}
        \pi_2 \in \gamma(\grad{\pi_2})
        \end{align}
        From completeness of the static system (see definition \ref{def:completeness}) it follows from \ref{frm:tmp6} that
        \begin{align}
        \label{frm:tmp7}
        \thoare{}{\phi_{1b}}{s}{\phi} ~\wedge~ \evalphiGen{\pi_1}{\phi_{1b}}
        \quad\quad\text{for some } \phi_{1b}, \phi \in \setFormula
        \end{align}
        Due to $\envs{\pi_1}$ being a filter we can derive from $\evalphiGen{\pi_1}{\phi_{1a}}$ (\ref{frm:tmp5}) and $\evalphiGen{\pi_1}{\phi_{1b}}$ (\ref{frm:tmp7}) that
        \begin{align}
        \label{frm:tmp8}
        \evalphiGen{\pi_1}{\phi_1} ~\wedge~ \phiImplies{\phi_1}{\phi_{1a}} ~\wedge~ \phiImplies{\phi_1}{\phi_{1b}}
        \quad\quad\text{for some } \phi_1 \in \setFormula
        \end{align}
        From monotonicity of $\thoare{}{\cdot}{\cdot}{\cdot}$ in its first argument (% REF %) we can deduce
        \begin{align}
        \label{frm:tmp9}
        \thoare{}{\phi_1}{s}{\phi'} ~\wedge~ \phiImplies{\phi'}{\phi}
        \quad\quad\text{for some } \phi_2 \in \setFormula
        \end{align}
        Applying the introduction rule for deterministic liftings we can deduce
        \begin{align}
        \label{frm:tmp10}
        \dgthoare{}{\phi_1}{s}{\grad{\phi}}
        \quad\quad\text{for some } \grad{\phi} \in \setGFormula
        \end{align}
        Applying the strength rule for deterministic liftings we can derive from \ref{frm:tmp9} and \ref{frm:tmp10} that
        \begin{align}
        \label{frm:tmp11}
        \thoare{}{\phi_1}{s}{\phi_2} ~\wedge~ \phiImplies{\phi_2}{\phi'}
        \quad\quad\text{for some } \phi_2 \in \gamma(\grad{\phi})
        \end{align}
        From soundness of the static system we can deduce (using \ref{frm:tmp10}, \ref{frm:tmp6}, \ref{frm:tmp8}) that
        \begin{align}
        \label{frm:tmp12}
        \evalphiGen{\pi_2}{\phi_2}
        \end{align}
        and therefore (using \ref{frm:tmp6x})
        \begin{align}
        \label{frm:tmp14}
        \evalgphiGen{\grad{\pi_2}}{\phi_2}
        \end{align}
        and therefore (using \ref{frm:tmp11})
        \begin{align}
        \label{frm:tmp13}
        \evalgphiGen{\grad{\pi_2}}{\grad{\phi}}
        \end{align}
        Now, using $\phi_1 \in \gamma(\grad{\phi_1})$ (implied from the fact that $\phi_{1a} \in \gamma(\grad{\phi_1})$, $\phiImplies{\phi_1}{\phi_{1a}}$) we can apply monotonicity of deterministic liftings to \ref{frm:tmp10}, we can derive
        \begin{align}
        \label{frm:tmp15}
        \grad{\phi} \mpt \grad{\phi_2}
        \end{align}
        and therefore
        \begin{align}
        \label{frm:tmp16}
        \evalgphiGen{\grad{\pi_2}}{\grad{\phi_2}}
        \end{align}
\end{proof}