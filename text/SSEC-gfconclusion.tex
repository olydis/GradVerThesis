Since lifted predicates and functions directly affect the gradual semantics of \gvl, they must adhere to certain rules in order to be sound.
What soundness means is a direct consequence of the gradual guarantee for gradual verification systems, which we derive from the gradual guarantee for gradual type systems by Siek et al. \cite{siek2015refined}.

For simplicity we will simply call programs “acceptable” if they are successfully verifiable by the gradual verifier.

\begin{definition}[Gradual Guarantee (Static Semantics)]
    \label{grad-guarantee-static}~\\
    Acceptable programs remain acceptable when reducing precision of any formula.
\end{definition}

\begin{definition}[Gradual Guarantee (Dynamic Semantics)]
    \label{grad-guarantee-dynamic}~\\
    Acceptable programs with a particular observational behavior (termination, values of variables, output, etc.) will have the same observational behavior after reducing precision of any formula.
\end{definition}

\begin{comment}


% "static" function?
% - here: always part of the concretization, ... later we will see counterexample


PROBABLY UNNECESSARY:\\
Because of its generality, we will pursue the approach introduced in section \ref{ssec:wildcard-with-upper} for the remainder of this chapter.
As concretization we chose the semantic version, as it is more flexible than the syntactic one in practice.
For reference, the full definitions:
\begin{align*} 
&\text{Syntax:}\\
&\grad{\phi} ::= \phi ~|~ \withqmGen{\phi}\\
\\
&\text{Concretization:}\\
&\gamma(\phi) = \{~ \phi ~\}     \quad\quad \forall \phi \in \setFormulaA\\
&\gamma(\withqmGen{\phi}) = \{~ \phi' \in \setFormulaA ~|~ \phiImplies{\phi'}{\phi} ~\}\\
&\gamma(\grad{\phi}) = \emptyset    \quad\textit{otherwise}
\end{align*}
\end{comment}
