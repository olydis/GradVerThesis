
Comparing gradual formulas (e.g. $\ttt{x = 3}$, $\withqmGen{x = 3}$, $\qm$) gives rise to a notion of “precision”.
Intuitively, $\ttt{x = 3}$ is more precise than $\withqmGen{x = 3}$ which is still more precise than $\qm$.
Using concretization, we can formalize this intuition.
\begin{definition}[Formula Precision]
    $$\grad{\phi_a} \sqsubseteq \grad{\phi_b}  \quad\iff\quad  \gamma(\grad{\phi_a}) \subseteq \gamma(\grad{\phi_b})$$
    Read: Formula $\grad{\phi_a}$ is “at least as precise as” $\grad{\phi_b}$.
\end{definition}
The strict version $\sqsubset$ is defined accordingly.

With the notion of precision, we can give a formal definition of the gradual guarantee (TODO: ref) that we are aiming to satisfy.
\begin{definition}[Gradual Guarantee (for Gradual Verification Systems)]
\label{grad-guarantee-def}
    %TODO
\end{definition}

% TODO: introduce ensureDyn-function? with some lemmas about when it returns true?