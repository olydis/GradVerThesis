The static semantics of \svlidf consist of typing rules and a Hoare logic making use of those typing rules.
All the rules are implicitly parameterized over some program $p \in \setProgram$, necessary for example to extract the type of a field in the following typing rules.

\begin{figure}[h]
    \boxed{\sType{\Gamma}{e}{T}}
    %% Inductive Semantics.hasStaticType
\begin{mathpar}
\inferrule* [Right=STValNum]
{
    ~
}
{
    \sType {\Gamma} {\ev{${{n}}$}} {\Tint}
}
\end{mathpar}

\begin{mathpar}
\inferrule* [Right=STValNull]
{
    ~
}
{
    \sType {\Gamma} {\ev{${\enull}$}} {{C}}
}
\end{mathpar}

\begin{mathpar}
\inferrule* [Right=STVar]
{
    {\Gamma(x)} = {{T}}
}
{
    \sType {\Gamma} {\ex{${x}$}} {T}
}
\end{mathpar}

\begin{mathpar}
\inferrule* [Right=STField]
{
    \sType {\Gamma} {e} {{C}} \\
    {\fieldType({C}, {f})} = {{T}}
}
{
    \sType {\Gamma} {\edot{${e}$}{${f}$}} {T}
}
\end{mathpar}


% Inductive Semantics.hasStaticType
\begin{mathpar}
\inferrule* [right=STValNum]
{
    ~
}
{
    \sType {\Gamma} {\ev{${{n}}$}} {\Tint}
}

\inferrule* [right=STValNull]
{
    ~
}
{
    \sType {\Gamma} {\ev{${\enull}$}} {{C}}
}

\inferrule* [right=STVar]
{
    {\Gamma(x)} = {{T}}
}
{
    \sType {\Gamma} {\ex{${x}$}} {T}
}

\inferrule* [right=STField]
{
    \sType {\Gamma} {e} {{C}} \\
    {\fieldType({C}, {f})} = {{T}}
}
{
    \sType {\Gamma} {\edot{${e}$}{${f}$}} {T}
}
\end{mathpar}


    \caption{\svlidf: Static Typing of Expressions}
\end{figure}

\begin{figure}[h!]
    \boxed{\thoare {\Gamma} {\phi_{pre}} {s} {\phi_{post}}}
    % Inductive Semantics.hoare
\begin{mathpar}
\inferrule* [Right=HAlloc]
{
    {\phi} \implies {\phi'} \\
    {\emptyset} \sfrmphi {\phi'} \\
    {x} \not \in {FV(\phi')} \\
    {\Gamma} \vdash {\ex{${x}$}} : {{C}} \\
    {\fields({C})} = {{\overline{f}}}
}
{
    {\Gamma} \hoare {\phi} {{\sAlloc {${x}$} {${C}$}}} {\phiCons{${\phi'}$}{${\phiCons{${\phiNeq {${\ex{${x}$}}$} {${\ev{${\vnull}$}}$}}$}{${\overline{\acc({x}, f_i)}}$}}$}}
}
\end{mathpar}

\begin{mathpar}
\inferrule* [Right=HFieldAssign]
{
    {\phi} \implies {\phiCons{${\phiAcc {${\ex{${x}$}}$} {${f}$}}$}{${\phi'}$}} \\
    {\emptyset} \sfrmphi {\phi'} \\
    {\Gamma} \vdash {\ex{${x}$}} : {{C}} \\
    {\Gamma} \vdash {\ex{${y}$}} : {T} \\
    \vdash {C}.{f} : {T}
}
{
    {\Gamma} \hoare {\phi} {{\sFieldAssign {${x}$} {${f}$} {${y}$}}} {\phiCons{${\phi'}$}{${\phiCons{${\phiAcc {${\ex{${x}$}}$} {${f}$}}$}{${\phiCons{${\phiNeq {${\ex{${x}$}}$} {${\ev{${\vnull}$}}$}}$}{${\ensuremath{{\phiEq {${\edot{${\ex{${x}$}}$}{${f}$}}$} {${\ex{${y}$}}$}}}}$}}$}}$}}
}
\end{mathpar}

\begin{mathpar}
\inferrule* [Right=HVarAssign]
{
    {\phi} \implies {\phi'} \\
    {\emptyset} \sfrmphi {\phi'} \\
    {x} \not \in {FV(\phi')} \\
    {x} \not \in {FV({e})} \\
    {\Gamma} \vdash {\ex{${x}$}} : {T} \\
    {\Gamma} \vdash {e} : {T} \\
    {\accFor {{e}}} \subseteq {\phi'}
}
{
    {\Gamma} \hoare {\phi} {{\sVarAssign {${x}$} {${e}$}}} {\phiCons{${\phi'}$}{${\ensuremath{{\phiEq {${\ex{${x}$}}$} {${e}$}}}}$}}
}
\end{mathpar}

\begin{mathpar}
\inferrule* [Right=HReturn]
{
    {\phi} \implies {\phi'} \\
    {\emptyset} \sfrmphi {\phi'} \\
    {\xresult} \not \in {FV(\phi')} \\
    {\Gamma} \vdash {\ex{${x}$}} : {T} \\
    {\Gamma} \vdash {\ex{${\xresult}$}} : {T}
}
{
    {\Gamma} \hoare {\phi} {{\sReturn {${x}$}}} {\phiCons{${\phi'}$}{${\ensuremath{{\phiEq {${\ex{${\xresult}$}}$} {${\ex{${x}$}}$}}}}$}}
}
\end{mathpar}

\begin{mathpar}
\inferrule* [Right=HCall]
{
    {\Gamma} \vdash {\ex{${y}$}} : {{C}} \\
    {\mmethod({C}, {m})} = {{\method {${T_r}$} {${m}$} {${T_p}$} {${z}$} {${\requires {\phi_{pre}};~\ensures {\phi_{post}};}$} {${\_}$}}} \\
    {\Gamma} \vdash {\ex{${x}$}} : {T_r} \\
    {\Gamma} \vdash {\ex{${z'}$}} : {T_p} \\
    {\phi} \implies {\phiCons{${\phiNeq {${\ex{${y}$}}$} {${\ev{${\vnull}$}}$}}$}{${\phiCons{${\phi_p}$}{${\phi'}$}}$}} \\
    {\emptyset} \sfrmphi {\phi'} \\
    {x} \not \in {FV(\phi')} \\
    x \neq y \wedge x \neq z' \\
    {\phi_p} = {{\phi_{pre}}[{y}, {z'} / {\xthis}, {{z}}]} \\
    {\phi_q} = {{\phi_{post}}[{y}, {z'}, {x} / {\xthis}, {{z}}, {\xresult}]}
}
{
    {\Gamma} \hoare {\phi} {{\sCall {${x}$} {${y}$} {${m}$} {${z'}$}}} {\phiCons{${\phi'}$}{${\phi_q}$}}
}
\end{mathpar}

\begin{mathpar}
\inferrule* [Right=HAssert]
{
    {\phi} \implies {\phi'}
}
{
    {\Gamma} \hoare {\phi} {{\sAssert {${\phi'}$}}} {\phi}
}
\end{mathpar}

\begin{mathpar}
\inferrule* [Right=HRelease]
{
    {\phi} \implies {\phiCons{${\phi_r}$}{${\phi'}$}} \\
    {\emptyset} \sfrmphi {\phi'}
}
{
    {\Gamma} \hoare {\phi} {{\sRelease {${\phi_r}$}}} {\phi'}
}
\end{mathpar}

\begin{mathpar}
\inferrule* [Right=HDeclare]
{
    {x} \not\in \dom({\Gamma}) \\
    {{\Gamma}, {x} : {T}} \hoare {\phiCons{${\phiEq {${\ex{${x}$}}$} {${\ev{${\texttt{defaultValue}({T})}$}}$}}$}{${\phi}$}} {\overline{s}} {\phi'}
}
{
    {\Gamma} \hoare {\phi} {{\sDeclare {${T}$} {${x}$}} {\overline{s}}} {\phi'}
}
\end{mathpar}

\begin{mathpar}
\inferrule* [Right=HHold]
{
    {\phi_f} \implies {\phiCons{${\phi_r}$}{${\phi'}$}} \\
    {\phi'} \implies {\phi} \\
    {\Gamma} \hoare {\phi_r} {\overline{s}} {\phi_r'}
}
{
    {\Gamma} \hoare {\phi_f} {{\sHold {${\phi}$} {${\overline{s}}$}}} {\phiCons{${\phi_r'}$}{${\phi'}$}}
}
\end{mathpar}

\begin{mathpar}
\inferrule* [Right=HSeq]
{
    {\Gamma} \hoare {\phi_p} {\overline{s_1}} {\phi_q} \\
    {\Gamma} \hoare {\phi_q} {\overline{s_2}} {\phi_r}
}
{
    {\Gamma} \hoare {\phi_p} {{\overline{s_1}}\ttt{;} {\overline{s_2}}} {\phi_r}
}
\end{mathpar}


    \caption{\svlidf: Hoare Logic} 
\end{figure}

\begin{comment}
Let $\wsp : \setStmt \rightarrow \PP(\setProgramState)$ be defined as
\newcommand{\tempDefPS}{\{~ \pi \in \setProgramState_s ~|~ \exists \phi_1, \phi_2 \in \setFormula,\, \Gamma \in \setTypeEnv.~ \thoare{\Gamma}{\phi_1}{s}{\phi_2} ~~\wedge~~ \evalphiGen{\pi}{\phi_1} ~\}}
\begin{flalign*}
& \wsp(s)                 & =~ & \tempDefPS       & ~ \\
& \wsp(s) & =~ &
\begin{cases}
\setProgramState_s                                                                                  & \text{if~} s = \sAlloc{$x$}{$C$}            \\
\{~ \pi \in \setProgramState_s ~|~ \evalphiGen{\pi}{\phiAcc{$x$}{$f$}} ~\}                          & \text{if~} s = \sFieldAssign{$x$}{$f$}{$y$} \\
\{~ \pi \in \setProgramState_s ~|~ \evalphiGen{\pi}{\accFor{$e$}} ~\}                               & \text{if~} s = \sVarAssign{$x$}{$e$}        \\
\setProgramState_s                                                                                  & \text{if~} s = \sReturn{$x$}                \\
\{~ \pi \in \setProgramState_s ~|~ \evalphiGen{\pi}{\phiCons{\phiNeq{$y$}{\enull}}{\mpre{$m$}}} ~\} & \text{if~} s = \sCall{$x$}{$y$}{$m$}{$z$}   \\
\{~ \pi \in \setProgramState_s ~|~ \evalphiGen{\pi}{\phi} ~\}                                       & \text{if~} s = \sAssert{$\phi$}             \\
\{~ \pi \in \setProgramState_s ~|~ \evalphiGen{\pi}{\phi} ~\}                                       & \text{if~} s = \sRelease{$\phi$}
\end{cases}
& ~ \\ 
\end{flalign*}
\end{comment}