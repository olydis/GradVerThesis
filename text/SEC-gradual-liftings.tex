With a concretization $\gamma$ at hand, we can use gradual lifting (see section \ref{sec:lifting-predicates-and}) to derive gradual versions of some commonly used predicates.
Furthermore, we deterministically lift (see section \ref{ssec:the-deterministic-approach}) predicates that appear as components in the Hoare logic (see figure \ref{fig:svl-sem-stat-hoare}).

Implementation strategies for all of the following predicates can be found in the appendix \ref{app:impl-strat}.

\begin{description}
    \item[Gradual Formula Semantics]~\\
    Let $~\evalgphiGen{\cdot}{\cdot}~ \subseteq \setProgramState \times \setGFormula$ be an optimal predicate lifting of $\evalphiGen{\cdot}{\cdot}$ 
    \\(defined in section \ref{ssec:formula-semantics})
    
    \item[Gradual Self-Framing]~\\
    Let $~\sfrmgphi \cdot~ \subseteq \setGFormula$ be an optimal predicate lifting of $\sfrmphi \cdot$ 
    \\(defined in section \ref{ssec:framing})
    
    \item[Gradual Implication of Static Formula]~\\
    Let $\dgrad{P}_{\phi_t} \subseteq \setGFormula \rightharpoonup \setGFormula$ be an optimal deterministic lifting of 
    \\$P_{\phi_t}(\phi_1, \phi_2) ~\defeq~ \phiImplies{\phi_1}{\phi_t} \wedge \phi_1 = \phi_2$ 
    \\(shape of \tset{HAssert}, found as component of \tset{HVarAssign} and \tset{HHold})
    \\We define a more intuitive notation that matches the original static predicate:
    \begin{displaymath}
    \gphiImpliesEv{\grad{\phi_1}}{\phi_t}{\grad{\phi_2}} ~\defiff~ \dgrad{P}_{\phi_t}(\grad{\phi_1}) = \grad{\phi_2}
    \end{displaymath}
    
    \item[Gradual Formula Extraction]~\\
    Let $\dgrad{P} \subseteq \setGFormula \times \setGFormula \rightharpoonup \setGFormula$ be an optimal deterministic lifting of 
    \\$P(\phi_1, \phi_t, \phi_2) ~\defeq~ \phiImplies{\phi_1}{\phiCons{$\phi_t$}{$\phi_2$}} ~\wedge \sfrmphi \phi_2$ 
    \\(shape of \tset{HRelease}, found as component of \tset{HFieldAssign}, \tset{HCall} and \tset{HHold})
    \\We define a more intuitive notation that reflects the dual nature with “\ttt{*}”:
    \begin{displaymath}
    \wo{\grad{\phi}}{\grad{\phi_t}} ~\defeq~ \dgrad{P}(\grad{\phi}, \grad{\phi_t})
    \end{displaymath}
    
    \begin{comment} % single variable
    \item[Gradual Variable Extraction]
    Let $\dgrad{P}_x \subseteq \setGFormula \rightharpoonup \setGFormula$ be an optimal deterministic lifting of 
    \\$P_x(\phi_1, \phi_2) ~\defeq~ \phiImplies{\phi_1}{\phi_2} ~~\wedge~~ x \not \in \FV(\phi_2) ~~\wedge~~ \sfrmphi \phi_2$ 
    \\(found as component of \tset{HAlloc}, \tset{HVarAssign}, \tset{HReturn}, \tset{HCall} and in a sense \tset{HHold}: the equality between the free variable sets can be reformulated as repeated application of $P_x$)
    \\We extend the domain of the extraction function $\wo{}{}$ to deal with variables:
    \begin{displaymath}
    \wo{\grad{\phi}}{x} ~\defeq~ \dgrad{P}_x(\grad{\phi})
    \end{displaymath}
    \end{comment}
    
    \item[Gradual Variable Extraction]~\\
    Let $\dgrad{P}_{\overline{x}} \subseteq \setGFormula \rightharpoonup \setGFormula$ be an optimal deterministic lifting of 
    \\$P_{\overline{x}}(\phi_1, \phi_2) ~\defeq~ \phiImplies{\phi_1}{\phi_2} ~~\wedge~~ \FV(\phi_2) \cap \overline{x} = \emptyset ~~\wedge~~ \sfrmphi \phi_2$ 
    \\(found as component of \tset{HAlloc}, \tset{HVarAssign}, \tset{HReturn}, \tset{HCall} and in a sense \tset{HHold}: the equality between the free variable sets can be reformulated as an application of $P_{\overline{x}}$)
    \\We extend the domain of the extraction function $\wo{}{}$ to deal with variables:
    \begin{displaymath}
    \wo{\grad{\phi}}{\overline{x}} ~\defeq~ \dgrad{P}_{\overline{x}}(\grad{\phi})
    \end{displaymath}
\end{description}