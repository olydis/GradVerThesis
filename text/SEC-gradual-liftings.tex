With a concretization $\gamma$ at hand, we can use gradual lifting as introduced in section \ref{sec:lifting-predicates-and} to derive gradual versions of some commonly used predicates.

\begin{lemma}[Optimal Lifting of Evaluation]~\\
    \label{ex:idf-opt-lift-evalphi}
    Let $~~\evalgphiGen{\cdot}{\cdot}~ \subseteq \setProgramState \times \setGFormula$ be defined inductively as
    \begin{mathpar}
        \inferrule* [Right=\gradT EvalStatic]
        {
            \evalphiGen{\pi}{\phi}
        }
        {
            \evalgphiGen{\pi}{\phi}
        }
    \end{mathpar}
    \begin{mathpar}
        \inferrule* [Right=\gradT EvalGrad]
        {
            \evalphiGen{\pi}{\phi}\\
            A_s \sfrmphi \phi\\
            \forall \langle e, f \rangle \in A_s.~ \evalphiGen{\pi}{\phiAcc{e}{f}}
        }
        {
            \evalgphiGen{\pi}{\withqm{\phi}}
        }
    \end{mathpar}
    
    Then $~\evalgphiGen{\cdot}{\cdot}~$ is an optimal lifting of $~\evalphiGen{\cdot}{\cdot}~$.
    Note the additional effort necessary in rule $\tset{\gradT EvalGrad}$ compared to the non-IDF version in lemma \ref{ex:opt-lift-evalphi}.
    The additional premises make sure that there exists a way to frame $\phi$ with the access provided by $\pi$.
\end{lemma}

\begin{lemma}[Optimal Lifting of Self-Framing]~\\
    \label{ex:idf-opt-lift-sfrmphi}
    Let $~~\sfrmgphi \cdot~ \subseteq \setGFormula$ be defined inductively as
    \begin{mathpar}
        \inferrule* [Right=\gradT SfrmStatic]
        {
            \sfrmphi \phi
        }
        {
            \sfrmgphi \phi
        }
    \end{mathpar}
    \begin{mathpar}
        \inferrule* [Right=\gradT SfrmGrad]
        {
            \evalphiGen{\pi}{\phi}
        }
        {
            \sfrmgphi \withqm{\phi}
        }
    \end{mathpar}
    
    Then $~\sfrmgphi \cdot~$ is an optimal lifting of $~\sfrmphi \cdot~$.
    The premise of \tset{\gradT SfrmGrad} makes sure that the concretization of $\withqm{\phi}$ is not empty.
\end{lemma}