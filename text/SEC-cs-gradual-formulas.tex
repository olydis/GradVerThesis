%% intro
The first step of deriving \gvlidf is defining its modified syntax.
As motivated in section \ref{sec:gradual-formulas} we first extend the formula syntax, afterwards everything that depends on it.

%% formula syntax & concretization
In our design of gradual formulas we aim for “gradual formulas with upper bound” as introduced in section \ref{ssec:wildcard-with-upper}:
\begin{description}
    \item[Syntax] 
    \begin{flalign*}
    	 & \grad{\phi} \quad::=\quad \phi ~|~ \withqm{\phi} &
    \end{flalign*}
    We pose $\qm \defeq \withqm{\phiTrue}$.
    
    \item[Concretization]
    \begin{flalign*}
    & \gamma(\phi) = \{~ \phi ~\}                                                         & ~ \\
    & \gamma(\withqm{\phi}) = \{~ \phi' \in \setFormulaB ~|~ \phiImplies{\phi'}{\phi} ~\} &
    \end{flalign*}
    
    Note that we do not require the static part of $\withqm{\phi}$ to be self-framing.
    On the contrary, we want concretizations to be able to provide framing for an otherwise unframed formula.
    (We decided to put $\qm$ in front of the static part to emphasize that fact.)
    This allows programmers to resort to gradual formulas when being uncertain or indifferent about the concrete framing of a heap-dependent formula.
    \begin{exmp}~\\
        The programmer may want to express $\phiEq{a.color}{b.color}$, but is indifferent about whether \ttt{a} and \ttt{b} alias or not.
        Fortunately, $\qm$ can be used to frame the formula, covering both alternatives:
        \begin{flalign*}
        	\phiCons{\phiCons{\phiAcc{a}{color}}{\phiAcc{b}{color}}}{\phiEq{a.color}{b.color}} & \in \gamma(\withqm{\phiEq{a.color}{b.color}}) & ~ \\
        	\phiCons{\phiCons{\phiEq{a}{b}}{\phiAcc{b}{color}}}{\phiEq{b.color}{b.color}}      & \in \gamma(\withqm{\phiEq{a.color}{b.color}}) &
        \end{flalign*}
    \end{exmp}
\end{description}

%% gradual syntax
We want to only allow gradual formulas in method contracts, resulting in the following gradual syntax:
\begin{figure}[h]
    \begin{align*}
	\grad{program}  & \in \setGProgram  &  & ::= \ttt{$\overline{\grad{cls}}$~$\grad{s}$}                         \\
	\grad{cls}      & \in \setGClass    &  & ::= \class {$C$} {$\overline{field}$} {$\overline{\grad{method}}$}   \\
	\grad{method}   & \in \setGMethod   &  & ::= \method {$T$} {$m$} {$T$} {$x$} {$\grad{contract}$} {$\grad{s}$} \\
	\grad{contract} & \in \setGContract &  & ::= \contract{$\grad{\phi}$}{$\grad{\phi}$}                          \\
	\grad{\phi}     & \in \setGFormula  &  & ::= \phi ~|~ \withqm{\phi}
\end{align*}
    \caption{\gvlidf: Syntax}
    \label{fig:gidf-syntax}
\end{figure}
Note how the small change propagates throughout other constructs, all the way up to gradual programs.
The change has no direct impact on statement syntax, allowing us to define $\setGStmt \defeq \setStmt$ and therefore $\setGProgramState \defeq \setProgramState$.
However, note that static and dynamic semantics of the call statement $\sCall {$x$} {$y$} {$m$} {$z$}$ will still have to be lifted as $m$ now references a gradual method contract (see section \ref{sec:gradual-statements} for detailed discussion).
