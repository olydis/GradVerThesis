In this section we give a number of implementation strategies, that can be used to implement the functions and predicates defined in section \ref{sec:gradual-liftings}.


\begin{description}
    \item[Gradual Formula Semantics]
    \begin{lemma}[Optimal Lifting of Formula Evaluation]\label{ex:idf-opt-lift-evalphi}~\\
        Let $~~\evalgphiGen{\cdot}{\cdot}~ \subseteq \setProgramState \times \setGFormula$ be defined inductively as
        \begin{mathpar}
            \inferrule* [Right=\gradT EvalStatic]
            {
                \evalphiGen{\pi}{\phi}
            }
            {
                \evalgphiGen{\pi}{\phi}
            }
        \end{mathpar}
        \begin{mathpar}
            \inferrule* [Right=\gradT EvalGrad]
            {
                \evalphiGen{\pi}{\phi}\\
                A_s \sfrmphi \phi\\
                \forall \langle e, f \rangle \in A_s.~ \evalphiGen{\pi}{\phiAcc{$e$}{$f$}}
            }
            {
                \evalgphiGen{\pi}{\withqm{\phi}}
            }
        \end{mathpar}
        
        Then $~\evalgphiGen{\cdot}{\cdot}~$ is an optimal lifting of $~\evalphiGen{\cdot}{\cdot}~$.
    \end{lemma}
    Note the additional effort necessary in rule $\tset{\gradT EvalGrad}$ compared to the non-IDF version in lemma \ref{ex:opt-lift-evalphi}.
    The additional premises make sure that there exists a way to frame $\phi$ with the access provided by $\pi$.
    
    \item[Gradual Self-Framing]
    \begin{lemma}[Optimal Lifting of Self-Framing]\label{ex:idf-opt-lift-sfrmphi}~\\
        Let $~~\sfrmgphi \cdot~ \subseteq \setGFormula$ be defined inductively as
        \begin{mathpar}
            \inferrule* [Right=\gradT SfrmStatic]
            {
                \sfrmphi \phi
            }
            {
                \sfrmgphi \phi
            }
        \end{mathpar}
        \begin{mathpar}
            \inferrule* [Right=\gradT SfrmGrad]
            {
                \evalphiGen{\pi}{\phi}
            }
            {
                \sfrmgphi \withqm{\phi}
            }
        \end{mathpar}
        
        Then $~\sfrmgphi \cdot~$ is an optimal lifting of $~\sfrmphi \cdot~$.
    \end{lemma}
    The premise of \tset{\gradT SfrmGrad} makes sure that the concretization of $\withqm{\phi}$ is not empty.
    
    \item[Gradual Implication of Static Formula]
    \begin{lemma}\label{lemma:impl-gimpl}~\\
        Let $\dgrad{P}_{\phi_t} : \setGFormula \rightharpoonup \setGFormula$ be defined as
        \begin{alignat*}{2}
        & \dgrad{P}_{\phi_t}(\phi)          & &= \phi  \quad\text{if } \phiImplies{\phi}{\phi_t} \\
        & \dgrad{P}_{\phi_t}(\withqm{\phi}) & &= \withqm{\phiCons{\norms{\phi}}{\norms{\phi_t}}}
        \end{alignat*}
        Then $\dgrad{P}_{\phi_t}$ is an optimal deterministic lifting of $P_{\phi_t} \subseteq \setFormula \times \setFormula$ with
        $$P_{\phi_t}(\phi_1, \phi_2) ~\defeq~ \phiImplies{\phi_1}{\phi_t} \wedge \phi_1 = \phi_2$$
    \end{lemma}
    The definition makes use of $\norms{\phi}$ which is a normal form of $\phi$ defined in section $\ref{ssec:gradual-normal-form}$.
    
    \item[Gradual Formula Extraction]
    \begin{lemma}\label{lemma:frm-extr}~\\
        There exists a function $\dgrad{P} : \setGFormula \times \setGFormula \rightharpoonup \setGFormula$ such that
        $\dgrad{P}$ is an optimal deterministic lifting of $P \subseteq \setFormula \times \setFormula \times \setFormula$ with
        $$P(\phi_1, \phi_t, \phi_2) ~\defeq~ \phiImplies{\phi_1}{\phiCons{$\phi_t$}{$\phi_2$}} ~\wedge \sfrmphi \phi_2$$
    \end{lemma}
    
    \begin{comment}
    Formula extraction is dual to the separating conjunction in a sense that it weakens given formula $\phi_1$ enough (yielding formula $\phi_2$) such that $\phiImplies{\phi_1}{\phiCons{$\phi_t$}{$\phi_2$}}$ holds for some $\phi_r$.\\
    Note that this operation requires that $\phiImplies{\phi_1}{\phi_t}$ holds, otherwise it is undefined.
    
    The operation would have no meaning in a classical setting because $\phiImplies{\phi_1}{\phiAnd{$\phi_t$}{$\phi_1$}}$ holds, i.e. $\phi_1$ does not require any weakening in order to be implied together with another conjunctive term.
    However the linear nature of implicit dynamic frames requires that $\phi_2$ and $\phi_t$ give access to disjoint memory locations.
    \end{comment}
    
    \begin{example}{Behavior of $\dgrad{P}$}
        \begin{alignat*}{3}
        	 & \dgrad{P}(\grad{\phi}, \phi)            && = \grad{\phi}    \quad\quad \text{if } \staticFP{\phi} = \emptyset & ~ \\
        	 %& \dgrad{P}(\grad{\phi}, \withqm{\phi_t}) && = \dgrad{P}(\grad{\phi}, \static{$\grad{\phi}$})                   & \\
             & \dgrad{P}(\ttt{acc(x.f)\:*\:(x.f = 3)}, \ttt{acc(x.f)}) && = \phiTrue                                         & \\
             & \dgrad{P}(\ttt{?\:*\:(a = x.f)\:*\:(x.f = b)}, \ttt{acc(x.f)}) && = \phiEq{a}{b}                                        & \\
        	 & \dgrad{P}(\phiAcc{y}{f}, \phiAcc{x}{f}) && \textit{undefined (extracted formula not implied)}                 &
        \end{alignat*}
    \end{example}
    
    \item[Gradual Variable Extraction]\label{lemma:var-extr}~\\
    \begin{lemma}~\\
        There exists a function $\dgrad{P}_{\overline{x}} \subseteq \setGFormula \rightharpoonup \setGFormula$ such that
        $\dgrad{P}$ is an optimal deterministic lifting of $P_{\overline{x}} \subseteq \setFormula \times \setFormula$ with
        $$P_{\overline{x}}(\phi_1, \phi_2) ~\defeq~ \phiImplies{\phi_1}{\phi_2} ~~\wedge~~ \FV(\phi_2) \cap \overline{x} = \emptyset ~~\wedge~~ \sfrmphi \phi_2$$
    \end{lemma}
    
    \begin{example}{Behavior of $\dgrad{P}_{\overline{x}}$}
        \begin{align*}
        &\dgrad{P}_{\emptyset}(\grad{\phi}) = \grad{\phi}\\
        &\dgrad{P}_{X}(\withqm{\phi}) = \withqm{\dgrad{P}_{X}(\phi)}\\
        &\dgrad{P}_{\{x\}}(\ttt{acc(x.f)\:*\:(x.f = 3)}) = \phiTrue\\
        &\dgrad{P}_{\{x\}}(\ttt{(a = x)\:*\:(x = b)}) = \phiEq{a}{b}
        \end{align*}
    \end{example}
\end{description}