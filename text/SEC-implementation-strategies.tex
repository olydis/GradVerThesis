
\begin{lemma}[Optimal Lifting of Evaluation]~\\
    \label{ex:idf-opt-lift-evalphi}
    Let $~~\evalgphiGen{\cdot}{\cdot}~ \subseteq \setProgramState \times \setGFormula$ be defined inductively as
    \begin{mathpar}
        \inferrule* [Right=\gradT EvalStatic]
        {
            \evalphiGen{\pi}{\phi}
        }
        {
            \evalgphiGen{\pi}{\phi}
        }
    \end{mathpar}
    \begin{mathpar}
        \inferrule* [Right=\gradT EvalGrad]
        {
            \evalphiGen{\pi}{\phi}\\
            A_s \sfrmphi \phi\\
            \forall \langle e, f \rangle \in A_s.~ \evalphiGen{\pi}{\phiAcc{$e$}{$f$}}
        }
        {
            \evalgphiGen{\pi}{\withqm{\phi}}
        }
    \end{mathpar}
    
    Then $~\evalgphiGen{\cdot}{\cdot}~$ is an optimal lifting of $~\evalphiGen{\cdot}{\cdot}~$.
    Note the additional effort necessary in rule $\tset{\gradT EvalGrad}$ compared to the non-IDF version in lemma \ref{ex:opt-lift-evalphi}.
    The additional premises make sure that there exists a way to frame $\phi$ with the access provided by $\pi$.
\end{lemma}

\begin{lemma}[Optimal Lifting of Self-Framing]~\\
    \label{ex:idf-opt-lift-sfrmphi}
    Let $~~\sfrmgphi \cdot~ \subseteq \setGFormula$ be defined inductively as
    \begin{mathpar}
        \inferrule* [Right=\gradT SfrmStatic]
        {
            \sfrmphi \phi
        }
        {
            \sfrmgphi \phi
        }
    \end{mathpar}
    \begin{mathpar}
        \inferrule* [Right=\gradT SfrmGrad]
        {
            \evalphiGen{\pi}{\phi}
        }
        {
            \sfrmgphi \withqm{\phi}
        }
    \end{mathpar}
    
    Then $~\sfrmgphi \cdot~$ is an optimal lifting of $~\sfrmphi \cdot~$.
    The premise of \tset{\gradT SfrmGrad} makes sure that the concretization of $\withqm{\phi}$ is not empty.
\end{lemma}





\begin{description}
    \item[Implication of fixed static formula: $P_{\phi_t}(\phi_1, \phi_2) ~\defeq~ \phiImplies{\phi_1}{\phi_t} \wedge \phi_1 = \phi_2$]~\\
    This predicate is not only a component of some rules but also the overall shape of \tset{HAssert}.
    \begin{lemma}~\\
        Let $\dgrad{P}_{\phi_t} : \setGFormula \rightharpoonup \setGFormula$ be defined as
        \begin{alignat*}{2}
        & \dgrad{P}_{\phi_t}(\phi)          & &= \phi  \quad\text{if } \phiImplies{\phi}{\phi_t} \\
        & \dgrad{P}_{\phi_t}(\withqm{\phi}) & &= \withqm{\phiCons{\norms{\phi}}{\norms{\phi_t}}}
        \end{alignat*}
        Then $\dgrad{P}_{\phi_t}$ is an optimal deterministic lifting of $P_{\phi_t}$.
    \end{lemma}
    We define a more intuitive notation that matches original static predicate:
    \begin{displaymath}
    \gphiImpliesEv{\grad{\phi_1}}{\phi_t}{\grad{\phi_2}} ~\defiff~ \dgrad{P}_{\phi_t}(\grad{\phi_1}) = \grad{\phi_2}
    \end{displaymath}
    
    \item[Formula extraction: $P_{\phi_t}(\phi_1, \phi_2) ~\defeq~ \phiImplies{\phi_1}{\phiCons{$\phi_t$}{$\phi_2$}} ~~\wedge~~ \sfrmphi \phi_2$]~\\
    This predicate is a component of \tset{HFieldAssign}, \tset{HCall}, \tset{HHold} and also is the overall shape of \tset{HRelease}.
    \begin{lemma}~\\
        Let $\dgrad{P}_{\phi_t} : \setGFormula \rightharpoonup \setGFormula$ be defined as
        \begin{alignat*}{2}
        & \dgrad{P}_{\phi_t}(\phi_1)          & &= \underset{\phiImplies{}{}}{\min} \{~ \phi_2 \in \setFormulaB ~|~ \phiImplies{\phi_1}{\phiCons{$\phi_t$}{$\phi_2$}} ~\} \\
        & \dgrad{P}_{\phi_t}(\withqm{\phi_1}) & &= \withqm{\dgrad{P}_{\phi_t}(\phi_1)}
        \end{alignat*}
        Then $\dgrad{P}_{\phi_t}$ is well-defined and an optimal deterministic lifting of $P_{\phi_t}$.
    \end{lemma}
    We define a more intuitive notation that reflects the dual nature with \ttt{*}:
    \begin{displaymath}
    \wo{\grad{\phi}}{\phi_t} ~\defeq~ \dgrad{P}_{\phi_t}(\grad{\phi})
    \end{displaymath}
    We extend the domain of $\wo{}{}$ to deal with gradual parameters:
    \begin{displaymath}
    \wo{\grad{\phi}}{(\withqm{\phi_t})} ~\defeq~ \wo{\grad{\phi}}{\static{$\grad{\phi}$}}
    \end{displaymath}
    Formally, this extension optimally lifts the function w.r.t. its second parameter.
    
    \item[Variable extraction: $P_x(\phi_1, \phi_2) ~\defeq~ \phiImplies{\phi_1}{\phi_2} ~~\wedge~~ x \not \in \FV(\phi_2) ~~\wedge~~ \sfrmphi \phi_2$]~\\
    This predicate is a component of \tset{HAlloc}, \tset{HVarAssign}, \tset{HReturn}, \tset{HCall} and in a sense \tset{HHold} (the equality between the free variable sets can be reformulated as repeated application of $P_x$).
    \begin{lemma}~\\
        Let $\dgrad{P}_x : \setGFormula \rightharpoonup \setGFormula$ be defined as
        \begin{alignat*}{2}
        & \dgrad{P}_x(\phi_1)          & &= \underset{\phiImplies{}{}}{\min} \{~ \phi_2 \in \setFormulaB ~|~ \phiImplies{\phi_1}{\phi_2} \wedge x \not \in \FV(\phi_2) ~\} \\
        & \dgrad{P}_x(\withqm{\phi_1}) & &= \withqm{\dgrad{P}_x(\phi_1)}
        \end{alignat*}
        Then $\dgrad{P}_x$ is well-defined and an optimal deterministic lifting of $P_x$.
    \end{lemma}
    We extend the domain of the extraction function $\wo{}{}$ to deal with variables:
    \begin{alignat*}{2}
    & \wo{\grad{\phi}}{x} ~\defeq~ \dgrad{P}_x(\grad{\phi}) & ~
    \end{alignat*}
    We further extend the function to deal with sets of variables element-wise
    \begin{alignat*}{2}
    & \wo{\grad{\phi}}{\emptyset}        & & ~\defeq~ \grad{\phi}                   \\
    & \wo{\grad{\phi}}{(\{ x \} \cup X)} & & ~\defeq~ \wo{(\wo{\grad{\phi}}{x})}{X}
    \end{alignat*}
\end{description}