Having introduced a lot of concepts that all play together, it is worth going through a gradual verification process in its entirety.
We assume that \gvl knows the Hoare rules lifted in section \ref{sssec:examples-lift-determ}.

Example \gvl program:
\begin{lstlisting}
int getFourA(int i)
    requires true;
    ensures  result = 4;
{
    i := 4;
    return i;
}

int getFourB(int i)
    requires ?; // too lazy to figure that one out, yet
    ensures  result = 4;
{
    i := i + 1;
    return i;
}
\end{lstlisting}
We assume that $\sReturn{i}$ is syntactic sugar for an assignment $\sVarAssign{result}{i}$ to the reserved variable \ttt{result}.

During compilation, the gradual verifier is expected to verify the method contracts which is equivalent to deducing
\begin{displaymath}
\gthoare{~}{\phiTrue}{\sSeq{\sVarAssign{i}{4}}{\sVarAssign{result}{i}}}{\phiEq{result}{4}}
\end{displaymath}
and
\begin{displaymath}
\gthoare{~}{\qm}{\sSeq{\sVarAssign{i}{i + 1}}{\sVarAssign{result}{i}}}{\phiEq{result}{4}}
\end{displaymath}

It does so by applying the deterministic lifting (see section \ref{sssec:examples-lift-determ}) and then using lemma \ref{lem:det2grad} to deduce the gradual lifting.

~\\~\\
Deducing $\gthoare{}{\phiTrue}{\sSeq{\sVarAssign{i}{4}}{\sVarAssign{result}{i}}}{\phiEq{result}{4}}$:
\begin{mathpar}
    \inferrule* [Right=Lemma \ref{lem:det2grad}]
    {
        \inferrule* [right=\dgradT HSeq]
        {
            \inferrule* [right=\dgradT HAssign1]
            {
                \ttt{i} \not \in \FV(\phiTrue) \\\\
                \ttt{i} \not \in \FV(\ttt{4})
            }
            {
                \mathcal{D}_1
            }
            \\
            \inferrule* [Right=\dgradT HAssign1]
            {
                \ttt{result} \not \in \FV({\phiAnd{\phiTrue}{\phiEq{i}{4}}}) \\\\
                \ttt{result} \not \in \FV(\ttt{i})
            }
            {
                \mathcal{D}_2
            }
        }
        {
            \mathcal{D}
        }\\
        \mathcal{I}
    }
    {
        \gthoare{}
        {\phiTrue}
        {\sSeq{\sVarAssign{i}{4}}{\sVarAssign{result}{i}}}
        {\phiEq{result}{4}}
    }
\end{mathpar}
where
\begin{flalign*}
\mathcal{D}_1 \equiv& 
\dgthoare{~}
{\phiTrue}
{\sVarAssign{i}{4}}
{\phiAnd{\phiTrue}{\phiEq{i}{4}}}
\\
\mathcal{D}_2 \equiv& 
\dgthoare{~}
{\phiAnd{\phiTrue}{\phiEq{i}{4}}}
{\sVarAssign{result}{i}}
{\phiAnd{\phiAnd{\phiTrue}{\phiEq{i}{4}}}{\phiEq{result}{i}}}
\\
\mathcal{D} \equiv& 
\dgthoare{~}
{\phiTrue}
{\sSeq{\sVarAssign{i}{4}}{\sVarAssign{result}{i}}}
{\phiAnd{\phiAnd{\phiTrue}{\phiEq{i}{4}}}{\phiEq{result}{i}}}
\\
\mathcal{I} \equiv&~ 
\gphiImplies{\phiAnd{\phiAnd{\phiTrue}{\phiEq{i}{4}}}{\phiEq{result}{i}}}{\phiEq{result}{4}}
\end{flalign*}


~\\~\\
Deducing $\gthoare{~}{\qm}{\sSeq{\sVarAssign{i}{i + 1}}{\sVarAssign{result}{i}}}{\phiEq{result}{4}}$:
\begin{mathpar}
    \inferrule* [Right=Lemma \ref{lem:det2grad}]
    {
        \inferrule* [right=\dgradT HSeq]
        {
            \inferrule* [right=\dgradT HAssign2]
            {~}
            {
                \mathcal{D}_1
            }
            \\
            \inferrule* [right=\dgradT HAssign2]
            {~}
            {
                \mathcal{D}_2
            }
        }
        {
            \mathcal{D}
        }\\
        \mathcal{I}
    }
    {
        \gthoare{~}
        {\qm}
        {\sSeq{\sVarAssign{i}{i + 1}}{\sVarAssign{result}{i}}}
        {\phiEq{result}{4}}
    }
\end{mathpar}
where
\begin{flalign*}
\mathcal{D}_1 \equiv& 
\dgthoare{~}
{\qm}
{\sVarAssign{i}{i + 1}}
{\qm}
\\
\mathcal{D}_2 \equiv& 
\dgthoare{~}
{\qm}
{\sVarAssign{result}{i}}
{\qm}
\\
\mathcal{D} \equiv& 
\dgthoare{~}
{\qm}
{\sSeq{\sVarAssign{i}{i + 1}}{\sVarAssign{result}{i}}}
{\qm}
\\
\mathcal{I} \equiv&~ 
\gphiImplies{\qm}{\phiEq{result}{4}}
\end{flalign*}
~\\~\\


As established in section \ref{ssec:gradual-soundness}, runtime checks have to be injected whenever such a judgment is made, in order to guarantee soundness.
Definition \ref{def:gsnd} of \tset{\gradT Soundness} suggests inserting a runtime assertion for the entire postcondition.
The compiled program (as executed by the small-step semantics) would correspond to the following source code:
\begin{lstlisting}
int getFourA(int i)
{
    i := 4;
    result := i;
    assert (result = 4);
}

int getFourB(int i)
{
    i := i + 1;
    result := i;
    assert (result = 4);
}
\end{lstlisting}

The runtime check in method \ttt{getFourA} looks redundant.
As motivated in section \ref{ssec:the-deterministic-approach}, the gradual verifier may formally reason about optimized runtime checks using the return values of the deterministic gradual Hoare logic in case \tset{\dgradT Soundness} holds.
Lemma \ref{lemma:runtime-checks-reduced} and corollary \ref{cor:runtime-checks-removed} give refined criteria for sufficient runtime checks.

Assuming that \tset{\dgradT Soundness} does indeed hold, one can use corollary \ref{cor:runtime-checks-removed} to deduce that no runtime check is necessary for method \ttt{getFourA} since
\begin{displaymath}
\envs{\phiAnd{\phiAnd{\phiTrue}{\phiEq{i}{4}}}{\phiEq{result}{i}}} \subseteq \envs{\phiEq{result}{4}}
\end{displaymath}
holds.

For method $\ttt{getFourB}$ it is valid to check for $P(\grad{\pi}) = \evalgphiGen{\grad{\pi}}{\phiEq{result}{4}}$ (according to lemma \ref{lemma:runtime-checks-reduced}), which again corresponds to executing a runtime assertion $\sAssert{\phiEq{result}{4}}$.

As a result, the compiled program corresponds to the following source code.
\begin{lstlisting}
int getFourA(int i)
{
    i := 4;
    result := i;
}

int getFourB(int i)
{
    i := i + 1;
    result := i;
    assert (result = 4);
}
\end{lstlisting}

\begin{comment}
Note that the gradual verifier has not yet verified above judgments.



~\\~\\
Comparing \ttt{getFourA} and \ttt{getFourB} one can observe a difference regarding the injected runtime assertions.
On the one hand, the check in \ttt{getFourA} is unnecessary since it will always succeed.
On the other hand, the check in \ttt{getFourB} seems necessary in order for the postcondition to be satisfied (for all executions that reach it).
This difference is also indicated in the deduction of both contracts.
Verifying \ttt{getFourA}, the deterministic lifting determined $\phiAnd{\phiAnd{\phiTrue}{\phiEq{i}{4}}}{\phiEq{result}{i}}$ as postcondition, before being weakened to $\phiEq{result}{4}$.
However, verifying \ttt{getFourB}, the deterministic lifting ends up with $\qm$ as knowledge, from which $\phiEq{result}{4}$ can only be implied by instantiating $\qm$ accordingly.

The intuition behind the negligible assertion can be formalized:
In case \tset{\dgradT Soundness} holds, we are guaranteed that the postcondition returned by the deterministic lifting actually holds for every execution.
In other words, it is static knowledge that can be used to reason about the outcome of runtime assertions at verification time.
In case of function \ttt{getFourA}, the compiler would be able to treat $\phiAnd{\phiAnd{\phiTrue}{\phiEq{i}{4}}}{\phiEq{result}{i}}$ as static knowledge that is usable to formally guarantee the success of $\sAssert{\phiEq{result}{4}}$.


\end{comment}
