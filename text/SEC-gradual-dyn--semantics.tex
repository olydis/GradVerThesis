The call statement is the only statement that is affected by the introduction of gradual formulas.
As we do not want the semantics of any of the other statements to change, we only have to adjust the rules for calls.
Figure \ref{fig:gvl-sem-dyn-sstep} shows those rules.
\begin{figure}
    \boxed{\gsstep{\pi}{\pi}}
    % Inductive Semantics.dynSem
\begin{mathpar}
\inferrule* [Right=\gradT ESCall]
{
    \evalex {H} {\rho} {\ex{${y}$}} {{o}} \\
    \evalex {H} {\rho} {\ex{${z}$}} {v} \\
    {H(o)} = {{\langle{C}, {\usc}\rangle}} \\
    {\mmethod{{C}, {m}}} = {{\method {${T_r}$} {${m}$} {${T}$} {${w}$} {${\contract {$\grad{\phi}$} {${\usc}$}}$} {$\grad{r}$}}} \\
    {\rho'} = {[{\eresult} \mapsto {\defaultValue{${T_r}$}}, {\ethis} \mapsto {{o}}, {{w}} \mapsto {v}]} \\
    \evalgphiGen {H, \rho', A} {\grad{\phi}} \\
    {A'} = {
    \begin{cases}
    	\dynamicFP {H} {\rho'} {\grad{\phi}} & \textit{if } \grad{\phi} \in \setFormula \\
    	A                                    & \textit{otherwise}
    \end{cases}}
}
{
    \sstep {\langle{H}, {{\langle{{\rho}, {A}}, {\sSeq{${\sCall {${x}$} {${y}$} {${m}$} {${z}$}}$}{$s$}}\rangle} \cdot {S}}\rangle} {\langle{H}, {{\langle{{\rho'}, {A'}}, \grad{r}\rangle} \cdot {{\langle{{\rho}, {{A} \backslash {A'}}}, {\sSeq{${\sCall {${x}$} {${y}$} {${m}$} {${z}$}}$}{$s$}}\rangle} \cdot {S}}}\rangle}
}

\inferrule* [Right=\gradT ESCallFinish]
{
    \evalex {H} {\rho} {\ex{${y}$}} {{o}} \\
    {H(o)} = {{\langle{C}, {\usc}\rangle}} \\
    {\mpost{{C}, {m}}} = {\grad{\phi}} \\
    \evalgphiGen {H, \rho', A'} {\grad{\phi}} \\
    \evalex {H} {\rho'} {\ex{${\eresult}$}} {v_r}
}
{
    \sstep {\langle{H}, {{\langle{{\rho'}, {A'}}, {\sSkip}\rangle} \cdot {{\langle{{\rho}, {A}}, {\sSeq{${\sCall {${x}$} {${y}$} {${m}$} {${z}$}}$}{$s$}}\rangle} \cdot {S}}}\rangle} {\langle{H}, {{\langle{{{\rho}[{x} \mapsto {v_r}]}, {{A} \cup {A'}}}, s\rangle} \cdot {S}}\rangle}
}
\end{mathpar}


    \caption{\gvlidf: Small-Step Semantics for Method Call}
    \label{fig:gvl-sem-dyn-sstep}
\end{figure}
For all other statements $\gsstep{\cdot}{\cdot}$ is identical to $\sstep{\cdot}{\cdot}$ (see figure \ref{fig:svl-sem-dyn-sstep}).

Furthermore, we design $\gsstep{\cdot}{\cdot}$ to be total.
We extend the set of final states $\setGProgramStateFin$ with a designated exceptional state $\pi_{EX}$, representing dynamic verification failure.
If no other derivation exists, we map to this state.

\begin{lemma}[$\gsstep{\cdot}{\cdot}$ Well-Defined]
    The small-step semantics of \gvlidf is well-defined.
\end{lemma}
\begin{proof}
    For $\gsstep{\cdot}{\cdot}$ to be well-defined, the inductive rules may allow deducing at most one return value for every input.
    For the most part, this is the case due to $\sstep{\cdot}{\cdot}$ being well-defined (see lemma \ref{lemma:ss-wd}).
    We show that the same is true for adjustments made in figure \ref{fig:gvl-sem-dyn-sstep}.
    Note the adjustments do not change the fact that the inductive rules are syntax-directed.
    
    Rule \tset{\gradT SsCall} is deterministic:
    $H, \rho, A, x, y, m, z, s, S$ are forwarded from the input, $\rho', A', r$ are uniquely determined by the premises.
    
    Rule \tset{\gradT SsCallFinish} is deterministic:
    $H, \rho, x, A, A', s, S$ are forwarded from the input, $v_r$ is uniquely determined by the premises.
\end{proof}

\begin{lemma}[\gvlidf: Sound Lifting of Small-Step-Semantics]
    The lifting we propose is sound as defined in section \ref{sssec:lifting-partial-functions}.
\end{lemma}
\begin{proof}
    For the rules copied from $\sstep{\cdot}{\cdot}$, the rules are trivially satisfied as there is no difference between gradual and non-gradual statements.
    Precision is only meaningful for call statements:

    \begin{description}
        \item[\tset{\gradT SsCall}]~
        \begin{description}
            \item[Introduction]~\\
            For static precondition, \tset{\gradT SsCall} is identical to \tset{SsCall}.
            
            \item[Monotonicity]~\\
            Reducing the precision of the precondition results in all permissions being passed to the topmost stack frame.
            Having more permissions than before cannot introduce runtime failures.
            Succeeding executions will thus be observationally identical after reducing precision.
        \end{description}
        
        \item[\tset{\gradT SsCallFinish}]~
        \begin{description}
            \item[Introduction]~\\
            For static postcondition, \tset{\gradT SsCallFinish} is identical to \tset{SsCallFinish}.
            
            \item[Monotonicity]~\\
            The return value of \tset{\gradT SsCallFinish} is independent of the postcondition thus monotonic w.r.t. to it.
            In terms of definedness, reducing the precision of the postcondition cannot result in premises that were satisfied before to be unsatisfied.
        \end{description}
    \end{description}
\end{proof}