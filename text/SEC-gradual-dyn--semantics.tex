We follow the pattern introduced in section \ref{ssec:dynamic-semantics} and define $\gsssem$ using a predicate $\gsstep{\cdot}{\cdot}$
\begin{displaymath}
\gsssem(\pi) = \pi' \defiff \gsstep{\pi}{\pi'}
\end{displaymath}
Note that the changes compared to $\sstep{\cdot}{\cdot}$ are minimal.
The call statement is the only one affected by the introduction of gradual formulas and therefore the only statement that requires adjusted rules:
\begin{figure}
    \boxed{\gsstep{\pi}{\pi}}
    % Inductive Semantics.dynSem
\begin{mathpar}
\inferrule* [Right=\gradT ESCall]
{
    \evalex {H} {\rho} {\ex{${y}$}} {{o}} \\
    \evalex {H} {\rho} {\ex{${z}$}} {v} \\
    {H(o)} = {{\langle{C}, {\usc}\rangle}} \\
    {\mmethod{{C}, {m}}} = {{\method {${T_r}$} {${m}$} {${T}$} {${w}$} {${\contract {$\grad{\phi}$} {${\usc}$}}$} {$\grad{r}$}}} \\
    {\rho'} = {[{\eresult} \mapsto {\defaultValue{${T_r}$}}, {\ethis} \mapsto {{o}}, {{w}} \mapsto {v}]} \\
    \evalgphiGen {H, \rho', A} {\grad{\phi}} \\
    {A'} = {
    \begin{cases}
    	\dynamicFP {H} {\rho'} {\grad{\phi}} & \textit{if } \grad{\phi} \in \setFormula \\
    	A                                    & \textit{otherwise}
    \end{cases}}
}
{
    \sstep {\langle{H}, {{\langle{{\rho}, {A}}, {\sSeq{${\sCall {${x}$} {${y}$} {${m}$} {${z}$}}$}{$s$}}\rangle} \cdot {S}}\rangle} {\langle{H}, {{\langle{{\rho'}, {A'}}, \grad{r}\rangle} \cdot {{\langle{{\rho}, {{A} \backslash {A'}}}, {\sSeq{${\sCall {${x}$} {${y}$} {${m}$} {${z}$}}$}{$s$}}\rangle} \cdot {S}}}\rangle}
}

\inferrule* [Right=\gradT ESCallFinish]
{
    \evalex {H} {\rho} {\ex{${y}$}} {{o}} \\
    {H(o)} = {{\langle{C}, {\usc}\rangle}} \\
    {\mpost{{C}, {m}}} = {\grad{\phi}} \\
    \evalgphiGen {H, \rho', A'} {\grad{\phi}} \\
    \evalex {H} {\rho'} {\ex{${\eresult}$}} {v_r}
}
{
    \sstep {\langle{H}, {{\langle{{\rho'}, {A'}}, {\sSkip}\rangle} \cdot {{\langle{{\rho}, {A}}, {\sSeq{${\sCall {${x}$} {${y}$} {${m}$} {${z}$}}$}{$s$}}\rangle} \cdot {S}}}\rangle} {\langle{H}, {{\langle{{{\rho}[{x} \mapsto {v_r}]}, {{A} \cup {A'}}}, s\rangle} \cdot {S}}\rangle}
}
\end{mathpar}


    \caption{\gvlidf: Small-Step Semantics for Method Call}
    \label{fig:gvl-sem-dyn-sstep}
\end{figure}

The other rules are identical to their counterparts in figure \ref{fig:svl-sem-dyn-sstep}.

\begin{lemma}[$\gsssem$ Well-Defined]
    The small-step semantics of \gvlidf is well-defined.
\end{lemma}

\begin{lemma}[\gvlidf: Sound Lifting of Small-Step-Semantics]
    The lifting we propose is sound as defined in section \ref{sssec:lifting-partial-functions}.
\end{lemma}