We choose the deterministic approach proposed in section \ref{ssec:the-deterministic-approach}.
Figure \ref{fig:gvl-sem-stat-hoare} shows a deterministic lifting of the Hoare logic (figure \ref{fig:svl-sem-stat-hoare}).
It uses a number of lifted components that we defined in section \ref{sec:gradual-liftings}.
\begin{figure}[h!]
    \boxed{\dgthoare {\Gamma} {\grad{\phi_{pre}}} {s} {\grad{\phi_{post}}}}
    \begin{mathpar}
    \inferrule* [right=\dgradT HSkip]
    {
        ~
    }
    {
        \dgthoare {\Gamma} {\grad{\phi}} {{\sSkip}} {\grad{\phi}}
    }
    
    \inferrule* [Right=\dgradT HAlloc]
    {
        {\wo {\grad{\phi}} {x}} = \grad{\phi'}\\
        \sType {\Gamma} {\ex{${x}$}} {{C}} \\
        {\fields{C}} = {{\overline{\field{$T$}{$f$}}}}
    }
    {
        \dgthoare {\Gamma} {\grad{\phi}} {{\sAlloc {${x}$} {${C}$}}} {\gphiCons{$\grad{\phi'}$}{${\gphiCons{${\phiNeq {${\ex{${x}$}}$} {${\ev{${\enull}$}}$}}$}{${\overline{\gphiCons{\phiAcc {$x$} {$f_i$}}{\phiEq {\edot {$x$} {$f_i$}} {\defaultValue {$T_i$}} }}}$}}$}}
    }
    
    \inferrule* [Right=\dgradT HFieldAssign]
    {
        \wo {\grad{\phi}} {\phiAcc{$x$}{$f$}} = \grad{\phi'}\\
        \sType {\Gamma} {\ex{${x}$}} {{C}} \\
        \sType {\Gamma} {\ex{${y}$}} {T} \\
        \vdash {C}.{f} : {T}
    }
    {
        \dgthoare {\Gamma} {\grad{\phi}} {{\sFieldAssign {${x}$} {${f}$} {${y}$}}} {\gphiCons{$\grad{\phi'}$}{${\gphiCons{${\phiAcc {${\ex{${x}$}}$} {${f}$}}$}{${\gphiCons{${\phiNeq {${\ex{${x}$}}$} {${\ev{${\enull}$}}$}}$}{${\ensuremath{{\phiEq {${\edot{${\ex{${x}$}}$}{${f}$}}$} {${\ex{${y}$}}$}}}}$}}$}}$}}
    }
    
    \inferrule* [Right=\dgradT HVarAssign]
    {
        \gphiImplies{\grad{\phi}} {\accFor {{e}}}\\
        {\wo {\grad{\phi}} {x}} = \grad{\phi'}\\
        {x} \not \in {\FV({e})} \\
        \sType {\Gamma} {\ex{${x}$}} {T} \\
        \sType {\Gamma} {e} {T}
    }
    {
        \dgthoare {\Gamma} {\grad{\phi}} {{\sVarAssign {${x}$} {${e}$}}} {\gphiCons{$\grad{\phi'}$}{${\ensuremath{{\phiEq {${\ex{${x}$}}$} {${e}$}}}}$}}
    }
    
    \inferrule* [Right=\dgradT HReturn]
    {
        {\wo {\grad{\phi}} {\eresult}} = \grad{\phi'}\\
        \sType {\Gamma} {\ex{${x}$}} {T} \\
        \sType {\Gamma} {\ex{${\eresult}$}} {T}
    }
    {
        \dgthoare {\Gamma} {\grad{\phi}} {{\sReturn {${x}$}}} {\gphiCons{$\grad{\phi'}$}{${\ensuremath{{\phiEq {${\ex{${\eresult}$}}$} {${\ex{${x}$}}$}}}}$}}
    }
    
    \inferrule* [Right=\dgradT HCall]
    {
        \wo {\wo {\grad{\phi}} {x}} {\grad{\phi_p}} = \grad{\phi'}\\
        \sType {\Gamma} {\ex{${y}$}} {{C}} \\
        {\mmethod{{C}, {m}}} = {{\method {${T_r}$} {${m}$} {${T_p}$} {${z}$} {${\contract {$\grad{\phi_{pre}}$} {$\grad{\phi_{post}}$}}$} {${\usc}$}}} \\
        \sType {\Gamma} {\ex{${x}$}} {T_r} \\
        \sType {\Gamma} {\ex{${z'}$}} {T_p} \\
        \gphiImplies{\grad{\phi}}{\gphiCons{${\phiNeq {${\ex{${y}$}}$} {${\ev{${\enull}$}}$}}$}{$\grad{\phi_p}$}} \\
        x \neq y \wedge x \neq z' \\
        \grad{\phi_p} = {\grad{\phi_{pre}}[{y}, {z'} / {\ethis}, {{z}}]} \\
        \grad{\phi_q} = {\grad{\phi_{post}}[{y}, {z'}, {x} / {\ethis}, {{z}}, {\eresult}]}
    }
    {
        \dgthoare {\Gamma} {\grad{\phi}} {{\sCall {${x}$} {${y}$} {${m}$} {${z'}$}}} {\gphiCons{$\grad{\phi'}$}{$\grad{\phi_q}$}}
    }
    
    \inferrule* [right=\dgradT HAssert]
    {
        \gphiImpliesEv{\grad{\phi}}{\phi_a}{\grad{\phi'}}
    }
    {
        \dgthoare {\Gamma} {\grad{\phi}} {{\sAssert {${\phi_a}$}}} {\grad{\phi'}}
    }
    
    \inferrule* [Right=\dgradT HRelease]
    {
        \gphiImpliesEv{\grad{\phi}}{\phi_r}{\grad{\phi'}}\\
        {\wo {\grad{\phi'}} {\phi_r}} = \grad{\phi''}
    }
    {
        \dgthoare {\Gamma} {\grad{\phi}} {{\sRelease {${\phi_r}$}}} {\grad{\phi''}}
    }
    
    \inferrule* [Right=\dgradT HDeclare]
    {
        {x} \not\in \dom{\Gamma} \\
        {x} \not \in {\FV(\grad{\phi})} \\
        \dgthoare {{\Gamma}, {x} : {T}} {\gphiCons{${\phiEq {${\ex{${x}$}}$} {${\ev{${\defaultValue{${T}$}}$}}$}}$}{$\grad{\phi}$}} {s} {\grad{\phi'}}
    }
    {
        \dgthoare {\Gamma} {\grad{\phi}} {\sSeq{\sDeclare {${T}$} {${x}$}}{$s$}} {\grad{\phi'}}
    }
    
    \inferrule* [Right=\dgradT HHold]
    {
        \sfrmphi {\phi} \\
        \gphiImpliesEv{\grad{\phi_f}}{\phi}{\grad{\phi_f'}}\\
        \wo {\grad{\phi_f'}} {\phi} = \grad{\phi_r}\\
        \wo {\wo {\grad{\phi_f'}} {\static{$\grad{\phi_r}$}}}{(\FV(\grad{\phi_f'}) \backslash \FV(\phi))} = \grad{\phi'}\\
        \mods(s) \cap \FV(\phi) = \emptyset \\
        \dgthoare {\Gamma} {\grad{\phi_r}} {s} {\grad {\phi_r'}}
    }
    {
        \dgthoare {\Gamma} {\grad{\phi_f}} {{\sHold {${\phi}$} {$s$}}} {\gphiCons{$\grad{\phi_r'}$}{$\grad{\phi'}$}}
    }
    
    \inferrule* [Right=\dgradT HSeq]
    {
        \dgthoare {\Gamma} {\grad{\phi_p}} {s_1} {\grad{\phi_q}} \\
        \dgthoare {\Gamma} {\grad{\phi_q}} {s_2} {\grad{\phi_r}}
    }
    {
        \dgthoare {\Gamma} {\grad{\phi_p}} {\sSeq{$s_1$}{$s_2$}} {\grad{\phi_r}}
    }
\end{mathpar}





% Let $\gsc$ behave like $\hsc$ if first operand is static - otherwise its regular concatenation.




    \caption{\gvl: Gradual Hoare Logic} 
    \label{fig:gvl-sem-stat-hoare}
\end{figure}

\begin{lemma}[\gvlidf: Sound Deterministic Lifting of Hoare Logic]
    The inductive definition we give induces a well-defined partial function.
    This function is a sound deterministic lifting (see \ref{ssec:the-deterministic-approach}) of the Hoare logic of \svlidf defined in section \ref{sec:static-semantics}.
\end{lemma}
\begin{proof}~
    \begin{description}
        \item[Well-definedness] The rules are syntax directed and only can deduce at most one result per input.
        \item[Deterministic lifting] Rule-wise.
        
        \tset{HSkip}:
        See lemma \ref{lemma:opt-lift-impl}.
        
        \tset{HAlloc}, \tset{HFieldAssign}, \tset{HReturn}, \tset{HAssert}:
        Composition of lifted components as defined in \ref{ssec:lifted-helper-functions}.
        
        Remaining rules analogous.
    \end{description}
\end{proof}

A sound predicate lifting $\gtHoare{\cdot}{\cdot}{\cdot}{\cdot}$ can be derived using lemma \ref{lem:det2grad}.