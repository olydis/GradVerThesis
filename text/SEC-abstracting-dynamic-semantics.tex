%% first: changes in source code!!!
% making formula gradual 
% - cannot let execution fail 
% - cannot satisfy less formulas (“observational compatibility”})

\begin{align*}
\grad{\pi_1} \sqsubseteq_{\pi} \grad{\pi_2}
~\defiff~
&\forall \phi \in \setFormula.~ \evalphiGen{\grad{\pi_1}}{\phi} \implies \evalphiGen{\grad{\pi_2}}{\phi} \\
~~\wedge~~
&\sqsubseteq \textit{f.a. statements}
\end{align*}

Choose ~$\gsssem : \setGProgramState \rightharpoonup \setGProgramState$~ as lifted version of ~$\sssem : \setProgramState \rightharpoonup \setProgramState$.

\begin{mathpar}
    \inferrule* [Right=Soundness]
    {
        \gthoare{~}{\grad{\phi_1}}{\grad{s}}{\grad{\phi_2}} \\ 
        \pi_{\grad{s}} \,\in \setGProgramState_{\grad{s}}\\
        \evalgphiGen{\pi_{\grad{s}}}{\grad{\phi_1}}
    }
    {
        \evalgphiGen{\gsssem^{\grad{s}}(\pi_{\grad{s}})}{\grad{\phi_2}}
    }
\end{mathpar}

%% intro
Static verification provides guarantees about the dynamic behavior of a program without changing it.
With the introduction of dynamic components to the assertion language, these guarantees potentially impossible to provide without altering the program.
In other words, it might be inevitable to enrich the program with runtime checks ensuring properties that were impossible to ensure at compile time.

%% example
Consider a (static) Hoare logic that can verify
\begin{displaymath}
\thoare{~}{\phiEq{x}{3}}{\sVarAssign{y}{4}}{\phiAnd{\phiEq{x}{3}}{\phiEq{y}{4}}}
\end{displaymath}
The corresponding gradual logic can thus verify
\begin{displaymath}
\gthoare{~}{\qm}{\sVarAssign{y}{4}}{\phiAnd{\phiEq{x}{3}}{\phiEq{y}{4}}}
\end{displaymath}
but without additional measures there is no guarantee that every program state satisfying the precondition (here: literally every program state) will satisfy the post condition after executing the assignment.
Adding a runtime assertion to ensure that $\phiEq{x}{3}$ holds (terminating exceptionally in case it does not) would be sufficient and minimal to reestablish soundness in the above example.

We will generalize and formalize this approach, resulting in gradual dynamic semantics that reestablish the soundness of the static verification system by adding minimal runtime checks.

% OUTLOOK on how I develop stuff

\subsection{Soundness of Gradual System}


\subsection{Runtime Approach}
\begin{displaymath}
\sssem
\end{displaymath}

, a dynamic check is required, making sure that $\phiEq{x}{3}$ holds after executing the assignment.


 , i.e. without altering a given runtime.
Often static verification formalisms are designed specifically to make 

In the presence of gradual formulas, additional measures are necessary to ensure soundness at runtime.

Our approach of ensuring soundness is motivated by a simple 



The static guarantee provided by fully static verification system is inherently lost 
Gradual static semantics inherently bears the risk of 




Gradualizing static semantics potentially introduced spots where progress is no longer guaranteed.

%%Naive:
Emit “runtimeAssert” prechecks.

%%Mention and explain evidence approach???
