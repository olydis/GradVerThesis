% intro!? NOT the same gradual formula syntax as for predicate examples

%% lAND
\begin{lemma}[Optimal Lifting of Or Operator]~\\
    Let $\setGFormula$ be extended using the “bounded unknown” approach (see section \ref{ssec:wildcard-with-upper}).
    We assume that $\vee$ is part of the formula syntax such that $\envs{\phi_1 \vee \phi_2} = \envs{\phi_1} \cup \envs{\phi_2}$.
    $\phiOr{\:}{\:}$ can be viewed as a binary function on formulas.
    Let $\gphiOr{\:}{\:} : \setGFormula \times \setGFormula \rightarrow \setGFormula$ be defined as
    \begin{flalign*}
    \gphiOr{$\phi_1$}{$\phi_2$} & \defeq \phiOr{$\phi_1$}{$\phi_2$} \\
    \gphiOr{$\phi_1$}{$(\withqmGen{\phi_2})$} & \defeq \\
    \gphiOr{$(\withqmGen{\phi_1})$}{$\phi_2$} & \defeq \\
    \gphiOr{$(\withqmGen{\phi_1})$}{$(\withqmGen{\phi_2})$} & \defeq \withqmGen{(\phiOr{$\phi_1$}{$\phi_2$})}
    \end{flalign*}
    Then $\gphiOr{\:}{\:}$ is an optimal lifting of $\phiOr{\:}{\:}$.
\end{lemma}
\begin{proof}
    Soundness
        Introduction
        \begin{align*}
        &\phiOr{$\phi_1$}{$\phi_2$}\\
        =
        &\gphiOr{$\phi_1$}{$\phi_2$}
        \end{align*}
        
        Monotonicity
        Large case analysis. Omitted.
    
    Optimality?
    
    \begin{comment} does not work due to non-existence of galois connection - not even partial for TOTAL and function (partial would work)
    Goal:
    \begin{displaymath}
    \forall \grad{\phi_1}, \grad{\phi_2} \in \setGFormula.~ \gphiAnd{$\grad{\phi_1}$}{$\grad{\phi_2}$} = \alpha(\{~ \phiAnd{$\phi_1$}{$\phi_2$} ~|~ \phi_1 \in \gamma(\grad{\phi_1}),\, \phi_2 \in \gamma(\grad{\phi_2}) ~\})
    \end{displaymath}
    
    Case $\grad{\phi_1} = \phi_1' \wedge \grad{\phi_2} = \phi_2'$:
    \begin{align*}
    &\alpha(\{~ \phiAnd{$\phi_1$}{$\phi_2$} ~|~ \phi_1 \in \gamma(\phi_1'),\, \phi_2 \in \gamma(\phi_2') ~\})\\
    =
    &\alpha(\{~ \phiAnd{$\phi_1'$}{$\phi_2'$} ~\})\\
    =
    &\phiAnd{$\phi_1'$}{$\phi_2'$}\\
    =
    &\gphiAnd{$\phi_1'$}{$\phi_2'$}\\
    \end{align*}
    
    Case $\grad{\phi_1} = \withqm{\phi_1'} \wedge \grad{\phi_2} = \phi_2'$:
    \begin{align*}
    &\alpha(\{~ \phiAnd{$\phi_1$}{$\phi_2$} ~|~ \phi_1 \in \gamma(\withqm{\phi_1'}),\, \phi_2 \in \gamma(\phi_2') ~\})\\
    =
    &\alpha(\{~ \phiAnd{$\phi_1$}{$\phi_2'$} ~|~ \phi_1 \in \setFormulaA ~\wedge~ \phiImplies{\phi_1}{\phi_1'} ~\})\\
    =
    &\alpha(\{~ \phiAnd{$\phi_1'$}{$\phi_2'$} ~\})\\
    =
    &\phiAnd{$\phi_1'$}{$\phi_2'$}\\
    =
    &\gphiAnd{$\phi_1'$}{$\phi_2'$}\\
    \end{align*}
    \end{comment}
\end{proof}

%% composite
\begin{lemma}[Sound Lifting of Composed Function]~\\
    Let $g, f : \setFormula \rightarrow \setFormula$ be arbitrary functions.
    
    Let $\grad{(g \circ f)} : \setGFormula \rightarrow \setGFormula$ be defined as
    \begin{displaymath}
    \grad{(g \circ f)} ~\defeq~ \grad{g} \circ \grad{f}
    \end{displaymath}
    with sound liftings $\grad{g}$ and $\grad{f}$.
    
    Then $\grad{(g \circ f)}$ is a sound lifting of $(g \circ f)$, i.e. “piecewise” lifting of composed functions is allowed.
    Optimality of $\grad{g}$ and $\grad{f}$ does not imply optimality of $\grad{(g \circ f)}$.
\end{lemma}
\begin{proof}
    Introduction
    \begin{align*}
    &g(f(\phi))\\
    \overset{Introduction~\grad{g}}{\mpt}
    &\grad{g}(f(\phi))\\
    \overset{\substack{Introduction~\grad{f}\\\&\\Monotonicity~\grad{g}}}{\mpt}
    &\grad{g}(\grad{f}(\phi))\\
    =
    &\grad{g}(\grad{f}(\phi))\\
    =
    &(\grad{g} \circ \grad{f})(\phi)\\
    =
    &\grad{(g \circ f)}(\phi)
    \end{align*}
    
    Monotonicity
    \begin{align*}
    &\grad{\phi_1} \mpt \grad{\phi_2}\\
    \overset{Monotonicity~\grad{f}}{\implies}
    &\grad{f}(\grad{\phi_1}) \mpt \grad{f}(\grad{\phi_2})\\
    \overset{Monotonicity~\grad{g}}{\implies}
    &\grad{g}(\grad{f}(\grad{\phi_1})) \mpt \grad{g}(\grad{f}(\grad{\phi_2}))\\
    \overset{Definition}{\implies}
    &\grad{(g \circ f)}(\grad{\phi_1}) \mpt \grad{(g \circ f)}(\grad{\phi_2})
    \end{align*}
\end{proof}

\begin{comment}
\begin{align*}
\grad{f}(\grad{\phi_1}, \grad{\phi_2}) = \alpha(\{~ \phiAnd{$\phi_1$}{$\phi_2$} ~|~ \phi_1 \in \gamma(\grad{\phi_1}) \wedge \phi_2 \in \gamma(\grad{\phi_2}) ~\})
\end{align*}
\end{comment}




% mention alpha(...), galois connection (does not always exist, make example... so partial GC instead (reference)...)

% function composition (soundness, optimality?), ...