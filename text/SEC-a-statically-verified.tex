% generic language introduction

Requirements:

\begin{description}
    \item[Syntax] ~\\
    We assume the existence of at least the following two syntactic categories:
    \begin{align*}
    	s    & \in \setStmt    \\
    	\phi & \in \setFormula
    \end{align*}
    % TODO: we make no other assumptions
    % To illustrate certain points, we will make ad-hoc assumptions about the syntax and semantics.
    
    \item[Formula Semantics]~\\
    Formulas are used to describe program states.
    For example, a method contract stating $\ttt{arg > 4}$ as precondition is supposed to make sure that the method is only entered, if $\ttt{arg}$ evaluates to a value larger than $4$ in the current program state.
    
    We assume that we are given a predicate
    \begin{displaymath}
    \evalphiGen{\cdot}{\cdot} ~\subseteq~ \setProgramState \times \setFormula
    \end{displaymath}
    that decides, whether a formula is satisfied given a concrete program state.
    
    % EXAMPLES examples examples
    
    This also induces notion of implication!!! % TODO
    % also talk about partially ordered set induced by that... name largest element "true" (expect that to exist!), etc.
    
    \item[Static Semantics]~\\
    We assume that there is an axiomatic semantics, i.e. some predicate 
    \begin{displaymath}
    \hoare {\cdot} {~\cdot~} {\cdot} ~\subseteq~ \setFormula \times \setStmt \times \setFormula
    \end{displaymath}
    describing which programs (together with pre- and postconditions about the program state) are accepted.
    In reality, this predicate might have further parameters. 
    For instance, a statically typed language might require a type context to safely deduce $$\thoare{\ex{x} : \Tint}{\phiTrue}{\sVarAssign{x}{3}}{\phiEq{x}{3}}$$
    As we will see later, further parameters are generally irrelevant for and immune to gradualization, so it is reasonable to omit them for now.
    
    \item[Dynamic Semantics]~\\
    We assume that there is a dynamic semantics (e.g. small-step) describing precisely whether and how program state is updated.
    This is required mainly for reasoning about soundness of the static semantics.
    % NOTE: allow for non-deterministic semantics!
    
    \item[Soundness]~\\
    We expect that given static semantics are sound w.r.t. given dynamic semantics.
    This means that programs satisfying ... 
\end{description}