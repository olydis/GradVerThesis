\begin{comment}
    \begin{alignat*}{3}
    	 & \rlap{\textbf{Syntax}}                        &  &                      & ~ \\
    	 & s                                             &  & \in \setStmt         &  \\
    	 & \phi                                          &  & \in \setFormula      &  \\
    	 &  \\
    	 & \rlap{\textbf{Program State}}                 &  &                      &  \\
    	 & \pi                                           &  & \in \setProgramState &  \\
    	 &  \\
    	 & \rlap{\textbf{Semantics}}                     &  &                      &  \\
    	 & \rlap{Static  ~~~~$\thoare{}{\phi}{s}{\phi}$} &  &                      &  \\
    	 & \rlap{Dynamic ~$\sstep{\pi}{\pi}$}            &  &                      &  \\
    	 & \rlap{Formula ~~$\evalphiGen{\pi}{\phi}$}     &  &                      &  \\
         & ~ \\
         & \rlap{\textbf{Soundness}}                     &  &                      &  \\
    \end{alignat*}

    \begin{alignat*}{3}
    	 & \rlap{\textbf{Syntax}}                                              &  &                       & ~ \\
    	 & \grad{s}                                                            &  & \in \setGStmt         &  \\
    	 & \grad{\phi}                                                         &  & \in \setGFormula      &  \\
    	 &  \\
    	 & \rlap{\textbf{Program State}}                                       &  &                       &  \\
    	 & \grad{\pi}                                                          &  & \in \setGProgramState &  \\
    	 &  \\
    	 & \rlap{\textbf{Semantics}}                                           &  &                       &  \\
    	 & \rlap{Static  ~~~~$\gthoare{}{\grad{\phi}}{\grad{s}}{\grad{\phi}}$} &  &                       &  \\
    	 & \rlap{Dynamic ~$\gsstep{\grad{\pi}}{\grad{\pi}}$}                   &  &                       &  \\
    	 & \rlap{Formula ~~$\evalgphiGen{\grad{\pi}}{\grad{\phi}}$}            &  &                       &  \\
    	 & ~ \\
    	 & \rlap{\textbf{Soundness}}                                           &  &                       &
    \end{alignat*}
\end{comment}



%% generic language introduction
While aiming to give a general procedure for deriving imperative gradually verified languages, we have to make certain assumptions about \svl in order to concisely describe our approach and reason about its correctness.
We believe that most imperative statically verified programming languages satisfy the following assumptions and thus qualify as starting point for our procedure.

\begin{description}
\item[Syntax]~\\
    We assume the existence of the following two syntactic categories
    \begin{align*}
    	s    & \in \setStmt    \\
    	\phi & \in \setFormula
    \end{align*}
    for statements and formulas.
    
    We assume that there is a sequence operator \ttt{;} such that $$\forall s_1, s_2 \in \setStmt.~~ s_1\ttt{;}s_2 \in \setStmt$$
    
\item[Program State]~\\
    Dynamic semantics (see below) are formalized as discrete transitions between program states.
    Therefore a program state contains all information necessary to evaluate expressions and determine the next program state.
    
    We assume that $\setProgramState$ is the set of all possible program states in \svl.
    
    \begin{example}{Program State: Primitive language with integer variables}
        \label{ex:ps-primitive}
        \begin{displaymath}
        \setProgramState ~=~ \underbrace{(\setVar \rightharpoonup \mathbb{Z})}_{\textit{variable memory}} ~\times~ \underbrace{\setStmt}_{\textit{continuation}} 
        \end{displaymath}
    \end{example}
    \begin{example}{Program State: Language with stack}
        \label{ex:ps-stacked}
        \begin{displaymath}
        \setProgramState ~=~ \bigcup_{i \in \mathbb{N}_+}{\underbrace{\Big((\setVar \rightharpoonup \mathbb{Z}) ~\times~ \setStmt \Big)}_{\textit{stack frame}}}^i                                                      
        \end{displaymath}
    \end{example}
    
    Note how these examples use statements to represent continuations (the “remaining work”), necessary for the operational semantics to deduce a state transition.
    In general, a different representation may be used to encode this continuation (e.g. a lower-level machine language emitted by the compiler).
    
    \begin{comment}
    In order to determine the next program state (or detect termination), a state must have a notion of “upcoming work”, usually represented by a statement internally.
    
    Let $\setProgramState_s$ (with $s \in \setStmt$) be the set of program states having $s$ as upcoming work.
    This notion will be necessary to define soundness of \svl's static semantics.
    
    Examples:
    \begin{description}
        \item[Primitive]
        \begin{flalign*}
        	 & \setProgramState ~=~ \underbrace{(\setVar \rightharpoonup \mathbb{Z})}_{\textit{variable memory}} ~\times~ \setStmt & ~ \\
        	 & \setProgramState_s ~=~ (\setVar \rightharpoonup \mathbb{Z}) ~\times~ \setStmt_s                                     &
        \end{flalign*}
        
        \item[Stack]
        \begin{flalign*}
        	 & \setProgramState ~=~ \bigcup_{i \in \mathbb{N}_+}{\underbrace{\Big((\setVar \rightharpoonup \mathbb{Z}) ~\times~ \setStmt \Big)}_{\textit{stack frame}}}^i                                                                                \\
        	 & \setProgramState_s ~=~ \Big((\setVar \rightharpoonup \mathbb{Z}) ~\times~ \setStmt_s\Big) ~\times~ \underbrace{\bigcup_{i \in \mathbb{N}_0}{\Big((\setVar \rightharpoonup \mathbb{Z}) ~\times~ \setStmt \Big)^i}}_{\textit{lower frames}}
        \end{flalign*}
    \end{description}
     
    \end{comment}
    
    
\item[Dynamic Semantics]~\\
    We assume that \svl has a structural operational semantics or small-step semantics.
    This semantics is formalized as $~\sstep{\cdot}{\cdot} : \setProgramState \rightharpoonup \setProgramState$, describing precisely how program state is updated.
    (Non-deterministic semantics would also be a potential area to investigate in the future, see section \ref{ssec:nd-semantics}.)
    
    %We assume that there is a small-step semantics $~\sssem \subseteq \setProgramState \rightarrow \PP(\setProgramState)$ describing precisely how program state can be updated.
    %Note that this signature allows for non-deterministic (e.g. concurrent) semantics.
    %Execution is considered stuck in state $\pi$ iff $\sssem(\pi) = \emptyset$
    As usual, 
    %taking $n$ steps is abbreviated as $\sstepGeneric{\cdot}{n}{\cdot}$ (undefinedness is propagated), 
    taking a finite amount of steps is abbreviated as $\ssteps{\cdot}{\cdot}$.
    Also, we write $\sstepStuck{\pi}$ to state that $\pi$ is a “stuck” state, i.e. that $\pi$ is not in the domain of the partial small-step function.
    
    Furthermore we define $\sstepConsume{s}{\cdot}{\cdot} \subseteq \setProgramState \times \setProgramState$ as a predicate indicating whether a program state is reachable from another program state by executing statement $s$.
    This is useful for abstracting from the precise implementation of both $\setProgramState$ (see examples \ref{ex:ps-primitive} and \ref{ex:ps-stacked}) and the small-step semantics.
    
    \begin{example}{Execution of Given Statement}
        Assume that program state is defined as in example \ref{ex:ps-primitive}.
        If $\pi_1 \in \setProgramState$ is defined as
        \begin{displaymath}
        \pi_1 = \langle [x \mapsto 4, y \mapsto 2],\,\, \sSeq{\sVarAssign{x}{8}}{\sVarAssign{y}{x}} \rangle
        \end{displaymath}
        then $\sstepConsume{\sVarAssign{x}{8}}{\pi_1}{\pi_2}$ is supposed to hold for
        \begin{displaymath}
        \pi_2 = \langle [x \mapsto 8, y \mapsto 2],\,\, \sVarAssign{y}{x} \rangle
        \end{displaymath}
        
        Note how the notation hides away implementation details like the number of steps taken.
        While the judgment $\ssteps{\pi_1}{\pi_2}$ is also independent from the number of steps taken, it is unable to encode what is supposed to happen during those steps:
        
        For $\pi_1$ defined as above
        \begin{displaymath}
        \pi_2 = \langle [x \mapsto 8, y \mapsto 8],\,\, \sSkip \rangle
        \end{displaymath}
        or
        \begin{displaymath}
        \pi_2 = \pi_1
        \end{displaymath}
        would be valid instantiations.
    \end{example}
    
    We assume that there is a designated non-empty set $$~\setProgramStateFin \subseteq \setProgramState~$$ of “final” states indicating regular termination of the program.
    W.l.o.g. we assume $\forall \pi \in \setProgramStateFin.~ \sstepStuck{\pi}$, i.e. final states are stuck.
    
    We assume that dynamic semantics of a sequence $\sSeq{$s_1$}{$s_2$}$ is implemented such that 
    \begin{align}
    \label{frm:assume-seq-ss}
    \sstepConsume{\sSeq{$s_1$}{$s_2$}}{\pi_1}{\pi_3} \iff \exists \pi_2 \in \setProgramState.~ \sstepConsume{s_1}{\pi_1}{\pi_2} \wedge \sstepConsume{s_2}{\pi_2}{\pi_3}
    \end{align}
    for all $\pi_1, \pi_3 \in \setProgramState$.
    
    
    % We say a statement “throws an exception” if its small-step semantics transitions into an exceptional state.
    \begin{comment}
    Optionally, there may be a subset $~\setProgramStateEx \subseteq \setProgramStateFin~$ of states indicating exceptional termination of the program.
    To simplify reasoning about exceptional states, we assume $$\forall \pi_X \in \setProgramStateEx, \phi \in \setFormula.~ \evalphiGen{\pi_X}{\phi}$$ and something with special statement set?
    
    %To indicate that states $\overline{\pi_2}$ can be obtained from state $\pi_1$ in $n$ steps, we write $\sssem^n(\pi_1)$.
    %If the specific number of steps is irrelevant, we write $\sssem^*(\pi_1, \pi_2)$.
    
    % Execution is considered to be stuck iff there are no more 
    \end{comment}
    
\item[Formula Semantics]~\\
    While types restrict which values or expressions are valid for a certain variable,
    formulas restrict which program states are valid for a certain point during execution. 

    They are used for annotations like method contracts or invariants.
    For example, a method contract stating $\ttt{arg > 4}$ as precondition is supposed to make sure that the method is only entered if $\ttt{arg}$ evaluates to a value larger than $4$.
    This is a restriction of program states corresponding to the very beginning of the method call.
    
    We assume that we are given a computable predicate
    \begin{displaymath}
    \evalphiGen{\cdot}{\cdot} ~\subseteq~ \setProgramState \times \setFormula
    \end{displaymath}
    that decides, whether a formula is satisfied given a concrete program state.
    
    From this predicate we can derive a number of concepts:
    
    \begin{definition}[Denotational Formula Semantics $\envs{\cdot}$]~\\
        \label{def:frm-den-sem}
        Let $\envs{\cdot} : \setFormula \rightarrow \PP^{\setProgramState}$ be defined as
        \begin{displaymath}
        \envs{\phi} \defeq \{~ \pi \in \setProgramState ~|~ \evalphiGen{\pi}{\phi} ~\}
        \end{displaymath}
    \end{definition}
        
    % EXAMPLES examples examples
    
    \begin{definition}[Formula Satisfiability]~\\
        A formula $\phi$ is \textbf{satisfiable} iff $$\envs{\phi} \neq \emptyset$$
        Let $\setFormulaA \subseteq \setFormula$ be the set of satisfiable formulas.
    \end{definition}
    
    \begin{definition}[Formula Implication]~\\
        \label{def:form-implication}
        A formula $\phi_1$ \textbf{implies} formula $\phi_2$ (written $\phiImplies{\phi_1}{\phi_2}$) iff
        \begin{displaymath}
        \envs{\phi_1} \subseteq \envs{\phi_2}
        \end{displaymath}
        We will consistently use $\implies$ to denote logical implication, whereas $\Rightarrow$ exclusively denotes formula implication as defined here.
        (Note the different lengths of the implication arrows.)
    \end{definition}
    
    \begin{definition}[Formula Equality]~\\
        \label{def:form-eq}
        Two formulas $\phi_1$ and $\phi_2$ are \textbf{equal} iff
        \begin{displaymath}
        \envs{\phi_1} = \envs{\phi_2}
        \end{displaymath}
    \end{definition}
    
    \begin{lemma}[Partial Order of Formulas]
        \label{lemma:po-form}~\\
        The implication predicate is a partial order on $\setFormula$.
    \end{lemma}
    
    We assume that there is a largest element $\phiTrue \in \setFormula$ with $\envs{\phiTrue} = \setProgramState$.
    
    \begin{comment}
    Note that the presence of an unsatisfiable formula (as invariant, pre-/postcondition, assertion, ...) in a sound verification system implies that the corresponding source code location is unreachable:
    Soundness guarantees that any reachable program state satisfies potentially annotated formulas, trivially ensuring that the formula is satisfiable.
    
    This property is true regardless of whether \svl forbids usage of unsatisfiable formulas entirely or whether it only fails when trying to use the corresponding code (which would involve proving that a satisfiable formula implies an unsatisfiable one).
    Therefore we will often restrict our reasoning on the satisfiable formulas $\setFormulaA$, without explicitly stating that the presence of an unsatisfiable formula would result in failure.
    \end{comment}
    
    With this semantics we can formalize the notion of semantically valid Hoare triples.
    \begin{definition}[Semantical Validity of Hoare Triples]\label{def:valid-hoare-triple}~\\
        A Hoare triple $\hoare{\phi_{pre}} {s} {\phi_{post}}$ is \textbf{valid}, written
        $\tHoare{}{\phi_{pre}} {s} {\phi_{post}}$
        iff
        \begin{flalign*}
        %& \tHoare {~} {\cdot} {\cdot} {\cdot} ~~~\subseteq~~~ \setFormula \times \setStmt \times \setFormula                                                                                            \\
        %& \tHoare {~} {\phi_{pre}} {s} {\phi_{post}} ~\defiff~ 
        \forall \pi_{pre}, \pi_{post} \in \setProgramState.~ \sstepConsume{s}{\pi_{pre}}{\pi_{post}} \implies (\evalphiGen{\pi_{pre}}{\phi_{pre}} \implies \evalphiGen{\pi_{post}}{\phi_{post}})
        \end{flalign*}
    \end{definition}
    \begin{lemma}[Compositional Validity of Hoare Triples]
        \label{lem:comp-tHoare}~
        \begin{align*}
        \forall \phi_1, \phi_2, \phi_3 \in \setFormula,\, s_1, s_2 \in \setStmt.~
        &\tHoare{}{\phi_1} {s_1} {\phi_2} ~\wedge~ \tHoare{}{\phi_2} {s_2} {\phi_3}\\
        \implies
        &\tHoare{}{\phi_1} {\sSeq{$s_1$}{$s_2$}} {\phi_3}
        \end{align*}
    \end{lemma}
    
    We assume that the set $\phis{\pi} \defeq \{~ \phi \in \setFormula ~|~ \evalphiGen{\pi}{\phi} ~\}$ is a filter, i.e.
    \begin{enumerate}
        \item It is non-empty. This ensures that every program state is describable by at least one formula.
        This is the case as we demanded that $\phiTrue$ describes every program state.
        \item If $\evalphiGen{\pi}{\phi_a}$ and $\evalphiGen{\pi}{\phi_b}$, then there exists $\phi \in \setFormula$ with  $$~~\evalphiGen{\pi}{\phi} ~~\wedge~~ \phiImplies{\phi}{\phi_a} ~~\wedge~~ \phiImplies{\phi}{\phi_b}$$
        Intuitively, this states that multiple formulas about the same program state can be combined.
        In case the formula syntax contains a logical conjunction operator, this criterion is met.
        \item If $\evalphiGen{\pi}{\phi_a}$ and $\phiImplies{\phi_a}{\phi_b}$ then $\evalphiGen{\pi}{\phi_b}$.
        This is true by definition of $\phiImplies{\cdot}{\cdot}$ (see \ref{def:form-implication}).
    \end{enumerate}
    
\item[Static Semantics]~\\
    We assume that there is a Hoare logic
    \begin{displaymath}
    \thoare {~} {\cdot} {\cdot} {\cdot} ~~~\subseteq~~~ \setFormula \times \setStmt \times \setFormula
    \end{displaymath}
    describing which programs (together with pre- and postconditions about the program state) are accepted.
    
    In practice, this predicate might also have further parameters. 
    For instance, a statically typed language might require a type context to safely deduce $$\thoare{\ex{x} : \Tint}{\phiTrue}{\sVarAssign{x}{3}}{\phiEq{x}{3}}$$
    As we will see later, further parameters are generally irrelevant for and immune to gradualization, so it is reasonable to omit them for now.
    
    %% SEQ
    \begin{comment}
    We assume that 
    \begin{mathpar}
        \inferrule* [Right=HoareSequence]
        {
            \thoare {~} {\phi_p} {{s_1}} {\phi_q} \\
            \thoare {~} {\phi_q} {{s_2}} {\phi_r}
        }
        {
            \thoare {~} {\phi_p} {\sSeq{$s_1$}{$s_2$}} {\phi_r}
        }
    \end{mathpar}
    is derivable from given Hoare rules.
    \end{comment}
    
    \begin{lemma}[Hoare Logic Monotonicity]
        \label{ass:hl-mono}~\\
        We assume that the Hoare predicate is monotonic in the precondition w.r.t. implication:
        \begin{align*}
        \forall s \in \setStmt.~& \\
        \forall \phi_1, \phi_2 \in \setFormula.~&\\
        \forall \phi_1' \in \setFormula.~&
        (\phiImplies {\phi_1} {\phi_2}) ~~\wedge~ \thoare {~} {\phi_1} {s} {\phi_1'}\\
        \implies
        \exists \phi_2' \in \setFormula.~&
        (\phiImplies {\phi_1'} {\phi_2'}) ~~\wedge~ \thoare {~} {\phi_2} {s} {\phi_2'}
        \end{align*}
    \end{lemma}
    Intuitively, this means that more knowledge about the initial program state cannot result in a loss of information about the final state.
    
    \begin{comment}
    %% "wsp" - wlp, but with static instead of dynamic semantics!!!
    \begin{definition}[Weakest Static Precondition]~\\
        Let $\wsp : \setStmt \rightarrow \PP(\setProgramState)$ be defined as
        \begin{displaymath}
        \wsp(s) = \{~ \pi \in \setProgramState_s ~|~ \exists \phi_1, \phi_2 \in \setFormula.~ \thoare{~}{\phi_1}{s}{\phi_2} ~~\wedge~~ \evalphiGen{\pi}{\phi_1} ~\}
        \end{displaymath}
    \end{definition}
    Intuitively, the $\wsp(s)$ is a predicate on program states, indicating whether we could deduce anything about the state after executing $s$, using only our Hoare rules.
    
    Example:
    \begin{itemize}
        \item 
        Given that
        \begin{mathpar}
            \inferrule* [Right=HoareAssign]
            {
                ~
            }
            {
                \thoare {~} {\phi[e/x]} {\sVarAssign {${x}$} {${e}$}} {\phi}
            }
        \end{mathpar}
        is the only Hoare rule for assignment, it follows that
        \begin{displaymath}
        \wsp(\sVarAssign {${x}$} {${e}$}) = \setProgramState
        \end{displaymath}
        
        \item 
        Given that
        \begin{mathpar}
            \inferrule* [Right=HoareStaticAssert]
            {
                \phiImplies{\phi}{\phi_a}
            }
            {
                \thoare {~} {\phi} {\sAssert {$\phi_a$}} {\phi}
            }
        \end{mathpar}
        is the only Hoare rule for assertions, it follows that
        \begin{displaymath}
        \wsp(\sAssert {$\phi_a$}) = \{~ \pi \in \setProgramState ~|~ \evalphiGen{\pi}{\phi_a} ~\}
        \end{displaymath}
    \end{itemize}
    
    % Note that this definition is similar to the weakest liberal precondition $\predicate{wlp}$
    
    % we use this static semantics to check e.g. method contracts
    \end{comment}
    
%\item[Notion of Well-Formedness]
    % formally that can be covered by static semantics!!!
    
\item[Soundness]~\\
    We expect that given static semantics are sound w.r.t. given dynamic semantics.
    This means that Hoare triples judged valid by the Hoare logic are also valid semantically:
    \begin{definition}[Soundness]\label{def:snd}~\\
        Let $\phi_1, \phi_2 \in \setFormula$ and $s \in \setStmt$.\\
        If $\thoare{}{\phi_1}{s}{\phi_2}$ then $\tHoare{}{\phi_1}{s}{\phi_2}$
    \end{definition}
\end{description}
