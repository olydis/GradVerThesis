For the following examples we assume that gradual formulas were defined as in section \ref{ssec:dedicated-wildcard-formula}:
\begin{align*}
&\grad{\phi} ::= \phi ~|~ \qm \\
~\\
&\gamma(\phi) = 
\begin{cases}
\{~ \phi ~\}  & \phi \in \setFormulaA\\
\emptyset     & \textit{otherwise}
\end{cases}\\
&\gamma(\qm) = \setFormulaA
\end{align*}

%% phiImplies
\begin{lemma}[Consistent Lifting of Implication]~\\
    Let $~~\gphiImplies{\cdot}{\cdot}~ \subseteq \setGFormula \times \setGFormula$ be defined as
    \begin{displaymath} 
    \gphiImplies {\grad{\phi_1}} {\grad{\phi_2}} ~\defiff~ \exists \phi_1 \in \gamma(\grad{\phi_1}), \phi_2 \in \gamma(\grad{\phi_2}).~ \phiImplies {\phi_1} {\phi_2}
    \end{displaymath}
    Then $~\gphiImplies{\cdot}{\cdot}~$ is a consistent lifting of $~\phiImplies{\cdot}{\cdot}~$.
\end{lemma}

%% evalphi
The first example is the 
\begin{lemma}[Optimal Lifting of Evaluation]~\\
    Let $~~\evalgphiGen{\cdot}{\cdot}~ \subseteq \setProgramState \times \setGFormula$ be defined as
    \begin{displaymath} 
    \evalgphiGen {\pi} {\grad{\phi}} ~\defiff~ \evalphiGen {\pi} {\static{$\phi$}}
    \end{displaymath}
    Then $~\evalgphiGen{\cdot}{\cdot}~$ is a consistent lifting of $~\evalphiGen{\cdot}{\cdot}~$.
\end{lemma}

We define $\setGFormulaA = \{~ \grad{\phi} \in \setGFormula ~|~ \exists \pi.~ \evalgphiGen {\pi} {\grad{\phi}} ~\}$ as the set of satisfiable gradual formulas.

\begin{lemma}[Restricted Domain of Concretization]~\\
    $\restr{\gamma}{\setGFormulaA}$ never returns the empty set.
\end{lemma}


% TODO: simplify? needs lifted functions!

