Semantics can be defined in terms of partial functions or even be a partial function as is the case for the small-step semantics of \svl.
We derive rules for lifting partial functions using the following decomposition:

\begin{lemma}[Partial Function Decomposition]
    Let $f : \setFormula \rightharpoonup \setFormula$ be a partial function.
    Then there exists a total function $f' : \setFormula \rightarrow \setFormula$ and a predicate $F \subseteq \setFormula$ such that
    \begin{flalign*}
    &f(\phi) = f'(\phi)  \quad\text{ if } F(\phi)\\
    &f ~\text{ undefined otherwise}
    \end{flalign*}
\end{lemma}

Composing $\grad{f}$ from the gradual liftings of $f$'s decomposition gives rise to the following rules for lifting partial functions.

\begin{description}
    \item[Introduction]~\\
    \begin{displaymath}
    \forall \phi \in \setFormula \cap \dom{f}.~ f(\phi) \sqsubseteq \grad{f}(\phi)
    \end{displaymath}
    
    \item[Monotonicity]~\\
    \begin{displaymath}
    \forall \grad{\phi_1}, \grad{\phi_2} \in \setGFormula.~ 
    \grad{\phi_1} \sqsubseteq \grad{\phi_2} \wedge \grad{\phi_1} \in \dom{\grad{f}} \implies \grad{f}(\grad{\phi_1}) \sqsubseteq \grad{f}(\grad{\phi_2})
    \end{displaymath}
\end{description}

Soundness and optimality are defined as usual.
