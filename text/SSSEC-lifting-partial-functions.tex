Semantics can be defined in terms of partial functions, the small-step semantics of \svl even is a partial function itself.
We derive rules for lifting partial functions by decomposing them into a total function and a predicate indicating the domain of the partial function.

\begin{definition}[Partial Function Decomposition]
    Let $f : \setFormula \rightharpoonup \setFormula$ be a partial function.
    Then we define a decomposition $\langle F, f' \rangle \in \PP^{\setFormula} \times (\setFormula \rightarrow \setFormula)$ where
    \begin{flalign*}
    &F = \dom{f}\\
    &f'(\phi) = f(\phi) ~\text{ if } F(\phi)\\
    &f'(\phi) = \phiTrue ~\text{ otherwise }
    \end{flalign*}
      
    Note that we treat $F$ as a predicate.
    Then $f$ can be defined in terms of $F$ and $f'$:
    \begin{flalign*}
    &f(\phi) = f'(\phi)  \quad\text{ if } F(\phi)\\
    &f ~\text{ undefined otherwise}
    \end{flalign*}
\end{definition}

We define lifting of partial functions as decomposing the function, lifting the parts and then recomposing them.
This process is equivalent to the following rules.

\begin{description}
    \item[Introduction]~\\
    \begin{displaymath}
    \forall \phi \in \setFormula \cap \dom{f}.~ f(\phi) \sqsubseteq \grad{f}(\phi)
    \end{displaymath}
    
    \item[Monotonicity]~\\
    \begin{displaymath}
    \forall \grad{\phi_1}, \grad{\phi_2} \in \setGFormula.~ 
    \grad{\phi_1} \sqsubseteq \grad{\phi_2} \wedge \grad{\phi_1} \in \dom{\grad{f}} \implies \grad{f}(\grad{\phi_1}) \sqsubseteq \grad{f}(\grad{\phi_2})
    \end{displaymath}
\end{description}

Soundness and optimality are defined as usual.
