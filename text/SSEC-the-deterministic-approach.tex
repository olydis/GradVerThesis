% proof by example: optimal deterministic lifting does not induce optimal predicate lifting!
% not verifiable with consistent predicate
% { ? }
% release acc(x.f)
% { ? * acc(x.f) }
%
% but
% { ? }
% release acc(x.f)
% { ? }
% =>
% { ? * acc(x.f) }
%
%
% not verifiable with consistent predicate
% { ? }
% x := random();
% { (x = 0) }
%
% but
% { ? }
% x := random();
% { ? }
% =>
% { (x = 0) }

The approach we propose is based on the idea to treat the Hoare predicate as a (multivalued) function, mapping preconditions to the set of possible/verifiable postconditions.
We can obtain a lifted version of this hypothetical construct and demand certain properties similar to the ones defined in section \ref{text/SSEC-lifting-functions}:

\begin{definition}[Deterministic Lifting]
    Given a binary predicate $P \subseteq \setFormula \times \setFormula$ we call a partial function $\dgrad{P} : \setFormula \rightharpoonup \setFormula$ \textbf{deterministic lifting} of $P$ if the following conditions are met:
    \begin{description}
        \item[Introduction]~\\
        \begin{displaymath}
        \forall (\phi_1, \phi_2) \in P.~ \phi_1 \in \dom(\dgrad{P})
        \end{displaymath}
        
        \item[Preservation]~\\
        \begin{mathpar}
            \forall \grad{\phi_1}, \grad{\phi_2} \in \setGFormula.~ 
            \dgrad{P}(\grad{\phi_1}) = \grad{\phi_2}\\
            \implies\\
            \forall \phi_1 \in \gamma(\grad{\phi_1}), \phi_2 \in \setFormula.~ P(\phi_1, \phi_2) \implies \exists \phi \in \gamma(\grad{\phi_2}).~ P(\phi_1, \phi) ~\wedge~ \phiImplies{\phi}{\phi_2}
        \end{mathpar}
        
        \item[Monotonicity]~\\
        Note: Identical to monotonicity condition of lifted partial functions. % say why? P/f doesn't show up.
        \begin{displaymath}
        \forall \grad{\phi_1}, \grad{\phi_2} \in \setGFormula.~ \grad{\phi_1} \sqsubseteq \grad{\phi_2} \wedge \grad{\phi_1} \in \dom(\dgrad{P}) \implies \dgrad{P}(\grad{\phi_1}) \sqsubseteq \dgrad{P}(\grad{\phi_2})
        \end{displaymath}
    \end{description}
\end{definition}

...assume we have obtained deterministic lifting $\dgthoare {~} {\cdot} {\cdot} {\cdot}$ of our Hoare triple. % TODO: notation talk
This gradual partial function has desirable properties: % use lemmas only? PROOFS


\begin{description}
    \item[Obtaining a Sound Gradual Lifting]
    \begin{lemma}[Deterministic Gradual Lifting]~\\
        Let $\dgrad{P}$ be a deterministic lifting of $P$.
        Then
        \begin{displaymath}
        \grad{P}(\grad{\phi_1}, \grad{\phi_2}) ~~\defiff~~ \exists \grad{\phi_2'}.~ \dgrad{P}(\grad{\phi_1}) = \grad{\phi_2'} \wedge \gphiImplies {\grad{\phi_2'}} {\grad{\phi_2}}
        \end{displaymath}
        is a sound gradual lifting of $P$.
    \end{lemma}
    
    \item[Determinism]~\\
    A verifier dealing with deterministic liftings has no more obligation of finding good intermediate formulas.
    % So also no more of the problems/dilemmas apply
    
    \item[Preservation]~\\
    A (gradual) postcondition returned by the lifted function is guaranteed to reflect the execution state after executing the statements in question (given that the precondition was met). Almost.
    Combines all the knowledge of static rules.
    % LEMMA?
    
    \item[Composability]~\\
    \begin{lemma}[Composability of Deterministic Lifting]~\\
        Let $\dgrad{P_1}, \dgrad{P_2}$ be deterministic liftings of predicates $P_1, P_2$.
        Then
        \begin{displaymath}
        \dgrad{P_3} ~~\defeq~~ \dgrad{P_2} \circ \dgrad{P_1}
        \end{displaymath}
        is a deterministic lifting of $P_3(\phi_1, \phi_3) = \exists \phi_2.~ P_1(\phi_1, \phi_2) \wedge P_2(\phi_2, \phi_3)$.
    \end{lemma}
\end{description}

% EXAMPLE LIFTINGS