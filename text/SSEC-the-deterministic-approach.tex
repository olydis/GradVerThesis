% proof by example: optimal deterministic lifting does not induce optimal predicate lifting!
% not verifiable with consistent predicate
% { ? }
% release acc(x.f)
% { ? * acc(x.f) }
%
% but
% { ? }
% release acc(x.f)
% { ? }
% =>
% { ? * acc(x.f) }
%
%
% not verifiable with consistent predicate
% { ? }
% x := random();
% { (x = 0) }
%
% but
% { ? }
% x := random();
% { ? }
% =>
% { (x = 0) }

%% intro
In the previous section we built an intuition about intermediate gradual formulas that are neither too weak (increasing the chance of hiding inconsistencies) nor too strict (introducing semantically invalid Hoare triples).
While this problem could certainly be solved by designing a sophisticated inference algorithm for gradual formulas, we propose solving the problem at the level of the gradual Hoare logic.

In this section we derive a “deterministic gradual Hoare logic” 
$$\dgrad{\vdash} : \setGFormula \times \setGStmt \rightharpoonup \setGFormula$$
from the static Hoare logic $\thoare{}{\cdot}{\cdot}{\cdot} \subseteq \setFormula \times \setStmt \times \setFormula$ in a process we call “deterministic lifting”.
Note the difference to the signature of the gradual Hoare logic predicate
$$\gthoare{}{\cdot}{\cdot}{\cdot} : \setGFormula \times \setGStmt \times \setGFormula$$

The deterministic gradual Hoare logic has the following desirable properties (formalized later):
A gradual lifting $\gthoare{}{\cdot}{\cdot}{\cdot}$ can be obtained from $\dgrad{\vdash}$, which is required for the gradual verifier.
This gradual lifting is not defined inductively and does not rely on variables being instantiated.
Instead it is syntax-directed, making the verification process deterministic.
Furthermore $\dgrad{\vdash}$ allows for a stronger notion of soundness that does not rely on runtime checks.

We may use the familiar notation $\dgthoare{}{\grad{\phi_1}}{\grad{s}}{\grad{\phi_2}}$ as an alias for $~\dgrad{\vdash}(\grad{\phi_1}, \grad{s}) = \grad{\phi_2}$

~\\
%% approach idea
Our approach is based on the idea to treat the Hoare predicate as a (multivalued) function, mapping preconditions to the set of possible/verifiable postconditions.
From this multivalued function we can obtain a lifted version, following similar rules to the ones for lifted partial functions (see \ref{sssec:lifting-partial-functions}).

%% def
\begin{definition}[Deterministic Lifting]~\\
    Given a binary predicate $P \subseteq \setFormula \times \setFormula$ we call a partial function $$\dgrad{P} : \setFormula \rightharpoonup \setFormula$$ \textbf{deterministic lifting} of $P$ if the following three conditions are met.\\
    (As before, the rules can be adapted to different parameter types and higher arity.
    We indicate an adaptation to Hoare logic in blue.)
\end{definition}

\begin{description}
    \item[Introduction]~\\
    The deterministic lifting should be defined whenever the underlying predicate is.
    \begin{displaymath}
    \forall \langle \phi_1, \phi_2 \rangle \in P.~ \phi_1 \in \dom{\dgrad{P}}
    \end{displaymath}
    \textcolor{blue}{
        \begin{displaymath}
        \forall \langle \phi_1, s, \phi_2 \rangle \in \thoare{}{\cdot}{\cdot}{\cdot}.~ \langle \phi_1, s \rangle \in \dom{\dgrad{\vdash}}
        \end{displaymath}
    }
    
    \item[Strength]~\\
    A return value of the deterministic lifting should agree with all instantiations of the underlying predicate.
    \begin{mathpar}
        \forall \grad{\phi_1}, \grad{\phi_2} \in \setGFormula.~ 
        \dgrad{P}(\grad{\phi_1}) = \grad{\phi_2}\\
        \implies\\
        \forall \phi_1 \in \gamma(\grad{\phi_1}),\, \phi \in \setFormula.~ P(\phi_1, \phi) \implies\\
        \exists \phi_2 \in \gamma(\grad{\phi_2}).~ P(\phi_1, \phi_2) ~\wedge~ (\phiImplies{\phi_2}{\phi})
    \end{mathpar}
    \textcolor{blue}{
        \begin{mathpar}
        \forall \grad{\phi_1}, \grad{\phi_2} \in \setGFormula,\, \grad{s} \in \setGStmt.~ 
        \dgrad{\vdash}(\grad{\phi_1}, \grad{s}) = \grad{\phi_2}\\
        \implies\\
        \forall \phi_1 \in \gamma(\grad{\phi_1}),\, s \in \gamma(\grad{s}),\, \phi \in \setFormula.~ \thoare{}{\phi_1}{s}{\phi} \implies\\
        \exists \phi_2 \in \gamma(\grad{\phi_2}).~ \thoare{}{\phi_1}{s}{\phi_2} ~\wedge~ (\phiImplies{\phi_2}{\phi})
        \end{mathpar}
    }
    
    For Hoare rules this means that the postcondition returned must be at least as strong as every postcondition returned by static Hoare logic (it might be less precise, though).
    The following example illustrates the effects of the rule:
    \begin{example}{Deterministic Lifting Strength}
        Assume that the following list contains all instantiations of $\thoare{}{\phiEq{2}{2}}{\sVarAssign{y}{2}}{\cdot}$.
        \begin{flalign*}
        \thoare{}{\phiEq{2}{2}}{\sVarAssign{y}{2}}{\phiEq{2}{2}}\\
        \thoare{}{\phiEq{2}{2}}{\sVarAssign{y}{2}}{\phiEq{y}{2}}\\
        \thoare{}{\phiEq{2}{2}}{\sVarAssign{y}{2}}{\phiEq{2}{y}}\\
        \thoare{}{\phiEq{2}{2}}{\sVarAssign{y}{2}}{\phiEq{y}{y}}
        \end{flalign*}
        Then valid return values for the deterministic lifting are
        \begin{flalign*}
        &\dgthoare{}{\phiEq{2}{2}}{\sVarAssign{y}{2}}{\phiEq{y}{2}}\\
        &\dgthoare{}{\phiEq{2}{2}}{\sVarAssign{y}{2}}{\phiEq{2}{y}}\\
        &\dgthoare{}{\phiEq{2}{2}}{\sVarAssign{y}{2}}{\qm}\\
        &\dgthoare{}{\phiEq{2}{2}}{\sVarAssign{y}{2}}{\withqmGen{\phiEq{y}{2}}}\\
        &\dgthoare{}{\phiEq{2}{2}}{\sVarAssign{y}{2}}{\withqmGen{\phiEq{2}{y}}}
        \end{flalign*}
        Not valid are weaker static values like
        \begin{flalign*}
        &\dgthoare{}{\phiEq{2}{2}}{\sVarAssign{y}{2}}{\phiEq{2}{2}}\\
        &\dgthoare{}{\phiEq{2}{2}}{\sVarAssign{y}{2}}{\phiEq{y}{y}}
        \end{flalign*}
        or stronger values like
        \begin{flalign*}
        &\dgthoare{}{\phiEq{2}{2}}{\sVarAssign{y}{2}}{\phiAnd{\phiEq{x}{3}}{\phiEq{y}{2}}}
        \end{flalign*}
        
        Likewise, with precondition $\qm$ (motivated by case “too strict” in section \ref{ssec:the-problem-with}) valid return values include
        \begin{flalign*}
        &\dgthoare{}{\qm}{\sVarAssign{y}{2}}{\qm}\\
        &\dgthoare{}{\qm}{\sVarAssign{y}{2}}{\withqmGen{\phiEq{y}{2}}}
        \end{flalign*}
        bot not
        \begin{flalign*}
        &\dgthoare{}{\qm}{\sVarAssign{y}{2}}{\phiAnd{\phiEq{x}{3}}{\phiEq{y}{2}}}
        \end{flalign*}
        which corresponds to the instantiation criticized in section \ref{ssec:the-problem-with}.
    \end{example}
    
    \item[Monotonicity]~\\
    This rule is identical to the monotonicity condition of lifted partial functions (see section \ref{sssec:lifting-partial-functions}). % say why? P/f doesn't show up.
    \begin{align*}
    \forall \grad{\phi_1}, \grad{\phi_2} \in \setGFormula.~ 
    &\grad{\phi_1} \sqsubseteq \grad{\phi_2} ~~\wedge~~ \grad{\phi_1} \in \dom{\dgrad{P}} \\
    \implies 
    &\dgrad{P}(\grad{\phi_1}) \sqsubseteq \dgrad{P}(\grad{\phi_2})
    \end{align*}
    \textcolor{blue}{
        \begin{align*}
        \forall \grad{\phi_1}, \grad{\phi_2} \in \setGFormula,\, \grad{s_1}, \grad{s_2} \in \setGStmt.~ 
        &\grad{\phi_1} \sqsubseteq \grad{\phi_2} ~~\wedge~~ \grad{s_1} \sqsubseteq \grad{s_2} ~~\wedge~~ \langle \grad{\phi_1}, \grad{s_1} \rangle \in \dom{\dgrad{\vdash}}\\
        \implies 
        &\dgrad{\vdash}(\grad{\phi_1}, \grad{s_1}) \sqsubseteq \dgrad{\vdash}(\grad{\phi_2}, \grad{s_2})
        \end{align*}
    }
\end{description}

Soundness and optimality are defined as usual (see sections \ref{ssec:lifting-predicates} or \ref{ssec:lifting-functions}).

%% application & properties
Assume we have obtained the deterministic lifting $\dgthoare {~} {\cdot} {\cdot} {\cdot}$ of our Hoare logic.
As mentioned before, this gradual partial function has desirable properties:
\begin{description}
    \item[(a) Obtaining a Sound Gradual Lifting]
    \begin{lemma}[Deterministic Lifting as Sound Lifting]
        \label{lem:det2grad}~\\
        Let $\dgrad{P}$ be a deterministic lifting of $P$.
        Then
        \begin{displaymath}
        \grad{P}(\grad{\phi_1}, \grad{\phi_2}) ~~\defiff~~ \exists \grad{\phi_2'}.~ \dgrad{P}(\grad{\phi_1}) = \grad{\phi_2'} \wedge \gphiImplies {\grad{\phi_2'}} {\grad{\phi_2}}
        \end{displaymath}
        is a sound gradual lifting of $P$.\\
        See lemma \ref{lemma:ol-impl} for definition of $\gphiImplies{\cdot}{\cdot}$.
    \end{lemma}
    
    This observation bridges the gap between $\dgthoare {} {\cdot} {\cdot} {\cdot}$ and the gradual verifier which is supposed to implement $\gthoare {} {\cdot} {\cdot} {\cdot}$.
    
    Optimality of the deterministic lifting does not imply optimality of the obtained gradual lifting.
    
    \begin{example}{Counterexample of Optimality Induced by Deterministic Lifting}
        \label{cex:opt-det2grad}~\\
        Assume that one can verify
        \begin{align}
        \label{frm:opt-det2grad1}
        \thoare{}{\phiEq{y}{4}}{\sVarAssign{x}{3}}{\phiAnd{\phiEq{y}{4}}{\phiEq{x}{3}}}
        \end{align}
        and
        \begin{align}
        \label{frm:opt-det2grad2}
        \thoare{}{\phiEq{y}{5}}{\sVarAssign{x}{3}}{\phiAnd{\phiEq{y}{5}}{\phiEq{x}{3}}}
        \end{align}
        using Hoare logic.
        
        Now, let $\setGFormula$ be extended using the “total unknown” approach (see section \ref{ssec:dedicated-wildcard-formula}).
        The deterministic lifting of the Hoare logic will verify
        \begin{displaymath}
        \dgthoare{}{\qm}{\sVarAssign{x}{3}}{\qm}
        \end{displaymath}
        The wildcard as postcondition is necessary, as anything else (e.g. $\phiEq{x}{3}$) would break the strength condition (take judgments \ref{frm:opt-det2grad1} and \ref{frm:opt-det2grad2}, both of their postconditions cannot be implied by any satisfiable static formula).
        
        Using \ref{lem:det2grad} we can deduce
        \begin{align}
        \label{frm:opt-det2grad4}
        \gthoare{}{\qm}{\sVarAssign{x}{3}}{\phiEq{x}{1}}
        \end{align}
        since $\gphiImplies{\qm}{\phiEq{x}{1}}$ holds.
        An optimal gradual lifting would not be able to deduce this judgment, as it requires the existing of some $\phi \in \gamma(\qm) = \setFormulaA$ such that
        \begin{displaymath}
        \thoare{}{\phi}{\sVarAssign{x}{3}}{\phiEq{x}{1}}
        \end{displaymath}
        holds.
        However, the only working instantiation according to Hoare logic is $\phi = \phiEq{3}{1}$ which is not satisfiable.
        It follows that the obtained lifting is not optimal, even though the deterministic lifting is.
        
        Note that the “bounded unknown” approach (see \ref{ssec:wildcard-with-upper}) would allow specifying $\withqmGen{\phiEq{x}{3}}$ as postcondition.
        With this more precise postcondition, one cannot deduce judgment \ref{frm:opt-det2grad4} (since $\gphiImplies{\withqmGen{\phiEq{x}{3}}}{\phiEq{x}{1}}$ does not hold).
        This example motivates the use of a more powerful gradual syntax, as they enable more optimal gradual liftings.
    \end{example}
    
    We conjecture that optimality of the deterministic lifting does imply optimality of the obtained gradual lifting if the gradual formula syntax is expressive enough to encode all knowledge about the dynamic semantics of a statement.
    
    \item[(b) Determinism of Verifier]~\\
    As the name suggests, deterministic liftings leave no room for choice, but are instead syntax-directed.
    For the gradual verifier this means that there is no need to infer intermediate formulas, averting the risk of choosing bad instantiations (as illustrated in section \ref{ssec:the-problem-with}).
        
    \item[(c) Soundness]~\\
    Deterministic liftings are designed in a way that allows for a stronger notion of soundness that does not rely on runtime checks.
    \begin{definition}[\dgradT Soundness]\label{def:dgsnd}~\\
        Let $\grad{\phi_1}, \grad{\phi_2} \in \setGFormula$ and $\grad{s} \in \setGStmt$.\\
        If $\dgthoare{}{\grad{\phi_1}}{\grad{s}}{\grad{\phi_2}}$ then $\gtHoare{}{\grad{\phi_1}}{\grad{s}}{\grad{\phi_2}}$
    \end{definition}
    Note that this rule is no longer a tautology, compliance also depends on the gradual dynamic semantics of \gvl.
    
    \begin{example}{\tset{\dgradT Soundness} Counterexample}
        \label{ex:cex-gdpres}~\\
        Assume that \svl contains an assertion statement with corresponding Hoare rule.
        \begin{mathpar}
            \inferrule* [Right=HAssert]
            {
                \phiImplies{\phi}{\phi_a}
            }
            {
                \thoare{~}{\phi}{\sAssert{$\phi_a$}}{\phi}
            }
        \end{mathpar}
        Soundness of the Hoare logic implies that assertions are guaranteed to hold at runtime.
        It is therefore reasonable for \svl to implement assertions as no-operations.
        It follows from the rules of gradual function lifting (applied to the small-step semantics) that \gvl implements them the same way.
        
        A valid deterministic lifting of \tset{HAssert} is able to verify
        \begin{displaymath}
        \dgthoare{~}{\qm}{\sAssert{$\phi_a$}}{\withqmGen{\phi_a}}
        \end{displaymath}
        However $\gtHoare{}{\qm}{\sAssert{$\phi_a$}}{\withqmGen{\phi_a}}$ does not hold for non-trivial $\phi_a$:
        
        Choose $\phi_a = \phiEq{x}{2}$.
        Let $\grad{\pi}, \grad{\pi'} \in \setGProgramState$ such that $\gsstepConsume{\sAssert{$\phi_a$}}{\grad{\pi}}{\grad{\pi'}}$ and $\evalgphiGen{\grad{\pi}}{\phiEq{x}{3}}$ holds (both assumptions are compatible since assertions are implemented as no-operations, i.e. especially not runtime check $\evalgphiGen{\grad{\pi}}{\phi_a}$ is performed).
        It follows that $\evalgphiGen{\grad{\pi'}}{\phiEq{x}{3}}$.
        According to definition \ref{def:valid-ghoare-triple}, we can deduce from $\gtHoare{}{\qm}{\sAssert{$\phi_a$}}{\withqmGen{\phi_a}}$ that $\evalgphiGen{\grad{\pi'}}{\withqmGen{\phiEq{x}{2}}}$.
        No such $\grad{\pi'}$ exists.
        
        On the other hand, a small-step semantics that is stuck or throws an exception if the assertion is violated would restore soundness.
        In general, small-step semantics that rely on guarantees given by static semantics may turn out too weak after gradualization of the static semantics.
    \end{example}
\end{description}


\begin{lemma}[Reduced Runtime Checks]
    \label{lemma:runtime-checks-reduced}~\\
    Assume that $\dgthoare{~}{\grad{\phi_1}}{\grad{s}}{\grad{\phi_2'}}$ holds for some $\grad{\phi_1}, \grad{s}, \grad{\phi_2'}$.
    If \tset{\dgradT Soundness} holds, the runtime check $\sAssert{$\grad{\phi_2}$}$ associated with a judgment $\gthoare{~}{\grad{\phi_1}}{\grad{s}}{\grad{\phi_2}}$ (as postulated by definition \ref{def:gsnd}) can be simplified or omitted.
    
    Specifically, it is sufficient to insert a runtime check (predicate) $P \subseteq \setGProgramState$ if
    \begin{gather*}
    \envs{\grad{\phi_2'}} \cap \envs{\grad{\phi_2}} \subseteq \envs{\grad{\phi_2'}} \cap P \subseteq \envs{\grad{\phi_2}}
    \end{gather*}
    holds where $\envs{\cdot} : \setGFormula \rightarrow \PP^{\setGProgramState}$ is defined analogous to definition \ref{def:frm-den-sem}:
    \begin{displaymath}
    \envs{\grad{\phi}} \defeq \{~ \grad{\pi} \in \setGProgramState ~|~ \evalgphiGen{\grad{\pi}}{\grad{\phi}} ~\}
    \end{displaymath}
\end{lemma}

\begin{example}{Reduced Runtime Check}~\\
    Let $\grad{\phi_2'} = \withqmGen{\phiEq{x}{3}}$ and $\grad{\phi_2} = \phiAnd{\phiEq{x}{3}}{\phiEq{y}{5}}$.
    
    Then $P(\pi) \defeq \evalgphiGen{\pi}{\phiEq{y}{5}}$ is a sufficient runtime check.
\end{example}

\begin{corollary}[Superfluous Runtime Checks]\label{cor:runtime-checks-removed}~\\
    $P(\pi) = \setProgramState$ is a sufficient runtime check if $\envs{\grad{\phi_2'}} \subseteq \envs{\grad{\phi_2}}$.\\
    In other words, no runtime check is necessary in this case.
\end{corollary}