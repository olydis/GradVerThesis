%% intro
Completeness as defined in definition \ref{def:completeness} tightly couples Hoare logic and small-step semantics.
In practice this can be impossible to achieve due to decidability, i.e. not all valid small-step derivations can be modeled using Hoare logic.
Similarly, semi-optimality of the gradual semantics might be hard to ensure.
Semi-optimality of the gradual Hoare logic requires deciding the existence of a Hoare logic derivation which involves first order logic for composite statements like sequences (which was a problem we originally wanted to avoid with deterministic liftings).
In this section we motivate that a weaker notion of completeness and semi-optimality may be sufficient to prove \tset{\dgradT Soundness}.

%% 
The sequence operator \ttt{;} is key to defining composite statements in most programming languages.
Fortunately, \tset{\dgradT Soundness} can be proved for sequences inductively.
\begin{lemma}[\tset{\dgradT Soundness} for Sequences]
    \label{lemma:gdpres-seq}~\\
    If \tset{\dgradT Soundness} holds for statements $s_1$ and $s_2$ then it holds for $s_1;s_2$
\end{lemma}

In the case study we will use this lemma to prove \tset{\dgradT Soundness} of a gradually verified language, but also show that this inductive approach can be applied to other composite statements like method calls (section \ref{sec:gradual-soundness}).
As a result, it is sufficient to prove \tset{\dgradT Soundness} for “primitive” statements, e.g. by using the approach introduced in the previous section:
Note that the definitions of completeness and semi-optimality are universally quantified over the set of all (gradual) statements.
Instead, they can be weakened to quantify only over a limited set of primitive statements.
The resulting proof of \tset{\dgradT Soundness} (lemma \ref{thm:compl-and-so-to-gdpres}) will apply only to this set of statements.

Again, we want to point out that we only give examples of sufficient criteria to prove \tset{\dgradT Soundness}.
It is possible that the approaches do not work for a certain programming language, or even that it is entirely impossible to satisfy \tset{\dgradT Soundness}.
However, recall that \tset{\dgradT Soundness} is not necessary for \gvl to be sound (i.e. satisfy \tset{\gradT Soundness}).