$\sssem'$ as defined previously requires a priori knowledge about whether Hoare logic can derive a proof for arbitrarily complex statements.
In \svl the sequence operator is key to defining more complex statements.

Assume $\sssem'$ is artificially made undefined, but not for sequences.
Example: If $\sssem^{\sSeq{$s_1$}{$s_2$}}(\pi_1, \pi_2)$ holds, then $\sssem'^{\sSeq{$s_1$}{$s_2$}}(\pi_1, \pi_2)$ may hold even if there exists no proof.

\tset{GDPreservation} still holds for sequences:
\begin{mathpar}
    \inferrule* [Right=seq]
    {
        \inferrule* [Right=GDPreservation]
        {
            \inferrule* [Right=inversion]
            {
                \dgthoare{~}{\grad{\phi_1}}{\sSeq{$\grad{s_1}$}{$\grad{s_2}$}}{\grad{\phi_3}}\\
            }
            {
                 \dgthoare{~}{\grad{\phi_1}}{\grad{s_1}}{\grad{\phi_2}}\\
                 \dgthoare{~}{\grad{\phi_2}}{\grad{s_2}}{\grad{\phi_3}}
            }
        }
        {
            \gtHoare{~}{\grad{\phi_1}}{\grad{s_1}}{\grad{\phi_2}}\\
            \gtHoare{~}{\grad{\phi_2}}{\grad{s_2}}{\grad{\phi_3}}
        }
    }
    {
        \gtHoare{~}{\grad{\phi_1}}{\sSeq{$\grad{s_1}$}{$\grad{s_2}$}}{\grad{\phi_3}}
    }
\end{mathpar}