%% intro
Completeness as defined before tightly couples Hoare logic and small-step semantics.
In practice this can be impossible to implement due to decidability, i.e. not all valid small-step derivations can be modeled using Hoare logic.
As a result, completeness is impossible to achieve.
Similarly, semi-optimality of the gradual semantics might be hard to ensure.
Semi-optimality of the gradual Hoare logic requires deciding the existence of a Hoare logic derivation which involves first order logic for composite statements like sequences (which was a problem we originally wanted to avoid with deterministic liftings).
In this section we motivate that a weaker notion of completion and semi-optimality may be sufficient to prove \tset{\dgradT Preservation}.

%% 
The sequence operator \ttt{;} is key to defining composite statements in most programming languages.
Fortunately, \tset{\dgradT Preservation} can be proved for sequences inductively.
\begin{lemma}[\tset{\dgradT Preservation} for Sequences]
    \label{lemma:gdpres-seq}~\\
    If \tset{\dgradT Preservation} holds for statements $s_1$ and $s_2$ then it holds for $s_1;s_2$
\end{lemma}
\begin{proof}
    \begin{mathpar}
        \inferrule* [Right=seq]
        {
            \inferrule* [Right=\dgradT Preservation]
            {
                \inferrule* [Right=inversion]
                {
                    \dgthoare{~}{\grad{\phi_1}}{\sSeq{$\grad{s_1}$}{$\grad{s_2}$}}{\grad{\phi_3}}\\
                }
                {
                    \dgthoare{~}{\grad{\phi_1}}{\grad{s_1}}{\grad{\phi_2}}\\
                    \dgthoare{~}{\grad{\phi_2}}{\grad{s_2}}{\grad{\phi_3}}
                }
            }
            {
                \gtHoare{~}{\grad{\phi_1}}{\grad{s_1}}{\grad{\phi_2}}\\
                \gtHoare{~}{\grad{\phi_2}}{\grad{s_2}}{\grad{\phi_3}}
            }
        }
        {
            \gtHoare{~}{\grad{\phi_1}}{\sSeq{$\grad{s_1}$}{$\grad{s_2}$}}{\grad{\phi_3}}
        }
    \end{mathpar}
\end{proof}

As we will show in the case study (section \ref{sec:gradual-soundness}) this inductive approach can be applied to other composite statements like method calls.
As a result, it is sufficient to prove \tset{\dgradT Preservation} for “primitive” statements, e.g. by using the approach introduced in the previous section:
Note that the definitions of completeness and semi-optimality are universally quantified over the set of all (gradual) statements.
Instead, they can be weakened to quantify only over a limited set of primitive statements.
The resulting proof of \tset{\dgradT Preservation} (lemma \ref{thm:compl-and-so-to-gdpres}) will apply only to this set of statements.

Again, we want to point out that we only give examples of sufficient criteria to prove \tset{\dgradT Preservation}.
It is possible that the approaches do not work for a certain programming language, or even that it is entirely impossible to satisfy \tset{\dgradT Preservation}.
However, recall that \tset{\dgradT Preservation} is not necessary for \gvl to be sound.