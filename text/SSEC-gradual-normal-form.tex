In this section we will introduce a normal form for partially unknown formulas that proves useful for implementing a gradual verifier for \gvlidf.
One example is the function postulated in lemma \ref{lemma:impl-gimpl}, another example is gradual formula precision $\sqsubseteq$:
Originally it is defined as comparing (possibly infinite) concretizations for inclusion (see definition \ref{def:mpt}), using the following theorem the problem is reducible to checking an implication (without proof).

%% normal form
\begin{theorem}[Gradual Normal Form]\label{thm:gnf}~\\
    There exists a computable function $\snorm{\cdot} : \setFormula \rightarrow \setFormula$, such that
    \begin{enumerate}
        \item $\norm{\withqm{\phi}} \defeq \withqm{\snorm{\phi}}$ is equivalent to $\withqm{\phi}$ ~~~(for all $\phi \in \setFormula$)
        \item $\phi \implies \snorm{\phi}$
        \item $\snorm{\phi}$ contains no accessibility-predicates
        \item $\withqm{\phi_1} \sqsubseteq \withqm{\phi_2}  \quad\iff\quad  \snorm{\phi_1} \implies \snorm{\phi_2}$
    \end{enumerate}
\end{theorem}

\begin{comment}
In the remainder of this section we try to give an intuition behind this normal form.

Recall that gradual formulas $\withqm{\phi_1}$ and $\withqm{\phi_2}$ are considered equal iff $\gamma(\withqm{\phi_1}) = \gamma(\withqm{\phi_1})$.
The normal form makes use of the fact that concretizations of partially unknown formulas contain only self-framed formulas (see section \ref{sec:cs-gradual-formulas}).

\begin{lemma}[Mentioning a field implies access]~\\
    For any formula $\phi$ mentioning field $\edot{$x$}{$f$}$:
    $$\forall \phi' \in \gamma(\withqm{\phi}), \phiImplies{\phi'}{\phiAcc{$x$}{$f$}}$$
\end{lemma}
In other words: Merely mentioning a field will make sure that the concretization contains appropriate framing.
This is a helpful observation for justifying removal of accessibility predicates from the static part.

Note, however, that simply dropping access from the static part may not result in an equivalent gradual formula for two reasons:
\begin{description}
    \item[1. No more mentions]~\\
    Dropping $\phiAcc{$x$}{$f$}$ might result in $\edot{$x$}{$f$}$ not being mentioned in the formula anymore, so there would be no more reason for the concretization to “restore” $\phiAcc{$x$}{$f$}$ in all concretizations.
    
    \item[2. Aliasing]~\\
    In general there are different ways in which access to multiple (syntactically different) fields can be restored.
    \begin{example}{Ambiguous Framing}~\\
        Dropping all accessibility predicates from $$\phiCons{\phiCons{\phiAcc{a}{f}}{\phiAcc{b}{f}}}{\phiCons{\phiEq{a.f}{3}}{\phiEq{b.f}{x}}}$$
        results in
        $$\phiCons{\phiEq{a.f}{3}}{\phiEq{b.f}{x}}$$
        However, $\gamma(\withqm{\phiCons{\phiEq{a.f}{3}}{\phiEq{b.f}{x}}})$ contains
        $$\phiCons{\phiCons{\phiAcc{a}{f}}{\phiEq{a}{b}}}{\phiCons{\phiEq{a.f}{3}}{\phiEq{a.f}{x}}}$$
        whereas $\gamma(\withqm{\phiCons{\phiCons{\phiAcc{a}{f}}{\phiAcc{b}{f}}}{\phiCons{\phiEq{a.f}{3}}{\phiEq{b.f}{x}}}})$
        does not
    \end{example}
    %In other words, the possibility of aliasing may result in a variety of re-framed formulas that are not equivalent with the original one.
    % Elaborate in more detail why this is bad?
    % Also: this is where dominators play in... draw the line? How far?
\end{description}

Fortunately, we can prevent both problems from occurring by carefully preparing the static part before dropping all access, resulting in the following two-step approach:

\begin{description}
    \item[1. Enhancement]~\\
    Enrich the static part to counteract above problems, i.e. to enforce that access is restored exactly the right way.
    This is achieved by simply spelling out certain implications of the access-terms:
    \begin{description}
        \item [$\phiAcc{$x$}{$f$} \implies \phiEq{\edot{$x$}{$f$}}{\edot{$x$}{$f$}}$]~\\
        Access to a field implicitly guarantees that it actually evaluates to some (arbitrary) value.
        Note that $\phiEq{\edot{$x$}{$f$}}{\edot{$x$}{$f$}}$ is not a logical tautology (i.e. it is not implies by $\phiTrue$), since it indeed makes sure that $\edot{$x$}{$f$}$ evaluates, whereas $\phiTrue$ does not.
        The bottom line is that $\edot{$x$}{$f$}$ is being mentioned even after dropping $\phiAcc{$x$}{$f$}$, therefore solving the first problem. 
        \item [$\phiCons{\phiAcc{$x$}{$f$}}{\phiAcc{$y$}{$f$}} \implies \phiNeq{$x$}{$y$}$]~\\
        Having access to the same field of different expressions actively prevents those expressions to ever alias.
        Spelling out this restriction by adding the corresponding inequality also prevents re-framing in a way that relies on aliasing.
        The bottom line is that any valid re-framing must restore two distinct access-terms, therefore solving the second problem.
    \end{description}
    We enhance the non-linear part of our formula by spelling out above implications in every possible way, i.e. accounting for all (pairs of) access-terms.
    It is worth noting that this enhancement preserves equality of the formula as only terms are added that were implied by the original formula, anyway.
    
    \item[2. Delinearization]~\\
    All access-terms are dropped.
\end{description}
\end{comment}