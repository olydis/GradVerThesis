%%intro
It is expected that gradual verification can induce runtime overhead.
Specifically, in the presence of unknown formulas runtime checks are necessary to guarantee soundness of the gradual system.

%%where runtime overhead can come from
With the methodology we introduced in chapter \ref{ch:gradualization-of-a} there are two potential causes for runtime checks.
The first cause is the gradual Hoare judgment, which in general requires runtime measures to ensure the postcondition dynamically (see section \ref{ssec:gradual-soundness}).
We have also discussed how those runtime checks can be reduced or even eliminated if \tset{\dgradT Soundness} holds (see lemma \ref{lemma:runtime-checks-reduced}).
The second cause is more subtle and originates in restricting the domain of the small-step semantics to achieve completeness as described in section \ref{ssec:perfect-knowledge}.
Restricting the domain usually means adding additional checks that determine whether the function is defined or not.
Example \ref{ex:cex-gdpres} illustrates this effect using the example of an assertion statement.
Note that a compiler may also simplify or omit these static checks using the knowledge provided by the deterministic lifting (given that  \tset{\dgradT Soundness} holds).

%% svlidf
In chapter \ref{ch:case-study--implicit} we designed the small-step semantics of \svlidf to be (partially) complete in the first place (see lemma \ref{lemma:pc-idf}).
Note however, that ???mark them gray or something??? were not necessary for \svlidf, yet they are necessary for \gvlidf.

%% static annotations
It is a desirable property of gradual systems to not impose any runtime overhead (compared to the static system) should all annotations be static.
We give sufficient criteria for such a gradual verification system.
\begin{lemma}[Overhead-Free Gradually Verified Language]
    \label{lemma:oh-free}~\\
    For fully statically annotated programs, no runtime overhead is necessary if \tset{\dgradT Soundness} and
    \begin{align}
    \label{lemma-frm:oh-free}
    \forall \phi_1 \in \setFormula,\, s \in \setStmt,\, \grad{\phi_2} \in \setGFormula.~ \dgthoare{}{\phi_1}{s}{\grad{\phi_2}} \implies \grad{\phi_2} \in \setFormula
    \end{align}
    holds.
    Intuitively, this condition states that the deterministic gradual Hoare logic is precise for all static input.
    
   % Optimality of the deterministic lifting does not necessarily imply \ref{lemma-frm:oh-free} if \tset{\dgradT Soundness} holds.
   % A stricter version of this lemma would thus read:
   % For fully statically annorated programs, no runtime overhead is necessary if \tset{\dgradT Soundness} and the deterministic gradual Hoare logic is an optimal deterministic lifting.
\end{lemma}

Note: \gvlidf satisfies the criteria given by lemma \ref{lemma:oh-free} (without proof) and can thus be implemented overhead-free for statically annotated programs.
