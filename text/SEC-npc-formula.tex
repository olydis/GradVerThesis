%% reasonable: formulas evaluable at runtime
Checking a formula at runtime, i.e. performing a runtime assertion check, is the integral part of dynamic verification and thus plays a role in gradual verification.
Formally, a runtime assertion check corresponds to evaluating a closed formula since the environment provides an instantiation of the formula's free variables.
It is reasonable to demand that this check can be performed in a time polynomial, if not linear to the formula's length (the specifics are up to the language designer, of course).

%% impact on formula syntax: quantification
Such a requirement effectively restricts the formula syntax.
For example, a syntax containing universal quantification generally violates above runtime limitations:
A formula $\forall x_1 \in M, x_2 \in M, ..., x_n \in M. P(x_1, x_2, ..., x_n)$ would require $|M|^n$ steps to evaluate.
As a result, the execution time is already exponential if $M$ is finite -- and unbounded otherwise.

%% impact on formula syntax: still pretty little restriction
Putting quantification (and therefore the introduction of new variables) aside, there are little restrictions to formula syntax, essentially allowing any predicates or operations that can be evaluated in linear/polynomial time.
This includes equality/inequality relations, arithmetic and even own predicates that might be recursive to some extent.

%% nevertheless: static verification undec.
Nevertheless such “easily” evaluable formulas are also subject to higher order reasoning in the static verification rules, including checks like satisfiability of or implication between formulas.
Those judgments basically introduce quantification of the free variables, whereas evaluation works on a concrete instantiation.
This makes static verification NP-hard in general:
\begin{description}
    \item[NPC] One can easily encode SAT instances as formulas, either directly (if the syntax covers boolean variables, conjunction and disjunction) or using arithmetic (if the syntax covers addition and a comparison relation like “greater-than”). Note that although evaluating such formulas is trivial, checking for satisfiability is NP-complete. % TODO: reference???
    \item[Undecidability] ...Paeno-arithmetic % TODO
\end{description}

%% our syntax
We chose the formula syntax of ... specifically to ensure that even static semantics are decidable in polynomial time.
This allowed applying the procedures of AGT directly, as they are based on a decidable type system, i.e. decidable .

\subsection{Impact of NP-hard Verification Predicates}
Let's assume that our rules for static verification indeed contain an NP-hard predicate $P$. (NOTE: need positive occurrence for following reasoning!)
The immediate consequence is that any working verifier would have to realize a conservative approximation of the actual predicate.

Under-approximation: for static guarantees to hold, verifier must under-approximate $P$... blabla

Over-approximation: for (det.) gradual lifting to be ?sound?, it must over-approximate $P$... blabla